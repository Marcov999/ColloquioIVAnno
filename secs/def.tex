\begin{frame}[t]
  \frametitle{Domini strettamente pseudoconvessi e metrica di Kobayashi}
  \only<1-2>{
  Sia $\mathbb{D}$ il disco unitario in $\mathbb{C}$. Su $\mathbb{D}$ possiamo mettere la distanza iperbolica $\omega$, indotta dalla metrica di Poincaré $\frac{|\diff z|}{1-|z|^2}$, che lo rende uno spazio iperbolico. \\

  }
  \only<2>{
  \,

  Setting: fissiamo $\Omega \subseteq \mathbb{C}^n, n \ge 2$ un dominio limitato con bordo $C^2$, cioè esiste $\rho \in C^2(\mathbb{C}^n)$ tale che $\Omega=\{\rho(z)<0\}$ e $\diff\rho\not=0$ in ogni punto di $\partial\Omega$.
  Come $\rho$ si può prendere $-\delta(x)$ per $x \in \Omega$ e $\delta(x)$ per $x \in \mathbb{C}^n\setminus\Omega$, dove $\delta(x)=\text{dist}(x,\partial\Omega)$.}
  \only<3-4>{
      \begin{defn}
        Dato $p \in \partial\Omega$, lo \textit{spazio tangente complesso} a $\partial\Omega$ in $p$ è $H_p\partial\Omega=\{Z \in \mathbb{C}^n \mid \langle \bar{\partial}\rho(p),Z\rangle=0\}$. Diciamo che $\Omega$ è \textit{strettamente pseudoconvesso} se la \textit{forma di Levi}
      $$L_{\rho}(p;Z)=\sum_{\nu,\mu=1}^n \frac{\partial^2\rho}{\partial z_\nu\partial\bar{z}_\mu}(p)Z_\nu\bar{Z}_\mu, \quad Z=(Z_1,\dots,Z_n) \in \mathbb{C}^n$$
      è definita positiva in $H_p\partial\Omega$ per ogni $p \in \partial\Omega$.

      \only<4>{La distanza indotta su $\partial\Omega$ è la \textit{distanza di Carnot-Carathéodory}
      \begin{gather*}
        d_H(p,q)=\inf\left\{\int_0^1L_{\rho}\big(\alpha(t);\dot{\alpha}(t)\big)^{1/2}\diff t \mid \alpha \text{ curva orizzontale tra $p$ e $q$}\right\}.
    \end{gather*}}
  \end{defn}}
  \only<5>{
  \begin{defn}
    Data $f:\mathbb{D} \longrightarrow \mathbb{C}^n$ olomorfa indichiamo con $Df(z)$ il differenziale di $f$ in $z \in \mathbb{D}$. La \textit{metrica di Kobayashi} su $\Omega$ limitato è
    $$K(x;Z)=\inf\{|v| \mid v \in \mathbb{C}, \text{ esiste }f:\mathbb{D} \longrightarrow \Omega$$
    $$\text{ olomorfa con } f(0)=x, Df(0)v=Z\},$$
    che induce la \textit{distanza di Kobayashi} $d_K$.

    Le funzioni olomorfe sono semicontrazioni rispetto a $d_K$.
  \end{defn}
  }
\end{frame}

\begin{frame}[t]
  \frametitle{Domini strettamente pseudoconvessi e iperbolicità di Gromov}
  \only<1-3>{\begin{defn}
    Sia $(X,d)$ uno spazio metrico proprio. Dati $x,y,w \in X$ il \textit{prodotto di Gromov} tra $x$ e $y$ con punto base $w$ è $(x,y)_w=\frac{1}{2}\big(d(x,w)+d(y,w)-d(x,y)\big)$. Dato $\delta \ge 0$, diciamo che $X$ è \textit{$\delta$-iperbolico} se
    $$(x,y)_w \ge \min\{(x,z)_w,(y,z)_w\}-\delta\text{ per ogni }x,y,z,w \in X.$$
    \only<2>{Fissato $w \in X$, il \textit{bordo iperbolico} $\partial_GX$ è costruito come classe di equivalenza delle successioni $(x_i)$ che convergono a infinito, cioè tali che $\displaystyle\lim_{i,j\rightarrow \infty}(x_i,x_j)_w=\infty$; due tali successioni $(x_i), (y_i)$ sono equivalenti se $\displaystyle\lim_{i\rightarrow \infty}(x_i,y_i)_w=\infty$.}
  \end{defn}}
  \only<3>{\begin{block}{Teorema (Balogh-Bonk, 2001)}\begin{itshape} Sia $\Omega$ un dominio limitato e strettamente pseudoconvesso, e sia $d_K$ la distanza di Kobayashi su $\Omega$. Allora $(\Omega,d_K)$ è Gromov-iperbolico, e il bordo iperbolico $\partial_G\Omega$ può essere identificato con il bordo euclideo $\partial\Omega$.
  \end{itshape}\end{block}}
\end{frame}

\begin{frame}
  \frametitle{Conseguenze: estensioni al bordo di funzioni olomorfe}
  \begin{cor}
    Siano $\Omega_1, \Omega_2 \subseteq \mathbb{C}^n$ domini limitati e strettamente pseudoconvessi, e sia $f:\Omega_1 \longrightarrow \Omega_2$ una funzione olomorfa propria. Allora $f$ si estende con continuità a $\bar{f}:\overline{\Omega}_1\longrightarrow\overline{\Omega}_2$ tale che $\bar{f}(\partial\Omega_1)\subseteq\partial\Omega_2$ e la restrizione al bordo è lipschitziana rispetto alle distanze di Carnot-Carathéodory sui bordi.
  \end{cor}
\end{frame}

\begin{frame}[t]
  \frametitle{Conseguenze: dinamica olomorfa}
  \only<1-4>{
  \begin{block}{Teorema (Wolff-Denjoy, 1926)}\begin{itshape}
    (Wolff-Denjoy) Sia $f:\mathbb{D} \longrightarrow \mathbb{D}$ olomorfa e senza punti fissi. Allora esiste un unico $\tau \in \partial\mathbb{D}$ tale che $f^k \longrightarrow \tau$ uniformemente sui compatti.
  \end{itshape}\end{block}
  }
  \only<2-4>{
  Karlsson, 2001: sotto opportune ipotesi, che sono soddisfatte dagli spazi Gromov-iperbolici, valgono dei risultati simili per semicontrazioni dallo spazio in sé.
  }

  \only<3-4>{
  Usando il teorema di Balogh-Bonk e il fatto che le funzioni olomorfe sono delle semicontrazioni rispetto a $d_K$, si ottiene una generalizzazione di Wolff-Denjoy per domini $\Omega$ limitati e strettamente pseudoconvessi.
  }
  \only<4>{
  \begin{block}{Corollario (Abate, 1991)}\begin{itshape}
    Sia $f:\Omega \longrightarrow \Omega$ una funzione olomorfa. Allora vale una delle seguenti: \\
    \begin{enumerate}
      \item le orbite di $f$ sono limitate;
      \item le orbite di $f$ convergono a un punto del bordo.
    \end{enumerate}
  \end{itshape}\end{block}
  }
\end{frame}

\begin{frame}
  \frametitle{Altre applicazioni dell'iperbolicità di Gromov}
  L'iperbolicità di Gromov per domini $\Omega$ segue anche da ipotesi diverse da quelle che abbiamo usato.
  \pause

  Zimmer, 2016: caratterizzazione dei domini convessi che sono Gromov-iperbolici con la metrica di Kobayashi; \pause seguono risultati, analoghi a quelli visti, di estensione al bordo e dinamica olomorfa.
  \pause

  Zimmer, 2022: per i domini limitati convessi Gromov-iperbolici (non necessariamente con bordo regolare) valgono delle stime subellittiche per le soluzioni del problema $\bar{\partial}$-Neumann, già estensivamente studiate per i domini strettamente pseudoconvessi (quindi con bordo regolare).
\end{frame}
