\begin{frame}[t]
  \frametitle{3. La metrica di Kobayashi soddisfa la disuguaglianza}
  \only<1->{\begin{prop}
    Per ogni $\epsilon>0$ esistono $\epsilon_0>0$ e $C \ge 0$ tali che per ogni $x \in \Omega$ con $\delta(x)<\epsilon_0$ e per ogni $Z \in \mathbb{C}^n$ si ha
    \begin{multline*}
      \big(1-C\delta^{1/2}(x)\big)\left(\frac{|Z_N|^2}{4\delta^2(x)}+(1-\epsilon)\frac{L_\rho\big(\pi(x);Z_H\big)}{\delta(x)}\right)^{1/2} \le K(x;Z) \\
      \le \big(1+C\delta^{1/2}(x)\big)\left(\frac{|Z_N|^2}{4\delta^2(x)}+(1+\epsilon)\frac{L_\rho\big(\pi(x);Z_H\big)}{\delta(x)}\right)^{1/2}.
    \end{multline*}
  \end{prop}}
  \only<2-4>{\textit{Traccia della dimostrazione:} si localizza a un intorno di un punto del bordo.}\only<3-4>{ Con un opportuno biolomorfismo, ci si sposta in un insieme che può essere stretto fra due ellissoidi complessi, uno contenuto e uno che lo contiene.}\only<4>{ Per gli ellissoidi complessi, la metrica di Kobayashi può essere calcolata esplicitamente. \qed}
  \only<5>{
  \begin{cor}
    Esiste $C \ge 0$ tale che per ogni $x,y \in \Omega$ si ha
    \begin{equation}\label{stimadistanzakobayashi}
      g(x,y)-C \le d_K(x,y) \le g(x,y)+C. \tag{3}
    \end{equation}
  \end{cor}
  }
\end{frame}

\begin{frame}[t]
  \frametitle{Dimostrazione dell'estensione al bordo}
  \only<1->{
  \begin{cor}
    Siano $\Omega_1, \Omega_2 \subseteq \mathbb{C}^n$ domini limitati e strettamente pseudoconvessi, e sia $f:\Omega_1 \longrightarrow \Omega_2$ una funzione olomorfa propria. Allora $f$ si estende con continuità a $\bar{f}:\overline{\Omega}_1\longrightarrow\overline{\Omega}_2$ tale che $\bar{f}(\partial\Omega_1)\subseteq\partial\Omega_2$ e la restrizione al bordo è lipschitziana rispetto alle distanze di Carnot-Carathéodory sui bordi.
  \end{cor}
  }
  \only<2-3>{
  \textit{Traccia della dimostrazione:} siano $d_1,d_2$ le metriche di Kobayashi su $\Omega_1, \Omega_2$, allora per ogni $x,y \in \Omega_1$ si ha
  $$d_2\big(f(x),f(y)\big)\le d_1(x,y);$$
  }
  \only<3>{
  inoltre, poiché $f$ è propria esiste $C_1 \ge 1$ tale che per ogni $x \in \Omega_1$ abbiamo
  $$\delta_1(x)/C_1 \le \delta_2\big(f(x)\big) \le C_1\delta_1(x),$$
  dove $\delta_j$ è la distanza dal bordo in $\Omega_j$.
  }
  \only<4->{
  Mettendo assieme queste due disuguaglianze e il Corollario, dette $d_H^j$ le rispettive distanze di Carnot-Carathéodory, troviamo che esiste $C_2 \ge 0$ tale che per ogni $x,y \in \Omega_1$ si ha
  $$d_H^2\Big(\pi\big(f(x)\big),\pi\big(f(y)\big)\Big) \le C_2\Big(d_H^1\big(\pi(x),\pi(y)\big)+\max\{\delta_1^{1/2}(x),\delta_1^{1/2}(y)\}\Big).$$
  }
  \only<5>{Da queste disuguaglianze è facile concludere. \qed}
\end{frame}

\begin{frame}
  \frametitle{Fine}
  \begin{center}
    \LARGE Grazie per l'attenzione!
  \end{center}
\end{frame}
