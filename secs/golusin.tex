\marginpar{c'è da farsi tutto il BM per arrivare a questa}
\begin{cor} \label{quasigolusin}
  Sia $f:\mathbb{D} \longrightarrow \mathbb{D}$ una funzione olomorfa che non è un automorfismo. Allora per ogni $z, w \in \mathbb{D}$ vale
  \begin{equation} \label{quasigol}
    d(f^h(z), f^h(w)) \le 2d(z,w).
  \end{equation}
\end{cor}

\marginpar{ricordati di definire $f^h$, con la notazione di BKR, quindi occhio quando scrivi tutti i risultati e le dimostrazioni in BM}

\begin{proof}
  Da scrivere.
\end{proof}

Ponendo $w=0$ in \eqref{quasigol} otteniamo la disuguaglianza di Golusin, che ci servirà per dimostrare il risultato a cui puntiamo.
