Lo scopo di questo paragrafo è mostrare che quasi tutti i domini di $\mathbb{C}$ sono biolomorfi al disco.

\begin{thm}
  (Osgood, Koebe) Sia $\Omega \subset \subset \mathbb{C}$ un dominio limitato, $z_0 \in \Omega$. Allora esiste un unico rivestimento olomorfo $f_0:\mathbb{D} \longrightarrow \Omega$ t.c. $f_0(0)=z_0$ e $f_0'(0) \in \mathbb{R}, f'(0)>0$.
\end{thm}

\begin{proof}
  Possiamo supporre $z_0=0 \in \Omega$ e $\Omega \subset \subset \mathbb{D}$. Sia $\mathcal{F} \subset \text{Hol}(\mathbb{D}, \mathbb{D})$ t.c.
  $\mathcal{F}=\{f \in \text{Hol}(\mathbb{D}, \mathbb{D}) \mid f(0)=0, f'(0) \in \mathbb{R}, f'(0)>0; \Omega \subseteq f(\mathbb{D});$ se $\Omega_f$ è la componente connessa per archi di $f^{-1}(\Omega)$ contenente $0$ allora $f\restrict{\Omega_f}:\Omega_f \longrightarrow \Omega$ è un rivestimento$\}$.
  Se esiste $f_0 \in \mathcal{F}$ con $\Omega_{f_0}=\mathbb{D}$ ci resta da dimostrare solo l'unicità. Poniamo per ogni $f \in \mathcal{F}$ $\displaystyle d_f=\inf_{z \not\in \Omega_f} |z|=\min_{z \in \partial\Omega_f} |z| \le 1$. Abbiamo $d_f=1 \iff \Omega_f=\mathbb{D}$.
  Dobbiamo trovare $f_0 \in \mathcal{F}$ con $d_{f_0}=1$. Sia $\displaystyle d=\sup_{f \in \mathcal{F}} d_f \le 1$. Sia $\{f_n\} \subset \mathcal{F}$ t.c. $d_{f_n} \longrightarrow d$.
  Per il teorema di Montel possiamo supporre che $f_n \longrightarrow f_0 \in \text{Hol}(\mathbb{D}, \mathbb{C})$ con immaggine in $\overline {\mathbb{D}}$. Vogliamo $f_0 \in \mathcal{F}$. Chiaramente $f_0(0)=0, f_0'(0) \in \mathbb{R}, f'(0) \ge 0$.
  \begin{enumerate}
    \item $f_0$ non è costante e $f'(0)>0$: sia $r>0$ t.c. $\mathbb{D}_r \subset \subset \Omega$. Siccome $f_n:\Omega_{f_n} \longrightarrow \Omega$ è un rivestimento e $0 \in \Omega_{f_n}$, esiste un unico $h_n:\mathbb{D}_r \longrightarrow \Omega_{f_n}$ olomorfa t.c. $f_n \circ h_n=\id_{\mathbb{D}_r}$ e $h_r(0)=0$.
    Sempre per il teorema di Montel, a meno di sottosuccessioni possiamo supporre $h_n \longrightarrow h_0 \in \text{Hol}(\mathbb{D}_r, \overline{\mathbb{D}})$ t.c. $f_0 \circ h_0=\id_{\mathbb{D}_r}$.
    Dunque $h_0$ è iniettiva, quindi per il teorema dell'applicazione aperta è aperta, perciò $h_0(\mathbb{D}_r) \subseteq \mathbb{D}$, e $f_0$ non è costante, dunque per il primo teorema di Hurwitz $f_0(\mathbb{D}) \subseteq \mathbb{D}$. $1=\id_{\mathbb{D}_r}'(0)=(f_0 \circ h_0)'(0)=f_0'(h_0(0))\cdot h_0'(0)=f_0'(0) \cdot h_0'(0) \implies f_0'(0) \not=0 \implies f_0'(0)>0$.
    \item $f_0(\Omega_{f_0})=\Omega$ dove $\Omega_{f_0}$ è la componente connessa di $f_0^{-1}(\Omega)$ contenente $0$. Sia infatti $z_0 \in \Omega$ e sia $\gamma$ una curva in $\Omega$ da $0$ a $z_0$.
    Ricopriamo $\gamma$ con dischi $D_0=\mathbb{D}_r, D_1, \dots, D_k$ con $D_j \cap D_{j+1} \not=\emptyset$ e $z_0 \in D_k$. Sia $h_{n, 0}$ l'inversa di $f_n$ su $D_0$ t.c. $h_{n,0}=0$.
    Per ogni $j$ sia $h_{n, j}$ l'inversa di $f_n$ su $D_j$ che coincide con $h_{n, j-1}$ su $D_{j-1} \cap D_j$. Per Montel, a meno di sottosuccessioni $h_{n, 0} \longrightarrow h_{0, 0} \in \text{Hol}(D_0, \mathbb{D})$.
    Per il teorema di Vitali, $h_{n, 1} \longrightarrow h_{0, 1} \in \text{Hol}(D_1, \mathbb{D})$ e, per induzione, $h_{n, k} \longrightarrow h_{0, k} \in \text{Hol}(D_k, \mathbb{D})$ con $f_0 \circ h_{0, k}=\id_{D_k} \implies f_0(h_{0, k}(z_0))=z_0 \implies z_0 \in f_0(\mathbb{D})$.
    In realtà $h_{0, k}(z_0) \in \Omega_{f_0}$ perché è immagine della curva ottenuta con $h_{0, j} \circ \gamma$ che parte da $0$ e quindi è contenuta nella componente connessa di $\Omega$ contenete $0$.
    \item $f_0:\Omega_{f_0} \longrightarrow \Omega$ è un rivestimento.
  \end{enumerate}
\end{proof}
