\begin{lm} \label{succ_K_nu}
  Sia $\Omega \subseteq \mathbb{C}$ aperto. Allora esiste una successione crescente $\{K_{\nu}\}$ di compatti in $\Omega$ t.c.: $K_{\nu}\subset \mathop {K_{\nu+1}}\limits^ \circ$, $\displaystyle \bigcup_{\nu} K_{\nu}=\Omega$ e $\widehat{(K_{\nu})}_{\Omega}=K_{\nu}$.
\end{lm}

\begin{proof}
  Poniamo $H_{\nu}=\{z \in \Omega \mid  d(z, \partial\Omega) \ge 1/\nu, |z| \le \nu\}$. $H_{\nu}$ è compatto, $H_{\nu}\subset \mathop {H_{\nu+1}}\limits^ \circ$ e $\displaystyle \bigcup_{\nu} H_{\nu}=\Omega$.
  Poniamo $K_1=\widehat{(H_1)}_{\Omega}$, che è compatto. Sia $\mu_1$ t.c. $K_1 \subset \mathop {H_{\mu_1}}\limits^ \circ$ e poniamo $K_2=\widehat{(H_{\mu_1})}_{\Omega}$. Continuando così abbiamo i $K_{\nu}$.
\end{proof}

\begin{thm}
  (Malgrange) Sia $\Omega \subseteq \mathbb{C}$ aperto, $\varphi \in C^{\infty}(\Omega)$. Allora esiste $u \in C^{\infty}(\Omega)$ t.c. $\dfrac{\partial u}{\partial\bar{z}}=\varphi$ $(\star)$ su $\Omega$.
\end{thm}

\begin{proof}
  Sappiamo che se $K \subset\subset \Omega$ compatto, allora esiste $v \in C^{\infty}(\Omega)$ che risolve $(\star)$ in un intorno di $K$. Infatti sia $\alpha \in C^{\infty}_C(\mathbb{C})$ con $\alpha \equiv 1$ in un intorno di $K$ e $0$ fuori da un intorno compatto e applichiamo quanto sappiamo a $\alpha\varphi$.
  Sia $\{K_{\nu}\}$ data dal lemma \ref{succ_K_nu}. Per ogni $\nu$ sia $v_{\nu} \in C^{\infty}(\Omega)$ soluzione di $(\star)$ in un intorno di $K_{\nu}$. Osserviamo che $v_{\nu+1}-v_{\nu} \in \mathcal{O}(K_{\nu})$.
  Per il primo teorema di Runge, esiste $h_{\nu} \in \mathcal{O}(\Omega)$ t.c. $\|v_{\nu+1}-v_{\nu}-h_{\nu}\|_{K_{\nu}}<2^{-\nu}$.
  Poniamo $\displaystyle u_{\nu}=v_{\nu}+\sum_{\mu \ge \nu} (v_{\mu+1}-v_{\mu}-h_{\mu})-\sum_{\mu=1}^{\nu-1} h_{\mu}$ su $K_{\nu}$. La serie è in $\mathcal{O}(K_{\nu})$ e la somma finita è in $\mathcal{O}(\Omega)$, quindi $u_{\nu}$ risolve $(\star)$ in un intorno di $K_{\nu}$.
  Ora, $u_{\nu}$ non dipende da $\nu$: $\displaystyle u_{\nu+1}=v_{\nu+1}+\sum_{\mu \ge \nu+1} (v_{\mu+1}-v_{\mu}-h_{\mu})-\sum_{\mu=1}^{\nu} h_{\mu}=v_{\nu}+(v_{\nu+1}-v_{\nu}-h_{\nu})+\sum_{\mu \ge \nu+1} (v_{\mu+1}-v_{\mu}-h_{\mu})-\sum_{\mu=1}^{\nu-1} h_{\mu}=v_{\nu}+\sum_{\mu \ge \nu} (v_{\mu+1}-v_{\mu}-h_{\mu})-\sum_{\mu=1}^{\nu-1} h_{\mu}=u_{\nu}$.
  Dunque abbiamo definito $u \in C^{\infty}(\Omega)$ ponendo $u\restrict{\mathop {K_{\nu}}\limits^ \circ}=u_{\nu}\restrict{\mathop {K_{\nu}}\limits^ \circ}$ e allora $\dfrac{\partial u}{\partial \bar{z}}\equiv \varphi$ su $\Omega$ in quanto vale su ciascun $K_{\nu}$.
\end{proof}

\begin{thm}
  (Mittag-Leffler) Sia $\Omega \subseteq \mathbb{C}$ aperto, $E \subset \Omega$ discreto e chiuso. Sia per ogni $a \in E$ $p_a \in \mathcal{O}(\mathbb{C}\setminus\{a\})$. Allora esiste $f \in \mathcal{O}(\Omega\setminus E)$ t.c. $f-p_a$ sia olomorfa in $a$ per ogni $a \in E$. In particolare, possiamo trovare $f \in \mathcal{M}(\Omega)$ con parti principali descritte.
\end{thm}

\begin{proof}
  Sia $\{K_{\nu}\}$ data dal lemma \ref{succ_K_nu}. Per ogni $\nu \ge 1$ poniamo $\displaystyle g_{\nu}=\sum_{a \in E \cap K_{\nu}} p_a$ (è una somma finita).
  Ora, $\displaystyle g_{\nu+1}-g_{\nu}=\sum_{a \in E \cap (K_{\nu+1}\setminus K_{\nu})} p_a \in \mathcal{O}(K_{\nu})$.
  Per il primo teorema di Runge, esiste $h_{\nu} \in \mathcal{O}(\Omega)$ t.c. $\|g_{\nu+1}-g_{\nu}-h_{\nu}\|_{K_{\nu}} \le 2^{-\nu}$.
  Sia $\displaystyle f=g_{\nu}+\sum_{\mu \ge \nu} (g_{\mu+1}-g_{\mu}-h_{\mu})-\sum_{\mu=1}^{\nu-1} h_{\mu}$.
  Come nella dimostrazione del teorema di Malgrange, $f$ non dipende da $\nu$ $\implies$ $f \in \mathcal{O}(\Omega\setminus E)$ (infatti, per ogni $\nu$, gli unici punti di $K_{\nu}$ dove $f$ può non essere olomorfa sono quelli di di $g_{\nu}$, dunque per definizione quelli di $E$).
  Sia $a \in E$ e sia $\nu \ge 1$ t.c. $a \in K_{\nu}$. La serie appartiene a $\mathcal{O}(K_{\nu})$, la somma finita appartiene a $\mathcal{O}(\Omega)$ $\implies$ $f-p_a=g_{\nu}-p_a+$qualcosa olomorfo in $K_{\nu}$ $\implies$ $f-p_a$ è olomorfa vicino ad $a$.
\end{proof}

\begin{cor}
  Sia $\Omega \subseteq \mathbb{C}$ aperto, $E \subset \Omega$ discreto e chiuso in $\Omega$. Siano dati per ogni $a \in E$ un intorno aperto $U_a \subset \Omega$ di $a$ e $\varphi_a \in \mathcal{O}(U_a \setminus \{a\})$. Allora esiste $f \in \mathcal{O}(\Omega\setminus E)$ t.c. $f-\varphi_a$ sia olomorfa vicino ad $a$ per ogni $a \in E$.
\end{cor}

\begin{proof}
  Sia $p_a$ la parte principale dello sviluppo di Laurent di $\varphi_a$ in $a$ $\implies$ $p_a \in \mathcal{O}(\mathbb{C}\setminus \{a\})$ e $\varphi_a-p_a$ è olomorfa in un intorno di $a$. Allora basta prendere $f$ data del teorema di Mittag-Leffler perché $f-\varphi_a=(f-p_a)-(\varphi_a-p_a)$.
\end{proof}

\begin{lm} \label{z-a/z-b}
  Sia $\Omega \subset \mathbb{C}$ aperto, $a, b \in \mathbb{C}\setminus\Omega$ appartenenti alla stessa comoponente connessa di $\mathbb{C}\setminus\Omega$. Allora esiste $f \in \mathcal{O}(\Omega)$ t.c. $e^{f(z)}=\dfrac{z-a}{z-b}$.
\end{lm}

\begin{proof}
  Per esercizio.
\end{proof}

\begin{thm}
  (Quarto teorema di Runge) Sia $\Omega \subseteq \mathbb{C}$ aperto, $K\subset\subset\Omega$ compatto t.c. $\widehat{K}_{\Omega}=K$. Sia $f \in \mathcal{O}(\Omega)(K)$ t.c. $f(z) \not=0$ per ogni $z \in K$. ù
  Allora per ogni $\epsilon>0$ esiste $F \in \mathcal{O}(\Omega)$ con $F(z) \not=0$ per ogni $z \in \Omega$ e $\|F-f\|_K<\epsilon$.
\end{thm}

\begin{proof}
  Siccome $f$ non si annulla su $K$, $\displaystyle \delta_0=\min_{z \in K} |f(z)|>0$. Quindi se $\tilde{f} \in \mathcal{O}(\Omega)$ t.c. $\|\tilde{f}-f\|_K<\delta_0/2$ allora $\tilde{f}(z)\not=0$ per ogni $z \in K$.
  Sappiamo che $\mathbb{C}\setminus K$ ha una componente connessa illimitata $U_0$ e un numero finito di componenti connesse limitate $U_1, \dots, U_p$ con $U_j \not\subset \Omega$; scegliamo per ogni $j=1, \dots, p$ $a_j \in U_j \setminus \Omega$. Possiamo approssimare $f$ con una funzione razionale $\tilde{f}$ con poli fuori da $K$; per l'osservazione fatta all'inizio della dimostrazione possiamo supporre che $\tilde{f}$ non si annulli mai in un intorno di $K$.
  Quindi $\displaystyle \tilde{f}(z)=c\prod_{\nu=1}^d (z-b_{\nu})^{m_{\nu}}$ con $c \in \mathbb{C}^*, m_{\nu} \in \mathbb{Z}^*, b_{\nu} \in \mathbb{C}\setminus K$. Fissiamo $R>0$ t.c. $K \subset\subset D(0,R)$ e poniamo $a_0=R \in U_0$.
  Per $j=0, \dots, p$ sia $A_j=\{\nu \mid b_{\nu} \in U_j\}$.
  Scriviamo $\displaystyle \tilde{f}(z)=cG(z)(z-R)^{n_0} \prod_{j=0}^p \prod_{\nu \in A_j} \left(\frac{z-b_{\nu}}{z-a_j}\right)^{m_{\nu}}$ dove $\displaystyle n_{j}=\sum_{\nu \in A_j} m_{\nu}$ e $\displaystyle G(z)=\prod_{j=1}^p (z-a_j)^{n_j}$.
  Se $\nu \in A_j$ allora $a_j$ e $b_{\nu}$ appartengono alla stessa componente connessa di $\mathbb{C} \setminus K$, dunque per il lemma \ref{z-a/z-b} esiste $\varphi_{\nu, j} \in \mathcal{O}(K)$ t.c. $\dfrac{z-b_{\nu}}{z-a_j}=e^{\varphi_{\nu, j}(z)}$.
  Inoltre esiste $\varphi_0 \in \mathcal{O}(D(0, R))$ t.c. $z-R=\exp(\varphi_0(z))$. Quindi esiste $h \in \mathcal{O}(K)$ t.c. $\tilde{f}(z)=cG(z)e^{h(z)}$. Per il primo teorema di Runge, per ogni $\delta>0$ esiste $H \in \mathcal{O}(\Omega)$ t.c. $\|H-h\|_K<\delta$.
  Poniamo $F=cGe^H \in \mathcal{O}(\Omega)$ e mai nulla su $\Omega$; inoltre $\|\tilde{f}-F\|_K=|c|\|G\|_K\|e^H-e^h\|_K$. A patto di prendere $\delta<<1$ possiamo rendere $\|\tilde{f}-F\|_K$ piccolo quanto vogliamo e quindi $\|f-F\|_K$ piccolo quanto vogliamo.
\end{proof}

\begin{exc}
  Sia $\{u_n\}$ una successione di funzioni complesse limitate definite su un insieme $S$ t.c. $\displaystyle \sum_{n} |u_n|$ converge uniformemente su $S$. Allora il prodotto $\displaystyle f(z)=\prod_{n=1}^{+\infty} (1+u_n(z))$ converge uniformemente su $S$ e $f(z_0)=0$ $\iff$ esiste $n \ge 1$ t.c. $u_n(z_0)=-1$.
\end{exc}

\begin{thm}
  (Weierstrass) Sia $\Omega \subseteq \mathbb{C}$ aperto, $E \subset \Omega$ discreto e chiuso in $\Omega$, $k:E \longrightarrow \mathbb{Z}$. Allora esiste $f \in \mathcal{M}(\Omega)$ t.c. $f \in \mathcal{O}(\Omega \setminus E)$, $f$ non ha zeri in $\Omega \setminus E$ e $(z-a)^{-k(a)}f(z)$ sia olomorfa mai nulla in un intorno di $a$ per ogni $a \in E$.
\end{thm}

\begin{proof}
  Sia $\{K_{\nu}\}$ data dal lemma \ref{succ_K_nu}. Poniamo $\displaystyle F_{\nu}(z)=\prod_{s \in E \cap K_{\nu}} (z-a)^{k(a)}$. In particolare, $F_{\nu+1}/F_{\nu} \in \mathcal{O}(K_{\nu})$ e non si annulla in $K_{\nu}$.
  Sia $\displaystyle \delta_{\nu}=\min_{z \in K_{\nu}} \left|\frac{F_{\nu+1}(z)}{F_{\nu}(z)}\right|>0$.
  Sia $g_{\nu} \in \mathcal{O}(\Omega)$ data dal quarto teorema di Runge mai nulla in $\Omega$ t.c. $\displaystyle \left\|\frac{F_{\nu+1}}{F_{\nu}}-g_{\nu}\right\|_{K_{\nu}}<\frac{2^{-\nu-1}\delta_{\nu}}{1+2^{-\nu-1}}$
  $\implies$ per ogni $z \in K_{\nu}$ $\displaystyle |g_{\nu}(z)| \ge \left|\frac{F_{\nu+1}(z)}{F_{\nu}(z)}\right|-\frac{2^{-\nu-1}\delta_{\nu}}{1+2^{-\nu-1}} \ge \delta_v-\frac{2^{-\nu-1}\delta_{\nu}}{1+2^{-\nu-1}}=\frac{\delta_{\nu}}{1+2^{-\nu-1}}$.
  Ponendo $h_{\nu}=\dfrac{1}{g_{\nu}} \in \mathcal{O}(\Omega)$ mai nulla in $\Omega$, $\left \|\dfrac{F_{\nu+1}}{F_{\nu}}h_{\nu}-1\right\|_{K_{\nu}}<2^{-\nu-1}$.
  Poniamo $\displaystyle f=F_{\nu}\prod_{\mu\ge\nu}\left(\frac{F_{\mu+1}}{F_{\mu}}h_{\mu}\right) \prod_{j=1}^{\nu-1}h_j$. $f\restrict{K_{\nu}}$ ha esattamente gli stessi zeri e poli di $F_{\nu}$; ma sempre per la stessa dimostrazione, $f$ non dipende da $\nu$. Quindi $f \in \mathcal{M}(\Omega)$ ed è come voluto.
\end{proof}

\begin{cor}
  Sia $\Omega \subset \mathbb{C}$ aperto. Allora ogni $q \in \mathcal{M}(\Omega)$ è il quoziente di due funzioni olomorfe in $\Omega$.
\end{cor}

\begin{proof}
  $E=\{$poli di $q\}$. Sia $k:E \longrightarrow \mathbb{Z}$ data da $k(a)=-ord_a(q)$. Allora il teorema di Weierstrass ci fornisce $g \in \mathcal{O}(\Omega)$ t.c. $gq \in \mathcal{O}(\Omega)$ $\implies$ $q=(gq)/g$ come voluto.
\end{proof}
