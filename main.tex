\documentclass{article}
\usepackage{mstyle}

\title{Appunti di Elementi di Analisi Complessa}
\date{}
\author{Marco Vergamini}

\begin{document}
\maketitle
\newpage
\tableofcontents
\newpage


\section*{Introduzione}
\addcontentsline{toc}{section}{Introduzione}
Da scrivere alla fine


\newpage

\section{Funzioni olomorfe in una variabile}

\subsection{Notazioni e prerequisiti}
Notazioni: $z=x+iy \in \mathbb{C}$ indica un numero complesso, $\bar{z}=x-iy$ il suo complesso coniugato. Con il termine \textit{dominio} si intende un aperto connesso. $\mathcal{O}(\Omega)=\{f: \Omega \rightarrow \mathbb{C} | f \text{ è olomorfa} \}$.
$\text{Hol}(\Omega_1, \Omega_2)=\{f: \Omega_1 \rightarrow \Omega_2 | f \text{ è olomorfa} \}$.
$\dfrac{\partial}{\partial z}=\dfrac{1}{2}\left(\dfrac{\partial}{\partial x}-i\dfrac{\partial}{\partial y}\right), \dfrac{\partial}{\partial \bar{z}}=\dfrac{1}{2}\left(\dfrac{\partial}{\partial x}+i\dfrac{\partial}{\partial y}\right)$. \\
$\diff z=\diff x+i\diff y, \diff\bar{z}=\diff x-i\diff y, \diff z\left(\dfrac{\partial}{\partial z}\right)=1, \diff z\left(\dfrac{\partial}{\partial \bar{z}}\right)=0$. \\

Daremo ora una definizione di funzione olomorfa basata su quattro definizioni, l'equivalenza delle quali è un prerequisito del corso e dovrebbe essere quindi nota agli studenti.

\begin{defn}
  Sia $\Omega \subseteq \mathbb{C}$ un aperto, $f: \Omega \in \mathbb{C}$ si dice \textsc{olomorfa} se vale una delle seguenti condizioni equivalenti:
  \begin{nlist}
      \item $f$ è \textit{$\mathbb{C}$-differenziabile}, cioè per ogni $a \in \Omega$ esiste $\displaystyle f'(a)=\lim_{z \rightarrow a}=\frac{f(z)-f(a)}{z-a}$;
      \item $f$ è \textit{analitica}, cioè per ogni $a \in \Omega$ esiste $U \subseteq \Omega$ aperto e intorno di $a$ e $\{c_n\}\subset \mathbb{C}$ t.c. per ogni $z \in U$ $\displaystyle f(z)=\sum_{n=0}^{+\infty} c_n(z-a)^n$;
      \item $f$ è \textit{olomorfa}, cioè $f$ è continua, $\partial f/\partial x$ e $\partial f/\partial y$ esistono su $\Omega$ e $\dfrac{\partial f}{\partial x}+i\dfrac{\partial f}{\partial y} \equiv 0$ (equazione di Cauchy-Riemann).
      Si noti che la condizione è $\dfrac{\partial f}{\partial \bar{z}} \equiv 0$, da cui si ricava $\dfrac{\partial f}{\partial z}=f'$;
      \item $f$ è continua e per ogni rettangolo (o disco) chiuso $D \subseteq \Omega$ si ha $\displaystyle \int_{\partial D} f\diff z=0$ (teorema di Cauchy-Goursat+Morera).
  \end{nlist}
\end{defn}

La seguente proposizione è anch'essa un risultato che dovrebbe essere noto agli studenti che seguono il corso.

\begin{prop}
  Sia $\{c_n\} \in \mathbb{C}$. Allora:
  \begin{nlist}
    \item esiste $R \in [0, +\infty]$ t.c. $\displaystyle \sum_{n=0}^{+\infty} c_nz^n$ converge per $|z|<R$ e diverge per $|z|>R$. $R$ è detto \textit{raggio di convergenza}. La convergenza +è uniforme su $\Delta_r=\{|z| \le r\}, r<R$. $\displaystyle \limsup_{n \rightarrow +\infty} |c_n|^{1/n}=\frac{1}{R}$;
    \item $\displaystyle \sum_{n=0}^{+\infty} nc_nz^{n-1}$ ha lo stesso raggio di convergenza;
    \item se $\displaystyle f(z)=\sum_{n=0}^{+\infty} c_n(z-a)^n$ allora $f'(z)=\sum_{n=0}^{+\infty} nc_n(z-a)^{n-1}$;
    \item se $f \in \mathcal{O}(\Omega)$ e $a \in \Omega$, allora $\displaystyle f(z)=\sum_{n=0}^{+\infty} \frac{1}{n!}f^{(n)}(a)(z-a)^n$. Questa formula è valida nel più grande disco aperto centrato in $a$ e contenuto in $\Omega$, cioè di raggio minore o uguale di $d(a, \partial \Omega)$.
  \end{nlist}
\end{prop}

\begin{thm}
  (Formula di Cauchy) Sia $\Omega$ aperto, $f \in \mathcal{O}(\Omega), D \subseteq \Omega$ disco/rettangolo chiuso. Per ogni $\displaystyle a \in D, f(a)=\frac{1}{2\pi i} \int_{\partial D} \frac{\zeta}{\zeta-a} \diff\zeta$.
  Si ha che $\displaystyle f^{(n)}(a)=\frac{n!}{2\pi i} \int_{\partial D} \frac{f(\zeta)}{(\zeta-a)^{n+1}}\diff \zeta$.
\end{thm}

\begin{cor}
  (Disuguaglianze di Cauchy) $f \in \mathcal{O}(\Omega), D=D(a, r) \subseteq \Omega$ disco di centro $a \in \Omega$ e raggio $r>0$. Sia $\displaystyle M=\max_{\zeta \in \partial D} |f(\zeta)|$. Allora per ogni $n \ge 1, |f^{(n)}(a)| \le \dfrac{n!}{r^n}M$.
\end{cor}

\begin{cor}
  (Teorema di Liouville) Sia $f \in \mathcal{O}(\mathbb{C})$ limitata. Allora $f$ è costante.
\end{cor}

\begin{proof}
  Le disuguaglianze di Cauchy danno, per ogni $r>0$, $|f'(a)| \le \dfrac{M}{r}$ dove $\displaystyle M=\sup_{z \in \mathbb{C}} |f(z)|<+\infty \implies
  f' \equiv 0$.
\end{proof}

\begin{thm}
  (Principio di identità o del prolungamento analitico) $\Omega \subseteq \mathbb{C}$ dominio, $f, g \in \mathcal{O}(\Omega)$. Se $\{z \in \Omega | f(z)=g(z)\}$ ha un punto di accumulazione in $\Omega$, allora $f \equiv g$.
\end{thm}

\begin{cor} \label{olo_discr}
  $\Omega \subseteq \mathbb{C}$ dominio, $f \in \mathcal{O}(\Omega)$ non identicamente nulla, allora $\{z \in \Omega | f(z)=0\}$ è discreto in $\Omega$.
\end{cor}

\begin{thm} \label{pr_max}
  (Principio del massimo) $\Omega \subseteq \mathbb{C}$ dominio, $f \in \mathcal{O}(\Omega)$. Allora:
  \begin{nlist}
    \item se $U$ è aperto e $U \subset \subset \Omega$ (si legge "$U$ relativamente compatto in $\Omega$" e si intende $\overline{U} \subset \Omega$ e $\overline{U}$ compatto) allora $\displaystyle \sup_{z \in U} |f(z)| \le \sup_{z \in \partial D} |f(z)|$. Inoltre, se $|f|$ ha un massimo locale in $U$, allora $f$ è costante in $\Omega$;
    \item la stessa affermazione vale per $\mathfrak{Re} f$ e $\mathfrak{Im} f$;
    \item se $\Omega$ è limitato poniamo $\displaystyle M=\sup_{x \in \partial D} \limsup_{z \rightarrow x} |f(z)| \in [0, +\infty]$. Allora per ogni $z \in \Omega$ $|f(z)| \le M$ con uguaglianza in un punto se e solo se $f$ è costante.
  \end{nlist}
\end{thm}

\begin{ex}
  Controesempio per vedere che serve $\Omega$ limitato per il punto (iii) del teorema \ref{pr_max}: $\Omega=\{z \in \mathbb{C} | \mathfrak{Re} z>0\}, f(z)=e^z$. $f \in \mathcal{O}(\mathbb{C}) \subset \mathcal{O}(\Omega)$.
  $z \in \partial\Omega \implies z=iy \implies |f(iy)|=|e^{iy}|=1$, ma $f$ è illimitata in $\Omega$. Per correggere questa cosa si aggiunge il punto all'infinito.
\end{ex}

\begin{thm}
  (Applicazione aperta) $f \in \mathcal{O}(\Omega)$ non costante $\implies$ $f$ è un'applicazione aperta.
\end{thm}

Siano $X, Y$ spazi topologici e indichiamo con $C^0(X, Y)$ le funzioni continue da $X$ in $Y$.

La \textit{topologia della convergenza puntuale} è la restrizione a $C^0(X, Y) \subset Y^X=\{f:X \rightarrow Y\}$ della topologia prodotto. Una prebase è data da $\mathcal{F}(x, U)=\{f \in C^0(X, Y) |f(x) \in U\}$ dove $x \in X$ e $U \subseteq Y$ è un aperto.

\begin{exc}
  $f_n \rightarrow f \in C^0(X, Y)$ per questa topologia se e solo se $f_n(x) \rightarrow f(x)$ per ogni $x \in X$.
\end{exc}

La \textit{topologia compatta-aperta} ha invece come prebase $\mathcal{F}(K, U)= \\ =\{f \in C^0(X, Y) | f(K) \subseteq U\}$ dove $U$ è preso come sopra e $K \subseteq X$ è un compatto.

\begin{prop}
  \begin{nlist}
    \item La topologia compatta-aperta è più fine della topologia della convergenza puntuale;
    \item $Y$ Hausdorff $\implies$ topologia compatta aperta Hausdorff.
  \end{nlist}
\end{prop}

\begin{proof}
  \begin{nlist}
    \item Ovvia (il singoletto è un compatto).
    \item Prendiamo $f \not\equiv g$ continue, allora esiste $x_0 \in X$ t.c. $f(x_0) \not= g(x_0)$, per cui, dato che $Y$ è Hausdorff,
    esistono $U, V \subset Y$ aperti disgiunti con $f(x_0) \in U, g(x_0) \in V \implies f \in \mathcal{F}(x_0, U), g \in \mathcal{F}(x_0, V), \mathcal{F}(x_0, U) \cap \mathcal{F}(x_0, V)=\emptyset$.
  \end{nlist}
\end{proof}

\begin{thm}
  (Ascoli-Arzelà) Siano $X, Y$ spazi metrici con $X$ localmente compatto, allora $\mathcal{F} \subseteq C^0(X, Y)$ è relativamente compatta rispetto alla topologia compatta-aperta se e solo se:
  \begin{nlist}
    \item per ogni $x \in X$ $\{f(x) | f \in \mathcal{F}\} \subset \subset Y$;
    \item $\mathcal{F}$ è equicontinua.
  \end{nlist}
\end{thm}

La topologia compatta-aperta viene detta anche topologia della \textit{convergenza uniforme sui compatti}: $\{f_n\} \subset C^0(X, \mathbb{R}^N)$. Se $K \subseteq X$ definiamo $\displaystyle \|f\|_K=\sup_{z \in K} \|f(z)\|$.
$f_n \rightarrow f$ uniformemente sui compatti se per ogni $K \subset \subset X$ compatto e per ogni $\epsilon>0$ esiste $n_0$ t.c. $n \ge n_0 \implies \|f_n-f\|_K<\epsilon$.

\begin{exc}
  $f_n \rightarrow f$ uniformemente sui compatti se e solo se $f_n \rightarrow f$ nella topologia compatta-aperta.
\end{exc}


\subsection{Risultati preliminari}
Vediamo ora alcuni risultati e definizioni preliminari, da considerarsi comunque come prerequisiti per altri risultati più interessanti che vedremo più avanti nel corso.

\begin{thm}
  (Weierstrass) Sia $\{f_n\} \subset \mathcal{O}(\Omega)$ t.c. $f_n \rightarrow f \in C^0(\Omega, \mathcal{C})$ uniformemente sui compatti. Allora:
  \begin{nlist}
    \item $f \in \mathcal{O}(\Omega)$;
    \item $f_n' \rightarrow f'$ uniformemente sui compatti.
  \end{nlist}
\end{thm}

\begin{proof}
  \begin{nlist}
    \item Sia $a \in \Omega$, $0<r<d(a, \partial\Omega)$ t.c. $D=D(a, r) \subset \subset \Omega$. $\displaystyle f_n(z)=\frac{1}{2\pi i} \int_{\partial D} \frac{f_n(\zeta)}{\zeta-z} \diff \zeta$ per ogni $z \in D(a, \rho)$ per ogni $0<\rho<r$.
    Allora $\dfrac{1}{|\zeta-z|} \le \dfrac{1}{r-\rho}$ per ogni $z \in D(a, \rho), \zeta \in \partial{D}$.
    Per ogni $\displaystyle z \in D(a, \rho), f(z)=\lim_{n \rightarrow +\infty} \frac{1}{2\pi i} \int_{\partial D} \frac{f_n(\zeta)}{\zeta-z} \diff \zeta$.
    Adesso, per uniforme convergenza e uniforme limitatezza si può portare il limite dentro, perciò $\displaystyle f(z)=\frac{1}{2\pi i}\int_{\partial D} \frac{f(\zeta)}{\zeta-z} \diff \zeta$, ma questo, per il teorema di Cauchy-Goursat+Morera, implica $f \in \mathcal{O}(\Omega)$.
    \item $\displaystyle f_n'(z)=\frac{1}{2\pi i} \int_{\partial D} \frac{f_n'(\zeta)}{(\zeta-z)^2}\diff\zeta \rightarrow \frac{1}{2\pi i} \int_{\partial D} \frac{f(\zeta)}{\zeta-z} \diff\zeta=f'(z)$.
    $f_n' \rightarrow f'$ uniformemente sui dischi e ogni compatto è coperto da un numero finito di dischi $\implies$ $f_n' \rightarrow f'$ uniformemente sui compatti.
  \end{nlist}
\end{proof}

\begin{thm}
  (Montel) $\Omega \subseteq \mathbb{C}$ aperto, $\mathcal{F} \subseteq \mathcal{O}(\Omega)$ t.c. per ogni $K \subset \subset C$ compatto esiste $M_K>0$ t.c. $\|f\|_K \le M_K$ per ogni $f \in \mathcal{F}$ (si diche che $\mathcal{F}$ è \textit{uniformemente limitata sui compatti}).
  Allora $\mathcal{F}$ è relativamente compatta in $\mathcal{O}(\Omega)$.
\end{thm}

\begin{proof}
  Basta vedere che ogni successione $\{f_n\} \subseteq \mathcal{F}$ ha una sottosuccessione convergente (segue dal fatto che, nelle ipotesi del teorema di Montel, la topologia compatta aperta è metrizzabile). \\
  Dati $a \in \Omega, 0<r<d(a, \partial \Omega), f \in \mathcal{O}(\Omega)$, sia $c_n(f)=\dfrac{f^{(n)}(a)}{n!}$, allora $\displaystyle f(z)=\sum_{n=0}^{+\infty} c_n(f)(z-a)^n$ in $\overline{D(a, r)}$.
  Inoltre, se $\|f\|_{\overline{D(a, r)}} \le M$, allora per le disuguaglianze di Cauchy $|c_n(f)| \le \dfrac{M}{r^n}$ per ogni $n \ge 0$.
  Sia $\{f_n\} \subseteq \mathcal{F}$. Per ipotesi, esiste $M$ t.c. $\|f_n\|_{\overline{D(a, r)}} \le M$ per ogni $n$ $\implies$ $|c_0(f_n)| \le M$ per ogni $n$ $\implies$ esiste una sottosuccessione $c_0(f_{n_j^{(0)}})$ che tende a $c_0 \in \mathbb{C}$.
  Per  induzione, da $\{f_{n_j^{(k-1)}}\}$ possiamo estrarre una sottosuccessione $\{f_{n_j^{(k)}}\}$ t.c. $c_k(f_{n_j^{(k)}}) \rightarrow c_k \in \mathbb{C}$. Consideriamo $\{f_{n_j^{(j)}}\}$, allora $c_k(f_{n_j^{(j)}}) \rightarrow c_k \in \mathbb{C}$ per ogni $k$.
  Sia $f_{\nu_j}=f_{n_j^{(j)}}$. Poniamo $D_a=\overline{D(a, r/2)}$ e sia $z \in D_a$. Vogliamo $\displaystyle f_{\nu_j} \rightarrow f(z)=\sum_{n=0}^{+\infty} c_n(z-a)^n$ in $D_a$. Basta vedere che $f_{\nu_j}$ è di Cauchy uniformemente in $D_a$.
  $\displaystyle |f_{\nu_h}(z)-f_{\nu_k}(z)| \le \sum_{n=0}^{+\infty} |c_n(f_{\nu_h})-c_n(f_{\nu_k})||z-a|^n=\sum_{n=0}^N |c_n(f_{\nu_h})-c_n(f_{\nu_k})||z-a|^n+\sum_{n>N} |c_n(f_{\nu_h})-c_n(f_{\nu_k})||z-a|^n$.
  Sappiamo che $|c_n(f_{\nu_k})| \le \dfrac{M}{r^n}$ e $z \in D_a \implies |z-a| \le \dfrac{r}{2}$.
  Allora $\displaystyle \sum_{n>N} |c_n(f_{\nu_h})-c_n(f_{\nu_k})||z-a|^n \le \sum_{n>N} \frac{2M}{r^n}\left(\frac{r}{2}\right)^n=\frac{M}{2^{N-1}}$.
  $\displaystyle \sum_{n=0}^{+\infty} |c_n(f_{\nu_h})-c_n(f_{\nu_k})||z-a|^n \le \sum_{n=0}^{+\infty} |c_n(f_{\nu_h})-c_n(f_{\nu_k})|\left(\dfrac{r}{2}\right)^n$.
  Dato $\epsilon>0$, scegliamo $N>>1$ t.c. $\dfrac{M}{2^{N-1}}<\epsilon/2$ e $n_0$ t.c. per ogni $h, k \ge n_0, |c_n(f_{\nu_h})-c_n(f_{\nu_k})|\left(\dfrac{r}{2}\right)^n<\dfrac{\epsilon}{2(N+1)}$
  (possiamo farlo, una volta fissato $N$, perché gli $n$ tra $0$ e $N$ sono in numero finito e le successioni $c_n(f_{\nu_j})$ convergono, dunque si sceglie un indice per ogni successione e si prende come $n_0$ il massimo di questi indici).
  Mettendo insieme le disuguaglianze si ha che per ogni $\epsilon>0$ esiste $n_0$ t.c. per ogni $h, k \ge n_0$ e per ogni $z \in D_a$, $|f_{\nu_k}(z)-f_{\nu_k}(z)|<\epsilon$, dunque la sottosuccessione $f_{\nu_j}$ è di Cauchy e converge uniformemente su $D_a$.
  Deve convergere a $f$ perché, per il teorema di Weierstrass, le derivate convergono al valore della derivata limite, e questo ci dice che i coefficienti della serie della funzione limite sono proprio quelli di $f$. \\
  $\Omega$ è a base numerabile, dunque possiamo estrarre un sottoricoprimento numerabile da $\{D_a | a \in \Omega\}$. Sia dunque $\{a_j\} \subseteq \Omega$ t.c. $\displaystyle \bigcup_j D_{a_j}=\Omega$. Per quanto dimostrato finora, possiamo estrarre da $\{f_n\}$ una sottosuccessione $\{f_{n_j^{(0)}}\}$ convergente uniformemente in $D_{a_0}$.
  Per induzione, da $\{f_{n_j^{(k-1)}}\}$ estraiamo una sottosuccessione $\{f_{n_j^{(k)}}\}$ convergente uniformemente in $D_{a_0} \cup \dots \cup D_{a_k}$. Prendiamo $\{f_{n_j^{(j)}}\}$ che converge uniformemente in ogni $D_{a_k}$.
  Adesso, ogni compatto è coperto da un numero finito di $D_{a_k}$, quindi (scegliendo per ogni $\epsilon$ il massimo degli indici t.c. le cose che vogliamo valgono in quei $D_{a_k}$) $\{f_{n_j^{(j)}}\}$ converge uniformemente sui compatti.
\end{proof}

\begin{thm}
  (Vitali) $\Omega \subseteq \mathbb{C}$ dominio, $A \subseteq \Omega$ con almeno un punto di accumulazione in $\Omega$. Sia $\{f_n\} \subset \mathcal{O}(\Omega)$ uniformemente limitata sui compatti. Supponiamo che, per ogni $a \in A$, $\{f_n(a)\}$ converge (cioe $f_n$ converge puntualmente). Allora esiste $f \in \mathcal{O}(\Omega)$ t.c. $f_n \rightarrow f$ uniformemente sui compatti di $\Omega$.
\end{thm}

\begin{proof}
  Facciamola per assurdo. Supponiamo che esistono $K \subset\subset \Omega, \\ \{n_k\}, \{m_k\} \subset \mathbb{N}, \{z_k\} \subset K, \delta>0$ t.c. $|f_{n_k}(z_k)-f_{m_k}(z_k)| \ge \delta$. A meno di sottosuccessioni, $z_k \rightarrow z_0 \in K$.
  Per il teorema di Montel, a meno di sottosuccessioni $f_{n_k} \rightarrow g_1 \in \Omega$ e $f_{m_k} \rightarrow g_2 \in \Omega$ con $|g_1(z_0)-g_2(z_0)| \ge \delta$ (passando al limite). Per ipotesi,  $g_1(a)=g_2(a)$ per ogni $a \in A$. Per il principio di identità, $g_1 \equiv g_2$, assurdo.
\end{proof}

\begin{thm}
  (Sviluppo di Laurent) Siano $0 \le r_1 < r_2 \le +\infty, A(r_1, r_2):=\{z \in \mathbb{C} | r_1 < |z| < r_2 \}$. Sia $f \in \mathcal{O}(A(r_1, r_2))$, allora $\displaystyle f(z)=\sum_{n=-\infty}^{+\infty} c_nz^n$ e converge uniformemente e assolutamente sui compatti di $A(r_1, r_2)$.
  In particolare, se $\Omega \subseteq \mathbb{C}$ aperto, $a \in \Omega$ e $\displaystyle f \in \mathcal{O}(\Omega \setminus \{a\}), f(z)=\sum_{n=-\infty}^{+\infty} c_n(z-a)^n$ in $\{0<|z-a|<r\} \subset \Omega$.
\end{thm}

\begin{cor}
  (Teorema di estensione di Riemann) $f \in \mathcal{O}(\Omega \setminus \{a\})$ si estende olomorficamente ad $a$ $\Leftrightarrow$ $\displaystyle \lim_{z \rightarrow a} (z-a)f(z)=0$.
\end{cor}

\begin{proof}
  Per lo sviluppo di Laurent, $\displaystyle (z-a)f(z)=\sum_{n=-\infty}^{+\infty} c_n(z-a)^{n+1} \rightarrow 0 \Leftrightarrow c_n=0$ per ogni $n \le -1$.
\end{proof}

\begin{thm} \label{biolo}
  \begin{nlist}
    \item $f \in \text{Hol}(\Omega, \Omega_1)$ biettiva $\implies$ $f^{-1}$ è olomorfa e $f'$ non si annulla mai;
    \item $f \in \mathcal{O}(\Omega)$ t.c. $f'(z_0) \not=0$ $\implies$ $f$ è iniettiva vicino a $z_0$.
  \end{nlist}
\end{thm}

\begin{proof}
  \begin{nlist}
    \item Per il teorema dell'applicazione aperta, $f$ è aperta $\implies$ $f$ omeomorfismo. $g=f^{-1}$. Sia $w_0 \in \Omega_1$ t.c. $f'(g(w_0)) \not=0$. Allora
    $$ \frac{g(w)-g(w_0)}{w-w_0}=\frac{1}{\frac{w-w_0}{g(w)-g(w_0)}}=\frac{1}{\frac{f(g(w))-f(g(w_0))}{g(w)-g(w_0)}}=\frac{1}{f'(g(w_0))}. $$
    Quindi $g$ è olomorfa in $\Omega_1 \setminus f(\{f'=0\})$. Per il corollario \ref{olo_discr} $\{f'=0\}$ è discreto in $\Omega$. $f$ omeomorfismo $\implies$ $f(\{f'=0\})$ discreto in $\Omega_1$.
    Ma $g$ è continua (quindi localmente limitata) in $\Omega_1$, dunque per il teorema di estensione di Riemann $g \in \mathcal{O}(\Omega_1)$. $(f' \circ g)g' \equiv 1$ su $\Omega_1 \setminus f(\{f'=0\})$ $\implies$ vale su $\Omega_1$ $\implies$ $f'\circ g \not=0$ sempre.
    \item Possiamo supporre $z_0=0$. $\displaystyle f(z)=\sum_{n=0}^{+\infty} c_nz^n$. Per ipotesi, $c_1 \not=0$. \\
    $\displaystyle f(z)-f(w)=c_1(z-w)+(z-w)\sum_{n=2}^{+\infty} c_n\sum_{k=1}^n w^{k-1}z^{n-k}$. \\
    $\displaystyle |f(z)-f(w)| \ge |c_1||z-w|-|z-w|\sum_{n=2}^{+\infty} |c_n|\sum_{k=1}^n |w|^{k-1}|z|^{n-k}$. \\
    Prendiamo $z, w \in D(0, r)$, allora \\
    $\displaystyle |c_1||z-w|-|z-w|\sum_{n=2}^{+\infty} |c_n|\sum_{k=1}^n |w|^{k-1}|z|^{n-k} \ge \\
    \ge |c_1||z-w|-|z-w|\sum_{n=2}^{+\infty} |c_n|nr^{n-1}=(|c_1|-\sum_{n=2}^{+\infty} |c_n|nr^{n-1})|z-w|$.
    Scegliamo $r$ t.c. $\displaystyle \sum_{n=2}^{+\infty} |c_n|nr^{n-1} \le \frac{|c_1|}{2}$, allora $\displaystyle (|c_1|-\sum_{n=2}^{+\infty} |c_n|nr^{n-1})|z-w| \ge \frac{|c_1|}{2}|z-w|$.
    Dato che $c_1 \not=0$, si ha quindi (concatenando le disuguaglianze) che $z \not=w \implies |f(z)-f(w)| \ge \dfrac{|c_1|}{2}|z-w|>0 \implies f(z) \not= f(w)$.
  \end{nlist}
\end{proof}

\begin{defn}
  Se $f: \Omega_1 \rightarrow \Omega_2$ è olomorfa e biettiva (quindi con inversa olomorfa per il teorema \ref{biolo}) si chiama \textsc{biolomorfismo}.
\end{defn}

\begin{defn}
  $f:\Omega_1 \rightarrow \mathbb{C}$ è un \textsc{biolomorfismo locale} se ogni $a \in \Omega_1$ ha un intorno $U \ni a$ t.c. $f\restrict{U}:U \rightarrow f(U)$ è un biolomorfismo.
\end{defn}

Per il teorema \ref{biolo}, $f$ è un biolomorfismo locale se e solo se $f'$ non si annulla mai.

\begin{defn}
  Sia $\displaystyle f(z)=\sum_{n=-\infty}^{+\infty} c_n(z-a)^n$ in $D^*=D(a, r) \setminus \{a\}$. $ord_a(f):=\inf\{n \in \mathbb{Z} | c_n \not=0\}$ è detto \textsc{ordine di $f$ in $a$}. $ord_a(f) \ge 0 \Leftrightarrow f$ è olomorfa in $a$.
  Se $0>ord_a(f)>-\infty$ diremo che $a$ è un \textsc{polo} di $f$. Se $ord_a(f)=-\infty$ $a$ è una \textsc{singolarità essenziale}.
\end{defn}

\begin{thm}
  (Casorati-Weierstrass) Se $a$ è una singolarità essenziale, $f(D^*)$ è denso in $\mathbb{C}$.
\end{thm}

\begin{defn}
  $c_{-1}=:res_f(a)$ è detto \textsc{residuo di $f$ in $a$}.
\end{defn}

\begin{oss} \label{int_res}
  $\gamma(t)=a+\rho e^{2\pi i t}, 0<\rho<r$. \\
  $\displaystyle \frac{1}{2\pi i} \int_{\gamma} f \diff z=\frac{1}{2\pi i} \sum_{n=-\infty}^{+\infty} c_n \int_{\gamma} (z-a)^n \diff z= \\ \frac{1}{2\pi i} \sum_{n=-\infty}^{+\infty} c_n \int_0^1 \rho^ne^{2\pi i n t} \rho 2 \pi i e^{2\pi i t} \diff t=\sum_{n=-\infty}^{+\infty} c_n \rho^{n+1} \int_0^1 e^{2\pi i (n+1)t} \diff t=c_{-1}$.
\end{oss}

\begin{prop}
  $\Omega \subseteq \mathbb{C}$ aperto, $E \subset \Omega$ discreto e chiuso in $\Omega$, $D \subset \subset \Omega$ disco chiuso t.c. $E \cap \partial D=\emptyset$, $f\in \mathcal{O}(\Omega \setminus E)$.
  Allora $\displaystyle \frac{1}{2\pi i}\int_{\partial D} f \diff z=\sum_{a \in D \cap E} res_f(a)$.
\end{prop}

\begin{proof}
  Traccia: si dimostra che $E \cap D$ è finito e si applica una versione leggermente più forte del teorema di Cauchy-Goursat+Morera, prendendo per ogni punto di $E$ un dischetto tutto contenuto in $D$ che lo isoli dagli altri e considerando la regione $D$ meno quei dischetti. Il bordo di questa regione è considerato il bordo di $D$ meno il bordo dei dischetti. Questo bordo, a meno di aggiungere dei tratti lineari che uniscono una circonferenza all'altra (che quindi verranno percorsi in entrambi i sensi nell'integrale e non daranno contributo), è percorribile con un solo cammino omotopo al cammino costante in $\Omega \setminus E$, il cui integrale fa $0$ per la versione forte del teorema di C-G+M, dunque l'integrale sul bordo di $D$ meno l'integrale sul bordo dei dischetti (occhio al verso di percorrenza di uno e degli altri!) deve essere uguale a $0$. Per l'osservazione \ref{int_res} si ha la tesi.
\end{proof}

\begin{oss} \label{wn}
  $\gamma:[0, 1] \rightarrow \mathbb{C}$ chiusa ($\gamma(0)=\gamma(1))$, $a \not\in \gamma([0, 1])$. $p_s: \mathbb{C} \rightarrow \mathbb{C} \setminus \{a\}$, $p_a(z)=a+e^z$ è un rivestimento.
  \begin{center}
    \begin{tikzcd}
            & \mathbb{C} \arrow[d, "p_a"]\\
            \left[0,1\right] \arrow[ru, "\tilde{\gamma}"] \arrow[r, "\gamma"'] & \mathbb{C} \setminus \{a\}
     \end{tikzcd}
  \end{center}
  Sia $\tilde{\gamma}$ un sollevamento di $\gamma$ rispetto a $p_a$, $p_a(\tilde{\gamma}(1))=\gamma(1)=\gamma(0)=p_a(\tilde{\gamma}(0)) \iff e^{\tilde{\gamma}(1)}=e^{\tilde{\gamma}(0)} \iff \tilde{\gamma}(1)-\tilde{\gamma}(0) \in 2\pi i \mathbb{Z}$.
\end{oss}

\begin{defn}
  L'\textsc{indice di avvolgimento $\gamma$ rispetto ad $a$} (\textit{winding number} in inglese) è dato dall'osservazione \ref{wn}: $n(\gamma, a):=\dfrac{1}{2\pi i}(\tilde{\gamma}(1)-\tilde{\gamma}(0)) \in \mathbb{Z}$.
\end{defn}

\begin{thm}
  \begin{nlist}
    \item $n(\gamma, a)$ dipende solo da $a$ e da $\gamma$ e non dal sollevamento scelto;
    \item $n(\gamma, a) \in \mathbb{Z}$;
    \item $\displaystyle n(\gamma, a)=\frac{1}{2\pi i}\int_{\gamma} \frac{1}{z-a} \diff z$;
    \item $a \mapsto n(\gamma, a)$ è costante sulle componente connesse di $\mathbb{C} \setminus \gamma([0, 1])$. In particolare $n(\gamma, a)=0$ sulla componente connessa illimitata di $\mathbb{C} \setminus \gamma([0, 1])$;
    \item $\gamma(t)=a_0+re^{2\pi i t} \implies n(a, \gamma)=1$ per ogni $a \in D(a_0, r)$;
    \item $\gamma_1$ e $\gamma_2$ chiuse con $\gamma_1(0)=\gamma_2(0)=p_0$ omotope (tramite omotopia che fissa il punto base $p_0$) e $a \not\in \gamma_1([0, 1]) \cup \gamma_2([0, 1])$, se l'omotopia è in $\mathbb{C} \setminus \{a\}$ allora $n(\gamma_1, a)=n(\gamma_2, a)$.
  \end{nlist}
\end{thm}

\begin{thm}
  (Teorema dei residui) $\Omega \subseteq \mathbb{C}$ aperto, $E \subset \Omega$ discreto e chiuso in $\Omega$, $\gamma$ curva chiusa in $\Omega \setminus E$ omotopa a una costante in $\Omega$.
  Allora per ogni $f \in \mathcal{O}(\Omega \setminus E)$ $\displaystyle \frac{1}{2\pi i} \int_{\gamma} f \diff z=\sum_{a \in E} res_f(a) \cdot n(\gamma, a)$.
\end{thm}

\begin{defn}
  $\Omega \subseteq \mathbb{C}$, $f$ è \textsc{meromorfa} in $\Omega$ se esiste $E \subset \Omega$ discreto e chiuso in $\Omega$ t.c. $f \in \mathcal{O}(\Omega \setminus E)$ e nessun punto di $E$ è una singolarità essenziale. Scriveremo che $f \in \mathcal{M}(\Omega)$.
\end{defn}

\begin{prop}
  \begin{nlist}
    \item $f \in \mathcal{O}(\Omega \setminus E)$ è meromorfa $\iff$ localmente è quoziente di due funzioni olomorfe;
    \item $f \in \mathcal{O}(\Omega \setminus E)$ è meromorfa $\iff$ per ogni $a \in E$ o $|f|$ è limitato vicino ad $a$ o $\displaystyle \lim_{z \rightarrow a} |f(z)|=+\infty$.
  \end{nlist}
\end{prop}

\begin{proof}
  \begin{nlist}
    \item ($\implies$) Se $a \in \Omega \setminus E$ banalmente $f=\dfrac{f}{1}$ vicino ad $a$. \\
    Se $a \in E$, $\displaystyle f(z)=\sum_{n \ge n_0} c_n(z-a)^n=(z-a)^{n_0}(c_{n_0}+h(z))$, $h$ olomorfa vicino ad $a$. Se $n_0<0$, $f(z)=\dfrac{c_{n_0}+h(z)}{(z-a)^{-n_0}}$.

    ($\Leftarrow$) Se $\displaystyle f(z)=\frac{h_1(z)}{h_2(z)}=\frac{\sum_{n \ge n_1} b_n(z-a)^n}{\sum_{m \ge n_2} c_m(z-a)^m}=(z-a)^{n_1-n_2}k(z)$, $k$ olomorfa vicino ad $a$.
    \item Per Casorati-Weierstrass, $a \in E$ è singolarità essenziale $\iff$ $\displaystyle \lim_{z \rightarrow a} |f(z)|$ non esiste. Per lo stesso motivo, è un polo $\iff$ $\displaystyle \lim_{z \rightarrow a} |f(z)|=+\infty$.
  \end{nlist}
\end{proof}

\begin{thm}
  (Principio dell'argomento) $\Omega \subseteq \mathbb{C}$, $f \in \mathcal{M}(\Omega)$. $Z_f:=\{\text{zeri di } f\}, P_f:=\{\text{poli di } f\}$. $\gamma$ curva chiusa in $\Omega \setminus (Z_f \cup P_f)$ omotopa a una costante in $\Omega$.
  Allora $\displaystyle \sum_{a \in Z_f \cup P_f} n(\gamma, a) \cdot ord_a(f)=\frac{1}{2\pi i}\int_{\gamma} \frac{f'}{f} \diff z$.
\end{thm}

\begin{proof}
  $ord_a(f)=res_{f'/f}(a)$. Infatti $f(z)=(z-a)^mh(z)$ con $m=ord_a(f)$, $h(a) \not=0$ e $h$ olomorfa. $f'(z)=m(z-a)^{m-1}h(z)+(z-a)^mh'(z)$. Allora $\dfrac{f'}{f}=\dfrac{m}{z-a}+\dfrac{h'(z)}{h(z)}$ e $\dfrac{h'}{h}$ è olomorfa $\implies$ $res_{f'/f}(a)=m=ord_a(f)$. La tesi segue allora dal teorema dei residui.
\end{proof}

\begin{prop}
  (Versione semplice del teorema di Rouché) $\Omega \subseteq \mathbb{C}$, $f, g \in \mathcal{O}(\Omega)$, $D$ disco con $\overline{D} \subset \Omega$. Supponiamo che $|f-g|<|g|$ su $\partial D$ (questo implica anche che non si annullano mai su $\partial D$). Allora $f$ e $g$ hanno lo stesso numero di zeri (contati con molteplicità) su $D$.
\end{prop}

\begin{proof}
  Per $t \in [0, 1]$ poniamo $f_t=g+t(f-g)$ ($f_0=g$, $f_1=f$). Se $z \in \partial D$, $0<|g(z)|-|f(z)-g(z)| \le |g(z)|-t|f(z)-g(z)| \le |f_t(z)|$. Sia $\displaystyle a_t=\sum_{a \in \overline{D}} ord_a(f_t)=\text{numero di zeri di } f_t \text{ in } \overline{D}$.
  Non ci sono poli, dunque $a_t \in \mathbb{N}$, quindi per il principio dell'argomento $\displaystyle a_t=\frac{1}{2\pi i} \int_{\partial D} \frac{f_t'}{f_t} \diff z=\frac{1}{2\pi i} \int_{\partial D} \frac{g'+t(f'-g')}{g+t(f-g)} \diff z$,
  che dipende con continuità da $t$ $\implies$ $a_t$ è costante (è a valori in $\mathbb{N}$) $\implies$ $a_0=a_1$ come voluto.
\end{proof}

\begin{cor}
  (Teorema di Ritt) Sia $h \in \mathcal{O}(\mathbb{D})$ ($\mathbb{D}:=\{|z|<1\}$) t.c. $h(\mathbb{D}) \subset \subset \mathbb{D}$. Allora $h$ ha un punto fisso.
\end{cor}

\begin{proof}
  Esiste $0<r<1$ t.c. $|h(z)|<r$ per ogni $z \in \mathbb{D}$. Sia $\mathbb{D}_r:=\{|z|<r\}$. Su $\partial \mathbb{D}_r$, $|z-(z-h(z))|=|h(z)|<r=|z|$.
  Per il teorema di Rouché su $g(z)=z, f(z)=z-h(z)$, $g$ e $f$ hanno lo stesso numero di zeri in $\mathbb{D}$, ma $g$ ha un unico zero $\implies$ $f=\id_{\mathbb{D}}-h$ ha un unico zero $z_0$ $\implies$ $h(z_0)=z_0$.
\end{proof}


\subsection{Teoremi di Hurwitz}
Vediamo ora qualche risultato interessante.

\begin{thm}
  (Primo teorema di Hurwitz) $\Omega \subseteq \mathbb{C}$ aperto, $\{f_n\} \subset \mathcal{O}(\Omega)$ convergente a $f\in \mathcal{O}(\Omega)$ uniformemente sui compatti. Supponiamo che $f$ non sia costante sulle componenti connesse di $\Omega$.
  Allora per ogni $z_0 \in \Omega$ esistono $n_1=n_1(z_0) \in \mathbb{N}$ e $z_n \in \Omega$ per ogni $n \ge n_1$ t.c. $f_n(z_n)=f(z_0)$ e $\displaystyle \lim_{n \longrightarrow +\infty} z_n=z_0$.
  Senza la tesi sul limite di $z_n$, si può dire che per ogni $w=f(z_0) \in f(\Omega)$ esiste $n_1=n_1(w)$ t.c. $w \in f_n(\Omega)$ per ogni $n \ge n_1$.
\end{thm}

\begin{proof}
  Vogliamo applicare Rocuhé a $f_n-w$ e $f-w$, $w=f(z_0)$ in dischetti centrati in $z_0$ di raggio arbitrariamente piccolo. $f$ non costante sulle componenti connesse $\implies$ $f^{-1}(w)$ è discreto $\implies$ esiste $\delta>0$ t.c. $0<|z-z_0| \le \delta$ $\implies$ $z \in \Omega$ e $f(z) \not=w$.
  Se $D=D(z_0, \delta)$ allora $\overline{D} \cap f^{-1}(w)=\{z_0\}$. Per ogni $k>0$, $\gamma_k=\partial D(z_0, \delta/k)$. Poniamo $\delta_k=\min\{|f(\zeta)-w| \mid \zeta \in \gamma_k\}>0$.
  Esiste $n_k \ge 1$ t.c. per ogni $n \ge n_k$ $\displaystyle \max_{\zeta \in \gamma_k} |f_n(\zeta)-f(\zeta)|<\frac{\delta_k}{2}$ ($f_n$ converge a $f$ uniformemente sui compatti). Possiamo supporre $n_1<n_2<n_3<\dots$.
  Fissato $k \ge 1$, se $n \ge n_k$ e $\zeta \in \gamma_k$, $|(f_n(\zeta)-w)-(f(\zeta)-w)|=|f_n(\zeta)-f(\zeta)|<\dfrac{\delta_k}{2}<\delta_k \le |f(\zeta)-w|$.
  Per il teorema di Rocuhé applicato a $f_n-w$ e $f-w$ in $\overline{D(z_0, \delta/k)}$, per ogni $n \ge n_k$ $f_n-w$ ha almeno uno zero in $D(z_0, \delta_k)$ $\implies$ esiste $z_n \in D(z_0, \delta/k)$ t.c. $f_n(z_n)=w$. $z_n \longrightarrow z_0$ per $n \longrightarrow +\infty$.
\end{proof}

\begin{cor}
  (Secondo teorema di Hurwitz) $\Omega \subseteq \mathbb{C}$ dominio, $\{f_n\} \subset \mathcal{O}(\Omega)$ t.c. $f_n \longrightarrow f \in \mathcal{O}(\Omega)$. Supponiamo che le $f_n$ non si annullino mai (o, in generale, esiste $w_0 \in \mathbb{C}$ t.c. $w_0 \not\in f_n(\Omega)$ per ogni $n$),
  allora o $f \equiv 0$ o $f$ non si annulla mai (in generale, o $f \equiv w_0$ o $w_0 \not\in f(\Omega)$).
\end{cor}

\begin{proof}
  Per assurdo, $w_0 \in f(\Omega)$. Allora o $f$ è costante ($f \equiv w_0$) oppure, per il primo teorema di Hurwitz, $w_0 \in f_n(\Omega)$ per ogni $n>>1$, assurdo.
\end{proof}

\begin{cor}
  (Terzo teorema di Hurwitz) $\Omega \subseteq \mathbb{C}$ dominio, $\{f_n\} \subset \mathcal{O}(\Omega)$ t.c. $f_n \longrightarrow f \in \mathcal{O}(\Omega)$. Supponiamo che le $f_n$ siano iniettive. Allora $f$ è costante o iniettiva.
\end{cor}

\begin{proof}
  Per assurdo, sia $f$ né costante né iniettiva. Allora esistono $z_1 \not=z_2$ t.c. $f(z_1)=f(z_2)$. Poniamo $h_n(z)=f_n(z)-f_n(z_2)$ e $h(z)=f(z)-f(z_2)$. $h_n \longrightarrow h$ e le $h_n$ non si annullano mai in $\Omega \setminus \{z_2\}$ (perché le $f_n$ sono iniettive). Dato che per ipotesi $f$ non è costante, pure $h$ non è costante, dunque per il secondo teorema di Hurwitz non si annulla mai in $\Omega \setminus \{z_2\}$, ma $h(z_1)=0$, assurdo.
\end{proof}


\subsection{La sfera di Riemann}
\begin{defn}
  La \textsc{sfera di Riemann} è l'insieme $\hat{\mathbb{C}}=\overline{\mathbb{C}}=\mathbb{C}_{\infty}=\mathbb{C} \cup \{\infty\}=\mathbb{P}^1(\mathbb{C})$ (l'ultimo è la retta proiettiva complessa).
  Per noi sarà $\hat{\mathbb{C}}=\mathbb{C} \cup \{\infty\}$ con la seguente topologia: ristretta a $\mathbb{C}$ è la topologia usuale, mentre gli intorni aperti di $\infty$ sono della forma $(\mathbb{C} \setminus K) \cup \{\infty\}$ con $K \subset \subset \mathbb{C}$ compatto.
\end{defn}

Siano $U_0=\mathbb{C}, U_1=\mathbb{C}^*\cup\{\infty\}$ (si noti che $U_1$ è un intorno aperto di $\infty$). Sia $\varphi_1:U_1 \longrightarrow \mathbb{C}$ definita come
$$\varphi_1(w)=\begin{cases} 1/w & \mbox{se }w\not=\infty \\ 0 & \mbox{se }w=\infty. \end{cases}$$
$\varphi_1$ è un omeomorfismo fra $U_1$ e $\mathbb{C}$.
Sia $\varphi_0:U_0 \longrightarrow \mathbb{C}$, $\varphi_0(z)=z$ (l'identità); è un omeomorfismo fra $U_0$ e $\mathbb{C}$. \\
$U_0 \cap U_1=\mathbb{C}^*$, $\varphi_1(U_0 \cap U_1)=\mathbb{C}^*=\varphi_0(U_0 \cap U_1)$. \\
$\varphi_0 \circ \varphi_1^{-1}, \varphi_1 \circ \varphi_0^{-1}:\mathbb{C}^* \longrightarrow \mathbb{C}^*$,
$(\varphi_0 \circ \varphi_1^{-1})(w)=\dfrac{1}{w}, (\varphi_1 \circ \varphi_0^{-1})(z)=\dfrac{1}{z}$ sono olomorfe. \\
$\varphi_0$ e $\varphi_1$ si chiamano \textit{carte}. Una funzione definita a valori in $\hat{\mathbb{C}}$ è olomorfa se lo è letta tramite carte. Vediamo nello specifico cosa significa.

Sia $\Omega \subseteq \hat{\mathbb{C}}$ aperto, $f:\Omega \longrightarrow \mathbb{C}$ continua; quando è olomorfa? \\
Risposta:
\begin{nlist}
  \item $f \restrict{\Omega \cap \mathbb{C}}$ è olomorfa in senso classico (notiamo che $\Omega \cap \mathbb{C}=\Omega \setminus \{\infty\}$);
  \item $f \circ \varphi_1^{-1}:\varphi_1(\Omega) \longrightarrow \mathbb{C}$ è olomorfa vicino a $0=\varphi_1(\infty)$. \\
  $(f \circ \varphi_1^{-1})(w)=f\left(\dfrac{1}{w}\right)$.
\end{nlist}

\begin{ex}
  $\displaystyle f(z)=\sum_{n=-\infty}^{+\infty} c_nz^n$. Quando è olomorfa in $\infty$? Se e solo se $f\left(\dfrac{1}{w}\right)$ è olomorfa in $0$.
  $\displaystyle f\left(\frac{1}{w}\right)=\sum_{n=-\infty}^{+\infty} c_nw^{-n}$ è olomorfa in $0$ $\iff$ $c_n=0$ per ogni $n>0$.
\end{ex}

\begin{oss}
  $f: \hat{\mathbb{C}} \longrightarrow \mathbb{C}$ è olomorfa se e solo se è costante. Infatti, $\hat{\mathbb{C}}$ compatto $\implies$ $f(\hat{\mathbb{C}})$ compatto, cioè chiuso e limitato in $\mathbb{C}$ $\implies$ $|f|$ ha max in $x_0 \in \hat{\mathbb{C}}$.
  Se $z_0 \in \mathbb{C}$, allora per il teorema di Liouville \ref{liou} $f \restrict{\mathbb{C}}$ è costante $\implies$ $f$ costante. Se $z_0=\infty$, $f(1/w)$ ha massimo in $0$, dunque ragionando come prima è costante.
\end{oss}

Sia $\Omega \subseteq \mathbb{C}$ aperto, $f:\Omega \longrightarrow \hat{\mathbb{C}}$ continua; quando è olomorfa? \\
Risposta:
\begin{nlist}
  \item $f$ è olomorfa in $\Omega \setminus f^{-1}(\infty)$ in senso classico;
  \item se $f(z_0)=\infty$, $\varphi_1 \circ f=\dfrac{1}{f}$ è olomorfa vicino a $z_0$.
\end{nlist}

\begin{ex}
  $\displaystyle f(z)=\sum_{n=-\infty}^{+\infty} c_nz^n$ è a valori in $\hat{\mathbb{C}}$ se $0$ è un polo (ci interessa il caso in cui $\infty$ sia effettivamente nell'immagine, altrimenti è una comune funzione olomorfa a valori in $\mathbb{C}$), cioè consideriamo $f(0)=\infty$.
  Supponiamo allora $\displaystyle f(z)=\sum_{n=k}^{+\infty} c_nz^n=z^{-k}\sum_{n=k}^{+\infty} c_nz^{n+k}=z^{-k}h(z)$, $h(0)=c_{-k}\not=0$, $h$ olomorfa. $\dfrac{1}{f}(z)=\dfrac{z^k}{h(z)}$ è olomorfa in $0$.
  Viceversa, se $f$ è olomorfa, $\dfrac{1}{f}$ è olomorfa in $0$ $\implies$ $\displaystyle \frac{1}{f}(z)=z^k\sum_{n=0}^{+\infty}c_nz^n$, $c_0 \not=0, k \ge 1$ (la condizion $k \ge 1$ segue dal fatto che siamo nell'ipotesi $f(0)=\infty \implies (1/f)(0)=0$).
  Allora $\dfrac{1}{f}(z)=z^kh(z)$ $\implies$ $f(z)=z^{-k}\frac{1}{h(z)}$ e quindi ha un polo in $0$.
\end{ex}

\begin{cor}
  $\Omega \subseteq \mathbb{C}$, $f:\Omega \longrightarrow \hat{\mathbb{C}}$ è olomorfa se e solo se è meromorfa.
\end{cor}

Possiamo ora dare una definizione generale.

\begin{defn}
  $f:\hat{\mathbb{C}} \longrightarrow \hat{\mathbb{C}}$ continua è \textit{olomorfa} se e solo se $f\left(\dfrac{1}{w}\right)$ è olomorfa vicino a $0$ e $\dfrac{1}{f}$ è olomorfa vicino a $f^{-1}(\infty)$ (e ovviamente dev'essere normalmente olomorfa in tutti gli altri punti). \\
  Se $f(\infty)=\infty$, la condizione è che $\dfrac{1}{f(1/w)}$ sia olomorfa in $0$.
\end{defn}

\begin{ex}
  $p(z)=a_0+a_1z+\dots+a_dz^d, a_d \not=0$ (un polinomio). $p(\infty)=\infty$. È olomorfo in $\infty$? Sì: $\displaystyle \frac{1}{p(1/z)}=\frac{1}{a_0+a_1z^{-1}+\dots+a_dz^{-d}}=\frac{1}{z^{-d}(a_0z^d+\dots+a_d)}=\frac{z^d}{a_d+\dots+a_0z^d}$ è olomorfo in $0$.
  $\dfrac{1}{p(1/z)}$ ha uno zero di ordine $d$ in $0$ $\iff$ $p$ ha un polo di ordine di $-d$ in $\infty$ (vedremo più avanti come è definito $ord_f(\infty)$).
\end{ex}

\begin{prop}
  $f \in \text{Hol}(\hat{\mathbb{C}}, \hat{\mathbb{C}})$ $\iff$ $f=\dfrac{P}{Q}$ con $P, Q \in \mathbb{C}[z]$ senza fattori comuni, cioè $f$ è una funzione razionale.
\end{prop}

\begin{proof}
  ($\Leftarrow$) Sappiamo che $\mathbb{C}[z] \subset \text{Hol}(\hat{\mathbb{C}}, \hat{\mathbb{C}})$ e quozienti di funzioni olomorfe sono olomorfi.

  ($\implies$) Sia $f: \hat{\mathbb{C}} \longrightarrow \hat{\mathbb{C}}$ olomorfa non costante. $Z_f=f^{-1}(0)$ è chiuso e discreto in $\hat{\mathbb{C}}$ che è compatto, dunque è finito, perciò $Z_f \cap \mathbb{C}=\{z_1, \dots, z_k\} \subset \mathbb{C}$.
  Analogamente $P_f=f^{-1}(\infty)=Z_{1/f}$, $P_f \cap \mathbb{C}=\{w_1, \dots, w_h\} \subset \mathbb{C}$. Sia $g(z)=\dfrac{(z-w_1)\cdots(z-w_h)}{(z-z_1) \cdots (z-z_k)}f(z)$, $g \in \text{Hol}(\hat{\mathbb{C}}, \hat{\mathbb{C}})$ (gli zeri e i poli compaiono con molteplicità nei prodotti al numeratore e al denominatore).
  In questo modo $g$ non ha né zeri né poli in $\mathbb{C}$. Se $g(\infty) \in \mathbb{C}$, $g \in \text{Hol}(\hat{\mathbb{C}}, \mathbb{C})$ $\implies$ $g$ costante, diciamo $g \equiv c$ $\implies$ $f(z)=c\dfrac{(z-z_1) \cdots (z-z_k)}{(z-w_1)\cdots(z-w_h)}$, come voluto.
  Se $g(\infty)=\infty$ $\implies$ $\dfrac{1}{g}(\infty)=0 \in \mathbb{C}$ $\implies$ $\dfrac{1}{g} \in \text{Hol}(\hat{\mathbb{C}}, \mathbb{C})$ $\implies$ $\dfrac{1}{g}$ costante e si conclude come sopra.
\end{proof}

\begin{defn}
  Sia $f=\dfrac{P}{Q} \in \text{Hol}(\hat{\mathbb{C}}, \hat{\mathbb{C}})$. Il \textsc{grado di $f$} è $\deg{f}=\max{\{\deg{P}, \deg{Q}\}}$. \\
  La definizione dell'ordine di zeri e poli in $\mathbb{C}$ ce l'abbiamo. \\
  $\displaystyle f(\infty)=\lim_{w \longrightarrow 0} \frac{P(1/w)}{Q(1/w)}=\lim_{w \longrightarrow 0} \frac{a_m\left(\frac{1}{w}\right)^m+\dots+a_0}{b_n\left(\frac{1}{w}\right)^n+\dots+b_0}=$ \\
  $$\lim_{w \longrightarrow 0} w^{n-m} \frac{a_m+\dots+a_0w^m}{b_n+\dots+b_0w^n}=\begin{cases} 0 & \mbox{se }n>m \\ \frac{a_m}{b_n} & \mbox{se }n=m \\ \infty & \mbox{se } n<m. \end{cases}$$
  Definiamo allora $ord_f(\infty))=n-m=\deg{Q}-\deg{P}$.
\end{defn}

\begin{defn}
  Siano $f \in \text{Hol}(\hat{\mathbb{C}}, \hat{\mathbb{C}})$, $z_0 \in \hat{\mathbb{C}}$.
  La \textsc{molteplicità di $f$ in $z_0$} è $\delta_f(z_0)$ definita come segue: se $f(z_0)=w_0 \in \mathbb{C}$, $z_0$ è uno zero di $f-w_0$ e poniamo $\delta_f(z_0)=ord_{f-w_0}(z_0)$; se $f(z_0)=\infty$, $z_0$ è un polo di $f$ e poniamo $\delta_f(z_0)=-ord_f(z_0)$. Si ha che $\delta_f(z_0) \in \mathbb{N}$.
\end{defn}

\begin{prop}
  Sia $f \in \text{Hol}(\hat{\mathbb{C}}, \hat{\mathbb{C}})$ non costante. Allora per ogni $q \in \hat{\mathbb{C}}$ $\displaystyle \sum_{f(p)=q} \delta_f(p)=\deg{f}$.
\end{prop}

\begin{proof}
  Sia $f=\dfrac{P}{Q}$. Se $q=0$, $\displaystyle \sum_{f(p)=0} \delta_f(p)=\sum_{\substack{f(p)=0 \\ p \in \mathbb{C}}} \delta_f(p)+c \cdot \delta_f(\infty)$ dove $c=1$ se $f(\infty)=0$ e $c=0$ altrimenti.
  Si noti che per il teorema fondamentale dell'algebra $\displaystyle \sum_{\substack{f(p)=0 \\ p \in \mathbb{C}}} \delta_f(p)=\deg{P}$. Per com'è definito $c$, $c \cdot \delta_f(\infty)=\max{\{0, \deg{Q}-\deg{P}\}}$.
  Allora $\displaystyle \displaystyle \sum_{f(p)=0} \delta_f(p)=\deg{P}+\max{\{0, \deg{Q}-\deg{P}\}}=\max{\{\deg{P}, \deg{Q}\}}=\deg{f}$.
  Se $q=\infty$, $\displaystyle \sum_{f(p)=\infty} \delta_f(p)=\sum_{\substack{f(p)=\infty \\ p \in \mathbb{C}}} \delta_f(p)+c \cdot \delta_f(\infty)$ dove stavolta $c=1$ se $f(\infty)=\infty$ e $c=0$ altrimenti.
  Dunque in questo caso la sommatoria vale, per il teorema fondamentale dell'algebra, $\deg{Q}$, mentre $c \cdot \delta_f(\infty)=\max{\{0, \deg{P}-\deg{Q}\}}$,
  per cui $\displaystyle \sum_{f(p)=\infty} \delta_f(p)=\deg{Q}+\max{\{0, \deg{P}-\deg{Q}\}}=\max{\{\deg{Q}, \deg{P}\}}=\deg{f}$.
  Se $q \in \mathbb{C}^*$, $\displaystyle \sum_{f(p)=q} \delta_f(p)=\sum_{f(p)=q} ord_{f-q}(p)=\sum_{f(p)-q} ord_{f-q}(p)=\deg{(f-q)}$. $f(z)-q=\dfrac{P(z)-qQ(z)}{Q(z)}$.
  $$\deg{(P-qQ)}\begin{cases} =\max{\{\deg{P}, \deg{Q}\}} & \mbox{se sono diversi} \\ \le \max{\{\deg{P}, \deg{Q}\}} & \mbox{se sono uguali} \end{cases}$$ $\implies$ $\deg{(f-q)}=\deg{f}$.
\end{proof}

\begin{cor}
  Siano $f \in \text{Hol}(\hat{\mathbb{C}}, \hat{\mathbb{C}})$ non costante, $w_0 \in \hat{\mathbb{C}}$. Allora $1 \le card(f^{-1}(w_0)) \le \deg{f}$.
\end{cor}

\begin{proof}
  $\ge 1$: se $w_0 \not=\infty$, $f(z)=w_0$ $\iff$ $f(z)-w_0=0$ $\iff$ $P(z)-w_0Q(z)=0$ e per il teorema fondamentale dell'algebra esiste $z$ che soddisfa; se $w_0=\infty$, si considera $1/f$. \\
  $\displaystyle card(f^{-1}(w_0)) \le \sum_{f(p)=w_0} \delta_f(p)=\deg{f}$.
\end{proof}

\begin{oss}
  $\delta_f(p)>1$ $\implies$ $f'(p)=0 \lor \left(\dfrac{1}{f}\right)'(p)=0$. Infatti, senza perdita di generalità $p=0$ e $f(p)=0$, allora se $\delta_f(p)=k>1$ si ha che $f(z)=z^kh(z)$ con $h$ olomorfa e $h(0) \not=0$ $\implies$ $f'(z)=[kz^{k-1}h(z)+z^kh'(z)]$. Ricordando che $k>1$, si ha che $f'(0)=0$.
\end{oss}

\begin{cor}
  $f \in \text{Aut}(\hat{\mathbb{C}})$ $\iff$ $\deg{f}=1$ $\iff$ $f(z)=\dfrac{az+b}{cz+d}$ con $ad-bc=1$.
\end{cor}

\begin{proof}
  ($\Leftarrow$) Ogni $f \in \text{Hol}(\hat{\mathbb{C}}, \hat{\mathbb{C}})$ è suriettiva per quanto appena dimostrato. Se $\deg{f}=1$, allora $f$ è iniettiva, quindi biettiva, per cui per il teorema \ref{biolo} $f \in \text{Aut}(\hat{\mathbb{C}})$. \\
  ($\implies$) $f$ automorfismo $\implies$ $f$ iniettiva $\implies$ $\displaystyle \sum_{f(p)=w_0} \delta_f(p)$ contiene un unico addento con molteplicità uno (da cui la tesi). Infatti, da $f$ non costante si ha che $f'$ ha un insieme di zeri discreto $C_f$ e $(1/f)'$ ha un insieme di zeri discreto $C_{1/f}$. Allora basta prendere $z_0 \not\in C_f \cup C_{1/f}$ per ottenere, dall'osservazione precedente, che $\delta_f(z_0)=1$. \\
  Il secondo se e solo se è un banale esercizio lasciato al lettore.
\end{proof}

\begin{oss}
  Siccome numeratore e denominatore sono definiti a meno di una costante moltiplicativa, possiamo suppore $ad-bc=1$.
\end{oss}

\begin{exc}
  $\text{Aut}(\hat{\mathbb{C}})$ è isomorfo a $\faktor{SL(2, \mathbb{C})}{\{\pm I_2\}}$.
\end{exc}

\begin{cor}
  $f \in \text{Aut}(\mathbb{C})$ $\iff$ $f(z)=az+b, a, b \in \mathbb{C}, a\not=0$.
\end{cor}

\begin{proof}
  ($\Leftarrow$) Ovvia.

  ($\implies$) $f \in \text{Aut}(\mathbb{C})$ $\implies$ $f$ è iniettiva, dunque per Casorati-Weierstrass $\infty$ è un polo di $f$ $\implies$ $f$ si estende a un automorfismo di $\hat{\mathbb{C}}$ con $f(\infty)=\infty$ $\implies$ $f(z)=\frac{a}{d}z+\frac{b}{d}$.
\end{proof}


\subsection{Il disco unitario}
Come abbiamo già visto, il disco unitario (aperto) è definito come $\mathbb{D}=\{z \in \mathbb{C} \mid |z|<1\}$.

\begin{lm}
  (Lemma di Schwarz) Sia $f \in \text{Hol}(\mathbb{D}, \mathbb{D})$ t.c. $f(0)=0$. Allora per ogni $z \in \mathbb{D}$ $|f(z)| \le |z|$ e $|f'(0)| \le 1$; inoltre, se vale l'uguale nella prima per $z \not=0$ oppure nella seconda allora $f(z)=e^{i\theta}z, \theta \in \mathbb{R}$, cioè $f$ è una rotazione.
\end{lm}

\begin{proof}
  $f(0)=0$ $\implies$ possiamo costruire $g \in \text{Hol}(\mathbb{D}, \mathbb{C})$ con $g(z)=\dfrac{f(z)}{z}$ estendendola per continuità in $0$ a $g(0)=f'(0)$. Fissiamo $0<r<1$.
  Per ogni $|z| \le r$, per il principio del massimo $\displaystyle |g(z)| \le \max_{|w|=r} |g(w)|=\max_{|w|=r} \frac{|f(w)|}{r} \le \frac{1}{r}$. Mandando $r$ a $1$ otteniamo che per ogni $z \in \mathbb{D}$ si ha $|g(z)| \le 1$, da cui $|f(z)|\le |z|$ e $|f'(0)| \le 1$. \\
  Se vale uno dei due uguali sopra, allora esiste $z_0 \in \mathbb{D}$ t.c. $|g(z_0)|=1$, per cui sempre per il principio del massimo $g$ è costantemente uguale a un valore di modulo $1$, cioè $g(z)=e^{i\theta}$ con $\theta \in \mathbb{R}$ da cui $f(z)=e^{i\theta}z$.
\end{proof}

\begin{cor}
  Se $f \in \text{Aut}(\mathbb{D})$ è t.c. $f(0)=0$, allora $f(z)=e^{i\theta}z$.
\end{cor}

\begin{proof}
  $f^{-1} \in \text{Aut}(\mathbb{D})$. $(f^{-1})'(0)=\dfrac{1}{f'(0)}$. Per il lemma di Schwarz, $|f'(0)| \le 1$ e $|(f^{-1})'(0)| \le 1$ $\implies$ $|f'(0)|=1$, da cui la tesi sempre per il lemma di Schwarz.
\end{proof}

\begin{lm} \label{az_gr}
  Sia $G$ un gruppo che agisce fedelmente su uno spazio $X$, cioè per ogni $g \in G$ è data una biezione $\gamma_g:X \rightarrow X$ t.c. $\gamma_{e}=\id$ e $\gamma_{g_1} \circ \gamma_{g_2} =\gamma_{g_1g_2}$, inoltre $\gamma_{g_1}=\gamma_{g_2} \iff g_1=g_2$.
  Sia $G_{x_0}$ il gruppo di isotropia di $x_0 \in X$, cioè $G_{x_0}=\{g \in G \mid \gamma_g(x_0)=x_0\}$. Supponiamo che per ogni $x \in X$ esiste $g_x \in G$ t.c. $\gamma_{g_x}(x)=x_0$ e sia $\Gamma=\{g_x \mid x \in X\}$.
  Allora $G$ è generato da $\Gamma$ e $G_{x_0}$, cioè ogni $g \in G$ è della forma $g=hg_x$ con $x \in X$ e $h \in G_{x_0}$.
\end{lm}

\begin{proof}
  Sia $g \in G$ e $x=\gamma_g(x_0)$. Allora $(\gamma_{g_x}\circ \gamma_g)(x_0)=x_0$ $\implies$ $\gamma_{g_x}\circ \gamma_g=\gamma_{g_xg}=\gamma_h$ con $h \in G_{x_0}$ $\implies$ $g_xg=h$ $\implies$ $g=g_x^{-1}h$.
  Partendo da $g^{-1}$ avremmo ottenuto $g^{-1}=g_x^{-1}h$ $\implies$ $g=h^{-1}g_x$ con $h \in G_{x_0}$.
\end{proof}

\begin{prop}
  $f \in \text{Aut}(\mathbb{D})$ $\iff$ esistono $\theta \in \mathbb{R}$ e $a \in \mathbb{D}$ t.c. $f(z)=e^{i\theta}\dfrac{z-a}{1-\bar{a}z}$.
\end{prop}

\begin{proof}
  ($\Leftarrow$) $1-\left|\dfrac{z-a}{1-\bar{a}z}\right|^2=\dfrac{(1-|a|^2)(1-|z|^2)}{|1-\bar{a}z|^2}$. Se $a, z \in \mathbb{D}$, $f(z) \in \mathbb{D}$.
  Se $a \in \mathbb{D}, z \in \partial\mathbb{D}$, $f(z) \in \partial\mathbb{D}$. L'inversa è $f^{-1}(z)=e^{-i\theta}\dfrac{z+ae^{i\theta}}{z+\bar{a}e^{-i\theta}z}$ ed è della stessa forma. Si noti che $f(a)=0$.

  ($\implies$) Scriviamo per semplicità $f_{a, \theta}=e^{i\theta}\dfrac{z-a}{1-\bar{a}z}$. Vediamo $\text{Aut}(\mathbb{D})$ come gruppo che agisce su $\mathbb{D}$. $\text{Aut}(\mathbb{D})_0$ è, per il corollario del lemma di Schwarz, $\{f_{0, \theta} \mid \theta \in \mathbb{R}\}$.
  $\Gamma=\{f_{a, 0} \mid a \in \mathbb{D}\}$ ($f_{a, 0}(a)=0$).
  Per il lemma \ref{az_gr}, $\text{Aut}(\mathbb{D})$ è generato da $\text{Aut}(\mathbb{D})$ e $\Gamma$, cioè ogni $\gamma \in \text{Aut}(\mathbb{D})$ è della forma $\gamma=f_{0, \theta} \circ f_{a, 0}=f_{a, \theta}$.
\end{proof}

\begin{cor}
  $\text{Aut}(\mathbb{D})$ agisce in modo transitivo su $\mathbb{D}$, cioè per ogni $z_0, z_1 \in \mathbb{D}$ esiste $\gamma \in \text{Aut}(\mathbb{D})$ t.c. $\gamma(z_0)=z_1$.
\end{cor}

\begin{proof}
  $\gamma=f_{z_1, 0}^{-1} \circ f_{z_0, 0}$.
\end{proof}

\begin{oss}
  Dati $z_0, z_1, w_0, w_1 \in \mathbb{D}$ ($z_0 \not=z_1, w_0 \not=w_1$), in generale non esiste $\gamma \in \text{Aut}(\mathbb{D})$ t.c. $\gamma(z_0)=w_0$ e $\gamma_(z_1)=w_1$.
  Infatti, se poniamo $z_0=w_0=0, z_1, w_1 \not=0$, abbiamo che $\gamma(0)=0$ $\implies$ $\gamma(z)=e^{i\theta}z$ $\implies$ $|\gamma(z_1)|=|z_1|$, per cui se $|w_1|\not=|z_1|$ non è possibile trovare un siffatto $\gamma$.
\end{oss}

\begin{exc}
  Per ogni $\sigma_0, \sigma_1, \tau_0, \tau_1 \in \partial\mathbb{D}$ con $\sigma_0\not=\sigma_1, \tau_0\not=\tau_1$ esiste $\gamma \in \text{Aut}(\mathbb{D})$ t.c. $\gamma(\sigma_0)=\tau_0$ e $\gamma(\sigma_1)=\tau_1$.
\end{exc}


\subsection{Dinamica del disco e del semipiano superiore}
Vogliamo adesso cercare di studiare qual è la "dinamica" delle funzioni olomorfe. Lo faremo nei casi del disco e del semipiano superiore.

\begin{prop}
  Sia $\gamma \in \text{Aut}(\mathbb{D}), \gamma \not=\id_{\mathbb{D}}$. Allora o
  \begin{nlist}
    \item $\gamma$ ha un unico punto fisso in $\mathbb{D}$ (si parla in questo caso di automorfismo \textit{ellittico}) o
    \item $\gamma$ non ha punti fissi in $\mathbb{D}$ e ha un unico punto fisso in $\partial\mathbb{D}$ (\textit{parabolico}) o
    \item $\gamma$ non ha punti fissi in $\mathbb{D}$ e ha due punti fissi distinti in $\partial\mathbb{D}$ (\textit{iperbolico}).
  \end{nlist}
\end{prop}

\begin{proof}
  $\gamma(z_0)=z_0 \iff e^{i\theta}(z_0-a)=(1-\bar{a}z_0)z_0 \iff \bar{a}z_0^2+(e^{i\theta}-1)z_0-e^{i\theta}a=0$, equazione di secondo grado con radici $z_1, z_2$ (può essere che $z_1=z_2$) t.c.
  $z_1 \cdot z_2=-e^{i\theta}\dfrac{a}{\bar{a}} \in \partial\mathbb{D} \implies |z_1||z_2|=1$.
  Se $z_1 \not=z_2$, o $z_1 \in \mathbb{D}$ e $z_2 \in \mathbb{C} \setminus \{\overline{\mathbb{D}}\}$ (caso ellittico) e $z_1, z_2 \in \partial\mathbb{D}$ (caso iperbolico). Se $z_1=z_2$, $|z_1|=|z_2|=1$ (caso iperbolico).
\end{proof}

\begin{oss}
  Se $f \in \text{Hol}(\mathbb{D}, \mathbb{D})$ è t.c. $f(z_1)=z_1$ e $f(z_2)=z_2$ con $z_1, z_2 \in \mathbb{D}, z_1 \not=z_2$, allora $f=\id_{\mathbb{D}}$.
  Infatti, possiamo supporre $z_1=0$ $\implies$ $f(0)=0$ e $f(z_2)=z_2$, quindi siamo nel caso del lemma di Schwarza in cui vale l'uguaglianza, per cui $f(z)=e^{i\theta}z$, ma $f(z_2)=z_2$ $\implies$ $e^{i\theta}=1$.
\end{oss}

\begin{ex}
  Esempio di automorfismo ellittico: la rotazione intorno a $0$ $\gamma_{0, \theta}(z)=e^{i\theta}z$. Più in generale, se $a \in \mathbb{D}$, $\gamma_{a, 0}(z)=\dfrac{z-a}{1-\bar{a}z}$, allora $\gamma_{a, 0}^{-1} \circ \gamma_{0, \theta} \circ \gamma_{a, 0}$ è ellittico con punto fisso $a$.
  Queste sono dette \textit{rotazioni non euclidee} e caratterizzano tutti gli automorfismi ellittici (lo si può vedere coniugando opportunamente con $\gamma_{a, 0}$ o $\gamma_{a, 0}^{-1}$).
\end{ex}

\begin{defn}
  Il \textsc{semipiano superiore} è $\mathbb{H}^+=\{w \in \mathbb{C} \mid \mathfrak{Im}w>0\}$. La \textsc{trasformata di Cayley} è $\Psi:\mathbb{D} \longrightarrow \mathbb{H}^+$ t.c. $\Psi(z)=i\dfrac{1+z}{1-z}$.
\end{defn}

Notiamo che possiamo vedere $\mathbb{H}^+ \subset \hat{\mathbb{C}}$ e in questo caso $\partial\mathbb{H}^+=\mathbb{R}\cup\{\infty\}$. $\Psi^{-1}(w)=\dfrac{w-i}{w+i}$. $\Psi(0)=1, \Psi(1)=\infty$. \\
$\mathfrak{Im}\Psi(z)=\mathfrak{Im}\left(i\dfrac{1+z}{1-z}\right)=\mathfrak{Re}\left(\dfrac{1+z}{1-z}\right)=\dfrac{1}{|1-z|^2}\mathfrak{Re}((1+z)(1-\bar{z}))=\dfrac{1-|z|^2}{|1-z|^2}$ che è $>0$ $\iff$ $z \in \mathbb{D}$ e $=0$ $\iff$ $z \in \partial\mathbb{D}\setminus\{1\}$.

$\Psi$ è un biolomorfismo fra $\mathbb{D}$ e $\mathbb{H}^+$ che si estende continua a $\partial\mathbb{D} \longrightarrow \partial\mathbb{H}^+$. Se abbiamo $f: \mathbb{D} \longrightarrow \mathbb{D}$, abbiamo anche $F=\Psi \circ f \circ \Psi^{-1}:\mathbb{H}^+ \longrightarrow \mathbb{H}^+$ e viceversa.

\begin{cor}
  $\gamma \in \text{Aut}(\mathbb{H}^+) \iff \gamma(w)=\dfrac{aw+b}{cw+s}$ con $ad-bc=1$ e $a, b, c, d \in \mathbb{R}$. Si ha allora che $\text{Aut}(\mathbb{H}^+) \cong \faktor{SL(2, \mathbb{R})}{\{\pm I_2\}}=PSL(2, \mathbb{R})$ (questo è detto \textit{gruppo speciale lineare proiettivo}).
\end{cor}

\begin{proof}
  $\gamma \in \text{Aut}(\mathbb{H}^+) \iff \Psi^{-1} \circ \gamma \circ \Psi \in \text{Aut}(\mathbb{D}) \iff (\Psi^{-1} \circ \gamma \circ \Psi)(z)=e^{i\theta}\dfrac{z-a}{1-\bar{a}z}$.
  Ponendo $\Psi(z)=w$, l'uguaglianza sopra equivale a $\gamma(w)=\Psi\left(e^{i\theta}\dfrac{z-a}{1-\bar{a}z}\right)=\Psi\left(e^{i\theta}\dfrac{\Psi^{-1}(w)-a}{1-\bar{a}\Psi^{-1}(w)}\right)$. Facendo il conto si trova l'enunciato.
\end{proof}

\begin{exc}
  $\gamma \in \text{Aut}(\mathbb{H}^+)$ è t.c. $\gamma(i)=i \iff \gamma(w)=\dfrac{w\cos{\theta}-\sin{\theta}}{w\sin{\theta}+\cos{\theta}}$.
\end{exc}

\begin{ex}
  Sia $\gamma \in \text{Aut}(\mathbb{H}^+)$, $\gamma(\infty)=\infty \iff \gamma(w)=\alpha w+\beta$ con $\alpha, \beta \in \mathbb{R}, \alpha>0$.
  Se lo vogliamo parabolico non deve avere altri punti fissi in $\mathbb{C}$ e questo è possibile se e solo se $\alpha w+\beta=w$ non ha altre soluzioni $\iff$ $\alpha=1, \beta \not=0$, cioè $\gamma(w)=w+\beta$.
  È una traslazione di $\mathbb{H}^+$ parallela al suo bordo.
\end{ex}

\begin{exc}
  Sia $\tau \in \partial\mathbb{D}$. Dimostrare che tutti gli automorfismi $\gamma$ parabolici di $\mathbb{D}$ con $\gamma(\tau)=\tau$ sono della forma $\gamma(z)=\sigma_0\dfrac{z+z_0}{1+\bar{z}_0z}$ con $z_0=\dfrac{ic}{2-ic}\tau$ e $\sigma_0=\dfrac{2-ic}{2+ic}$ con $c \in \mathbb{R}$.
  Hint: a meno di una rotazione, $\tau=1$.
\end{exc}

\begin{ex}
  $\gamma \in \text{Aut}(\mathbb{H}^+)$ è iperbolico con $\gamma(\infty)=\infty$ e $\gamma(0)=0$ $\iff$ $\gamma(w)=\alpha w$ con $\alpha>0$.
\end{ex}

Passiamo ora alla \textsc{dinamica di funzioni iterate}. Abbiamo uno spazio generico $X$ e una funzione $f:X \longrightarrow X$. Le sue \textit{iterate} sono $f^2=f \circ f$ e, induttivamente, $f^{k+1}=f \circ f^{k-1}$. Vogliamo capire il comportamento asintotico di $\{f^k\}$ (in relazione alla struttura presente su $X$), per esempio, capire cosa succede all'\textit{orbita} $O^+(x)=\{f^k(x)\}$ con $x \in X$.

\begin{ex}
  $\gamma(w)=\alpha w \implies \gamma^2(w)=\alpha(\alpha w)=\alpha^2 w \implies \gamma^k(w)=\alpha^k w$.
  Quindi: se $0<\alpha<1$, $\gamma^k(w) \longrightarrow 0$ per $k \longrightarrow +\infty$ $\implies$ $\gamma^k \longrightarrow 0$ (costante) uniformemente sui compatti; se $\alpha>1$, $\gamma^k(w) \longrightarrow \infty$ per $k \longrightarrow +\infty$ $\implies$ $\gamma^k \longrightarrow \infty$ (costante) uniformemente sui compatti.
\end{ex}

\begin{ex}
  $\gamma(w)=w+\beta \implies \gamma^k(w)=w+k\beta \implies \gamma^k(w) \longrightarrow \infty$ per $k \longrightarrow +\infty$ $\implies$ $\gamma^k \longrightarrow \infty$ (costante) uniformemente sui compatti.
\end{ex}

\begin{oss}
  \begin{center}
    \begin{tikzcd}
      X \arrow[r, "f"] & X\\
      Y \arrow[u, "\Psi", "\cong" right] \arrow[r, "F"] & Y \arrow[u, "\cong", "\Psi" right]
    \end{tikzcd}
  \end{center}
  $\Psi$ bigezione (omeomorfismo/biolomorfismo/eccetera), $F=\Psi^{-1} \circ f \circ \Psi$, cioè \textit{$F$ è coniugata a $f$}. Allora $f^k=\Psi^{-1} \circ f^k \circ \Psi$, cioè $F^k$ è coniugata a $f^k$ per ogni $k$. In particolare, la "dinamica di $F$" è "uguale" alla "dinamica di $f$".
\end{oss}

\begin{cor}
  Sia $\gamma \in \text{Aut}(\mathbb{D})$ parabolico o iperbolico, allora $\gamma^k$ converge uniformemente sui compatti a una funzione costantemente uguale a un punto fisso di $\gamma$ sul bordo.
\end{cor}

\begin{proof}
  A meno di coniugio possiamo supporre $\text{Fix}(\gamma)=\{1\}$ nel caso parabolico e $\{1, -1\}$ nel caso iperbolico. Coniughiamo con $\Psi$ e usiamo gli esempi.
\end{proof}

Sia $\gamma \in \text{Aut}(\mathbb{D})$ ellittico, a meno di coniugio $\gamma(0)=0 \implies \gamma(z)=e^{2\pi i \theta}z \implies \gamma^k(z)=e^{2k\pi i \theta}z$.
Se $\theta \in \mathbb{Q}$, esiste $k_0$ t.c. $k_0\theta \in \mathbb{Z} \implies \gamma^{k_0}(z) \equiv z \iff \gamma^{k_0}=\id_{\mathbb{D}}$.

\begin{exc}
  Se $\theta \not\in \mathbb{Q}$, $\gamma^k(z) \not=z$ per ogni $z \not=0$ e $k \in \mathbb{N}^*$, da cui si ha anche che $\gamma^k(z)\not=\gamma^h(z)$ per ogni $z \not=0$ e $h \not=k$.
\end{exc}

\begin{exc}
  Se $\theta \not\in \mathbb{Q}$, $\{\gamma^k(z_0) \mid k \in \mathbb{N}\}$ è densa nella circonferenza $\{|z|=|z_0|\}$.
\end{exc}

Vogliamo adesso studiare la dinamica di una $f \in \text{Hol}(\mathbb{D}, \mathbb{D})$ qualunque.

\begin{defn}
  Sia $f \in \text{Hol}(\Omega, \Omega)$, un \textit{punto limite} di $\{f^k\}$ è $g \in \text{Hol}(\Omega, \mathbb{C})$ t.c. è il limite di una sottosuccessione $\{f^{k_{\nu}}\}$, cioè $f^{k_{\nu}} \longrightarrow g$.
\end{defn}

\begin{lm} \label{pli}
  Sia $\Omega \subseteq \mathbb{C}$ dominio, $f \in \text{Hol}(\Omega, \Omega)$. Se $\id_{\Omega}$ è un punto limite di $\{f^k\}$, allora $f \in \text{Aut}(\Omega)$.
\end{lm}

\begin{proof}
  $f^{k_{\nu}} \longrightarrow \id_{\Omega}$ $\implies$ $f$ è iniettiva (se $z_1 \not=z_2$ sono t.c. $f(z_1)=f(z_2)$, allora $f^{k_{\nu}}(z_1)=f^{k_{\nu}}(z_2)$, ma la prima tende a $z_1$ e la seconda a $z_2$, che sono diversi, assurdo).
  Se $z_0 \in \Omega$, $z_0=\id_{\Omega}(z_0)$. Per il primo teorema di Hurwitz, $\id_{\Omega}(z_0) \in f^{k_{\nu}}(\Omega)$ per $\nu \gg 1$, ma $f^{k_{\nu}}(\Omega) \subseteq f(\Omega)$ $\implies$ $f$ è suriettiva.
\end{proof}

\begin{prop} \label{lim_aut}
  Sia $\Omega \subset \subset \mathbb{C}$ un dominio limitato, $f \in \text{Hol}(\Omega, \Omega)$. Sia $h \in \text{Hol}(\Omega, \mathbb{C})$ un punto limite di $\{f^k\}$ (che esiste per il teorema di Montel). Allora o
  \begin{nlist}
    \item $h \equiv c \in \overline{\Omega}$ oppure
    \item $h \in \text{Aut}(\Omega)$ e in questo caso $f \in \text{Aut}(\Omega)$.
  \end{nlist}
\end{prop}

\begin{proof}
  Sia $\displaystyle h=\lim_{\nu \longrightarrow +\infty} f^{k_{\nu}}$. Poniamo $m_{\nu}=k_{\nu+1}-k_{\nu}$. Possiamo supporre $m_{\nu} \longrightarrow +\infty$. Per Montel, a meno di una sottosuccessione possiamo supporre $f^{m_{\nu}} \xrightarrow{\nu \longrightarrow +\infty} g \in \text{Hol}(\Omega, \mathbb{C})$.
  Se $h$ è costante abbiamo finito. Se $h$ non è costante, per il teorema dell'applicazione aperta $h$ è aperta $\implies$ $h(\Omega)$ è aperto e per il primo teorema di Hurwitz è contenuto in $\Omega$.
  Se $z \in \Omega$, $\displaystyle g(h(z))=\lim_{\nu \longrightarrow +\infty} f^{m_{\nu}}(f^{k_{\nu}}(z))=\lim_{\nu \longrightarrow +\infty} f^{k_{\nu+1}}(z)=h(z) \implies g\restrict{h(\Omega)}=\id_{\Omega}$,
  ma per il principio di identità questo ci dà $g \equiv \id_{\Omega}$, dunque per il lemma \ref{pli} abbiamo che $f \in \text{Aut}(\Omega)$. A meno di sottosuccessioni è facile vedere che $f^{-k_{\nu}}=(f^{-1})^{k_{\nu}}$ converge a $h^{-1}$.
\end{proof}

\begin{prop}
  Sia $f \in \text{Hol}(\mathbb{D}, \mathbb{D})$, $f(z_0)=z_0, z_0 \in \mathbb{D}$, $f\not\in \text{Aut}(\mathbb{D})$. Allora $f^k \longrightarrow z_0$ (costante) uniformemente sui compatti.
\end{prop}

\begin{proof}
  A meno di coniugio possiamo supporre $z_0=0$. Per il lemma di Schwarz, $|f(z)|<|z|$ per ogni $z \in \mathbb{D}\setminus\{0\}$. Fissiamo $0<r<1$. In $\overline{\mathbb{D}}_r$, $\left|\dfrac{f(z)}{z}\right|$ ha un massimo $\lambda_r<1$.
  Per ogni $z \in \overline{\mathbb{D}}_r$, $|f(z)| \le \lambda_r|z| \implies |f^2(z)| \le \lambda_r|f(z)| \le \lambda_r^2|z| \implies |f^k(z)| \le \lambda_r^k|z| \le \lambda_r^kr \longrightarrow 0$ per $k \longrightarrow +\infty$ $\implies$ $f^k \longrightarrow 0$ (costante) uniformemente sui compatti.
\end{proof}

\begin{defn}
  Chiamiamo \textit{orociclo} di centro $\tau \in \partial\mathbb{D}$ e raggio $R>0$ l'insieme $E(\tau, R)=\left\{z \in \mathbb{D} \, \bigg| \, \dfrac{|\tau-z|^2}{1-|z|^2}<R \right\}$. Geometricamente, è un disco di raggio $\dfrac{R}{R+1}$ tangente a $\partial \mathbb{D}$ in $\tau$.
\end{defn}

\begin{exc}
  $\displaystyle E(\tau, R)=\left\{z \in \mathbb{D} \, \bigg| \, \lim_{w \longrightarrow \tau} [\omega(z, w)-\omega(0, w)]<\frac{1}{2}\log{R}\right\}$.
\end{exc}

\begin{lm}
  (Lemma di Wolff) Sia $f \in \text{Hol}(\mathbb{D}, \mathbb{D})$ senza punti fissi. Allora esiste un unico $\tau \in \partial\mathbb{D}$ t.c. per ogni $z \in \mathbb{D}$ $\dfrac{|\tau-f(z)|^2}{1-|f(z)|^2} \le \dfrac{|\tau-z|^2}{1-|z|^2}$ $(\star)$.
  In altre parole, per ogni $R>0$ $f(E(\tau, R)) \subseteq E(\tau, R)$.
\end{lm}

\begin{proof}
  Unicità: se ce ne fossero due, $\tau$ e $\tau_1$, prendiamo un orociclo centrato in $\tau$ e uno centrato in $\tau_1$ tangenti, allora il punto di tangenza verrebbe mandato in sé e sarebbe dunque un punto fisso in $\mathbb{D}$, assurdo.

  Esistenza: prendiamo $\{r_{\nu}\} \subset (0, 1)$ t.c. $r_{\nu} \nearrow 1^{-}$ e poniamo $f_{\nu}=r_{\nu}f$ $\implies$ $f_{\nu}(\mathbb{D}) \subseteq \mathbb{D}_{r_{\nu}} \subset\subset \mathbb{D}$,
  allora per il teorema di Ritt esiste $w_{\nu} \in \mathbb{D}$ t.c. $f_{\nu}(w_{\nu})=w_{\nu}$. A meno di sottosuccessioni, possiamo suppore $w_{\nu} \longrightarrow \in \overline{\mathbb{D}}$.
  Se $\tau \in \mathbb{D}$, $\displaystyle f(\tau)=\lim_{\nu \longrightarrow +\infty} f_{\nu}(w_{\nu})=\lim_{\nu \longrightarrow +\infty} w_{\nu}=\tau$, assurdo $\implies$ $\tau \in \partial\mathbb{D}$.
  Per Schwarz-Pick, $\left|\dfrac{f_{\nu}(z)-w_{\nu}}{1-\bar{w}_{\nu}f_{\nu}(z)}\right|^2 \le \left|\dfrac{z-w_{\nu}}{1-\bar{w}_{\nu}z}\right|^2 \implies 1-\left|\dfrac{f_{\nu}(z)-w_{\nu}}{1-\bar{w}_{\nu}f_{\nu}(z)}\right|^2 \ge 1-\left|\dfrac{z-w_{\nu}}{1-\bar{w}_{\nu}z}\right|^2 \implies \dfrac{|1-\bar{w}_{\nu}f_{\nu}(z)|^2}{1-|f_{\nu}(z)|^2} \le \dfrac{|1-\bar{w}_{\nu}z|^2}{1-|z|^2}$.
  Mandando $\nu \longrightarrow +\infty$ otteniamo $\dfrac{|1-\bar{\tau}f(z)|^2}{1-|f(z)|^2} \le \dfrac{|1-\bar{\tau}z|^2}{1-|z|^2}$ che moltiplicando per $\tau$ ($\tau\bar{\tau}=1$) dà la tesi.
\end{proof}

\begin{exc}
  Si ha l'uguaglianza in $(\star)$ nel lemma di Wolff $\iff$ $f$ è un automorfismo parabolico con punto fisso $\tau$ $\iff$ vale l'uguaglianza in $(\star)$ per ogni $z \in \mathbb{D}$.
\end{exc}

\begin{thm}
  (Wolff-Denjoy) Sia $f \in \text{Hol}(\mathbb{D}, \mathbb{D})$ senza punti fissi in $\mathbb{D}$. Allora esiste un unico $\tau \in \partial\mathbb{D}$ t.c. $f^k \longrightarrow \tau$ (costante) uniformemente sui compatti.
\end{thm}

\begin{proof}
  Se $f \in \text{Aut}(\mathbb{D})$ parabolico o iperbolico l'abbiamo già visto. Supponiamo $f \not\in \text{Aut}(\mathbb{D})$. Per Montel, $\{f^k\}$ è relativamente compatta in $\text{Hol}(\mathbb{D}, \mathbb{C})$. Useremo il seguente risultato di topologia che viene lasciato come esercizio.

  \begin{exc} \label{sct}
    Sia $X$ spazio topologico di Hausdorff. Sia $\{x_k\} \subset X$ con $\overline{\{x_k\}}$ compatta in $X$. Supponiamo che esista un unico $\bar{x} \in X$ t.c. ogni sottosuccessione convergente di $\{x_k\}$ converge a $\bar{x}$. Allora $x_k \longrightarrow \bar{x}$.
  \end{exc}

  Sia $\tau \in \partial\mathbb{D}$ dato dal lemma di Wolff. Sia $\displaystyle h=\lim_{\nu \longrightarrow +\infty} f^{k_{\nu}}$ un punto limite di $\{f^k\}$ (che esiste per Montel). Per la proposizione \ref{lim_aut}, $h \equiv \sigma \in \overline{\mathbb{D}}$.
  Se $\sigma \in \mathbb{D}$, $\displaystyle f(\sigma)=\lim_{\nu \longrightarrow +\infty} f(f^{k_{\nu}}(\sigma))=\lim_{\nu \longrightarrow +\infty} f^{k_{\nu}}(f(\sigma))=\sigma$, assurdo. Quindi $h \equiv \sigma \in \partial\mathbb{D}$.
  Vogliamo $\sigma=\tau$. Per il lemma di Wolff $f^{k_{\nu}}(E(\tau, R)) \subseteq E(\tau, R)$ per ogni $R>0$ $\implies$ $\{\sigma\}=h(E(\tau, R)) \subseteq \overline{E(\tau, R)} \cap \partial\mathbb{D}=\{\tau\}$ $\implies$ $\sigma=\tau$.
  Si conclude allora per l'esercizio \ref{sct}.
\end{proof}


\subsection{Germi e prolungamenti analitici}
\begin{defn}
  Sia $\gamma:[0, 1] \longrightarrow \mathbb{C}$ un cammino continuo.
  Se esistono $0=t_0<t_1<\dots<t_r=1$, intorni $U_0, \dots, U_j, \dots, U_r$ di $\gamma(t_j)$ e $f_j:U_j \longrightarrow \mathbb{C}$ olomorfe t.c. $f_j\restrict{U_j \cap U_{j+1}} \equiv f_{j+1}\restrict{U_j \cap U_{j+1}}$ diremo che \textsc{$f_0$ si prolunga olomorficamente lungo $\gamma$}.
\end{defn}

\begin{ex}
  $\gamma(t)=e^{2\pi i t}, \gamma(0)=\gamma(1)=1$. $z=|z|e^{2\pi i \theta}, \theta \in \mathbb{R}$. $U_0=D(1, 1/2), f_0:U_0: \longrightarrow \mathbb{C}, f_0(z)=z^{1/2}=|z|^{1/2}e^{2\pi i(\theta/2)}$ ($\theta \in (-\pi, \pi)$).
  $f_0 \in \mathcal{O}(U_0)$. È possibile prolungare olomorficamente $f_0$ lungo $\gamma$ con $f(\gamma(t))=e^{2\pi i(t/2)}$ $\implies$ $f(\gamma(1))=e^{2\pi i/2}=e^{\pi i}=-1$. $f(\gamma(0))=1$.
\end{ex}

\begin{defn}
  Sia $a \in \mathbb{C}$ e consideriamo le coppie $(U, f)$ dove $U \subseteq \mathbb{C}$ è un intorno aperto di $a$ e $f \in \mathcal{O}(U)$. Definiamo la seguente relazione di equivalenza: $(U, f) \sim (V, g)$ se esiste $W \subseteq U \cap V$ intorno aperto di $a$ t.c. $f\restrict{W}=g\restrict{W}$. \\
  $\mathcal{O}_a:=\faktor{\{(U, f)\}}{\sim}$ è detta \textsc{spiga dei germi di funzioni olomorfe in $a$}. \\
  $\underline{f_a} \in \mathcal{O}_a$ si dice \textsc{germe} di funzione olomorfa. \\
  $(U,f) \in \underline{f_a}$ si dice \textsc{rappresentante} di $\underline{f_a}$. \\
  $\displaystyle \mathcal{O}:=\bigcup_{a \in \mathbb{C}} \mathcal{O}_a$ si dice \textsc{fascio dei germi} di funzioni olomorfe. \\
  Dato $\Omega \subseteq \mathbb{C}$ aperto, definiamo anche $\displaystyle \mathcal{O}_{\Omega}:=\bigcup_{a \in \Omega} \mathcal{O}_a$.
\end{defn}

\begin{exc}
  $\sim$ appena definita è una relazione di equivalenza.
\end{exc}

\begin{exc}
  $\mathcal{O}_a$ è una $\mathbb{C}$-algebra ($\underline{f_a}+\underline{g}_a$ è il germe rappresentato da $(U \cap V, (f+g)\restrict{U \cap V})$ dove $(U,f) \in \underline{f_a}$ e $(V, g) \in \underline{g}_a$).
\end{exc}

\begin{oss}
  Possiamo definire per ogni $k \ge 0$ $\underline{f_a}^{(k)}(a) \in \mathbb{C}$ ponendo $\underline{f_a}^{(k)}(a)=f^{(k)}(a)$ con $(U, f) \in \underline{f_a}$.
\end{oss}

\begin{defn}
  Definiamo $p$ come la \textit{proiezione}
  \begin{align*}
    p: \mathcal{O} &\longrightarrow \mathbb{C}\\
    \underline{f_z} &\longmapsto z
  \end{align*}
  Vale che $p(\mathcal{O}_a)=\{a\}$. Vogliamo rendere $p$ "quasi" un rivestimento (vedremo che, per i soliti esempio stupidi, non può essere un rivestimento).
\end{defn}

Vogliamo definire una topologia su $\mathcal{O}$. Definiamo un sistema fondamentale di intorni.

\begin{defn}
  Gli intorni del sistema fondamentale sono i seguenti: dati $U \subseteq \mathbb{C}$ aperto, $f \in \mathcal{O}(U)$ l'intorno associato è $N(U, f)=\{\underline{f_z} \mid z \in U, (U, f) \in \underline{f_z}\}$.
\end{defn}

\begin{exc}
  Esiste un'unica topologia su $\mathcal{O}$ t.c. $\{N(U, f)\}$ siano un sistema fondamentale di intorni.
\end{exc}

\begin{oss}
  $p\restrict{N(U, f)}:N(U, f) \longrightarrow U$ è una bigezione.
\end{oss}

\begin{prop}
  $\mathcal{O}$ è uno spazio di Hausdorff.
\end{prop}

\begin{proof}
  Siano $\underline{f_a} \not \underline{g_b}$. Se $a \not= b$, esistono $(U, f) \in \underline{f_a}, (V, g) \in \underline{g_b}$ con $U \cap V=\emptyset$ $\implies$ $N(U, f) \cap N(V, g)=\emptyset$.
  Se $a=b$, siano $(U, f) \in \underline{f_a}, (V, g) \in \underline{g_a}$, $D \subset U \cap V$ disco aperto di centro $a$. Vogliamo $N(D, f) \cap N(D, g)=\emptyset$.
  Per assurdo, sia $\underline{h}_z \in N(D, f) \cap N(D, g)$ $\implies$ $z \in D$ e $\underline{h_z}=\underline{f_z}$ e $\underline{h_z}=\underline{g_z}$ $\implies$ $\underline{f_z}=\underline{g_z}$ $\implies$ esiste un aperto $W \subseteq D$ intorno di $z$ t.c. $f\restrict{W}=g\restrict{W}$ e per il principio di identità si avrebbe $f \equiv g$ su $D$ $\implies$ $\underline{f_a}=\underline{g_a}$, assurdo.
\end{proof}

\begin{prop}
  $p: \mathcal{O} \longrightarrow \mathbb{C}$ è continua, aperta e omeomorfismo locale.
\end{prop}

\begin{proof}
  Sia $V \subseteq \mathbb{C}$, $\displaystyle p^{-1}(V)=\bigcup\{N(W, f) \mid W \subseteq V \text{ aperto}, f \in \mathcal{O}(W)\}$ è aperto. $p(N(U, f))=U$ $\implies$ $p$ è aperta.
  $p\restrict{N(U, f)}$ è invertibile: $p^{-1}(z)=\underline{f_z}$ $\implies$ $p\restrict{N(U, f)}$ è un omeomorfismo $\implies$ $p$ è un omeomorfismo locale.
\end{proof}

\begin{defn}
  Una \textit{sezione} di $\mathcal{O}$ su un $\Omega \subset \mathbb{C}$ aperto è una $\underline{f}:\Omega \longrightarrow \mathcal{O}$ continua t.c. $p \circ \underline{f}=\id_{\Omega}$, cioè $\underline{f}(z) \in \mathcal{O}_z$ per ogni $z \in \Omega$.
\end{defn}

\begin{exc}
  L'insieme delle sezioni di $\mathcal{O}$ su $\Omega$ è in corrispondenza biunivoca con lo spazio $\mathcal{O}(\Omega)$ delle funzioni olomorfe su $\Omega$.
\end{exc}

\begin{defn}
  Siano $a \in \mathbb{C}, \underline{f_a} \in \mathcal{O}_a$. Sia $\gamma:[0, 1] \longrightarrow \mathbb{C}$ una curva continua con $\gamma(0)=a$.
  Un \textsc{prolungamento analitico di $\underline{f_a}$ lungo $\gamma$} è un sollevamento $\tilde{\gamma}:[0, 1] \longrightarrow \mathcal{O}$ di $\gamma$ (cioè $p \circ \tilde{\gamma}=\gamma$) t.c. $\tilde{\gamma}(0)=\underline{f_a}$.
\end{defn}

\begin{oss}
  $p$ non è un rivestimento perché non tutte le curve possono essere sollevate. Vediamo un esempio.
\end{oss}

\begin{ex}
  $a=1, \underline{f_a}=(\mathbb{C}^*, 1/z), \gamma(t)=1-t$. Non esiste alcun sollevamento di $\gamma$ che parte da $\underline{f_a}$.
\end{ex}

\begin{defn}
  Sia $\diff:\mathcal{O} \longrightarrow \mathcal{O}$ così definita: dato $\underline{f_a} \in \mathcal{O}_a$, $\diff\underline{f_a}$ è il germe in $a$ rappresentato dalla derivata di un rappresentante di $\underline{f_a}$, cioè se $(U, f) \in \underline{f_a}$, $\diff\underline{f_a}$ è rappresentato da $(U, f')$.
\end{defn}

\begin{lm} \label{primitiva}
  Sia $D \subseteq \mathbb{C}$ un disco aperto. Allora ogni $f \in \mathcal{O}(D)$ ha una primitiva in $D$, e due primitive differiscono per una costante additiva.
\end{lm}

\begin{proof}
  Se $a \in D$ è il centro, $\displaystyle f(z)=\sum_{n=0}^{+\infty} c_n(z-a)^n$. Una primitiva è data da $\displaystyle F(z)=\sum_{n=0}^{+\infty} c_n(z-a)^n$. È chiaro che due primitive differiscono per una costante additiva.
\end{proof}

\begin{prop}
  $\diff:\mathcal{O} \longrightarrow \mathcal{O}$ è un rivestimento.
\end{prop}

\begin{proof}
  Dati $\underline{f_a} \in \mathcal{O}_a$, $(U, f) \in \underline{f_a}$, $D \subseteq U$ un disco centrato in $a$, poniamo $\mathcal{D}=N(D, f)$, intorno aperto di $\underline{f_a}$. Sia $F$ una primitiva di $f$ su $D$ che esiste per il lemma \ref{primitiva}, per ogni $c \in \mathbb{C}$ poniamo $\mathcal{D}_c=N(D, F+c)$. Vogliamo dimostrare che:
  \begin{nlist}
    \item $\displaystyle \diff^{-1}(\mathcal{D})=\bigcup_{c \in \mathbb{C}} \mathcal{D}_c$;
    \item $\diff\restrict{\mathcal{D}_c}:\mathcal{D}_c \longrightarrow \mathcal{D}$ è un omeomorfismo;
    \item $c_1\not=c_2 \implies \mathcal{D}_{c_1} \cap \mathbb{D}_{c_2} \emptyset$.
  \end{nlist}
  (i), (ii), (iii) $\implies$ $\diff$ è un rivestimento. Procediamo con la dimostrazione.
  \begin{nlist}
    \item Sia $z \in D$ e $\underline{f_z} \in \mathcal{D}$. Sia $\underline{g_z} \in \mathcal{O}_z$ t.c. $\diff\underline{g_z}=\underline{f_z}$ $\implies$ esiste $(W, g) \in \underline{g_z}$ t.c. $g'=f$;
    possiamo supporre che $W \subseteq D$, il disco, quindi sempre per il lemma \ref{primitiva} esiste $c \in \mathbb{C}$ t.c. $g\restrict{W}=F\restrict{W}+c$ $\implies$ $\underline{g_z} \in \mathcal{D}_c$. È banale vedere che $\underline{g_z} \in \mathcal{D}_c \implies \diff\underline{g_z} \in \mathcal{D}$.
    \item È ovvio che $\diff(\mathcal{D}_c)=\mathcal{D}$ (per definizione di $\diff$ e $\mathcal{D}_c$). Questo più il punto (i) ci danno che $\diff$ è continua e aperta: infatti,\ gli insiemi della forma $\mathcal{D}$ formano un sistema fondamentale di intorni e la loro preimmagine, unione di aperti, è aperta; anche gli insiemi $\mathcal{D}_c$ sono un sistema fondamentale di intorni (ogni funzione olomorfa è la primitiva della sua derivata) e la loro immagine, come abbiamo visto, è un aperto.
    $\diff\restrict{\mathcal{D}_c}: \mathcal{D}_c \longrightarrow \mathcal{D}$ è, come visto sopra, suriettiva, ma anche iniettiva perché $\displaystyle \mathcal{D}_c=\bigcup_{z \in D} \mathcal{D}_c \cap \mathcal{O}_z, \mathcal{D}=\bigcup_{z \in D} \mathcal{D} \cap \mathcal{O}_z$,
     ma per ogni $z \in D$, $\mathcal{D}_c\cap \mathcal{O}_z$ e $\mathcal{D}\cap \mathcal{O}_z$ contengono un unico germe e $\diff(\mathcal{O}_z) \subseteq \mathcal{O}_z$, da cui appunto segue l'iniettività ($z\not=z' \implies \mathcal{O}_z\cap\mathcal{O}_{z'}=\emptyset$).
     \item Se $\underline{F_z} \in \mathcal{D}_{c_1}\cap \mathcal{D}_{c_2}$ $\implies$ $\underline{F_z}$ è rappresentanto sia da $(D, F+c_1)$ che da $(D, F+c_2)$ $\implies$ $F+c_1\equiv F+c_2$ vicino a $z$ $\implies$ $c_1=c_2$.
  \end{nlist}
\end{proof}

\begin{thm}
  Sia $\Omega \subseteq \mathbb{C}$ un aperto semplicemente connesso. Allora ogni $f \in \mathcal{O}(\Omega)$ ammette una primitiva.
\end{thm}

\begin{proof}
  Sia $\varphi:\Omega \longrightarrow \mathcal{O}$ la sezione corrispondente a $f$. Sia $\Phi$ un sollevamento di $\varphi$, cioè $d \circ \Phi=\varphi$ (che esiste per la teoria generali dei rivestimenti).
  \begin{center}
    \begin{tikzcd}
      & \mathcal{O} \arrow[d, "\diff"]\\
      \Omega \arrow[ru, "\Phi"] \arrow[r, "\varphi"] & \mathcal{O}
    \end{tikzcd}
  \end{center}
  Anche $\Phi$ è una sezione di $\mathcal{O}$: infatti, siccome $p \circ \diff=p$, $p \circ \Phi=p\circ\diff\circ\Phi=p\circ\varphi=\id_{\Omega}$ $\implies$ la $F \in \mathcal{O}(\Omega)$ associata a $\Phi$ è una primitiva di $f$.
\end{proof}

Concludiamo il paragrafo definendo logaritmo e radice $n$-esima su insiemi semplicemente connessi.

\begin{cor}
  Sia $\Omega \subseteq \mathbb{C}$ aperto semplicemente connesso, $f \in \text{Hol}(\Omega, \mathbb{C}^*)$. Allora esiste $g \in \mathcal{O}(\Omega)$ t.c. $f=\exp(g)$. Inoltre $g$ è unica a meno di costanti additive della forma $2k\pi i$ con $k \in \mathbb{Z}$.
\end{cor}

\begin{proof}
  Sia $g_0$ una primitiva di $f'/f$. $\dfrac{\diff}{\diff z}(fe^{-g_0})=f'e^{-g_0}+f(-e^{-g_0}g_0')=f'e^{-g_0}+f(-e^{-g_0}f'/f)=e^{-g_0}(f'-f')=0$ $\implies$ $f \cdot e^{-g_0}=\text{costante diversa da zero}=e^{c_0}$ $\implies$ $f\equiv e^{c_0+g_0}$.
  Per l'unicità a meno di costanti additive, $\exp(g_1)=\exp(g_2) \implies \exp(g_1-g_2)\equiv 1 \implies$ $g_1-g_2$ è continua a valori in $2\pi i \mathbb{Z}$ discreto $\implies$ $g_1-g_2=2k\pi i$ con $k \in \mathbb{Z}$ costante.
\end{proof}

\begin{cor}
  Sia $\Omega \subseteq \mathbb{C}$ aperto semplicemente connesso, $f \in \text{Hol}(\Omega, \mathbb{C}^*)$, $n \in \mathbb{Z}^*$. Allora esiste $h \in \mathcal{O}(\Omega)$ t.c. $f=h^n$. Inoltre $h$ è unica a meno di costanti moltiplicative della forma $e^{2\pi ik/n}$ con $k \in \mathbb{Z}$.
\end{cor}

\begin{proof}
  Sia $g \in \mathcal{O}(\Omega)$ t.c. $f=\exp(g)$, allora $h=\exp(g/n)$ soddisfa le condizioni richieste. Poi, $h_1^n=h_2^n \iff (h_1/h_2)^n\equiv 1 \implies h_1=e^{2\pi i k/n}h_2$ con $k \in \mathbb{Z}$ costante.
\end{proof}


\subsection{Teorema di uniformizzazione di Riemann}
Lo scopo di questo paragrafo è mostrare che quasi tutti i domini semplicemente connessi di $\mathbb{C}$ sono biolomorfi al disco. Il teorema di uniformizzazione di Riemann, di cui riporteremo solo l'enunciato, caratterizza i biolomorfismi delle superfici di Riemann, in particolare caratterizza completamente i biolomorfismi di quelle semplicemente connesse.

\begin{lm} \label{esistefinF}
  Sia $\Omega \subset \mathbb{D}$ dominio limitato con $\Omega \not=\mathbb{D}$ e $0 \in \Omega$.
  Allora esiste $f \in \text{Hol}(\mathbb{D}, \mathbb{D})$ t.c. $f(0)=0, f'(0) \in \mathbb{R}, f'(0)>0$, $\Omega \subseteq f(\mathbb{D})$ e, se $\Omega_f$ è la componente connessa di $f^{-1}(\Omega)$ contentente $0$, $f\restrict{\Omega_f}:\Omega_f \longrightarrow \Omega$ è un rivestimento.
  Inoltre, $\displaystyle d_1=\inf_{z \not\in \Omega_f} |z|>\inf_{z \not\in \Omega} |z|=d$.
\end{lm}

\begin{proof}
  Sia $a \in \mathbb{D}\setminus\Omega$, $b \in \mathbb{D}$ t.c. $b^2=-a$. Siano $\varphi, \psi \in \text{Aut}(\mathbb{D})$, $\varphi(z)=\dfrac{z+a}{1+\bar{a}z}, \psi(z)=\dfrac{z+b}{1+\bar{b}z}$.
  Poniamo $f \in \text{Hol}(\mathbb{D}, \mathbb{D})$ t.c. $f(z)=\dfrac{\bar{b}}{|b|}\varphi(\psi(z)^2)$. $f(\mathbb{D})=\mathbb{D}, f(0)=0$. $f'(0)=2|b|\dfrac{1-|b|^2}{1-|a|^2}>0$.
  Siccome $w \longmapsto w^2$ è un rivestimento di $\mathbb{D}^*=\mathbb{D}\setminus\{0\}$ e $\varphi^{-1}(\Omega) \subseteq \mathbb{D}^*$, $f$ è un rivestimento da $\Omega_f$ a $\Omega$. Se $d_1=1$ abbiamo finito in quanto $d \le |a|<1$.
  Se invece $d_1<1$, esiste $z_1 \in \partial\Omega_f \cap \mathbb{D}$ con $|z_1|=d_1 \implies f(z_1) \not\in \Omega \implies |f(z_1)| \ge d$. Allora per il lemma di Schwarz abbiamo $d \le |f(z_1)|<|z_1|=d_1$.
\end{proof}

\begin{thm}
  (Osgood, Koebe) Sia $\Omega \subset \subset \mathbb{C}$ un dominio limitato, $z_0 \in \Omega$. Allora esiste un unico rivestimento olomorfo $f_0:\mathbb{D} \longrightarrow \Omega$ t.c. $f_0(0)=z_0$ e $f_0'(0) \in \mathbb{R}, f'(0)>0$.
\end{thm}

\begin{proof}
  Possiamo supporre $z_0=0 \in \Omega$ e $\Omega \subset \subset \mathbb{D}$. Sia $\mathcal{F} \subset \text{Hol}(\mathbb{D}, \mathbb{D})$ t.c.
  $\mathcal{F}=\{f \in \text{Hol}(\mathbb{D}, \mathbb{D}) \mid f(0)=0, f'(0) \in \mathbb{R}, f'(0)>0; \Omega \subseteq f(\mathbb{D});$ se $\Omega_f$ è la componente connessa di $f^{-1}(\Omega)$ contenente $0$ allora $f\restrict{\Omega_f}:\Omega_f \longrightarrow \Omega$ è un rivestimento$\}$.
  Se esiste $f_0 \in \mathcal{F}$ con $\Omega_{f_0}=\mathbb{D}$ ci resta da dimostrare solo l'unicità. Poniamo per ogni $f \in \mathcal{F}$ $\displaystyle d_f=\inf_{z \not\in \Omega_f} |z|=\min_{z \in \partial\Omega_f} |z| \le 1$. Abbiamo $d_f=1 \iff \Omega_f=\mathbb{D}$.
  Dobbiamo trovare $f_0 \in \mathcal{F}$ con $d_{f_0}=1$. Sia $\displaystyle d=\sup_{f \in \mathcal{F}} d_f \le 1$. Sia $\{f_n\} \subset \mathcal{F}$ t.c. $d_{f_n} \longrightarrow d$.
  Per il teorema di Montel possiamo supporre che $f_n \longrightarrow f_0 \in \text{Hol}(\mathbb{D}, \mathbb{C})$ con immaggine in $\overline {\mathbb{D}}$. Vogliamo $f_0 \in \mathcal{F}$. Chiaramente $f_0(0)=0, f_0'(0) \in \mathbb{R}, f'(0) \ge 0$.
  \begin{enumerate}
    \item $f_0$ non è costante e $f'(0)>0$: sia $r>0$ t.c. $\mathbb{D}_r \subset \subset \Omega$. Siccome $f_n:\Omega_{f_n} \longrightarrow \Omega$ è un rivestimento e $0 \in \Omega_{f_n}$, esiste un unico $h_n:\mathbb{D}_r \longrightarrow \Omega_{f_n}$ olomorfa t.c. $f_n \circ h_n=\id_{\mathbb{D}_r}$ e $h_r(0)=0$.
    Sempre per il teorema di Montel, a meno di sottosuccessioni possiamo supporre $h_n \longrightarrow h_0 \in \text{Hol}(\mathbb{D}_r, \overline{\mathbb{D}})$ t.c. $f_0 \circ h_0=\id_{\mathbb{D}_r}$.
    Dunque $h_0$ è iniettiva, quindi per il teorema dell'applicazione aperta è aperta, perciò $h_0(\mathbb{D}_r) \subseteq \mathbb{D}$, e $f_0$ non è costante, dunque per il primo teorema di Hurwitz $f_0(\mathbb{D}) \subseteq \mathbb{D}$. $1=\id_{\mathbb{D}_r}'(0)=(f_0 \circ h_0)'(0)=f_0'(h_0(0))\cdot h_0'(0)=f_0'(0) \cdot h_0'(0) \implies f_0'(0) \not=0 \implies f_0'(0)>0$.
    \item $f_0(\Omega_{f_0})=\Omega$ dove $\Omega_{f_0}$ è la componente connessa di $f_0^{-1}(\Omega)$ contenente $0$. Sia infatti $z_0 \in \Omega$ e sia $\gamma$ una curva in $\Omega$ da $0$ a $z_0$.
    Ricopriamo $\gamma$ con dischi $D_0=\mathbb{D}_r, D_1, \dots, D_k$ con $D_j \cap D_{j+1} \not=\emptyset$ e $z_0 \in D_k$. Sia $h_{n, 0}$ l'inversa di $f_n$ su $D_0$ t.c. $h_{n,0}=0$.
    Per ogni $j$ sia $h_{n, j}$ l'inversa di $f_n$ su $D_j$ che coincide con $h_{n, j-1}$ su $D_{j-1} \cap D_j$. Per Montel, a meno di sottosuccessioni $h_{n, 0} \longrightarrow h_{0, 0} \in \text{Hol}(D_0, \mathbb{D})$.
    Per il teorema di Vitali, $h_{n, 1} \longrightarrow h_{0, 1} \in \text{Hol}(D_1, \mathbb{D})$ e, per induzione, $h_{n, k} \longrightarrow h_{0, k} \in \text{Hol}(D_k, \mathbb{D})$ con $f_0 \circ h_{0, k}=\id_{D_k} \implies f_0(h_{0, k}(z_0))=z_0 \implies z_0 \in f_0(\mathbb{D})$.
    In realtà $h_{0, k}(z_0) \in \Omega_{f_0}$ perché è immagine della curva ottenuta con $h_{0, j} \circ \gamma$ che parte da $0$ e quindi è contenuta nella componente connessa di $\Omega$ contenete $0$.
    \item $f_0:\Omega_{f_0} \longrightarrow \Omega$ è un rivestimento. Sia $z_0 \in \Omega$, $D \subseteq \Omega$ un disco di centro $z_0$.
    Per ogni $w_0 \in f_0^{-1}(z_0) \cap \Omega_{f_0}$ vogliamo un intorno $U_{w_0} \subseteq \Omega_{f_0}$ t.c. $f_0\restrict{U_{w_0}}:U_{w_0} \longrightarrow D$ è un biolomorfismo e $w_0 \not=w_0' \implies U_{w_0}\cap U_{w_0'}=\emptyset$.
    Per il primo teorema di Hurwitz, fissato $w_0$ esiste $n_1 \ge 1$ t.c. per ogni $n \ge n_1$ esiste $w_n \in \Omega_{f_n}$ t.c. $g_n(w_n)=z_0$ e $w_n \longrightarrow w_0$. Sia $h_n \in \text{Hol}(D, \mathbb{D})$ l'inversa di $f_n$ su $D$ con $h_n(z_0)=w_n$ (possiamo usare $D$ per tutte le $f_n$ perché, dalla teoria dei rivestimenti, dato un rivestimento un aperto semplicemente connesso dell'insieme di arrivo è sempre ben rivestito).
    Per Montel, a meno di sottosuccessioni $h_n \longrightarrow h_0 \in \text{Hol}(D, \mathbb{D})$ t.c. $f_0 \circ h_0=\id_D$ (perché $f_n \circ h_n=\id_D$ per ogni $n$), $h_0(z_0)=w_0$.
    Poniamo $U_{w_0}=h_0(D)$. È aperto perché $h_0$ non costante $\implies$ $h_0$ aperta, inoltre $U_{w_0} \subseteq \Omega_{f_0}$ perché, essendo immagine di un connesso, è connesso, e contiene $z_0=h_0(z_0)$, dunque deve stare nella componente connessa di $z_0$. Sia $w_0' \in f_0^{-1}(z_0) \cap \Omega_{f_0}$, costruiamo $h_0', U_{w_0'}$ come prima, senza perdita di generalità con la stessa sottosuccessione.
    Per assurdo, esiste $w \in U_{w_0} \cap U_{w_0'}$. Allora esistono $z_1, z_1' \in D$ t.c. $w=h_0(z_1)=h_0'(z_1')$.
    Applichiamo $f_0$ a entrambi i membri: $z_1=f_0(h_0(z_1))=f_0(w)=f_0(h_0(z_1'))=z_1' \implies z_1=z_1'$ è uno zero di $h_0-h_0'$, dunque per il primo teorema di Hurwitz o $h_0-h_0' \equiv 0 \implies w_0=w_0' \implies U_{w_0}=U_{w_0'}$ oppure per ogni $n>>1$ $h_n-h_n'$ ha uno zero, ma $h_n$ è l'inversa di $f_n$ su $D$ con $w_n \in h_n(D)$, $h_n'$ è l'inversa di $f_n$ su $D$ con $w_n' \in h_n'(D)$.
    $w_0 \not=w_0' \implies w_n \not=w_n'$ per ogni $n>>1 \implies h_n(D)\cap h_n'(D)=\emptyset$ perché inverse di un rivestimento che mandano lo stesso punto in due punti diversi, assurdo.
  \end{enumerate}

  Per il lemma \ref{esistefinF}, $\mathcal{F}\not=\emptyset$, quindi la costruzione che abbiamo fatto di $f_0$ ha senso. Per assurdo, $\displaystyle d_{f_0}=\sup_{f \in \mathcal{F}} d_f<1 \implies \Omega_{f_0} \subset \mathbb{D}$ con $\Omega_{f_0} \not= \mathbb{D}$.
  Sempre per il lemma \ref{esistefinF} esiste $f_1 \in \text{Hol}(\mathbb{D}, \mathbb{D})$, $f_1(0)=0, f_1'(0) \in \mathbb{R}, f_1'(0)>0$, $\Omega_{f_0} \subset f_1(\mathbb{D})$, $f_1\restrict{\Omega_{f_1}}:\Omega_{f_1} \longrightarrow \Omega_{f_0}$ rivestimento con $\Omega_{f_1}$ la componente connessa di $f_1^{-1}(\Omega_{f_0})$ contentente $0$ e $d_{f_1}>d_{f_0}$.
  Ma allora, se $\tilde{f}=f_0 \circ f_1:\Omega_{f_1} \longrightarrow \Omega$, abbiamo $\tilde{f} \in \mathcal{F}$ con $d_{\tilde{f}}=d_{f_1}>d_{f_0}$, assurdo.

  Per l'unicità si veda l'esercizio \ref{un_biolo}.
\end{proof}

\begin{exc} \label{un_biolo}
  Se $f_1, f_2:\mathbb{D} \longrightarrow \Omega$ sono rivestimenti con $f_1(0)=f_1(0)$ e $f_1'(0), f_2'(0)>0$, allora $f_1 \equiv f_2$. Hint: dato che sono rivestimenti, si sfruttano $h$ e $h^{-1}$ date dal seguente diagramma commutativo:
  \begin{center}
    \begin{tikzcd}
      & \mathbb{D} \arrow[dl, "h"', shift right] \arrow[d, "f_1"]\\
      \mathbb{D} \arrow[ru, "h^{-1}" right, shift right] \arrow[r, "f_2"'] & \Omega
    \end{tikzcd}
  \end{center}
\end{exc}

\begin{thm}
  (Riemann) Se $\Omega \subset \mathbb{C}$ è un dominio semplicemente connesso con $\Omega \not=\mathbb{C}$, allora $\Omega$ è biolomorfo a $\mathbb{D}$.
\end{thm}

\begin{proof}
  Basta far vedere che $\Omega$ è biolomorfo a un dominio limitato: quest'ultimo sarebbe semplicemente connesso e rivestito da $\mathbb{D}$ che è connesso, dunque per la teoria generale dei rivestimenti il rivestimento in questione sarebbe un omeomorfismo, ma dato che era anche un olomorfismo, allora è un biolomorfismo. Prendiamo $a \in \mathbb{C}\setminus \Omega$, $h \in \mathcal{O}(\Omega)$ t.c. $h(z)^2=z-a$ per ogni $z \in \Omega$. $h$ è iniettiva, ma vale di più: $h(z_1)=\pm h(z_2) \implies z_1-a=h(z_1)^2=h(z_2)^2=z_2-a \implies z_1=z_2$.
  Dunque $h$ è iniettiva e $h(\Omega) \cap (-h(\Omega))=\emptyset$. Fissato $z_0 \in \Omega$, sia $r>0$ t.c. $D=D(h(z_0), r) \subset h(\Omega) \implies D \cap(-h(\Omega))=\emptyset \implies |h(z)+h(z_0)| \ge r$ per ogni $z \in \Omega \implies 2|h(z_0)| \ge r$.
  Sia $f \in \mathcal{O}(\Omega)$ data da $f(z)=\dfrac{r}{4}\dfrac{1}{|h(z_0)|}\dfrac{h(z)-h(z_0)}{h(z)+h(z_0)}$. $f(z_0)=0$ e $f$ è iniettiva: $f(z_1)=f(z_2) \implies \dfrac{h(z_2)-h(z_0)}{h(z_1)+h(z_0)}=\dfrac{h(z_2)-h(z_0)}{h(z_2)+h(z_0)} \implies h(z_1)=h(z_2) \implies z_1=z_2$
  $\implies$ $f$ è un biolomorfismo tra $\Omega$ e $f(\Omega)$ e $f(\Omega) \subseteq \mathbb{D}$ in quanto $\left|\dfrac{h(z)-h(z_0)}{h(z)+h(z_0)}\right|=|h(z_0)|\left|\dfrac{1}{h(z_0)}-\dfrac{2}{h(z)+h(z_0)}\right| \le |h(z_0)|\left|\dfrac{1}{|h(z_0)|}+\dfrac{2}{|h(z)+h(z_0)|}\right| \le \dfrac{4|h(z_0)|}{r}$.
\end{proof}

Per il teorema di Liouville abbiamo che $\mathbb{C}$ non è biolomorfo a $\mathbb{D}$. Dato che $\widehat{\mathbb{C}}$ è compatta, abbiamo che non è biolomorfa né a $\mathbb{D}$ né a $\mathbb{C}$.

\begin{thm}
  (Uniformizzazione di Riemann) Se $X$ è una superficie di Riemann semplicemente connessa, allora $X$ è biolomorfo a $\widehat{\mathbb{C}}$, $\mathbb{C}$ o $\mathbb{D}$. Più in generale, se $X$ è una superficie di Riemann qualsiasi e $\pi:\widetilde{X} \longrightarrow X$ è un rivestimento universale, allora $\widetilde{X}$ è una superficie di Riemann e
  \begin{nlist}
    \item se $\widetilde{X}$ è biolomorfo a $\widehat{\mathbb{C}}$, allora anche $X$ è biolomorfo a $\widehat{\mathbb{C}}$ (caso ellittico);
    \item se $\widetilde{X}$ è biolomorfo a $\mathbb{C}$, allora $X$ è biolomorfo a $\mathbb{C}$, $\mathbb{C}^*$, oppure un toro $T_{\tau}=\faktor{\mathbb{C}}{(\mathbb{Z}+\tau \mathbb{Z})}$ con $\mathfrak{Im}\tau>0$ (caso parabolico);
    \item in tutti gli altri casi $\widetilde{X}$ è biolomorfo a $\mathbb{D}$ (caso iperbolico).
  \end{nlist}
\end{thm}

Quindi, se $\Omega \subset\subset \mathbb{C}$ è limitato e semplicemente connesso, abbiamo per il teorema di Riemann che esiste $f:\mathbb{D} \longrightarrow \Omega$ biolomorfismo. Domanda: possiamo estendere $f$ a un omeomorfismo da $\overline{\mathbb{D}}$ a $\overline{\Omega}$?

\begin{thm}
  (Carathéodory) Un biolomorfismo $\mathbb{D} \longrightarrow \Omega$ si estende continuo da $\overline{\mathbb{D}}$ a $\overline{\Omega}$ se e solo se $\partial \Omega$ è localmente connesso.
\end{thm}

\begin{cor}
  Si estende a un omeomorfismo se e solo se $\partial\Omega$ è una curva di Jordan (cioè immagine omeomorfa di $S^1$).
\end{cor}

Esiste una condizione su $\partial\Omega$ diversa che è equivalente all'estendibilità di $f^{-1}:\Omega \longrightarrow \mathbb{D}$ al bordo.


\subsection{Teoremi di Runge}
\begin{lm}
  Sia $K \subset \subset \mathbb{C}$ compatto, $V \supset K$ un intorno aperto. Allora esiste $g \in C^{\infty}(\mathbb{C})$ t.c. $g\restrict{K}=1$ e $supp(g) \subset V$ ($\implies g\restrict{\mathbb{C}\setminus V}\equiv 0$) [ricordiamo che $supp(g)=\overline{\{z \in \mathbb{C} \mid g(z)\not=0\}}$].
\end{lm}

\begin{proof}
  Sia $h: \mathbb{R} \longrightarrow \mathbb{R}$ data da $h(t)=\begin{cases}
    0 & \mbox{se }t\le 0\\ e^{-1/t} & \mbox{se }t>0
\end{cases}$, $h \in C^{\infty}(\mathbb{R})$. Sia $\eta: \mathbb{C} \longrightarrow \mathbb{C}$ data da $\eta(z)=\dfrac{h(1-|z|^2)}{h(1-|z|^2)+h(|z|^2-1/4)}$. $\eta \in C^{\infty}(\mathbb{C}), \eta(\mathbb{C})=[0, 1]$.
$\eta\restrict{D(0, 1/2)}\equiv 1$ e $\eta\restrict{\mathbb{C}\setminus \mathbb{D}} \equiv 0$. Dato $p \in K$, sia $r_p>0$ t.c. $D(p, 2r_p) \subset V$.
Allora, per compattezza di $K$, esistono $p_1, \dots, p_k \in K$ t.c. $\displaystyle K \subset \bigcup_{j=1}^k D(p_j, r_{p_j}/2) \subset \bigcup_{j=1}^k D(p_j, 2r_{p_j}) \subset V$. Poniamo $\displaystyle W=\bigcup_{j=1}^k D(p_j, r_{p_j})$.
Sia $g_j:\mathbb{C} \longrightarrow \mathbb{R}$, $g_j=\begin{cases}
  \eta\left(\dfrac{z-p_j}{r_{p_j}}\right) & \mbox{se }z\in D(p_j, 2r_{p_j})\\ 0 & \mbox{se }z\in\mathbb{C} \setminus \overline{D(p_j, r_{p_j})}
\end{cases}$, che è ben definita per come è definita $\eta$. $g_j \in C^{\infty}(\mathbb{C})$. Sia $g: \mathbb{C} \longrightarrow \mathbb{R}$, $\displaystyle g(z)=1-\prod_{j=1}^k (1-g_j(z))$. $g \in C^{\infty}(\mathbb{C})$.
Se $z \in K$, esiste $j$ t.c. $z \in D(p_j, r_{p_j}/2) \implies g_j(z)=1 \implies g(z)=1$. Se $z \not\in \overline{W}$, $z \not\in\overline{D(p_j, r_{p_j})}$ per ogni $j=1, \dots, k$ $\implies$ $g_j(z)=0$ per ogni $j$ $\implies$ $g(z)=0$ $\implies$ $supp(g) \subseteq \overline{W} \subset V$.
\end{proof}

\begin{thm}
  (Teorema di Cauchy generalizzato) Sia $\Omega \subset \subset \mathbb{C}$ un dominio limitato t.c. $\partial\Omega$ sia un numero finito di curve di Jordan. Sia $u \in C^1(\overline{\Omega}, \mathbb{C})$, cioè esiste $U$ intorno aperto di $\overline{\Omega}$ su cui $u$ si estende di classe $C^1$.
  Allora per ogni $w \in \Omega$ $\displaystyle u(w)=\int_{\partial \Omega} \dfrac{u(z)}{z-w}\diff z+\dfrac{1}{2\pi i}\int_{\Omega} \dfrac{\partial u/\partial\bar{z}}{z-w}\diff z \wedge \diff \bar{z}$.
\end{thm}

\begin{proof}
  Usando la formula di Gauss-Green abbiamo che $\displaystyle \int_{\partial \Omega}(f\diff x+g\diff y)=\int_{\Omega}\left(\dfrac{\partial g}{\partial x}-\dfrac{\partial f}{\partial y}\right)\diff x\diff y$. Sia $v \in C^1(\overline{\Omega}, \mathbb{C}), f=\mathfrak{Re}(v), g=\mathfrak{Im}(v)$.
  $v=f+ig, \diff z=\diff x+i\diff y \implies v\diff z=(f\diff x-g\diff y)+i(g\diff x+f\diff y)$.
  Per Gauss-Green, $\displaystyle \int_{\partial \Omega} v\diff z=\int_{\Omega} \left(-\dfrac{\partial g}{\partial x}-\dfrac{\partial f}{\partial y}\right)\diff x\diff y+i\int_{\Omega} \left(\dfrac{\partial f}{\partial x}-\dfrac{\partial g}{\partial y}\right)\diff x\diff y$.
  $\dfrac{\partial v}{\partial\bar{z}}=\dfrac{1}{2}\left(\dfrac{\partial v}{\partial x}+i\dfrac{\partial v}{\partial y}\right)=\dfrac{1}{2}\left(\dfrac{\partial f}{\partial x}-\dfrac{\partial g}{\partial y}\right)+\dfrac{1}{2}i\left(\dfrac{\partial f}{\partial y}+\dfrac{\partial g}{\partial x}\right)$.
  $\diff \bar{z}\wedge\diff z=(\diff x-i\diff y)\wedge(\diff x+i\diff y)=2i\diff x\wedge\diff y$.
  $\dfrac{\partial v}{\partial\bar{z}}\diff\bar{z}\wedge \diff z=\left[-\left(\dfrac{\partial f}{\partial y}+\dfrac{\partial g}{\partial x}\right)+i\left(\dfrac{\partial f}{\partial x}-\dfrac{\partial g}{\partial y}\right)\right]\diff x\wedge\diff y$.
  Allora $\displaystyle \int_{\partial \Omega}v\diff z=\int_{\Omega} \dfrac{\partial v}{\partial\bar{z}}\diff\bar{z}\wedge\diff z$ $(\star)$.
  Con il teorema di Stokes, $\displaystyle \int_{\partial \Omega}v\diff z=\int_{\Omega} \diff(v\diff z)=\int_{\Omega} \diff v\wedge\diff z=\int_{\Omega} \left(\dfrac{\partial v}{\partial z}\diff z+\dfrac{\partial v}{\partial\bar{z}}\diff\bar{z}\right)\wedge\diff z=\int_{\Omega} \dfrac{\partial v}{\partial\bar{z}} \diff \bar{z}\wedge\diff z$.
  Fissato $w \in \Omega$ poniamo $v(z)=\dfrac{u(z)}{z-w}$ su $\Omega_{\epsilon}=\Omega \cap \{|z-w|>\epsilon\}$ dove $\epsilon>0$ è sufficientemente piccolo. $\partial\Omega_{\epsilon}=\partial\Omega\cup\partial D(w, \epsilon)$. $v \in C^1(\overline{\Omega}_{\epsilon})$.
  Per $(\star)$, $\displaystyle \int_{\partial\Omega_{\epsilon}} \dfrac{u(z)}{z-w}\diff z=\int_{\Omega_{\epsilon}} \dfrac{\partial u/\partial\bar{z}}{z-w}\diff\bar{z}\wedge\diff z$.
  Parametrizziamo $\partial D(w, \epsilon)$ con $\gamma(t)=w+\epsilon e^{it}, t \in [0, 2\pi]$, abbiamo allora che
  $\displaystyle \int_{\partial\Omega_{\epsilon}} \dfrac{u(z)}{z-w}\diff z=\int_{\partial\Omega} \dfrac{u(z)}{z-w}\diff z-\int_{\partial D(w, \epsilon)} \dfrac{u(z)}{z-w}\diff z=\int_{\partial\Omega} \dfrac{u(z)}{z_{\epsilon}-w}\diff z-\int_0^{2\pi} u(w+\epsilon e^{it})i\diff t$.
  Mandando $\epsilon$ a $0$, dato che $u$ è continua, otteniamo $\displaystyle \int_{\partial\Omega} \dfrac{u(z)}{z-w}\diff z-2\pi iu(w)$, mentre $\displaystyle \int_{\Omega_{\epsilon}} \dfrac{\partial u/\partial\bar{z}}{z-w}\diff\bar{z}\wedge\diff z$ tende a $\displaystyle \int_{\Omega} \dfrac{\partial u/\partial\bar{z}}{z-w}\diff\bar{z}\wedge\diff z$ perché $\dfrac{1}{z-w}$ è integrabile sugli aperti limitati di $\mathbb{C}$. Mettendo tutto insieme si ha la tesi.
\end{proof}

\begin{defn}
  L'\textit{equazione di Cauchy-Riemann non omogenea} è $\dfrac{\partial u}{\partial\bar{z}}=\varphi$ dove l'incognita è $u$.
\end{defn}

\begin{lm} \label{misura}
  Sia $K \subset\subset \mathbb{C}$ compatto e $\mu$ una misura con $supp(\mu)=K$. Allora l'integrale $\displaystyle u(w)=\int \dfrac{1}{w-z}\diff\mu(z)$ definisce una funzione $u \in \mathcal{O}(\mathbb{C}\setminus K)$.
\end{lm}

\begin{proof}
  Una misura $\mu$ con $supp(\mu)=K$ è un elemento $\mu \in (C^0(K))^*$ continuo. $\displaystyle \int \dfrac{1}{w-z} \diff\mu(z)=\mu\left(\dfrac{1}{w-\cdot}\right), \dfrac{1}{w-\cdot} \in C^0(K)$ se $w\not\in K$.
  Sia $a \not\in K$, $r>0$ t.c. $\overline{D(a, r)}\cap K=\emptyset$, $w \in D(a, r)$.
  $\displaystyle \dfrac{1}{w-z}=\dfrac{1}{(a-z)\left(1-\dfrac{a-w}{a-z}\right)}=\sum_{n \ge 0} \dfrac{(a-w)^n}{(a-z)^{n+1}}$ in quanto $\left|\dfrac{a-w}{a-z}\right|<1$ per ogni $z \in K, w \in D(a, r)$, quindi $\displaystyle \mu\left(\dfrac{1}{w-z}\right)=\sum_{n\ge 0} (a-w)^n\mu\left(\dfrac{1}{(a-z)^{n+1}}\right)$ è una serie di potenze in $w$ che converge in $D(a,r)$ $\implies$ $u \in \mathcal{O}(\mathbb{C}\setminus K)$.
\end{proof}

\begin{thm}
  Sia $\varphi \in C^k(\mathbb{C})$ a supporto compatto ($\varphi \in C^k_{\text{\textsc{c}}}(\mathbb{C})$). Allora esiste $u \in C^k(\mathbb{C})$ t.c. $\dfrac{\partial u}{\partial\bar{z}}=\varphi$.
\end{thm}

\begin{proof}
  Poniamo $\displaystyle u(w)=\dfrac{1}{2\pi i}\int \dfrac{\varphi(z)}{w-z} \diff\bar{z}\wedge\diff z$. È la $u$ data dal lemma \ref{misura} con $\mu(2\pi i)^{-1}\varphi(\diff\bar{z}\wedge\diff z)=-\dfrac{1}{\pi}\varphi\diff y\diff x$.
  $u \in \mathcal{O}(\mathbb{C}\setminus K)$. Facciamo un cambiamento di variabile: $\zeta=w-z \implies z=w-\zeta, \diff\zeta=-\diff z, \diff\bar{\zeta}=-\diff\bar{z}$.
  $u(w)=\dfrac{1}{2\pi i} \int_{\mathbb{C}} \dfrac{\varphi(w-\zeta)}{\zeta}\diff\bar{\zeta}\wedge\diff\zeta$. Siccome $1/\zeta$ è integrabile sui compatti di $\mathbb{C}$ e $\varphi \in C^k_{\text{\textsc{c}}}(\mathbb{C})$ possiamo derivare sotto il segno di integrale e le derivate sono continue, $u \in C^k(\mathbb{C})$.
  $\displaystyle \dfrac{\partial u}{\partial \bar{w}}(w)=\dfrac{1}{2\pi i}\int_{\mathbb{C}} \dfrac{\frac{\partial \varphi}{\partial \bar{z}}(w-\zeta)}{\zeta} \diff \bar{z}\wedge\diff z=-\dfrac{1}{2\pi i}\int_{\mathbb{C}} \dfrac{\frac{\partial\varphi}{\partial\bar{z}}(z)}{z-w}\diff\bar{z}\wedge\diff z=\dfrac{1}{2\pi i}\int_{\mathbb{C}} \dfrac{\frac{\partial\varphi}{\partial\bar{z}}(z)}{z-w}\diff z\wedge\diff\bar{z}=\dfrac{1}{2\pi i}\int_{\Omega} \dfrac{\frac{\partial\varphi}{\partial\bar{z}}(z)}{z-w}\diff z\wedge\diff\bar{z}$ dove $\Omega \subset \subset \mathbb{C}$ è un disco con $K \subset\subset \Omega$.
  Quindi $\varphi\restrict{\partial \Omega}\equiv 0$ e il teorema di Cauchy generalizzato ci dà $\dfrac{\partial u}{\partial\bar{w}}(w)=\varphi(w)$ per ogni $w \in K$.
\end{proof}

\begin{oss}
  Se $K=supp(\varphi)$, $u \in \mathcal{O}(\mathbb{C} \setminus K)$.
\end{oss}

\begin{oss}
  Non è detto che $u$ abbia supporto compatto.
\end{oss}

\begin{oss}
  $u$ è unica a meno di funzioni in $\mathcal{O}(\mathbb{C})$.
\end{oss}

\begin{defn}
  Ricordiamo che se $K \subset\subset \mathbb{C}$ è compatto e $f \in C^0(K)$, allora poniamo $\|f\|_K=\sup_{z \in K}|f(z)|$.
  Definiamo adesso $\mathcal{O}(K)=\{f \in C^0(K) \mid$ esiste $(U, \tilde{f})$ dove $U \supset K$ è un intorno aperto di $K$, $\tilde{f} \in \mathcal{O}(U)$ e $\tilde{f}\restrict{K}\equiv f\}$.
\end{defn}

\begin{ex}
  Se $w \not\in K, f(z)=\dfrac{1}{w-z}$, allora $f \in \mathcal{O}(K)$.
\end{ex}

\begin{thm}
  (Primo teorema di Runge) Sia $K \subset\subset \mathbb{C}$ compatto, $\Omega \subseteq \mathbb{C}$ un intorno aperto di $K$. Le seguenti sono equivalenti:
  \begin{nlist}
    \item ogni $f \in \mathcal{O}(K)$ può essere approssimata uniformemente su $K$ da funzioni in $\mathcal{O}(\Omega)$;
    \item $\Omega \setminus K$ non ha componenti connesse relativamente compatte in $\Omega$;
    \item per ogni $z \in \Omega \setminus K$ esiste $f \in \mathcal{O}(\Omega)$ t.c. $|f(z)|>\|f\|_K$.
  \end{nlist}
\end{thm}

\begin{proof}
  (iii) $\implies$ (ii) Se (ii) è falso, esiste $U$ componente connessa di $\Omega \setminus K$ con $\overline{U} \subset \Omega$ e $\partial U \subseteq K$, dunque per il principio del massimo abbiamo, per ogni $g \in \mathcal{O}(\Omega)$ e per ogni $z \in U$, che $|g(z)|\le \max_{\zeta \in \partial U} |g(\zeta)| \le \|g\|_K$, contro (iii).

  (i) $\implies$ (ii) Se (ii) è falso, esiste $U$ componente connessa di $\Omega \setminus K$ con $\overline{U} \subset \Omega$ e $\partial U \subseteq K$. Sia $w \in U$, $f(z)=\dfrac{1}{w-z}$ e $f \in \mathcal{O}(K)$.
  Se (i) fosse vera esisterebbe $\{f_n\} \in \mathcal{O}(\Omega)$ t.c. $\|f_n-f\|_K \longrightarrow 0$ $\implies$ $\|f_m-f_n\|_K \longrightarrow 0$ per $m, n \longrightarrow +\infty$.
  Sempre per il principio del massimo, per ogni $z \in U$ $|g(z)|\le \max_{\zeta \in \partial U} |g(\zeta)| \le \|g\|_K$ $\implies$ $\|f_m-f_n\|_{\overline{U}} \longrightarrow 0$ per $m,n \longrightarrow +\infty$ $\implies$ $\{f_n\}$ è di Cauchy in $C^0(U \cup K)$ $\implies$ converga a una $F \in C^0(U \cup K) \subset C^0(\overline{U})$.
  Per il teorema di Weierestrass, $F \in \mathcal{O}(U)$. Su $K$ abbiamo che $(w-z)F(z)\equiv 1$. Ma allora applicando il principio del massimo a $(w-z)F(z)-1 \in \mathcal{O}(U) \cap C^0(\overline{U})$ otteniamo $(w-z)F(z)-1 \equiv 0$ su tutto $U$, impossibile (in $w$ fa $-1$).

  (ii) $\implies$ (i) Si vedrà.

  (i)+(ii) $\implies$ (iii) Fissiamo $z_0 \in \Omega \setminus K$. Sia $D \subset \Omega \setminus K$ un disco chiuso di centro $z_0$. Le componenti connesse di $\Omega \setminus (K \cup D)$ sono lo stesse di $\Omega \setminus K$ con una a cui è stato tolto $D$. In particolare $K \cup D$ soddisfa (ii).
  La funzione $g$ che è $0$ su $K$ e $1$ su $D$ appartiene a $\mathcal{O}(K \cup D)$, dunque per (i) può essere approssimata da funzioni in $\mathcal{O}(\Omega)$ $\implies$ esiste $f \in \mathcal{O}(\Omega)$ t.c. $\|f\|_K<1/2$ e $\|f-1\|_D<1/2$ $\implies$ $|f(z_0)|>1-1/2=1/2$ $\implies$ $\|f\|_K<1/2<|f(z_0)|$.
\end{proof}


\subsection{Applicazioni dei teoremi di Runge}
\begin{lm} \label{succ_K_nu}
  Sia $\Omega \subseteq \mathbb{C}$ aperto. Allora esiste una successione crescente $\{K_{\nu}\}$ di compatti in $\Omega$ t.c.: $K_{\nu}\subset \mathop {K_{\nu+1}}\limits^ \circ$, $\displaystyle \bigcup_{\nu} K_{\nu}=\Omega$ e $\widehat{(K_{\nu})}_{\Omega}=K_{\nu}$.
\end{lm}

\begin{proof}
  Poniamo $H_{\nu}=\{z \in \Omega \mid  d(z, \partial\Omega) \ge 1/\nu, |z| \le \nu\}$. $H_{\nu}$ è compatto, $H_{\nu}\subset \mathop {H_{\nu+1}}\limits^ \circ$ e $\displaystyle \bigcup_{\nu} H_{\nu}=\Omega$.
  Poniamo $K_1=\widehat{(H_1)}_{\Omega}$, che è compatto. Sia $\mu_1$ t.c. $K_1 \subset \mathop {H_{\mu_1}}\limits^ \circ$ e poniamo $K_2=\widehat{(H_{\mu_1})}_{\Omega}$. Continuando così abbiamo i $K_{\nu}$.
\end{proof}

\begin{thm}
  (Malgrange) Sia $\Omega \subseteq \mathbb{C}$ aperto, $\varphi \in C^{\infty}(\Omega)$. Allora esiste $u \in C^{\infty}(\Omega)$ t.c. $\dfrac{\partial u}{\partial\bar{z}}=\varphi$ $(\star)$ su $\Omega$.
\end{thm}

\begin{proof}
  Sappiamo che se $K \subset\subset \Omega$ compatto, allora esiste $v \in C^{\infty}(\Omega)$ che risolve $(\star)$ in un intorno di $K$. Infatti sia $\alpha \in C^{\infty}_C(\mathbb{C})$ con $\alpha \equiv 1$ in un intorno di $K$ e $0$ fuori da un intorno compatto e applichiamo quanto sappiamo a $\alpha\varphi$.
  Sia $\{K_{\nu}\}$ data dal lemma \ref{succ_K_nu}. Per ogni $\nu$ sia $v_{\nu} \in C^{\infty}(\Omega)$ soluzione di $(\star)$ in un intorno di $K_{\nu}$. Osserviamo che $v_{\nu+1}-v_{\nu} \in \mathcal{O}(K_{\nu})$.
  Per il primo teorema di Runge, esiste $h_{\nu} \in \mathcal{O}(\Omega)$ t.c. $\|v_{\nu+1}-v_{\nu}-h_{\nu}\|_{K_{\nu}}<2^{-\nu}$.
  Poniamo $\displaystyle u_{\nu}=v_{\nu}+\sum_{\mu \ge \nu} (v_{\mu+1}-v_{\mu}-h_{\mu})-\sum_{\mu=1}^{\nu-1} h_{\mu}$ su $K_{\nu}$. La serie è in $\mathcal{O}(K_{\nu})$ e la somma finita è in $\mathcal{O}(\Omega)$, quindi $u_{\nu}$ risolve $(\star)$ in un intorno di $K_{\nu}$.
  Ora, $u_{\nu}$ non dipende da $\nu$: $\displaystyle u_{\nu+1}=v_{\nu+1}+\sum_{\mu \ge \nu+1} (v_{\mu+1}-v_{\mu}-h_{\mu})-\sum_{\mu=1}^{\nu} h_{\mu}=v_{\nu}+(v_{\nu+1}-v_{\nu}-h_{\nu})+\sum_{\mu \ge \nu+1} (v_{\mu+1}-v_{\mu}-h_{\mu})-\sum_{\mu=1}^{\nu-1} h_{\mu}=v_{\nu}+\sum_{\mu \ge \nu} (v_{\mu+1}-v_{\mu}-h_{\mu})-\sum_{\mu=1}^{\nu-1} h_{\mu}=u_{\nu}$.
  Dunque abbiamo definito $u \in C^{\infty}(\Omega)$ ponendo $u\restrict{\mathop {K_{\nu}}\limits^ \circ}=u_{\nu}\restrict{\mathop {K_{\nu}}\limits^ \circ}$ e allora $\dfrac{\partial u}{\partial \bar{z}}\equiv \varphi$ su $\Omega$ in quanto vale su ciascun $K_{\nu}$.
\end{proof}

\begin{thm}
  (Mittag-Leffler) Sia $\Omega \subseteq \mathbb{C}$ aperto, $E \subset \Omega$ discreto e chiuso. Sia per ogni $a \in E$ $p_a \in \mathcal{O}(\mathbb{C}\setminus\{a\})$. Allora esiste $f \in \mathcal{O}(\Omega\setminus E)$ t.c. $f-p_a$ sia olomorfa in $a$ per ogni $a \in E$. In particolare, possiamo trovare $f \in \mathcal{M}(\Omega)$ con parti principali descritte.
\end{thm}

\begin{proof}
  Sia $\{K_{\nu}\}$ data dal lemma \ref{succ_K_nu}. Per ogni $\nu \ge 1$ poniamo $\displaystyle g_{\nu}=\sum_{a \in E \cap K_{\nu}} p_a$ (è una somma finita).
  Ora, $\displaystyle g_{\nu+1}-g_{\nu}=\sum_{a \in E \cap (K_{\nu+1}\setminus K_{\nu})} p_a \in \mathcal{O}(K_{\nu})$.
  Per il primo teorema di Runge, esiste $h_{\nu} \in \mathcal{O}(\Omega)$ t.c. $\|g_{\nu+1}-g_{\nu}-h_{\nu}\|_{K_{\nu}} \le 2^{-\nu}$.
  Sia $\displaystyle f=g_{\nu}+\sum_{\mu \ge \nu} (g_{\mu+1}-g_{\mu}-h_{\mu})-\sum_{\mu=1}^{\nu-1} h_{\mu}$.
  Come nella dimostrazione del teorema di Malgrange, $f$ non dipende da $\nu$ $\implies$ $f \in \mathcal{O}(\Omega\setminus E)$ (infatti, per ogni $\nu$, gli unici punti di $K_{\nu}$ dove $f$ può non essere olomorfa sono quelli di di $g_{\nu}$, dunque per definizione quelli di $E$).
  Sia $a \in E$ e sia $\nu \ge 1$ t.c. $a \in K_{\nu}$. La serie appartiene a $\mathcal{O}(K_{\nu})$, la somma finita appartiene a $\mathcal{O}(\Omega)$ $\implies$ $f-p_a=g_{\nu}-p_a+$qualcosa olomorfo in $K_{\nu}$ $\implies$ $f-p_a$ è olomorfa vicino ad $a$.
\end{proof}

\begin{cor}
  Sia $\Omega \subseteq \mathbb{C}$ aperto, $E \subset \Omega$ discreto e chiuso in $\Omega$. Siano dati per ogni $a \in E$ un intorno aperto $U_a \subset \Omega$ di $a$ e $\varphi_a \in \mathcal{O}(U_a \setminus \{a\})$. Allora esiste $f \in \mathcal{O}(\Omega\setminus E)$ t.c. $f-\varphi_a$ sia olomorfa vicino ad $a$ per ogni $a \in E$.
\end{cor}

\begin{proof}
  Sia $p_a$ la parte principale dello sviluppo di Laurent di $\varphi_a$ in $a$ $\implies$ $p_a \in \mathcal{O}(\mathbb{C}\setminus \{a\})$ e $\varphi_a-p_a$ è olomorfa in un intorno di $a$. Allora basta prendere $f$ data del teorema di Mittag-Leffler perché $f-\varphi_a=(f-p_a)-(\varphi_a-p_a)$.
\end{proof}

\begin{lm} \label{z-a/z-b}
  Sia $\Omega \subset \mathbb{C}$ aperto, $a, b \in \mathbb{C}\setminus\Omega$ appartenenti alla stessa comoponente connessa di $\mathbb{C}\setminus\Omega$. Allora esiste $f \in \mathcal{O}(\Omega)$ t.c. $e^{f(z)}=\dfrac{z-a}{z-b}$.
\end{lm}

\begin{proof}
  Per esercizio.
\end{proof}

\begin{thm}
  (Quarto teorema di Runge) Sia $\Omega \subseteq \mathbb{C}$ aperto, $K\subset\subset\Omega$ compatto t.c. $\widehat{K}_{\Omega}=K$. Sia $f \in \mathcal{O}(\Omega)(K)$ t.c. $f(z) \not=0$ per ogni $z \in K$. ù
  Allora per ogni $\epsilon>0$ esiste $F \in \mathcal{O}(\Omega)$ con $F(z) \not=0$ per ogni $z \in \Omega$ e $\|F-f\|_K<\epsilon$.
\end{thm}

\begin{proof}
  Siccome $f$ non si annulla su $K$, $\displaystyle \delta_0=\min_{z \in K} |f(z)|>0$. Quindi se $\tilde{f} \in \mathcal{O}(\Omega)$ t.c. $\|\tilde{f}-f\|_K<\delta_0/2$ allora $\tilde{f}(z)\not=0$ per ogni $z \in K$.
  Sappiamo che $\mathbb{C}\setminus K$ ha una componente connessa illimitata $U_0$ e un numero finito di componenti connesse limitate $U_1, \dots, U_p$ con $U_j \not\subset \Omega$; scegliamo per ogni $j=1, \dots, p$ $a_j \in U_j \setminus \Omega$. Possiamo approssimare $f$ con una funzione razionale $\tilde{f}$ con poli fuori da $K$; per l'osservazione fatta all'inizio della dimostrazione possiamo supporre che $\tilde{f}$ non si annulli mai in un intorno di $K$.
  Quindi $\displaystyle \tilde{f}(z)=c\prod_{\nu=1}^d (z-b_{\nu})^{m_{\nu}}$ con $c \in \mathbb{C}^*, m_{\nu} \in \mathbb{Z}^*, b_{\nu} \in \mathbb{C}\setminus K$. Fissiamo $R>0$ t.c. $K \subset\subset D(0,R)$ e poniamo $a_0=R \in U_0$.
  Per $j=0, \dots, p$ sia $A_j=\{\nu \mid b_{\nu} \in U_j\}$.
  Scriviamo $\displaystyle \tilde{f}(z)=cG(z)(z-R)^{n_0} \prod_{j=0}^p \prod_{\nu \in A_j} \left(\frac{z-b_{\nu}}{z-a_j}\right)^{m_{\nu}}$ dove $\displaystyle n_{j}=\sum_{\nu \in A_j} m_{\nu}$ e $\displaystyle G(z)=\prod_{j=1}^p (z-a_j)^{n_j}$.
  Se $\nu \in A_j$ allora $a_j$ e $b_{\nu}$ appartengono alla stessa componente connessa di $\mathbb{C} \setminus K$, dunque per il lemma \ref{z-a/z-b} esiste $\varphi_{\nu, j} \in \mathcal{O}(K)$ t.c. $\dfrac{z-b_{\nu}}{z-a_j}=e^{\varphi_{\nu, j}(z)}$.
  Inoltre esiste $\varphi_0 \in \mathcal{O}(D(0, R))$ t.c. $z-R=\exp(\varphi_0(z))$. Quindi esiste $h \in \mathcal{O}(K)$ t.c. $\tilde{f}(z)=cG(z)e^{h(z)}$. Per il primo teorema di Runge, per ogni $\delta>0$ esiste $H \in \mathcal{O}(\Omega)$ t.c. $\|H-h\|_K<\delta$.
  Poniamo $F=cGe^H \in \mathcal{O}(\Omega)$ e mai nulla su $\Omega$; inoltre $\|\tilde{f}-F\|_K=|c|\|G\|_K\|e^H-e^h\|_K$. A patto di prendere $\delta<<1$ possiamo rendere $\|\tilde{f}-F\|_K$ piccolo quanto vogliamo e quindi $\|f-F\|_K$ piccolo quanto vogliamo.
\end{proof}

\begin{exc}
  Sia $\{u_n\}$ una successione di funzioni complesse limitate definite su un insieme $S$ t.c. $\displaystyle \sum_{n} |u_n|$ converge uniformemente su $S$. Allora il prodotto $\displaystyle f(z)=\prod_{n=1}^{+\infty} (1+u_n(z))$ converge uniformemente su $S$ e $f(z_0)=0$ $\iff$ esiste $n \ge 1$ t.c. $u_n(z_0)=-1$.
\end{exc}

\begin{thm}
  (Weierstrass) Sia $\Omega \subseteq \mathbb{C}$ aperto, $E \subset \Omega$ discreto e chiuso in $\Omega$, $k:E \longrightarrow \mathbb{Z}$. Allora esiste $f \in \mathcal{M}(\Omega)$ t.c. $f \in \mathcal{O}(\Omega \setminus E)$, $f$ non ha zeri in $\Omega \setminus E$ e $(z-a)^{-k(a)}f(z)$ sia olomorfa mai nulla in un intorno di $a$ per ogni $a \in E$.
\end{thm}

\begin{proof}
  Sia $\{K_{\nu}\}$ data dal lemma \ref{succ_K_nu}. Poniamo $\displaystyle F_{\nu}(z)=\prod_{s \in E \cap K_{\nu}} (z-a)^{k(a)}$. In particolare, $F_{\nu+1}/F_{\nu} \in \mathcal{O}(K_{\nu})$ e non si annulla in $K_{\nu}$.
  Sia $\displaystyle \delta_{\nu}=\min_{z \in K_{\nu}} \left|\frac{F_{\nu+1}(z)}{F_{\nu}(z)}\right|>0$.
  Sia $g_{\nu} \in \mathcal{O}(\Omega)$ data dal quarto teorema di Runge mai nulla in $\Omega$ t.c. $\displaystyle \left\|\frac{F_{\nu+1}}{F_{\nu}}-g_{\nu}\right\|_{K_{\nu}}<\frac{2^{-\nu-1}\delta_{\nu}}{1+2^{-\nu-1}}$
  $\implies$ per ogni $z \in K_{\nu}$ $\displaystyle |g_{\nu}(z)| \ge \left|\frac{F_{\nu+1}(z)}{F_{\nu}(z)}\right|-\frac{2^{-\nu-1}\delta_{\nu}}{1+2^{-\nu-1}} \ge \delta_v-\frac{2^{-\nu-1}\delta_{\nu}}{1+2^{-\nu-1}}=\frac{\delta_{\nu}}{1+2^{-\nu-1}}$.
  Ponendo $h_{\nu}=\dfrac{1}{g_{\nu}} \in \mathcal{O}(\Omega)$ mai nulla in $\Omega$, $\left \|\dfrac{F_{\nu+1}}{F_{\nu}}h_{\nu}-1\right\|_{K_{\nu}}<2^{-\nu-1}$.
  Poniamo $\displaystyle f=F_{\nu}\prod_{\mu\ge\nu}\left(\frac{F_{\mu+1}}{F_{\mu}}h_{\mu}\right) \prod_{j=1}^{\nu-1}h_j$. $f\restrict{K_{\nu}}$ ha esattamente gli stessi zeri e poli di $F_{\nu}$; ma sempre per la stessa dimostrazione, $f$ non dipende da $\nu$. Quindi $f \in \mathcal{M}(\Omega)$ ed è come voluto.
\end{proof}

\begin{cor}
  Sia $\Omega \subset \mathbb{C}$ aperto. Allora ogni $q \in \mathcal{M}(\Omega)$ è il quoziente di due funzioni olomorfe in $\Omega$.
\end{cor}

\begin{proof}
  $E=\{$poli di $q\}$. Sia $k:E \longrightarrow \mathbb{Z}$ data da $k(a)=-ord_a(q)$. Allora il teorema di Weierstrass ci fornisce $g \in \mathcal{O}(\Omega)$ t.c. $gq \in \mathcal{O}(\Omega)$ $\implies$ $q=(gq)/g$ come voluto.
\end{proof}

\begin{defn}
  $\Omega \subset \mathbb{C}$ è il \textsc{dominio di esistenza} di $f \in \mathcal{O}(\Omega)$ se per ogni $p \in \partial\Omega$ non esiste $D=D(p,r)$ per cui esiste $F \in \mathcal{O}(D)$ t.c. $F\restrict{U}\equiv f\restrict{U}$ dove $U$ è la componente connessa di $D \cap \Omega$ t.c. $p \in \partial U$.
\end{defn}

\begin{prop}
  Sia $\Omega \subset \mathbb{C}$ un dominio. Allora $\Omega$ è il dominio di esistenza di una $f \in \mathcal{O}(\Omega)$.
\end{prop}

\begin{proof}
  Sia $\{K_{\nu}\}$ data dal lemma \ref{succ_K_nu}. Sia $\{D_n\}$ una successione di dischi aperti t.c.:
  \begin{enumerate}
    \item $\overline{D_n} \subset \Omega$;
    \item $K_1 \subset D_1 \cup \dots \cup D_{n_1}$ e $K_{\nu+1}\setminus\mathop {K_{\nu}}\limits^ \circ \subset D_{n_{\nu}+1}\cup \dots \cup D_{n_{\nu+1}}$;
    \item se $n \ge n_{\nu+1}+1$ allora $D_n \cap K_{\nu}=\emptyset$;
    \item il raggio di $D_n$ è minore di $1/\nu$ per $n_{\nu}+1 \le n \le n_{\nu+1}$.
  \end{enumerate}
  $\displaystyle \Omega=\bigcup_n D_n$, il raggio di $D_n$ tende a $0$, $\{D_n\}$ è localmente finita, cioè ogni punto ha un intorno che interseca solo un numero finito di $D_n$ ($\mathop {K_{\nu}}\limits^ \circ$). Sia $\{a_n\}$ una successione di punti distinti t.c. $a_n \in D_n$. $\{a_n\}$ è discreto e chiuso in quanto non ha punti di accumulazione in $\Omega$.
  Per il teorema di Weierstrass, esiste $f \in \mathcal{O}(\Omega)$ i cui zeri sono esattamente $\{a_n\}$ (in particolare $f \not\equiv 0$). Vogliamo mostrare che $\Omega$ è il dominio di esistenza di $f$. Per assurdo, sia $p \in \partial\Omega$, $D=D(p,\rho)$, $U$ la componente connessa di $D \cap \Omega$ con $p \in \partial U$ e sia $F \in \mathcal{O}(D)$ t.c. $F\restrict{U}\equiv f\restrict{U}$.
  Poniamo $D'=D(p,\rho/2)$. Si ha $p \in \partial \Omega \cap \partial (D' \cap U)$, per cui $D' \cap U$ non è relativamente compatto in $\Omega$ $\implies$ $D'\cap U$ interseca un numero infinito di dischi $D_{n_k}$.
  Siccome il raggio di $D_{n_k}$ tende a $0$, è definitivamente minore di $\rho/4$ $\implies$ un numero infinito di dischi $D_{n_k}$ sono contenuti in $D$ e intersecano $U$, ma $U$ è una componente connessa di $D \cap \Omega$ e i dischi $D_{n_k}$ sono connessi e stanno in $D \cap \Omega$, quindi infiniti $D_{n_k}$ sono contenuti in $D \cap U$. Più precisamente, $D_{n_k} \subset D(p, 3\rho/4) \cap U$.
  Quindi $F$ ha infiniti zeri distinti in $D(p, 3\rho/4) \subset \subset D(p,\rho)$ $\implies$ gli zeri di $F$ hanno un punto di accumulazione in $D$ $\implies$ $F \equiv 0$ $\implies$ $f \restrict{U} \equiv 0$ $\implies$ $f \equiv 0$ su $\Omega$, assurdo.
\end{proof}


\newpage

\section{Funzioni olomorfe in più variabili}

\subsection{Notazioni e definizione}
$z=(z_1, \dots, z_n) \in \mathbb{C}^n$. Se $\alpha \in \mathbb{N}^n$ è un multi-indice, $z^{\alpha}=z_1^{\alpha_1}\cdot \ldots \cdot z_n^{\alpha_n}$, $|\alpha|=\alpha_1+\dots+\alpha_n$, $\alpha!=\alpha_1!\cdot\ldots\cdot\alpha_n!$.
$z_j=x_j+iy_j$, $x_j,y_j \in \mathbb{R}$. $\dfrac{\partial}{\partial z_j}=\dfrac{1}{2}\left(\dfrac{\partial}{\partial x_j}-i\dfrac{\partial}{\partial y_j}\right)$, $\dfrac{\partial}{\partial\bar{z}_j}=\dfrac{1}{2}\left(\dfrac{\partial}{\partial x_j}+i\dfrac{\partial}{\partial y_j}\right)$.
$\diff z_j=\diff x_j+i\diff y_j$, $\diff\bar{z}_j=\diff x_j-i\diff y_j$. $\|z\|^2=|z_1|^2+\dots+|z_n|^2$. $\displaystyle \partial f=\sum_{j=1}^n \dfrac{\partial f}{\partial z_j}\diff z_j, \bar{\partial} f=\sum_{j=1}^n \dfrac{\partial f}{\partial \bar{z}_j}\diff \bar{z}_j$.

\begin{exc}
  $\partial+\bar{\partial}=\diff$, cioè $\displaystyle \partial f+\bar{\partial}f=\sum_{j=1}^n \left(\dfrac{\partial f}{\partial x_j}\diff x_j+\dfrac{\partial f}{\partial y_j}\diff y_j\right)$.
\end{exc}

$\dfrac{\partial^{|\beta|}}{\partial z^\beta}=\dfrac{\partial^{|\beta|}}{\partial z_1^{\beta_1}\dots\partial z_n^{\beta_n}}$.

\begin{exc}
  $\dfrac{\partial^{|\beta|}}{\partial z^\beta}(z-z^0)^\alpha=\dfrac{\alpha!}{(\alpha-\beta)!}(z-z^0)^{\alpha-\beta}$.
\end{exc}

$\diff x_1 \wedge \diff y_1 \wedge \dots \wedge \diff x_n \wedge \diff y_n=\left(\dfrac{1}{2i}\right)^n(\diff\bar{z}_1\wedge \diff z_1)\wedge \dots \wedge (\diff\bar{z}_n \wedge \diff z_n)$.
Dominio=aperto connesso. \textit{Palle aperte}: $B(z^0, r)=\{z \in \mathbb{C}^n \mid \|z-z^0\|<r\}$. $B^n=B(0,1)$.
\textit{Polidischi di poliraggio $\underline{r}=(r_1,\dots,r_n) \in (\mathbb{R}^+)^n$} ($r=(r, \dots, r) \in (\mathbb{R}^+)^n$): $P(z^0, \underline{r})=\{z \in \mathbb{C}^n \mid |z_j-z_j^0|<r_j\}=D(z_1^0, r_1) \times \dots \times D(z_n^0, r_n)$.
\textit{Polidisco unitario}: $\displaystyle \mathbb{D}^n=P(0, \underline{1})=\{z \in \mathbb{C}^n \mid \max_j |z_j|<1\}$. $(z_1, \dots, \hat{z}_j,\dots, z_n)=(z_1, \dots, z_{j-1}, z_{j+1}, \dots, z_n)$.

\begin{defn}
  Sia $\Omega \subset \mathbb{C}^n$ un dominio. $f: \Omega \in \mathbb{C}$ è \textsc{olomorfa} se soddisfa una delle seguenti condizioni equivalenti:
  \begin{nlist}
    \item per ogni $j$ e per ogni $(z_1, \dots, \hat{z}_j,\dots, z_n) \in \mathbb{C}^{n-1}$ la funzione che manda $\zeta \longmapsto f(z_1, \dots, z_{j-1}, \zeta, z_{j+1}, \dots, z_n)$ è olomorfa dove definita (\textit{olomorfa separatamente in ciascuna variabile});
    \item $f$ è $C^1$ in ciascuna variabile e $\dfrac{\partial f}{\partial \bar{z}_j} \equiv 0$ per ogni $j$ ($\bar{\partial} f\equiv 0$; \textit{Cauchy-Riemann});
    \item per ogni $z^0 \in \Omega$ esiste $r>0$ t.c. $P(z^0, r) \subset \Omega$ dove $\displaystyle f(z)=\sum_{\alpha \in \mathbb{N}^n} a_{\alpha}(z-z^0)^{\alpha}$ e la serie converge assolutamente (\textit{analitica});
    \item $f$ è $C^0$ in ciascuna variabile, localmente limitata e per ogni $z^0 \in \Omega$ esiste $r>0$ t.c. $P(z^0, r) \subset \Omega$ e
    $$f(z)=\dfrac{1}{(2\pi i)^n} \int_{|\zeta_1-z_1^0|=r}\dots\int_{|\zeta_n-z_n^0|=r} \frac{f(\zeta_1,\dots,\zeta_n)}{(\zeta_1-z_1)\dots(\zeta_n-z_n)}\diff\zeta_1\dots\diff\zeta_n$$ per ogni $z \in P(z^0, r)$ (\textit{formula di Cauchy}).
  \end{nlist}
\end{defn}


\subsection{Prime differenze con le funzioni in una variabile}
Prima di studiare quali risultati per le funzioni olomorfe in una variabile si mantengono nel caso in più variabili, vediamo prima un po' di differenze semplici da dimostrare ma molto distintive. Cominciamo con il \textit{fenomeno di Hartogs}: l'ultima cosa che abbiamo visto per le funzioni in una variabile è che ogni dominio è dominio di esistenza per una certa funzione olomorfa. Questo è in generale falso per funzioni in più variabili.

\begin{prop}
  (Hartogs) Sia $D=\mathbb{D}^2\setminus P(0,1/2)$. Ogni $f \in \mathcal{O}(D)$ si estende a una $\tilde{f} \in \mathcal{O}(\mathbb{D}^2)$.
\end{prop}

\begin{proof}
  Se $|z|<3/4$ e $1/2<|w|<1$, per Cauchy in una variabile abbiamo che $\displaystyle f(z,w)=\frac{1}{2\pi i}\int_{\partial D(0, 3/4)} \frac{f(\zeta, w)}{\zeta-z}\diff z$. L'integrale è ben definito per $|z|<3/4$ e $|w|<1$ ed è olomofo in $z$ e $w$, dunque estende $f$ a tutto $\mathbb{D}^2$ (che coincida con $f$ anche sui punti di $D$ che non erano stati considerati prima di fare l'integrale discende dal principio di identità).
\end{proof}

Problema: caratterizzare i domini di esistenza di funzioni olomorfe in più variabili (\textit{domini di olomorfia}). Non vedremo molto in questo senso. Vediamo invece alcune cose per quanto riguarda l'essere biolomorfi. In una variabile, il teorema di uniformizzazione di Riemann ci dava una caratterizzazione dei domini tra loro biolomorfi basata esclusivamente sulla topologia del dominio. Vedremo che questo non è possibile in più variabili. Ci sono problemi tra domini "lisci" (in termini di differenziabilità) e non, ma anche piccolissime variazioni lisce possono causare problemi. Ecco un paio di risultati in questo senso.

\begin{ex}
  Non vale nulla che assomigli al teorema di uniformizzazione di Riemann poiché $B^n$ non è biolomorfa e $\mathbb{D}^n$ (Poincaré).
\end{ex}

\begin{ex}
  (Greene-Krantz) "I foruncoli fanno male": se si prende un dominio liscio, come ad esempio $B^n$, è possibile fare una modifica "minuscola", cioè localizzata in un intorno di un punto, e che mantenga comunque il dominio liscio, tale che quello che si ottiene non è biolomorfo al dominio originale. Vale di più: esiste un'infinità più che numerabile di domini omeomorfi alla palla che a due a due non sono biolomorfi.
\end{ex}

\begin{ex}
  Non esistono zeri isolati. Infatti, se $f \in \mathcal{O}(\Omega)$ ha uno zero isolato $z^0$, $1/f$ sarebbe olomorfa in $P(z^0,r) \setminus P(z^0, r/2)$ per $r<<1$, ma non estendibile a $P(z^0, r)$, contro Hartogs, assurdo.
\end{ex}

Un'altra cosa che cambia sono i domini di convergenza delle serie di potenze: in una variabile sappiamo che sono dischi, in più variabili invece vediamo.

\begin{ex}
  $\displaystyle \sum_{n \ge 0} (z_1+z_2)^n$ converge in $|z_1+z_2|<1$; \\
  $\displaystyle \sum_{n \ge 0} (z_1z_2)^n$ converge in $|z_1z_2|<1$, che è un insieme illimitato; \\
  $\displaystyle \sum_{n \ge 0} z_1^n$ converge in $|z_1|<1$, cioè $\mathbb{D}\times \mathbb{C}$.
\end{ex}

Un'altra differenza è l'equazione di Cauchy-Riemann non omogenea.

\begin{ex}
  In una variabile abbiamo risolto $\bar{\partial}u=\psi$ con $\psi \in C^\infty_C(\mathbb{C})$. In generale, però, non c'è una soluzione $u \in C^\infty_C(\mathbb{C})$. Infatti, supponento che esista, $u$ avrebbe supporto compatto, per cui $supp(u) \subset D(0,R)$.
  Allora $\displaystyle 0=\int_{\partial D(0,r)} u(z)\diff \zeta$, che per Gauss-Green o Stokes è uguale a $\displaystyle \int_{D(0,R)} \dfrac{\partial u}{\partial \bar{\zeta}} \diff\bar{\zeta}\wedge\diff\zeta=2i\int_{\partial D(0,R)} \psi \diff x\wedge\diff y$, che in generale è diverso da $0$. Quindi se $\displaystyle \int_{\mathbb{C}} \psi \diff x \diff y\not=0$ allora $u$ non può avere supporto compatto.
  Invece, se $n \ge 2$, $\psi_1, \dots, \psi_n \in C^\infty_C(\mathbb{C}^n)$, $\psi=\psi_1\diff \bar{z}_1+\dots+\psi_n\diff \bar{z}_n$ con condizioni di compatibilità ($\frac{\partial \psi_h}{\partial \bar{z}_k}=\frac{\partial \psi_k}{\partial \bar{z}_h}$), allora esiste $u \in C^\infty_C(\mathbb{C}^n)$ t.c. $\bar{\partial}u=\psi$.
  Le condizioni di compatibilità sono necessarie, infatti se $u$ è una soluzione $\dfrac{\partial \psi_j}{\partial \bar{z}_l}=\dfrac{\partial^2u}{\partial \bar{z}_l\partial \bar{z}_j}=\dfrac{\partial^2u}{\partial \bar{z}_j\partial \bar{z}_l}=\dfrac{\partial \psi_l}{\partial \bar{z}_j}$.
\end{ex}


Per dimostrare il riultato appena enunciato useremo dei risultati che sappiamo essere veri in una variabile e che nella prossima sezione dimostreremo anche per funzioni in più variabili.

\begin{thm} \label{equazionaccia}
  Se $n \ge 2$, $\psi_1, \dots, \psi_n \in C^\infty_C(\mathbb{C}^n)$, $\psi=\psi_1\diff \bar{z}_1+\dots+\psi_n\diff \bar{z}_n$, $\dfrac{\partial \psi_j}{\partial \bar{z}_l}=\dfrac{\partial \psi_l}{\partial \bar{z}_j}$, allora esiste $u \in C^{\infty}_C(\mathbb{C}^n)$ t.c. $\bar{\partial}u=\psi$.
\end{thm}

\begin{proof}
  Fissiamo $j\not=1$ e poniamo $\displaystyle u_j(z)=\frac{1}{2\pi i}\int_{\mathbb{C}} \dfrac{\psi_j(z_1, \dots, z_{j-1}, \zeta, z_{j+1}, \dots, z_n)}{\zeta-z_j}\diff\bar{\zeta}\wedge\diff\zeta$.
  $\displaystyle \frac{\partial u_j}{\partial \bar{z}_l}(z)=\frac{1}{z\pi i}\frac{\partial}{\partial\bar{z}_l}\int_{\mathbb{C}} \dfrac{\psi_j(z_1, \dots, z_{j-1}, \zeta+z_j, z_{j+1}, \dots, z_n)}{\zeta}\diff\bar{\zeta}\wedge\diff\zeta=$\\
  $\displaystyle =\frac{1}{z\pi i}\int_{\mathbb{C}} \dfrac{\frac{\partial\psi_j}{\partial\bar{z}_l}(z_1, \dots, z_{j-1}, \zeta+z_j, z_{j+1}, \dots, z_n)}{\zeta}\diff\bar{\zeta}\wedge\diff\zeta$.
  Dalle condizioni di compatibilità, è uguale a $\displaystyle \frac{1}{z\pi i}\int_{\mathbb{C}} \dfrac{\frac{\partial\psi_l}{\partial\bar{z}_j}(z_1, \dots, z_{j-1}, \zeta+z_j, z_{j+1}, \dots, z_n)}{\zeta}\diff\bar{\zeta}\wedge\diff\zeta=$ \\
  $\displaystyle =\frac{1}{z\pi i}\int_{\mathbb{C}} \dfrac{\frac{\partial\psi_l}{\partial\bar{z}_j}(z_1, \dots, z_{j-1}, \zeta, z_{j+1}, \dots, z_n)}{\zeta-z_j}\diff\bar{\zeta}\wedge\diff\zeta$.
  Dato che $supp(\psi_l)$ è compatto, possiamo limitare l'integrale a un disco abbastanza grande da contenerlo tutto e avere $\psi_l$ nulla sul bordo di tale disco. Allora per il teorema di Cauchy generalizzato l'ultimo integrale viene proprio $\psi_l(z)$. $\psi_l$ a supporto compatto $\implies$ $\dfrac{\partial u_j}{\partial \bar{z}_l}$ a supporto compatto $\implies$ $\dfrac{\partial u_j}{\partial \bar{z}_l}=0$ per ogni $l$ se $\|z\|>>1$ $\implies$ $u_j(z)$ è olomorfa per $\|z\|>>1$.
  Ma $\psi_j(z)\equiv 0$ non appena $|z_1|>>1$ (ricordiamo che $j\not=1$) $\implies$ $u_j(z) \equiv 0$ non appena $|z_1|>>1$ (per definizione), dunque per il principio di identità $u_j(z) \equiv 0$ se $\|z\|>1$ $\implies$ $supp(u_j)$ è compatto.
\end{proof}

\begin{oss}
  $u$ è unica. Infatti, se $u_1, u_2$ sono soluzioni a supporto compatto, $\bar{\partial}(u_1-u_2) \equiv 0$ $\implies$ $u_1-u_2 \in \mathcal{O}(\mathbb{C}^n)$ a supporto compatto, dunque per il principio di identità $u_1-u_2 \equiv 0$.
\end{oss}

\begin{thm}
  (Serre, Ehrenpreiss) Sia $\Omega \subseteq \mathbb{C}^n$ un dominio, $K \subset \subset \Omega$ compatto t.c. $\Omega \setminus K$ sia connesso. Allora ogni $f \in \mathcal{O}(\Omega\setminus K)$ si estende olomorficamente a tutto $\Omega$.
\end{thm}

\begin{proof}
  Sia $\varphi \in C^{\infty}(\mathbb{C}^n)$ con $\varphi \equiv 0$ in un intorno $U$ di $K$, $\varphi \equiv 1$ in un intorno $V$ di $\mathbb{C}^n\setminus \Omega$ e con $supp(1-\varphi) \subset \subset \Omega$.
  Data $f \in \mathcal{O}(\Omega\setminus K)$ poniamo $\tilde{f}(z)=\begin{cases} \varphi f & \mbox{in }\Omega\setminus K \\ 0 & \mbox{in }U\end{cases}$ $\implies$ $\tilde{f} \in C^{\infty}(\Omega)$.
  Poniamo $\psi=\bar{\partial}\tilde{f}=\psi_1\diff\bar{z}_1+\dots+\psi_n\diff\bar{z}_n$.
  $\psi_1, \dots, \psi_n \in C^{\infty}(\mathbb{C}^n)$ perché $\tilde{f}\equiv f$ vicino a $\partial\Omega$ $\implies$ $\psi \equiv \bar{\partial} f\equiv 0$ vicino a $\partial\Omega$ $\implies$ possiamo estenderla $\equiv 0$ fuori da $\Omega$ ed è $C^\infty$.
  Inoltre $\dfrac{\partial \psi_j}{\partial \bar{z}_l}=\dfrac{\partial^2 \tilde{f}}{\partial\bar{z}_l\partial\bar{z}_j}=\dfrac{\partial^2 \tilde{f}}{\partial\bar{z}_j\partial\bar{z}_l}=\dfrac{\partial \psi_l}{\partial \bar{z}_j}$, dunque le condizioni di compatibilità sono soddisfatte.
  Infine $\psi$ è a supporto compatto perché $supp(\psi) \subset supp(1-\varphi)$. Per il teorema \ref{equazionaccia}, esiste $u \in C^{\infty}_C(\mathbb{C}^n)$ t.c. $\bar{\partial}u=\psi$. Siccome è a supporto compatto, $u \equiv 0$ in un intorno $W$ della componente connessa di $\partial\Omega$ che interseca la componente connessa illimitata di $\mathbb{C}^n\setminus K$.
  Poniamo $F=\tilde{f}-u \in C^{\infty}(\Omega)$. Chiaramente $\bar{\partial}F \equiv 0$ $\implies$ $F \in \mathcal{O}(\Omega)$.
  $u \equiv 0$ su $W \cap \Omega$ $\implies$ $F\restrict{W \cap V \cap \Omega}\equiv \tilde{f}\restrict{W \cap V \cap \Omega}\equiv f\restrict{W \cap V \cap \Omega}$, ma $W \cap V \cap \Omega \subseteq \Omega\setminus K$ ed è aperto, quindi per connessione di $\Omega \setminus K$ e per il principio di identità $F\restrict{\Omega\setminus K}\equiv f\restrict{\Omega \setminus K}$.
\end{proof}


\subsection{Risultati analoghi a quelli in una variabile}
Vediamo ora alcuni risultati analoghi a quelli che conosciamo per le funzioni olomorfe in una variabile. Abbiamo già usato uno di questi risultati, il principio di identità, nella sezione precedente. Inizialmente prendiamo come definizione di funzione olomorfa quella di analitica, poi tra le altre cose vedremo l'equivalenza con le altre definizioni date.

\begin{lm}
  (Abel) Data $\{c_{\alpha}\}_{\alpha \in \mathbb{N}^n} \subset \mathbb{C}$, supponiamo che esistano $\rho_1, \dots, \rho_n>0$ ($\underline{\rho}=(\rho_1, \dots, \rho_n)$), $M>0$ t.c. $|c_{\alpha}|\rho_1^{\alpha_1}\dots\rho_n^{\alpha_n} \le M$ per ogni $\alpha \in \mathbb{N}^n$.
  Allora $\displaystyle \sum_{\alpha \in \mathbb{N}^n} c_{\alpha}(z-z^0)^{\alpha}$ converge assolutamente in $P(z^0, \underline{\rho})$ e uniformemente in $\overline{P(z^0, \theta\underline{\rho})}$ per ogni $0<\theta<1$.
  Inoltre, la stessa convergenza vale per $\displaystyle \sum_{\alpha \in \mathbb{N}^n} c_{\alpha} \dfrac{\partial}{\partial z^\beta}(z-z^0)^\alpha$ per ogni $\beta \in \mathbb{N}^n$.
\end{lm}

\begin{cor} \label{coeff_anal_multi}
  $\mathcal{O}(\Omega) \subset C^\infty(\Omega)$ e $c_\alpha=\dfrac{1}{\alpha!}\dfrac{\partial f}{\partial z^\alpha}(z_0)$.
\end{cor}

\begin{cor}
  $f \in \mathcal{O}(\Omega)$ $\implies$ $\bar{\partial}f \equiv 0$.
\end{cor}

\begin{proof}
  $\displaystyle f(z)=f(z^0)+\sum_{j=1}^n \dfrac{\partial f}{\partial z_j}(z^0)(z_j-z_j^0)+o(\|z-z^0\|)$. Il pezzo $\displaystyle f(z^0)+\sum_{j=1}^n \dfrac{\partial f}{\partial z_j}(z^0)(z_j-z_j^0)$ ha ovviamemte $\bar{\partial}(\dots)=0$ perché sono termini costanti o lineari nelle variabili $z_j$, mentre per un $o$-piccolo di termini di ordine $1$ qualunque derivata in $z^0$ vale $0$.
\end{proof}

\begin{cor}
  $f \in \mathcal{O}(\Omega)$ $\implies$ $f$ è olomorfa in ciascuna variabile.
\end{cor}

\begin{prop}
  (Principio di identità) Sia $\Omega \subseteq \mathbb{C}^n$ un dominio e $f \in \mathcal{O}(\Omega)$. Se $\{z \in \Omega \mid f(z)=0\}$ ha parte interna non vuota, $f \equiv 0$.
\end{prop}

\begin{proof}
  Per ogni $\alpha \in \mathbb{N}^n$, sia $E_\alpha=\{z \in \Omega \mid \dfrac{\partial f}{\partial z^\alpha}(z)=0\}$ e $\displaystyle E=\bigcap_{\alpha \in \mathbb{N}^n} E_\alpha$.
  $E$ è ovviamente chiuso, ma è anche aperto per analiticità e per il corollario \ref{coeff_anal_multi}, ed è non vuoto per ipotesi, dunque essendo $\Omega$ connesso $E=\Omega$.
\end{proof}

\begin{prop}
  (Formula di Cauchy) Sia $\Omega \subset \mathbb{C}^n$ un dominio, $f \in \mathcal{O}(\Omega)$, $\overline{P(z^0,\underline{r})} \subset \Omega$. Allora per ogni $z \in P(z^0, \underline{r})$ $\displaystyle f(z)=$\\
  $\displaystyle =\dfrac{1}{(2\pi i)^n} \int_{|\zeta_1-z_1^0|=r}\dots\int_{|\zeta_n-z_n^0|=r} \frac{f(\zeta_1,\dots,\zeta_n)}{(\zeta_1-z_1)\dots(\zeta_n-z_n)}\diff\zeta_1\dots\diff\zeta_n=: \dfrac{1}{(2\pi i)^n} \int_{|\zeta-z^0|=\underline{r}} \dfrac{f(\zeta)}{(\zeta-z)}\diff \zeta$.
\end{prop}

\begin{proof}
  Poi
\end{proof}

\begin{oss}
  $\hat{\partial}P:=\{|\zeta-z^0|=\underline{r}\} \subset \partial P(z^0, \underline{r})$, in particolare è strettamente contenuto. È chiamato \textit{bordo di Šilov di $P(z^0, \underline{r})$}.
\end{oss}

\begin{cor}
  Sia $f \in \mathcal{O}(\Omega)$, $\overline{P(z^0, \underline{r})} \subset \Omega, \alpha \in \mathbb{N}^n$. Allora $\displaystyle \frac{\partial f}{\partial z^\alpha}=\dfrac{\alpha!}{(2\pi i)^n} \int_{|\zeta-z^0|=\underline{r}} \frac{f(\zeta)}{(\zeta-z)^{\alpha+1}}\diff \zeta$ su $P(z^0, \underline{r})$.
\end{cor}

\begin{proof}
  Basta derivare la formula di Cauchy.
\end{proof}


\subsection{Domini di convergenza delle serie di potenze}
\begin{defn}
  Sia $\displaystyle \sum_{\alpha} a_{\alpha}z^{\alpha}$ una serie di potenze in $\mathbb{C}^n$. Il \textsc{dominio di convergenza di $F$} è $\displaystyle \mathcal{C}=int\{z \in \mathbb{C}^n \mid \sum_{\alpha} |a_{\alpha}||z^{\alpha}|<+\infty\}$ ($int$=parte interna).
\end{defn}

\begin{oss}
  Per il lemma di Abel, $\displaystyle \mathcal{C}=int\{z \in \mathbb{C}^n \mid \sup_{\alpha} |a_{\alpha}||z^{\alpha}|<+\infty\}$.
\end{oss}

\begin{defn}
  Un insieme $S \subseteq \mathbb{C}^n$ è \textit{circolare} se per ogni $\theta \in \mathbb{R}$ $z \in S \implies e^{i\theta} \in S$.
  È \textsc{di Reinhardt} se per ogni $\theta_1, \dots, \theta_n \in \mathbb{R}$ $z \in S \implies (e^{i\theta_1}z_1, \dots, e^{i\theta_n}z_n) \in S$.
  È \textsc{circolare completo} se per ogni $\zeta_1, \dots, \zeta_n \in \overline{\mathbb{D}}$ $z \in S \implies (\zeta_1z_1,\dots,\zeta_nz_n) \in S$.
\end{defn}

\begin{oss}
  $S$ circolare completo $\implies$ $0 \in S$. Circolare completo $\implies$ di Reinhardt $\implies$ circolare. In una variabile, circolare $\implies$ di Reinhardt.
\end{oss}

\begin{oss}
  Un dominio di convergenza di una serie di potenze è circolare completo.
\end{oss}

\begin{defn}
  Sia $S \subseteq \mathbb{C}^n$. L'\textit{immagine logaritmica di $S$} è $\displaystyle \log{|S|}=\{(\log{|z_1|}, \dots, \log{|z_n|}) \mid z=(z_1, \dots, z_n) \in S \setminus \bigcup_{j=1}^n \{z_j=0\} \} \subseteq \mathbb{R}^n$.
\end{defn}

\begin{defn}
  $S$ è \textsc{logaritmicamente convesso} se $\log{|S|}$ è convesso.
\end{defn}

\begin{prop}
  Il dominio di convergenza $\mathcal{C}$ di una serie di potenze $F$ è logaritmicamente convesso.
\end{prop}

\begin{proof}
  Siano $w, w' \in \mathcal{C}$ e sia $\epsilon>0$ t.c. $P(0, |w|+\epsilon), P(0, |w'|+\epsilon) \subset \mathcal{C}$ ($|w|+\epsilon=(|w_1|+\epsilon, \dots, |w_n|+\epsilon)$). La condizione di logaritmica convessità di $\mathcal{C}$ segue se dimostriamo che per ogni $\lambda \in [0,1]$ $\lambda\log{|w|}+(1-\lambda)\log{|w'|} \in \log{|\mathcal{C}|}$.
  Questo equivale a: per ogni $\lambda \in [0,1]$ $(\log{(|w_1|^{\lambda}|w_1'|^{1-\lambda})},\dots,\log{(|w_n|^{\lambda}|w_n'|^{1-\lambda})}) \in \log{|\mathcal{C}|}$, che a sua volta è come dire che per ogni $\lambda \in [0,1]$ $(|w_1|^{\lambda}|w_1'|^{1-\lambda},\dots,|w_n|^{\lambda}|w_n'|^{1-\lambda}) \in \mathcal{C}$.
  Scriviamo $\displaystyle F=\sum_{\alpha} a_{\alpha}z^{\alpha}$. Per le disuguaglianze di Cauchy, $|a_{\alpha}| \le \dfrac{c}{\max\{(|w|+\epsilon)^{\alpha},(|w'|+\epsilon)^{\alpha}\}}$ per un certo $c$ che dipende da $|w|+\epsilon$ e $|w'|+\epsilon$.
  Poiché la funzione $t \longmapsto a^tb^{1-t}$ su $[0,1]$ è convessa per ogni $a, b>0$, per ogni $\lambda \in [0,1]$ e per ogni $j=1,\dots,n$ si ha $\max\{|w_j|+\epsilon, |w_j'|+\epsilon\} \ge (|w_j|+\epsilon)^{\lambda}(|w_j'|+\epsilon)^{1-\lambda} \ge |w_j|^{\lambda}|w_j'|^{1-\lambda}+\epsilon'$ per qualche $0<\epsilon'\le\epsilon$ perché la funzione $(a+\epsilon)^{\lambda}(b+\epsilon)^{1-\lambda}-a^{\lambda}b^{1-\lambda}$ è continua e $>0$ su $[0,1]$ compatto, dunque ammette minimo $>0$.
  Quindi per ogni $\alpha_j$ $\max\{(|w_j|+\epsilon)^{\alpha_j},(|w_j'|+\epsilon)^{\alpha_j}\} \ge (|w_j|^{\lambda}|w_j'|^{1-\lambda}+\epsilon')^{\alpha_j} \ge (|w_j|^{\lambda}|w_j'|^{1-\lambda})^{\alpha_j} \implies |a_{\alpha}| \le \dfrac{c}{\prod_j (|w_j|^{\lambda}|w_j'|^{1-\lambda})^{\alpha_j}} \implies (|w_1|^{\lambda}|w_1'|^{1-\lambda}, \dots, |w_n|^{\lambda}|w_n'|^{1-\lambda}) \in \mathcal{C}$.
\end{proof}

\begin{ftt}
  Viceversa, ogni dominio circolare completo logaritmicamente convesso è il dominio di convergenza di una serie di potenze.
\end{ftt}

\begin{defn}
  Sia $S \subseteq \mathbb{C}^n$ di Reinhardt, indichiamo con $\hat{C} \subset \mathbb{R}^n$ l'\textit{inviluppo convesso di $\log{|S|}$}, cioè il più piccolo convesso che contiene $\log{|S|}$, che equivale all'intersezione di tutti i convessi che contengono $\log{|S|}$.
\end{defn}

\begin{oss}
  $S$ aperto $\implies$ $\log{|S|}$ aperto $\implies$ $\hat{C}$ aperto.
\end{oss}

\begin{defn}
  Sia $\hat{S} \subseteq \mathbb{C}^n$ l'unico insieme di Reinhardt t.c. $\log{|\hat{S}|}=\hat{C}$. $\hat{S}$ è l'\textsc{inviluppo logaritmico di $S$}.
\end{defn}

\begin{prop}
  Sia $\Omega \subseteq \mathbb{C}^n$ dominio di Reinhardt con $0 \in \Omega$ e $f \in \mathcal{O}(\Omega)$. Allora lo sviluppo in serie di $f$ in $0$ converge in $\hat{\Omega}$.
\end{prop}

\begin{proof}
  Per ogni $j \ge 1$ sia $\Omega_j$ la componente connesse di $\{z \in \Omega \mid d(z,\partial\Omega)>\|z\|/j\}$ contenente $0$. Fissiamo $j, z \in \Omega_j$.
  Allora $(\zeta_1, \dots, \zeta_n) \longmapsto f(\zeta_1z_1,\dots,\zeta_nz_n)$ è ben definita per $|z_1|=\dots=|\zeta_n|=1+1/j$ poiché $\Omega$ è di Reinhardt, dunque $\displaystyle f_z(w)=\frac{1}{(2\pi i)^n} \int_{|\zeta|=1+1/j} \frac{f(\zeta_1z_1, \dots, \zeta_nz_n)}{(\zeta_1-w_1)\dots(\zeta_n-w_n)}\diff \zeta$ $(\star)$ è olomorfa in $P(0, 1+1/j)$.
  Quando $\|z\|<<1$, siccome $\Omega$ è aperto e di Reinhardt e $0 \in \Omega$, abbiamo che $(\zeta_1z_1,\dots,\zeta_nz_n) \in \overline{P(0,1+1/j)}$ per ogni $\zeta$ (rivedere da qui)
\end{proof}


\subsection{Domini di olomorfia}
\begin{defn}
  Sia $\Omega \subset \mathbb{C}^n$ un dominio. $P \in \partial \Omega$ è \textit{essenziale} se esiste $u \in \mathcal{O}(\Omega)$ t.c. per ogni intorno aperto connesso $\Omega_2$ di $P$ e per ogni aperto connesso $\Omega_1 \subseteq \Omega \cap \Omega_2$ con $\Omega_1\not=\emptyset,\Omega$ non esiste $u_2 \in \mathcal{O}(\Omega)$ con $u_2\restrict{\Omega_1}=u\restrict{\Omega_1}$.
  Diremo che $\Omega$ è un \textsc{dominio di olomorfia} se ogni $P \in \partial D$ è essenziale.
\end{defn}

\begin{defn}
  Un \textsc{funzionale di Minkovski} è una funzione $\mu:\mathbb{C}^n \longrightarrow \mathbb{R}^+\cup\{0\}$ continua t.c.
  \begin{nlist}
    \item $\mu(z)=0 \iff z=0$;
    \item $\mu(\zeta z)=|\zeta|\mu(z)$ per ogni $z \in \mathbb{C}^n$ e per ogni $\zeta \in \mathbb{C}$.
  \end{nlist}
\end{defn}

\begin{ex}
  $\mu(z)=\|z\|_p=(|z_1|^p+\dots+|z_n|^p)^{1/p}$ con $p>0$ e $\mu(z)=\|z\|_{\infty}=\max\{|z_1|,\dots,|z_n|\}$ sono funzionali di Minkovski.
\end{ex}

\begin{exc}
  $\Omega \subseteq \mathbb{C}^n$ dominio con $0 \in \Omega$ è circolare e stellato rispetto a $0$ (cioè per ogni $\zeta \in \mathbb{D}, z \in \Omega$ anche $\zeta z \in \Omega$) $\iff$ esiste $\mu$ funzionale di Minkovski t.c. $\Omega=\{z \in \mathbb{C}^n \mid \mu(z)<1\}$. Hint: una freccia è ovvia, per l'altra si consideri $\mu(z)=\inf\{r>0 \mid z/r \in \Omega \}$.
\end{exc}

Se $\Omega \subset \mathbb{C}^n$ è un dominio poniamo $\displaystyle \mu_\Omega(z)=\inf_{w \in \mathbb{C}^n\setminus\Omega} \mu(z-w)=$"$\mu$-distanza da $\partial\Omega$". Se $X \subseteq \Omega$ poniamo $\displaystyle \mu_\Omega(X)=\inf_{z \in X} \mu_\Omega(z)$.

\begin{defn}
  Sia $\Omega \subseteq \mathbb{C}^n$ un dominio, $\mathcal{F}$ una famiglia di funzioni su $\Omega$. Sia $K \subset \Omega$. Il \textsc{$\mathcal{F}$-inviluppo di $K$ in $\Omega$} è $\hat{K}_{\mathcal{F}}=\{z \in \Omega \mid |f(z)| \le \|f\|_K \text{ per ogni } f \in \mathcal{F}\}$.
  Se $\mathcal{F}=\mathcal{O}(\Omega)$, $\hat{K}_{\mathcal{F}}=\hat{K}_{\Omega}$ è \textit{inviluppo olomorfo di $K$}, che abbiamo già incontrato in una variabile.
\end{defn}

Diremo che $\Omega$ è \textit{$\mathcal{F}$-convesso} se e solo se per ogni $K \subset \subset \Omega$ compatto anche $\hat{K}_{\mathcal{F}} \subset \subset \Omega$ compatto. Se $\mathcal{F}=\mathcal{O}(\Omega)$, diremo che $\Omega$ è \textsc{olomorficamente convesso}.

\begin{oss}
  Ogni aperto di $\mathbb{C}$ è olomorficamente convesso.
\end{oss}

\begin{exc}
  Siano $\Omega=P^2(0,1)\setminus\overline{P^2(0,1/2)}, K=\{(0,3e^{i\theta}/4) \mid \theta \in \mathbb{R}\}$. Dimostrare che $\hat{K}_\Omega=\{0, te^{i\theta} \mid \theta \in \mathbb{R}, 1/2 < t \le 3/4\}$.
\end{exc}

\begin{exc}
  Se $\Omega \subseteq \mathbb{R}^n$, $\mathcal{F}=\text{funzioni lineari su }\Omega$, allora $\hat{K}_{\mathcal{F}}=\text{inviluppo convesso di }K$, e $\Omega$ è $\mathcal{F}$-convesso $\iff$ è convesso (nella disuguaglianza che definisce gli insiemi in $\mathbb{R}$ c'è una differenza: non si prende il modulo [notare quindi che anche al posto della norma infinito c'è un $\sup$]).
\end{exc}

\begin{lm}
  Sia $\Omega \subseteq \mathbb{C}^n$ un dominio, $K \subset \Omega$ limitato $\implies$ $\hat{K}_{\Omega}$ è limitato.
\end{lm}

\begin{proof}
  $K$ è limitato $\iff$ per ogni $j=1,\dots,n$ esiste $M_j$ t.c. $|z_j| \le M_j$ per ogni $z \in K$. Dato che le proiezioni alle singole coordinate sono funzioni olomorfe, otteniamo che per ogni $z \in \hat{K}_\Omega$ $|z_j| \le M_j$ $\implies$ $\hat{K}_\Omega$ limitato.
\end{proof}

\begin{oss}
  Se $\mathcal{F} \subset C^0(\Omega)$, allora $\hat{K}_{\mathcal{F}}$ è chiuso in $\Omega$.
\end{oss}

\begin{lm}
  Sia $\Omega \subseteq \mathbb{C}^n$ un dominio, $K \subset \Omega$ $\implies$ $\hat{K}_\Omega$ è contenuto nella chiusura dell'inviluppo convesso di $K$.
\end{lm}

\begin{proof}
  $\mathcal{O}(\Omega)$ contiene le funzioni $e^{L(z)}$ con $L: \mathbb{C}^n \longrightarrow \mathbb{C}$ lineare. $|e^{L(z)}|=e^{\mathfrak{Re}L(z)}$. $|e^{L(z)}| \le \|e^L\|_K$ $\iff$ $\displaystyle \mathfrak{Re}L(z) \le \sup_{w \in K} \mathfrak{Re}L(w)$.
  Ogni $l: \mathbb{C}^n \longrightarrow \mathbb{R}$ $\mathbb{R}$-lineare è la parte reale di $L:\mathbb{C}^n \longrightarrow \mathbb{C}$ $\mathbb{C}$-lineare.
\end{proof}

\begin{thm}
  Sia $\Omega \subseteq \mathbb{C}^n$ un dominio. Sono equivalenti:
  \begin{nlist}
    \item esiste $h \in \mathcal{O}(\Omega)$ che non può essere estesa olomorficamente a un qualsiasi aperto $\Omega' \supsetneq \Omega$;
    \item $\Omega$ è un dominio di olomorfia;
    \item $\Omega$ è olomorficamente convesso;
    \item per ogni $\mu$ funzionale di Minkovski, per ogni $f \in \mathcal{O}(\Omega)$ e per ogni $K \subset\subset \Omega$ compatto, $|f| \le \mu_{\Omega}$ su $K$ $\implies$ $|f| \le \mu_\Omega$ su $\hat{K}_\Omega$;
    \item per ogni $\mu$ funzionale di Minkovski e per ogni $K \subset\subset \Omega$ compatto, $\mu_{\Omega}(\hat{K}_\Omega)=\mu_\Omega(K)$;
    \item per ogni $\mu$ funzionale di Minkovski, per ogni $f \in \mathcal{O}(\Omega)$ e per ogni $K \subset\subset \Omega$ compatto, $\displaystyle \sup_{z \in K} \frac{|f(z)|}{\mu_\Omega(z)}=\sup_{z\in\hat{K}_\Omega} \frac{|f(z)|}{\mu_\Omega(z)}$;
    \item esiste $\mu$ funzionale di Minkovski, per ogni $f \in \mathcal{O}(\Omega)$ e per ogni $K \subset\subset \Omega$ compatto, $|f| \le \mu_{\Omega}$ su $K$ $\implies$ $|f| \le \mu_\Omega$ su $\hat{K}_\Omega$;
    \item esiste $\mu$ funzionale di Minkovski e per ogni $K \subset\subset \Omega$ compatto, $\mu_{\Omega}(\hat{K}_\Omega)=\mu_\Omega(K)$;
    \item esiste $\mu$ funzionale di Minkovski, per ogni $f \in \mathcal{O}(\Omega)$ e per ogni $K \subset\subset \Omega$ compatto, $\displaystyle \sup_{z \in K} \frac{|f(z)|}{\mu_\Omega(z)}=\sup_{z\in\hat{K}_\Omega} \frac{|f(z)|}{\mu_\Omega(z)}$.
  \end{nlist}
\end{thm}

\begin{proof}
  (i) $\implies$ (ii) è ovvia (basta prendere $h$ per tutti i $p \in \partial\Omega$). (iv) $\implies$ (vii) è ovvia. (vii) $\implies$ (ix) è ovvia (basta dividere $f$ per $\displaystyle \sup_{z \in K} \frac{|f(z)|}{\mu_\Omega(z)}$). (ix) $\implies$ (vii) è ovvia (basta prendere $f \equiv 1$). (iv) $\implies$ (vi) $\implies$ (v) sono analoghe a (vii) $\implies$ (xi) $\implies$ (viii).
  (v) $\implies$ (viii) è ovvia. Per concludere mostriamo che (ii) $\implies$ (viii) $\implies$ (iii) $\implies$ (i) $\implies$ (iv).

  (ii) $\implies$ (viii) Poniamo $\mu=\|\cdot\|_{\infty}$. Supponiamo, per assurdo, che (viii) non valga: allora esiste $K \subset\subset \Omega$ compatto t.c. $\mu_{\Omega}(\hat{K}_\Omega)<\mu_\Omega(K)$. Scegliamo $\mu_\Omega(\hat{K}_\Omega)< \delta_1<\delta_2<\mu_\Omega(K)$.
  Sia $z^0 \in \hat{K}_\Omega$ t.c. $\mu_\Omega(z^0)<\delta_1$.
  Poniamo $\displaystyle K_{\delta_2}=\bigcup_{z \in K} \overline{P(z, \delta_2)}=\{w \in \mathbb{C}^n \mid \min_{z \in K} \|z-w\|_\infty \le \delta_2\}$ chiuso e $\subset\subset \Omega$.
  Dalle disuguaglianze di Cauchy otteniamo che per ogni $f \in \mathcal{O}(\Omega)$ e per ogni $\alpha \in \mathbb{N}^n$ vale $\left|\dfrac{\partial^{|\alpha|}f}{\partial z^\alpha}(z)\right| \le \dfrac{\alpha!}{\delta_2^{|\alpha|}}\|f\|_{K_{\delta_2}}$ $(*)$ per ogni $z \in K$.
  Ma allora lo sviluppo in serie di $f$ in $z^0$ converge in $P(z^0, (\delta_1+\delta_2)/2)$ perché $\delta_1<(\delta_1+\delta_2)/2<\delta_2$ $\implies$ $f$ si estende olomorficamente a tutto $P(z^0, (\delta_1+\delta_2)/2)$, ma $P(z^0, (\delta_1+\delta_2)/2) \cap \partial\Omega\not=\emptyset$, assurdo.

  (viii) $\implies$ (iii) $K \subset\subset \Omega$ compatto $\iff$ $K$ è chiuso, limitato e $\mu_\Omega(K)>0$. Se $K \subset\subset \Omega$ compatto, allora $\hat{K}_\Omega$ è chiuso, limitato e, per (viii), $\mu_\Omega(\hat{K}_\Omega)=\mu_\Omega(K)>0$ $\implies$ $\hat{K}_\Omega \subset\subset \Omega$ compatto.

  (iii) $\implies$ (i) Sia $\{w_k\}_{j \in \mathbb{N}^*} \subset \Omega$ una successione ovunque densa e che ripete ogni punto infinite volte. Per ogni $j$ sia $P_j=P(w_j,r_j)$ il polidisco di centro $w_j$ più grande contenuto in $\Omega$ ($\iff$ $P(w_j,r_j) \subset \Omega$ ma $\overline{P(w_j,r_j)}\cap\partial\Omega\not=\emptyset$). In particolare, $P_j$ non è $\subset\subset\Omega$.
  Sia $\{K_j\}$ una successione di compatti che invade $\Omega$, cioè $K_j \subset\subset \Omega$ compatto, $K_j \subset \mathop {K_{j+1}}\limits^ \circ$ e $\displaystyle \bigcup_j K_j=\Omega$.
  (iii) $\implies$ $\widehat{(K_j)}_\Omega \subset\subset \Omega$ $\implies$ esiste $z_j \in P_j\setminus\widehat{(K_j)}_\Omega$ $\implies$ esiste $h_j \in \mathcal{O}(\Omega)$ t.c. $|h_j(z_j)|>\|h_j\|_{K_j}$. Possiamo suppore $h_j(z_j)=1$ e (a meno di sostituire $h_j^{M_j}$ con $M_j>>1$) possiamo supporre $\|h_j\|_{K_j}<2^{-j}$.
  Poniamo $\displaystyle h(z)=\prod_{j=1}^{+\infty} (1-h_j(z))^j$. $h \in \mathcal{O}(\Omega)$ perché $\displaystyle \sum_j \frac{j}{2^j}<+\infty$. Inoltre $h \not\equiv 0$ perché non lo è su $K_1$. Ogni $P_j$ contiene infiniti $z_l$ che si accumulano a $z_j^0 \in \overline{P_j}$.
  $h$ si annulla di ordine almeno $l$ in $z_l$ (segue dalla definizione). Se $z_j^0 \in \Omega$ allora $h$ dovrebbe annullarsi di ordine $\infty$ in $z_j^0$ $\implies$ $h \equiv 0$, assurdo $\implies$ $z_j^0 \in \partial\Omega$.
  Gli $\{z_j^0\}$ sono densi in $\partial\Omega$; se non lo fossero, esisterebbe $w_{j_0}$ t.c. $\overline{P(w_{j_0},r_{j_0})} \cap \partial\Omega$ non contiene alcun $z_j^0$, assurdo. Se $h$ si estendesse a $\Omega' \supsetneq \Omega$ allora si dovrebbe estendere a un intorno di qualche $z_j^0$ $\implies$ $h \equiv 0$, assurdo.

  (i) $\implies$ (iv) Fissiamo $\underline{r}=(r_1,\dots,r_n) \in (\mathbb{R}^+)^n$ e poniamo $\mu^{\underline{r}}(z)=\max\left\{\dfrac{|z_j|}{r_j}\right\}$.
  Vogliamo dimostrare che vale (iv) per $\mu^{\underline{r}}$. Siano $f \in \mathcal{O}(\Omega)$ e $K\subset\subset\Omega$ compatto t.c. $|f(z)| \le \mu_{\Omega}^{\underline{r}}(z)$ per ogni $z \in K$.
  \begin{ftt} \label{2.5.13}
    Data $f$ come sopra, per ogni $g \in \mathcal{O}(\Omega)$ e $w \in \hat{K}_\Omega$ $g$ ha un'espansione in serie di potenze centrata in $w$ e convergente in $P(w, |f(w)|\underline{r})=\{z \in \mathbb{C}^n \mid \mu^{\underline{r}}(z-w)<|f(w)|\}$.
  \end{ftt}
  \begin{proof}
    Fissiamo $0<t<1$, $\displaystyle W_t=\bigcup_{z \in K} P(z, |f(z)|t\underline{r})$. Siccome $t|f(z)|<u_\Omega^{\underline{r}}(z)$ per ogni $z \in K$ $\implies$ $W_t \subset\subset \Omega$. Sia $g \in \mathcal{O}(\Omega)$. Esiste $M>0$ t.c. $\|g\|_{W_t} \le M$.
    Per le disuguaglianze di Cauchy, per ogni $z \in K$ si ha $\left|\dfrac{\partial^{|\alpha|}g}{\partial z^\alpha}(z)\right| \le \dfrac{\alpha!M}{t^{|\alpha|}|f(z)|^{|\alpha|}\underline{r}^\alpha} \iff \left|f(z)^{|\alpha|}\dfrac{\partial^{|\alpha|}g}{\partial z^\alpha}(z)\right| \le \dfrac{\alpha!M}{t^{|\alpha|}\underline{r}^\alpha} \implies$
    per ogni $w \in \hat{K}_\Omega$ si ha $\left|f(w)^{|\alpha|}\dfrac{\partial^{|\alpha|}g}{\partial z^\alpha}(w)\right| \le \dfrac{\alpha!M}{t^{|\alpha|}\underline{r}^\alpha} \implies \left|\dfrac{\partial^{|\alpha|}g}{\partial z^\alpha}(w)\right| \le \dfrac{\alpha!M}{t^{|\alpha|}|f(w)|^{|\alpha|}\underline{r}^\alpha}$ $\implies$
    lo sviluppo in serie di $g$ in $w$ converge in $P(w,|f(w)|t\underline{r})$. Mandando $t$ a $1$ si ottiene quanto voluto.
  \end{proof}
  Usiamo il fatto \ref{2.5.13} per dimostrare che vale (iv) per $\mu^{\underline{r}}$.
  Per assurdo, se esiste $w \in \hat{K}_\Omega$ t.c. $|f(w)|>\mu_\Omega^{\underline{r}}(w)$ allora $P(w,|f(w)|\underline{r}) \cap \partial \Omega \not=\emptyset$, dunque per il fatto \ref{2.5.13} ogni $g \in \mathcal{O}(\Omega)$ si estende olomorficamente in $P(w,|f(w)|\underline{r})$ contro (i), assurdo.
  Sia adesso $\mu$ un funzionale di Minkovski qualunque.
\end{proof}

\begin{cor} \label{conv->dom_olo}
  $\Omega$ convesso $\implies$ $\Omega$ dominio di olomorfia.
\end{cor}

\begin{proof}
  Sia $P \in \partial \Omega$. Essendo $\Omega$ convesso, esiste $L: \mathbb{C}^n \longrightarrow \mathbb{C}$ $\mathbb{C}$-lineare t.c. $\mathfrak{Re}L(z) < \mathfrak{Re}L(P)$ per ogni $z \in \Omega$. Sia $f(z)=\dfrac{1}{L(z)-L(P)}$. Allora $f \in \mathcal{O}(\Omega)$ e non si estende oltre $P$ $\implies$ $P$ è essenziale.
\end{proof}

\begin{defn}
  $P \in \partial\Omega$ è un \textit{punto di picco} se esiste $f \in \mathcal{O}(\Omega) \cap C^0(\overline{\Omega})$ t.c. $f(P)=1$ ma $\|f\|_\Omega<1$.
\end{defn}

\begin{ex}
  Nel caso del corollario \ref{conv->dom_olo}, $f(z)=e^{L(z)-L(P)}$.
\end{ex}

\begin{exc}
  Se ogni punto di $\partial\Omega$ è di picco, allora $\Omega$ è dominio di olomorfia.
\end{exc}


\end{document}
