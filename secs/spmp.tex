Adesso possiamo procedere a dimostrare la serie di disuguaglianze di \cite{BM}, che coinvolgono la distanza di Poincaré $\omega$ e le funzioni olomorfe dal disco in sé che non sono automorfismi.

\begin{prop} \label{24}
  Siano $f \in \text{\normalfont{Hol}}(\mathbb{D},\mathbb{D})\setminus\text{\normalfont{Aut}}(\mathbb{D})$ e $v \in \mathbb{D}$. Allora per ogni $z \in \mathbb{D}$ si ha che $f^*(z,v) \in \mathbb{D}$ e la funzione $z \longmapsto f^*(z,v)$ è olomorfa.
\end{prop}

\begin{proof}
  Per quanto riguarda l'olomorfia, dalla definizione sappiamo che l'unico punto che potrebbe dar problemi è $v$; abbiamo però visto che la funzione ammette limite finito per $z \longrightarrow v$, perciò $v$ è una singolarità rimovibile. Per il lemma di Schwarz-Pick, $|f^*(z,w)| \le 1$; inoltre, vale l'uguaglianza in qualche punto solo se $f$ è un automorfismo. Dunque le ipotesi su $f$ assicurano che vale la disuguaglianza stretta sempre, cioè $f^*(z,v) \in \mathbb{D}$ per ogni $z \in \mathbb{D}$.
\end{proof}

\begin{thm} \label{31}
  (Beardon-Minda, 2004) Sia $f \in \text{\normalfont{Hol}}(\mathbb{D},\mathbb{D})\setminus\text{\normalfont{Aut}}(\mathbb{D})$. Allora per ogni $z, w, v \in \mathbb{D}$ vale
  \begin{equation} \label{3.1}
    \omega\bigl(f^*(z,v),f^*(w,v)\bigr) \le \omega(z,w).
  \end{equation}
\end{thm}

\begin{proof}
  Poiché $f$ non è un automorfismo, per la Proposizione \ref{24} la funzione $z \longmapsto f^*(z,v)$ è olomorfa dal disco unitario in sé; perciò il membro sinistro della disuguaglianza \eqref{3.1} è ben definito e la tesi segue dal lemma di Schwarz-Pick e dall'osservazione \ref{oss1}, punto (ii).
\end{proof}

\begin{cor} \label{32}
  Sia $f \in \text{\normalfont{Hol}}(\mathbb{D},\mathbb{D})\setminus\text{\normalfont{Aut}}(\mathbb{D})$. Allora per ogni $z, w, v \in \mathbb{D}$ vale
  \begin{equation}
    \omega\bigl(0, f^*(z,v)\bigr) \le \omega\bigl(0,f^*(w,v)\bigr)+\omega(z,w).
  \end{equation}
\end{cor}

\begin{proof}
  Si ha
  \begin{align*}
    \omega\bigl(0,f^*(z,v)\bigr) & \le \omega\bigl(0,f^*(w,v)\bigr)+\omega\bigl(f^*(w,v),f^*(z,v)\bigr) \\
    & \le \omega\bigl(0,f^*(w,v)\bigr)+\omega(z,w),
  \end{align*}
  dove la prima è la disuguaglianza triangolare per la distanza $\omega$ e la seconda segue dal Teorema \ref{31}.
\end{proof}

\begin{cor} \label{33}
  Sia $f \in \text{\normalfont{Hol}}(\mathbb{D},\mathbb{D})\setminus\text{\normalfont{Aut}}(\mathbb{D})$. Allora per ogni $z, w, v, u \in \mathbb{D}$ vale
  \begin{equation}
    \omega\bigl(0, f^*(z,v)\bigr) \le \omega\bigl(0, f^*(u,w)\bigr)+\omega(z,w)+\omega(v,u).
  \end{equation}
\end{cor}

\begin{proof}
  Si ha
  \begin{align*}
    \omega\bigl(0,f^*(z,v)\bigr) & \le \omega\bigl(0,f^*(w,v)\bigr)+\omega(z,w) \\
    & =\omega\bigl(0,|f^*(w,v)|\bigr)+\omega(z,w) \\
    & =\omega\bigl(0,|f^*(v,w)|\bigr)+\omega(z,w) \\
    & =\omega\bigl(0,f^*(v,w)\bigr)+\omega(z,w) \\
    & \le \omega\bigl(0,f^*(u,w)\bigr)+\omega(z,w)+\omega(v,u),
  \end{align*}
  dove le due disuguaglianze seguono dal Corollario \ref{32}.
\end{proof}

\marginpar{Se avanza tempo, inserire Corollary 3.5 di BM (serve dare la definizione di geodetica, da decidere come imbastirla)}

\begin{cor} \label{36}
  Sia $f \in \text{\normalfont{Hol}}(\mathbb{D},\mathbb{D})\setminus\text{\normalfont{Aut}}(\mathbb{D})$ tale che $f(0)=0$. Allora
  \begin{equation}
    \omega\bigl(f^h(0),f^h(z)\bigr) \le 2\omega(0,z).
  \end{equation}
\end{cor}

\begin{proof}
  Notiamo che $f(0)=0 \implies f^*(z,0)=f^*(0,z)$, dunque si ha
  \begin{align*}
    \omega\bigl(f^h(0),f^h(z)\bigr) & = \omega\bigl(f^*(0,0),f^*(z,z)\bigr) \\
    & \le \omega\bigl(f^*(0,0),f^*(z,0)\bigr)+\omega\bigl(f^*(0,z),f^*(z,z)\bigr) \\
    & \le 2\omega(0,z)
  \end{align*}
  dove la prima è la disuguaglianza triangolare per $\omega$ e la seconda segue applicando il Teorema \ref{31}.
\end{proof}

\marginpar{caso di uguaglianza in \ref{36}: se aggiungi Corollary 3.5 di BM, riprova ad aggiungerlo. C'erano problemi di impaginazione}

\begin{cor} \label{quasigolusin}
  Sia $f \in \text{\normalfont{Hol}}(\mathbb{D},\mathbb{D})\setminus\text{\normalfont{Aut}}(\mathbb{D})$. Allora per ogni $z, w \in \mathbb{D}$ vale
  \begin{equation} \label{quasigol}
    \omega\bigl(|f^h(z)|, |f^h(w)|\bigr) \le 2\omega(z,w).
  \end{equation}
\end{cor}

\begin{proof}
  Siano $z, w \in \mathbb{D}$; senza perdita di generalità possiamo supporre $|f^h(z)| \ge |f^h(w)|$. Allora
  \begin{align*}
    \omega\bigl(|f^h(z)|, |f^h(w)|\bigr) & =\frac{1}{2}\log\left(\frac{1+\frac{|f^h(z)|-|f^h(w)|}{1-|f^h(w)||f^h(z)|}}{1-\frac{|f^h(z)|-|f^h(w)|}{1-|f^h(w)||f^h(z)|}}\right) \\
    & =\frac{1}{2}\log\left(\frac{1-|f^h(w)||f^h(z)|+|f^h(z)|-|f^h(w)|}{1-|f^h(w)||f^h(z)|+|f^h(w)|-|f^h(z)|}\right) \\
    & =\frac{1}{2}\log\left(\frac{1+|f^h(z)|}{1-|f^h(z)|}\cdot\frac{1-|f^h(w)|}{1+|f^h(w)|}\right) \\
    & =\frac{1}{2}\log\left(\frac{1+|f^h(z)|}{1-|f^h(z)|}\right)-\frac{1}{2}\log\left(\frac{1+|f^h(w)|}{1-|f^h(w)|}\right) \\
    & =\omega\bigl(0,|f^h(z)|\bigr)-\omega\bigl(0,|f^h(w)|\bigr) \\
    & =\omega\bigl(0,f^h(z)\bigr)-\omega\bigl(0,f^h(w)\bigr) \le 2\omega(z,w),
  \end{align*}
  dove l'ultima disuguaglianza segue dal Corollario \ref{33} prendendo $u=w$ e $v=z$.
\end{proof}

Concludiamo la sezione con due lemmi sulle funzioni in $\text{Hol}(\mathbb{D},\mathbb{D})$, per i quali l'approccio dal punto di vista dell'articolo di Beardon e Minda semplifica le dimostrazioni.

\marginpar{aggiungere delle reference}

Notazione: dati $E \subset \mathbb{D}$ e $z \in \mathbb{D}$, scriviamo $zE=\{zw \mid w \in E\}$. Inoltre, dati $\gamma \subset \mathbb{D}$ e $r>0$, scriviamo $\Sigma(\gamma,r)=\{w \mid \omega(z,w)<r, z \in \gamma\}$.

\begin{lm}
  (lemma di Rogosinski) Sia $f \in \normalfont{\text{Hol}}(\mathbb{D},\mathbb{D})\setminus\normalfont{\text{Aut}}(\mathbb{D},\mathbb{D})$ tale che $f(0)=0$ e $f'(0) \in \mathbb{R}$. Allora per ogni $z \in \mathbb{D}$ si ha $f(z) \in z\Sigma\bigl((-1,1),\omega(0,z)\bigr)$.
\end{lm}

\begin{proof}
  Sia $g$ definita come nella dimostrazione del lemma di Schwarz, il quale ci dice anche che $g \in \text{Hol}(\mathbb{D},\mathbb{D})$. La tesi è vera per $z=0$, supponiamo dunque $z\not=0$. Per il lemma di Schwarz-Pick si ha
  \begin{gather*}
    \omega\bigl(g(0),g(z)\bigr) \le \omega(0,z) \\
    \omega\bigl(f'(0),f(z)/z\bigr) \le \omega(0,z).
  \end{gather*}
  Per il lemma di Schwarz dev'essere $f'(0) \in (-1,1)$. Abbiamo quindi che $f(z)/z \in \Sigma\bigl((-1,1),\omega(0,z)\bigr) \implies f(z) \in z\Sigma\bigl((-1,1),\omega(0,z)\bigr)$, come voluto.
\end{proof}

\begin{lm}
  (lemma di Dieudonné)
\end{lm}
