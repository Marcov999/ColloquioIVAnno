\begin{frame}
  \frametitle{Derivata e rapporto iperbolici}
  Scriviamo $[z,w]:=\dfrac{z-w}{1-\bar{w}z}$. \pause
  \begin{defn}
    Data $f \in \text{Hol}(\mathbb{D},\mathbb{D})$, la \textit{derivata iperbolica} è definita come
    $$f^h(w):=\lim_{z \longrightarrow w} \frac{[f(z),f(w)]}{[z,w]}=\frac{f'(w)(1-|w|^2)}{1-|f(w)|^2}.$$
  \end{defn}
  \pause
  \begin{defn}
    Il \textit{rapporto iperbolico} è definito come
    $$f^*(z,w):=\begin{cases}
      \frac{[f(z),f(w)]}{[z,w]} & \mbox{per }z\not=w \\
      f^h(w) & \mbox{per }z=w.
    \end{cases}$$
  \end{defn}
\end{frame}

\begin{frame}[t]
  \frametitle{Regioni di Stolz e settori}
  \only<1-3>{\begin{defn}
    Dati $\alpha \in (0,\pi/2)$ e $\sigma \in \partial\mathbb{D}$, chiamiamo \textit{settore di vertice $\sigma$ e angolo $2\alpha$} l'insieme $S(\sigma,\alpha) \subset \mathbb{D}$ tale che per ogni $z \in S(\sigma,\alpha)$ l'angolo compreso tra la retta congiungente $\sigma$ e $0$ e la retta congiungente $\sigma$ e $z$ ha modulo minore di $\alpha$.
  \end{defn}}
  \only<2-3>{\begin{defn}
    Dati $\sigma \in \partial \mathbb{D}$ e $M>1$, chiamiamo \textit{regione di Stolz $K(\sigma,M)$} l'insieme $\left\{z \in \mathbb{D} \mid \dfrac{|\sigma-z|}{1-|z|} < M\right\}$.
  \end{defn}}
  \only<3>{Vicino a $\sigma$, regioni di Stolz e settori sono ``intercambiabili''.}
  \only<4>{\begin{tikzpicture}[line cap=round,line join=round,>=triangle 45,x=2.5cm,y=2.5cm]
    \draw[->,color=black] (-1.13,0) -- (1.15,0);
    \foreach \x in {-1,1}
    \draw[shift={(\x,0)},color=black] (0pt,2pt) -- (0pt,-2pt);
    \draw[->,color=black] (0,-1.11) -- (0,1.12);
    \foreach \y in {-1,1}
    \draw[shift={(0,\y)},color=black] (2pt,0pt) -- (-2pt,0pt);
    \clip(-1.13,-1.11) rectangle (1.15,1.12);
    \draw(0,0) circle (2.5cm);
    \draw[name path=A] (0.49,0.87)-- (1,0);
    \draw[name path=B] (0.49,-0.87)-- (1,0);
    \tikzfillbetween[of=A and B]{blue, opacity=0.25};
    \draw[name path=C,smooth,samples=100,domain=-1:0.491] plot(\x,{sqrt(1-(\x)^2)});
    \draw[name path=D,smooth,samples=100,domain=-1:0.491] plot(\x,{0-sqrt(1-(\x)^2)});
    \tikzfillbetween[of=C and D]{blue, opacity=0.25};
  \end{tikzpicture}
  \begin{tikzpicture}[line cap=round,line join=round,>=triangle 45,x=2.5cm,y=2.5cm]
    \draw[->,color=black] (-1.16,0) -- (1.18,0);
    \foreach \x in {-1,1}
    \draw[shift={(\x,0)},color=black] (0pt,2pt) -- (0pt,-2pt);
    \draw[->,color=black] (0,-1.13) -- (0,1.15);
    \foreach \y in {-1,1}
    \draw[shift={(0,\y)},color=black] (2pt,0pt) -- (-2pt,0pt);
    \clip(-1.16,-1.13) rectangle (1.18,1.15);
    \draw(0,0) circle (2.5cm);
    \draw[name path=A, smooth,samples=100,domain=-0.1715705740864879:1] plot(\x,{sqrt(7-4*sqrt(3-2*(\x))-2*(\x)-(\x)^2)});
    \draw[name path=B, smooth,samples=100,domain=-0.1715705740864879:1] plot(\x,{0-sqrt(7-4*sqrt(3-2*(\x))-2*(\x)-(\x)^2)});
    \tikzfillbetween[of=A and B]{red, opacity=0.25};
  \end{tikzpicture}
  A sinistra, il settore $S(1,2\pi/3)$; a destra, la regione di Stolz $K(1,2)$.}
\end{frame}

\begin{frame}
  \frametitle{Limiti non tangenziali}
  \begin{defn}
    Diciamo che una funzione $f:\mathbb{D} \longrightarrow \mathbb{C}$ ha \textit{limite non-tangenziale} $L \in \mathbb{C}$ in $\sigma \in \partial\mathbb{D}$ e scriviamo
    $$\substack{\text{nt--lim} \\ z \longrightarrow \sigma} \, f(z)=L$$
    se  per ogni $M>1$ si ha $\displaystyle \lim_{\substack{z \longrightarrow \sigma, \\ z \in K(\sigma,M)}} f(z)=L$.
  \end{defn}
  \pause
  \begin{defn}
    Date tre funzioni $f,g,h: \mathbb{D} \longrightarrow \mathbb{C}$ scriviamo che $f(z)=g(z)+o\bigl(h(z)\bigr)$ per $z \longrightarrow \sigma$ \textit{non tangenzialmente} se
    $$\substack{\text{nt--lim} \\ z \longrightarrow \sigma} \, \frac{f(z)-g(z)}{h(z)}=0.$$
  \end{defn}
  \pause
  Notiamo che la definizione di limite non tangenziale è più debole di quella di limite classico; nel nostro caso rende il risultato più forte.
\end{frame}
