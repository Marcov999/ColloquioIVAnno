\begin{frame}
  \frametitle{Regioni di Stolz e settori}
  \begin{defn}
    Dati $\alpha \in (0,\pi/2)$ e $\sigma \in \partial\mathbb{D}$, chiamiamo \textit{settore di vertice $\sigma$ e angolo $2\alpha$} l'insieme $S(\sigma,\alpha) \subset \mathbb{D}$ tale che per ogni $z \in S(\sigma,\alpha)$ l'angolo compreso tra la retta congiungente $\sigma$ e $0$ e la retta congiungente $\sigma$ e $z$ ha modulo minore di $\alpha$.
  \end{defn}
  \pause
  \begin{defn}
    Dati $\sigma \in \partial \mathbb{D}$ e $M>1$, chiamiamo \textit{regione di Stolz $K(\sigma,M)$} l'insieme $\left\{z \in \mathbb{D} \mid \dfrac{|\sigma-z|}{1-|z|} < M\right\}$.
  \end{defn}
\end{frame}

\begin{frame}
  \frametitle{regioni di Stolz e settori}
  \begin{tikzpicture}[line cap=round,line join=round,>=triangle 45,x=2.5cm,y=2.5cm]
    \draw[->,color=black] (-1.13,0) -- (1.15,0);
    \foreach \x in {-1,1}
    \draw[shift={(\x,0)},color=black] (0pt,2pt) -- (0pt,-2pt);
    \draw[->,color=black] (0,-1.11) -- (0,1.12);
    \foreach \y in {-1,1}
    \draw[shift={(0,\y)},color=black] (2pt,0pt) -- (-2pt,0pt);
    \clip(-1.13,-1.11) rectangle (1.15,1.12);
    \draw(0,0) circle (2.5cm);
    \draw[name path=A] (0.49,0.87)-- (1,0);
    \draw[name path=B] (0.49,-0.87)-- (1,0);
    \tikzfillbetween[of=A and B]{blue, opacity=0.25};
    \draw[name path=C,smooth,samples=100,domain=-1:0.491] plot(\x,{sqrt(1-(\x)^2)});
    \draw[name path=D,smooth,samples=100,domain=-1:0.491] plot(\x,{0-sqrt(1-(\x)^2)});
    \tikzfillbetween[of=C and D]{blue, opacity=0.25};
  \end{tikzpicture}
  \begin{tikzpicture}[line cap=round,line join=round,>=triangle 45,x=2.5cm,y=2.5cm]
    \draw[->,color=black] (-1.16,0) -- (1.18,0);
    \foreach \x in {-1,1}
    \draw[shift={(\x,0)},color=black] (0pt,2pt) -- (0pt,-2pt);
    \draw[->,color=black] (0,-1.13) -- (0,1.15);
    \foreach \y in {-1,1}
    \draw[shift={(0,\y)},color=black] (2pt,0pt) -- (-2pt,0pt);
    \clip(-1.16,-1.13) rectangle (1.18,1.15);
    \draw(0,0) circle (2.5cm);
    \draw[name path=A, smooth,samples=100,domain=-0.1715705740864879:1] plot(\x,{sqrt(7-4*sqrt(3-2*(\x))-2*(\x)-(\x)^2)});
    \draw[name path=B, smooth,samples=100,domain=-0.1715705740864879:1] plot(\x,{0-sqrt(7-4*sqrt(3-2*(\x))-2*(\x)-(\x)^2)});
    \tikzfillbetween[of=A and B]{red, opacity=0.25};
  \end{tikzpicture}
  A sinistra, il settore $S(1,2\pi/3)$; a destra, la regione di Stolz $K(1,2)$.
\end{frame}

\begin{frame}[t]
  \frametitle{Relazione tra regioni di Stolz e settori}
  \only<1->{\begin{prop} \label{settori-stolz}
    Dato $M>1$, sia $\alpha=\text{\normalfont{arctan}}\sqrt{M^2-1} \in (0,\pi/2)$. Per ogni $\alpha'<\alpha$ esiste $\epsilon>0$ tale che, detto $B(\sigma,\epsilon)=\{z \in \mathbb{C} \mid |\sigma-z|<\epsilon\}$, si ha
    $$S(\sigma,\alpha')\cap B(\sigma,\epsilon) \subset K(\sigma,M) \subset S(\sigma,\alpha).$$
  \end{prop}}
  \only<2>{\textit{Traccia della dimostrazione:} senza perdita di generalità $\sigma=1$. Possiamo scrivere $S(1,\alpha)=\left\{z \in \mathbb{D} \mid |\mathfrak{Im}(z)|<(\tan{\alpha})\bigl(1-\mathfrak{Re}(z)\bigr)\right\}$. Se $z \in K(1,M)$, da
  $$M>\frac{|1-z|}{1-|z|}\ge \frac{|1-z|}{1-\mathfrak{Re}(z)}$$
  troviamo
  $$\frac{|\mathfrak{Im}(z)|}{1-\mathfrak{Re}(z)}<\sqrt{M^2-1}=\tan{\alpha};$$
  questo mostra la seconda inclusione.}
  \only<3>{\textit{Traccia della dimostrazione:} sia $\alpha'<\alpha$ e supponiamo per assurdo che per ogni $\epsilon>0$ esista $z \in S(1,\alpha')\cap B(1,\epsilon)$ tale che $z \not\in K(1,M)$. Si ha allora
  \begin{equation} \label<3>{star1}
    \dfrac{1-|z|}{|1-z|} \le \dfrac{1}{M} \text{ e } \frac{|\mathfrak{Im}(z)|}{1-\mathfrak{Re}(z)}<\tan{\alpha'}.
  \end{equation}
  Dalla seconda disuguaglianza in \eqref{star1} si ottiene
  \begin{equation} \label<3>{star2}
    \frac{|1-z|}{1-\mathfrak{Re}(z)}<\sqrt{\tan^2{\alpha'}+1}=:M'<M;
  \end{equation}
  moltiplicando la \eqref{star2} per la prima disuguaglianza della \eqref{star1} troviamo}
  \only<4>{\textit{Traccia della dimostrazione:} $\dfrac{1-|z|}{1-\mathfrak{Re}(z)}<\dfrac{M'}{M}<1$. Tuttavia, ponendo $x=\mathfrak{Re}(z)$ e $y=\mathfrak{Im}(z)$ e riscrivendo la condizione $z \in S(1,\alpha')$ come $y/(1-x)<\tan{\alpha'}$, vediamo facilmente che
  $$\lim_{\substack{z \longrightarrow 1, \\ z \in S(1,\alpha')}} \frac{1-|z|}{1-\mathfrak{Re}(z)}=1,$$
  da cui otteniamo una contraddizione. \qed}
\end{frame}
