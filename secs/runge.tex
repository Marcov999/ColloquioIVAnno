\begin{lm}
  Sia $K \subset \subset \mathbb{C}$ compatto, $V \supset K$ un intorno aperto. Allora esiste $g \in C^{\infty}(\mathbb{C})$ t.c. $g\restrict{K}=1$ e $supp(g) \subset V$ ($\implies g\restrict{\mathbb{C}\setminus V}\equiv 0$) [ricordiamo che $supp(g)=\overline{\{z \in \mathbb{C} \mid g(z)\not=0\}}$].
\end{lm}

\begin{proof}
  Sia $h: \mathbb{R} \longrightarrow \mathbb{R}$ data da $h(t)=\begin{cases}
    0 & \mbox{se }t\le 0\\ e^{-1/t} & \mbox{se }t>0
\end{cases}$, $h \in C^{\infty}(\mathbb{R})$. Sia $\eta: \mathbb{C} \longrightarrow \mathbb{C}$ data da $\eta(z)=\dfrac{h(1-|z|^2)}{h(1-|z|^2)+h(|z|^2-1/4)}$. $\eta \in C^{\infty}(\mathbb{C}), \eta(\mathbb{C})=[0, 1]$.
$\eta\restrict{D(0, 1/2)}\equiv 1$ e $\eta\restrict{\mathbb{C}\setminus \mathbb{D}} \equiv 0$. Dato $p \in K$, sia $r_p>0$ t.c. $D(p, 2r_p) \subset V$.
Allora, per compattezza di $K$, esistono $p_1, \dots, p_k \in K$ t.c. $\displaystyle K \subset \bigcup_{j=1}^k D(p_j, r_{p_j}/2) \subset \bigcup_{j=1}^k D(p_j, 2r_{p_j}) \subset V$. Poniamo $\displaystyle W=\bigcup_{j=1}^k D(p_j, r_{p_j})$.
Sia $g_j:\mathbb{C} \longrightarrow \mathbb{R}$, $g_j=\begin{cases}
  \eta\left(\dfrac{z-p_j}{r_{p_j}}\right) & \mbox{se }z\in D(p_j, 2r_{p_j})\\ 0 & \mbox{se }z\in\mathbb{C} \setminus \overline{D(p_j, r_{p_j})}
\end{cases}$, che è ben definita per come è definita $\eta$. $g_j \in C^{\infty}(\mathbb{C})$. Sia $g: \mathbb{C} \longrightarrow \mathbb{R}$, $\displaystyle g(z)=1-\prod_{j=1}^k (1-g_j(z))$. $g \in C^{\infty}(\mathbb{C})$.
Se $z \in K$, esiste $j$ t.c. $z \in D(p_j, r_{p_j}/2) \implies g_j(z)=1 \implies g(z)=1$. Se $z \not\in \overline{W}$, $z \not\in\overline{D(p_j, r_{p_j})}$ per ogni $j=1, \dots, k$ $\implies$ $g_j(z)=0$ per ogni $j$ $\implies$ $g(z)=0$ $\implies$ $supp(g) \subseteq \overline{W} \subset V$.
\end{proof}

\begin{thm}
  (Teorema di Cauchy generalizzato) Sia $\Omega \subset \subset \mathbb{C}$ un dominio limitato t.c. $\partial\Omega$ sia un numero finito di curve di Jordan. Sia $u \in C^1(\overline{\Omega}, \mathbb{C})$, cioè esiste $U$ intorno aperto di $\overline{\Omega}$ su cui $u$ si estende di classe $C^1$.
  Allora per ogni $w \in \Omega$ $\displaystyle u(w)=\int_{\partial \Omega} \dfrac{u(z)}{z-w}\diff z+\dfrac{1}{2\pi i}\int_{\Omega} \dfrac{\partial u/\partial\bar{z}}{z-w}\diff z \wedge \diff \bar{z}$.
\end{thm}

\begin{proof}
  Usando la formula di Gauss-Green abbiamo che $\displaystyle \int_{\partial \Omega}(f\diff x+g\diff y)=\int_{\Omega}\left(\dfrac{\partial g}{\partial x}-\dfrac{\partial f}{\partial y}\right)\diff x\diff y$. Sia $v \in C^1(\overline{\Omega}, \mathbb{C}), f=\mathfrak{Re}(v), g=\mathfrak{Im}(v)$.
  $v=f+ig, \diff z=\diff x+i\diff y \implies v\diff z=(f\diff x-g\diff y)+i(g\diff x+f\diff y)$.
  Per Gauss-Green, $\displaystyle \int_{\partial \Omega} v\diff z=\int_{\Omega} \left(-\dfrac{\partial g}{\partial x}-\dfrac{\partial f}{\partial y}\right)\diff x\diff y+i\int_{\Omega} \left(\dfrac{\partial f}{\partial x}-\dfrac{\partial g}{\partial y}\right)\diff x\diff y$.
  $\dfrac{\partial v}{\partial\bar{z}}=\dfrac{1}{2}\left(\dfrac{\partial v}{\partial x}+i\dfrac{\partial v}{\partial y}\right)=\dfrac{1}{2}\left(\dfrac{\partial f}{\partial x}-\dfrac{\partial g}{\partial y}\right)+\dfrac{1}{2}i\left(\dfrac{\partial f}{\partial y}+\dfrac{\partial g}{\partial x}\right)$.
  $\diff \bar{z}\wedge\diff z=(\diff x-i\diff y)\wedge(\diff x+i\diff y)=2i\diff x\wedge\diff y$.
  $\dfrac{\partial v}{\partial\bar{z}}\diff\bar{z}\wedge \diff z=\left[-\left(\dfrac{\partial f}{\partial y}+\dfrac{\partial g}{\partial x}\right)+i\left(\dfrac{\partial f}{\partial x}-\dfrac{\partial g}{\partial y}\right)\right]\diff x\wedge\diff y$.
  Allora $\displaystyle \int_{\partial \Omega}v\diff z=\int_{\Omega} \dfrac{\partial v}{\partial\bar{z}}\diff\bar{z}\wedge\diff z$ $(\star)$.
  Con il teorema di Stokes, $\displaystyle \int_{\partial \Omega}v\diff z=\int_{\Omega} \diff(v\diff z)=\int_{\Omega} \diff v\wedge\diff z=\int_{\Omega} \left(\dfrac{\partial v}{\partial z}\diff z+\dfrac{\partial v}{\partial\bar{z}}\diff\bar{z}\right)\wedge\diff z=\int_{\Omega} \dfrac{\partial v}{\partial\bar{z}} \diff \bar{z}\wedge\diff z$.
  Fissato $w \in \Omega$ poniamo $v(z)=\dfrac{u(z)}{z-w}$ su $\Omega_{\epsilon}=\Omega \cap \{|z-w|>\epsilon\}$ dove $\epsilon>0$ è sufficientemente piccolo. $\partial\Omega_{\epsilon}=\partial\Omega\cup\partial D(w, \epsilon)$. $v \in C^1(\overline{\Omega}_{\epsilon})$.
  Per $(\star)$, $\displaystyle \int_{\partial\Omega_{\epsilon}} \dfrac{u(z)}{z-w}\diff z=\int_{\Omega_{\epsilon}} \dfrac{\partial u/\partial\bar{z}}{z-w}\diff\bar{z}\wedge\diff z$.
  Parametrizziamo $\partial D(w, \epsilon)$ con $\gamma(t)=w+\epsilon e^{it}, t \in [0, 2\pi]$, abbiamo allora che
  $\displaystyle \int_{\partial\Omega_{\epsilon}} \dfrac{u(z)}{z-w}\diff z=\int_{\partial\Omega} \dfrac{u(z)}{z-w}\diff z-\int_{\partial D(w, \epsilon)} \dfrac{u(z)}{z-w}\diff z=\int_{\partial\Omega} \dfrac{u(z)}{z_{\epsilon}-w}\diff z-\int_0^{2\pi} u(w+\epsilon e^{it})i\diff t$.
  Mandando $\epsilon$ a $0$, dato che $u$ è continua, otteniamo $\displaystyle \int_{\partial\Omega} \dfrac{u(z)}{z-w}\diff z-2\pi iu(w)$, mentre $\displaystyle \int_{\Omega_{\epsilon}} \dfrac{\partial u/\partial\bar{z}}{z-w}\diff\bar{z}\wedge\diff z$ tende a $\displaystyle \int_{\Omega} \dfrac{\partial u/\partial\bar{z}}{z-w}\diff\bar{z}\wedge\diff z$ perché $\dfrac{1}{z-w}$ è integrabile sugli aperti limitati di $\mathbb{C}$. Mettendo tutto insieme si ha la tesi.
\end{proof}

\begin{defn}
  L'\textit{equazione di Cauchy-Riemann non omogenea} è $\dfrac{\partial u}{\partial\bar{z}}=\varphi$ dove l'incognita è $u$.
\end{defn}

\begin{lm} \label{misura}
  Sia $K \subset\subset \mathbb{C}$ compatto e $\mu$ una misura con $supp(\mu)=K$. Allora l'integrale $\displaystyle u(w)=\int \dfrac{1}{w-z}\diff\mu(z)$ definisce una funzione $u \in \mathcal{O}(\mathbb{C}\setminus K)$.
\end{lm}

\begin{proof}
  Una misura $\mu$ con $supp(\mu)=K$ è un elemento $\mu \in (C^0(K))^*$ continuo. $\displaystyle \int \dfrac{1}{w-z} \diff\mu(z)=\mu\left(\dfrac{1}{w-\cdot}\right), \dfrac{1}{w-\cdot} \in C^0(K)$ se $w\not\in K$.
  Sia $a \not\in K$, $r>0$ t.c. $\overline{D(a, r)}\cap K=\emptyset$, $w \in D(a, r)$.
  $\displaystyle \dfrac{1}{w-z}=\dfrac{1}{(a-z)\left(1-\dfrac{a-w}{a-z}\right)}=\sum_{n \ge 0} \dfrac{(a-w)^n}{(a-z)^{n+1}}$ in quanto $\left|\dfrac{a-w}{a-z}\right|<1$ per ogni $z \in K, w \in D(a, r)$, quindi $\displaystyle \mu\left(\dfrac{1}{w-z}\right)=\sum_{n\ge 0} (a-w)^n\mu\left(\dfrac{1}{(a-z)^{n+1}}\right)$ è una serie di potenze in $w$ che converge in $D(a,r)$ $\implies$ $u \in \mathcal{O}(\mathbb{C}\setminus K)$.
\end{proof}

\begin{thm}
  Sia $\varphi \in C^k(\mathbb{C})$ a supporto compatto ($\varphi \in C^k_{\text{\textsc{c}}}(\mathbb{C})$). Allora esiste $u \in C^k(\mathbb{C})$ t.c. $\dfrac{\partial u}{\partial\bar{z}}=\varphi$.
\end{thm}

\begin{proof}
  Poniamo $\displaystyle u(w)=\dfrac{1}{2\pi i}\int \dfrac{\varphi(z)}{w-z} \diff\bar{z}\wedge\diff z$. È la $u$ data dal lemma \ref{misura} con $\mu(2\pi i)^{-1}\varphi(\diff\bar{z}\wedge\diff z)=-\dfrac{1}{\pi}\varphi\diff y\diff x$.
  $u \in \mathcal{O}(\mathbb{C}\setminus K)$. Facciamo un cambiamento di variabile: $\zeta=w-z \implies z=w-\zeta, \diff\zeta=-\diff z, \diff\bar{\zeta}=-\diff\bar{z}$.
  $u(w)=\dfrac{1}{2\pi i} \int_{\mathbb{C}} \dfrac{\varphi(w-\zeta)}{\zeta}\diff\bar{\zeta}\wedge\diff\zeta$. Siccome $1/\zeta$ è integrabile sui compatti di $\mathbb{C}$ e $\varphi \in C^k_{\text{\textsc{c}}}(\mathbb{C})$ possiamo derivare sotto il segno di integrale e le derivate sono continue, $u \in C^k(\mathbb{C})$.
  $\displaystyle \dfrac{\partial u}{\partial \bar{w}}(w)=\dfrac{1}{2\pi i}\int_{\mathbb{C}} \dfrac{\frac{\partial \varphi}{\partial \bar{z}}(w-\zeta)}{\zeta} \diff \bar{z}\wedge\diff z=-\dfrac{1}{2\pi i}\int_{\mathbb{C}} \dfrac{\frac{\partial\varphi}{\partial\bar{z}}(z)}{z-w}\diff\bar{z}\wedge\diff z=\dfrac{1}{2\pi i}\int_{\mathbb{C}} \dfrac{\frac{\partial\varphi}{\partial\bar{z}}(z)}{z-w}\diff z\wedge\diff\bar{z}=\dfrac{1}{2\pi i}\int_{\Omega} \dfrac{\frac{\partial\varphi}{\partial\bar{z}}(z)}{z-w}\diff z\wedge\diff\bar{z}$ dove $\Omega \subset \subset \mathbb{C}$ è un disco con $K \subset\subset \Omega$.
  Quindi $\varphi\restrict{\partial \Omega}\equiv 0$ e il teorema di Cauchy generalizzato ci dà $\dfrac{\partial u}{\partial\bar{w}}(w)=\varphi(w)$ per ogni $w \in K$.
\end{proof}

\begin{oss}
  Se $K=supp(\varphi)$, $u \in \mathcal{O}(\mathbb{C} \setminus K)$.
\end{oss}

\begin{oss}
  Non è detto che $u$ abbia supporto compatto.
\end{oss}

\begin{oss}
  $u$ è unica a meno di funzioni in $\mathcal{O}(\mathbb{C})$.
\end{oss}

\begin{defn}
  Ricordiamo che se $K \subset\subset \mathbb{C}$ è compatto e $f \in C^0(K)$, allora poniamo $\|f\|_K=\sup_{z \in K}|f(z)|$.
  Definiamo adesso $\mathcal{O}(K)=\{f \in C^0(K) \mid$ esiste $(U, \tilde{f})$ dove $U \supset K$ è un intorno aperto di $K$, $\tilde{f} \in \mathcal{O}(U)$ e $\tilde{f}\restrict{K}\equiv f\}$.
\end{defn}

\begin{ex}
  Se $w \not\in K, f(z)=\dfrac{1}{w-z}$, allora $f \in \mathcal{O}(K)$.
\end{ex}

\begin{thm}
  (Primo teorema di Runge) Sia $K \subset\subset \mathbb{C}$ compatto, $\Omega \subseteq \mathbb{C}$ un intorno aperto di $K$. Le seguenti sono equivalenti:
  \begin{nlist}
    \item ogni $f \in \mathcal{O}(K)$ può essere approssimata uniformemente su $K$ da funzioni in $\mathcal{O}(\Omega)$;
    \item $\Omega \setminus K$ non ha componenti connesse relativamente compatte in $\Omega$;
    \item per ogni $z \in \Omega \setminus K$ esiste $f \in \mathcal{O}(\Omega)$ t.c. $|f(z)|>\|f\|_K$.
  \end{nlist}
\end{thm}

\begin{proof}
  (iii) $\implies$ (ii) Se (ii) è falso, esiste $U$ componente connessa di $\Omega \setminus K$ con $\overline{U} \subset \Omega$ e $\partial U \subseteq K$, dunque per il principio del massimo abbiamo, per ogni $g \in \mathcal{O}(\Omega)$ e per ogni $z \in U$, che $|g(z)|\le \max_{\zeta \in \partial U} |g(\zeta)| \le \|g\|_K$, contro (iii).

  (i) $\implies$ (ii) Se (ii) è falso, esiste $U$ componente connessa di $\Omega \setminus K$ con $\overline{U} \subset \Omega$ e $\partial U \subseteq K$. Sia $w \in U$, $f(z)=\dfrac{1}{w-z}$ e $f \in \mathcal{O}(K)$.
  Se (i) fosse vera esisterebbe $\{f_n\} \in \mathcal{O}(\Omega)$ t.c. $\|f_n-f\|_K \longrightarrow 0$ $\implies$ $\|f_m-f_n\|_K \longrightarrow 0$ per $m, n \longrightarrow +\infty$.
  Sempre per il principio del massimo, per ogni $z \in U$ $|g(z)|\le \max_{\zeta \in \partial U} |g(\zeta)| \le \|g\|_K$ $\implies$ $\|f_m-f_n\|_{\overline{U}} \longrightarrow 0$ per $m,n \longrightarrow +\infty$ $\implies$ $\{f_n\}$ è di Cauchy in $C^0(U \cup K)$ $\implies$ converga a una $F \in C^0(U \cup K) \subset C^0(\overline{U})$.
  Per il teorema di Weierestrass, $F \in \mathcal{O}(U)$. Su $K$ abbiamo che $(w-z)F(z)\equiv 1$. Ma allora applicando il principio del massimo a $(w-z)F(z)-1 \in \mathcal{O}(U) \cap C^0(\overline{U})$ otteniamo $(w-z)F(z)-1 \equiv 0$ su tutto $U$, impossibile (in $w$ fa $-1$).

  (ii) $\implies$ (i) Si vedrà.

  (i)+(ii) $\implies$ (iii) Fissiamo $z_0 \in \Omega \setminus K$. Sia $D \subset \Omega \setminus K$ un disco chiuso di centro $z_0$. Le componenti connesse di $\Omega \setminus (K \cup D)$ sono lo stesse di $\Omega \setminus K$ con una a cui è stato tolto $D$. In particolare $K \cup D$ soddisfa (ii).
  La funzione $g$ che è $0$ su $K$ e $1$ su $D$ appartiene a $\mathcal{O}(K \cup D)$, dunque per (i) può essere approssimata da funzioni in $\mathcal{O}(\Omega)$ $\implies$ esiste $f \in \mathcal{O}(\Omega)$ t.c. $\|f\|_K<1/2$ e $\|f-1\|_D<1/2$ $\implies$ $|f(z_0)|>1-1/2=1/2$ $\implies$ $\|f\|_K<1/2<|f(z_0)|$.
\end{proof}
