\begin{frame}[t]
  \frametitle{Teoremi di Bracci-Kraus-Roth e di Burns-Krantz}
  \only<1-2>{\begin{thm}
    (Bracci-Kraus-Roth, 2020) Sia $f \in \text{\normalfont{Hol}}(\mathbb{D},\mathbb{D})$ tale che
    \begin{equation}
      |f^h(z_n)|=1+o\bigl((|z_n|-1)^2\bigr)
    \end{equation}
    per qualche successione $\{z_n\}_{n \in \mathbb{N}} \subset \mathbb{D}$ con $|z_n| \longrightarrow 1$. Allora $f \in \text{\normalfont{Aut}}(\mathbb{D})$.
  \end{thm}}
  \only<2>{\begin{thm}
    (Burns-Krantz, 1994) Siano $f \in \text{\normalfont{Hol}}(\mathbb{D},\mathbb{D})$ e $\sigma \in \partial\mathbb{D}$ tali che
    \begin{equation}
      f(z)=\sigma+(z-\sigma)+o\bigl((z-\sigma)^3\bigr)
    \end{equation}
    per $z \longrightarrow \sigma$ non tangenzialmente. Allora $f$ è l'identità del disco.
  \end{thm}}
  \only<3-5>{Ricordiamo il lemma di Schwarz-Pick.}
  \only<4-5>{\begin{block}{Lemma di Schwarz-Pick}
    \begin{itshape}
      Sia $f \in \text{\normalfont{Hol}}(\mathbb{D},\mathbb{D})$. Allora per ogni $z, w \in \mathbb{D}$ si ha
      $$|f^*(z,w)| \le 1.$$
      Inoltre, se vale l'uguaglianza per $z_0, w_0 \in \mathbb{D}$ allora $f \in \text{\normalfont{Aut}}(\mathbb{D})$ e vale sempre l'uguaglianza.
    \end{itshape}
  \end{block}}
  \only<5>{\begin{oss}
    I due Teoremi sono risultati di rigidità simili alla parte di unicità del lemma di Schwarz-Pick, ma per un punto sul bordo del disco; lo stesso Lemma è il punto di partenza per la dimostrazione elementare dei Teoremi.
\end{oss}}
\end{frame}
