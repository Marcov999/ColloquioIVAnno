Adesso possiamo procedere a dimostrare la serie di disuguaglianze di \cite{BM}, che coinvolgono la distanza di Poincaré $\omega$ e le funzioni olomorfe dal disco in sé che non sono automorfismi.

\begin{defn}
  La \textit{rotazione iperbolica} di ordine due attorno a un punto $a \in \mathbb{D}$ è la funzione $r_a \in \text{Aut}(\mathbb{D})$ data da
  $$r_a(z)=-\frac{z-\frac{2a}{1+|a|^2}}{1-\frac{2\bar{a}}{1+|a|^2}z}.$$
  Chiamiamo $a$ il \textit{centro di rotazione} di $r_a$. Con semplici passaggi algebrici, si può mostrare che $r_a$ è caratterizzata dall'equazione $[r_a(z),a]=-[z,a]$, da cui $r_a\circ r_a=\id$.
\end{defn}

\begin{defn}
  Dati $a_1,\dots,a_n \in \mathbb{D}$ e $\theta \in \mathbb{R}$, chiamiamo \textit{prodotto di Blaschke} di grado $n$ la funzione
  $$e^{i\theta}\prod_{j=1}^n \frac{z-a_j}{1-\bar{a}_jz}.$$
  Indichiamo con $\mathcal{B}_n$ i prodotti di Blaschke di grado $n$.
\end{defn}

\begin{oss}
  I prodotti di Blaschke sono funzioni olomorfe sul disco unitario, con zeri assegnati. Quelli di grado $1$ sono $\text{Aut}(\mathbb{D})$.
\end{oss}

\begin{lm} \label{Blaschke-car}
  Tra le funzioni $f$ continue in $\overline{\mathbb{D}}$ e olomorfe in $\mathbb{D}$, i prodotti di Blaschke di grado $n$ sono caratterizzati dalle seguenti proprietà:
  \begin{nlist}
    \item se $|z|=1$ allora $|f(z)|=1$;
    \item $f$ ha esattamente $n$ zeri in $\mathbb{D}$ contati con molteplicità.
  \end{nlist}
\end{lm}

\begin{proof}
  Se $f$ è un prodotto di Blaschke di grado $n$, che soddisfi (ii) è ovvio e che soddisfi (i) segue dall'Osservazione \ref{dom}.

  Fissiamo ora $f$ che soddisfi (i) e (ii); consideriamo $B$ il prodotto di Blaschke definito con $\theta=0$ e $a_n$ gli zeri in $\mathbb{D}$ di $f$, contati con molteplicità. Allora $f/B$ e $B/f$ sono due funzioni olomorfe su $\mathbb{D}$ e continue in $\overline{\mathbb{D}}$, di modulo $1$ sul bordo. Per il principio del massimo per funzioni olomorfe, deve essere $|f/B| \le 1$ e $|B/f| \le 1$ sul disco unitario, da cui $|f/B|=1$ e $f/B$ è costante in $\mathbb{D}$.
\end{proof}

\begin{lm} \label{z^2}
  Sia $f \in \mathcal{B}_2$, esistono $S, T \in \text{Aut}(\mathbb{D})$ tali che $S\circ f\circ T(z)=z^2$.
\end{lm}

\begin{proof}
  Per transitività di $\text{Aut}(\mathbb{D})$ su $\mathbb{D}$, ci basta dimostrare che esiste un punto $c \in \mathbb{D}$ tale che $f(z)-c$ abbia uno zero doppio nel disco. È sufficiente trovare $z_0 \in \mathbb{D}$ con $f'(z_0)=0$, dato che sappiamo già $f(z_0) \in \mathbb{D}$. Scriviamo $f(z)=e^{i\theta}\cdot\dfrac{z-a_1}{1-\bar{a}_1z}\cdot\dfrac{z-a_2}{1-\bar{a}_2z}$.
  Sempre per transitività, usando $T$ possiamo supporre $a_1=0$ e poniamo $a_2=a$. Si ha
  $$f'(z)=e^{i\theta}\left(\frac{z-a}{1-\bar{a}z}+z\cdot\frac{1-|a|^2}{1-\bar{a}z}\right).$$
  Con qualche passaggio algebrico, l'equazione $f'(z)=0$ diventa
  $$\bar{a}z^2-2z+a=0.$$
  Le soluzioni sono $(1 \pm \sqrt{1-|a|^2})/\bar{a}$; da $|a|<1$, abbiamo che quella con il più sta fuori da $\mathbb{D}$ mentre quella con il meno sta dentro, dunque è la soluzione cercata.
\end{proof}

\begin{defn}
  Sia $f \in \mathcal{B}_2$. Indichiamo con $R_f$ la rotazione iperbolica di ordine due attorno al punto $z_0$, trovato nella dimostrazione del Lemma \ref{z^2}.
\end{defn}

\begin{cor} \label{rotazioni}
  Sia $f \in \mathcal{B}_2$. Allora $f^*\bigl(R_f(w),w\bigr)=0$.
\end{cor}

\begin{proof}
  Siano $S, T \in \text{Aut}(\mathbb{D})$ date dal Lemma \ref{z^2} tali che $S\circ f\circ T=F$ con $F(z)=z^2$. Abbiamo $[R_F(z),0]=-[z,0]$, dunque per l'Osservazione \ref{muu} troviamo $\mu[T\bigl(R_F(z)\bigr),T(0)]=-\mu[T(z),T(0)]$.
  Per costruzione, $T(0)$ dev'essere il punto attorno al quale avviene la rotazione $R_f$, che quindi è caratterizzata dall'equazione $[R_f(z),T(0)]=-[z,T(0)]$. Ne segue che $T\bigl(R_F(z)\bigr)=R_f\bigl(T(z)\bigr)$. Perciò abbiamo $f=S^{-1}\circ F\circ T^{-1}$ e $R_f=T\circ R_F\circ T^{-1}$, inoltre $F\circ R_F=F$. Ne deduciamo che
  $$f\circ R_f=S^{-1}\circ F\circ R_F\circ T^{-1}=S^{-1}\circ F\circ T^{-1}=f,$$
  che implica la tesi.
\end{proof}

\begin{prop} \label{blaschke-prop}
  Valgono le seguenti affermazioni:
  \begin{nlist}
    \item date $F \in \text{Hol}(\mathbb{D},\mathbb{D})$ e $S \in \text{Aut}(\mathbb{D})$, si ha che $F \in \mathcal{B}_n$ se e solo se $S\circ F \in \mathcal{B}_n$;
    \item $f \in \mathcal{B}_{n+1}$ se e solo se $f^*(z,w) \in \mathcal{B}_n$, con $w$ un qualsiasi elemento di $\mathbb{D}$ fissato.
  \end{nlist}
\end{prop}

\begin{proof}
  Per dimostrare (i), basta mostrare che $S$ conserva le proprietà del Lemma \ref{Blaschke-car}. La prima segue dall'Osservazione \ref{dom}. Per transitività di $\text{Aut}(\mathbb{D})$, la seconda corrisponde a dover dimostrare che, dato $c \in \mathbb{D}$, l'equazione $F(z)=c$ ha esattamente $n$ zeri contati con molteplicità in $\mathbb{D}$, dove $F \in B_n$. Scriviamo $F(z)=\displaystyle e^{i\theta}\prod_{j=1}^n \frac{z-a_j}{1-\bar{a}_jz}$. Allora $F(z)=c$ si riscrive come
  \begin{equation} \label{pol-eq}
    c\prod_{j=1}^n (1-\bar{a}_jz)=e^{i\theta}\prod_{j=1}^n(z-a_j).
  \end{equation}
  Stiamo uguagliando due polinomi di grado $n$ con coefficienti direttivi diversi: quello al membro sinistro è di modulo minore di $1$, mentre quello al membro destro ha modulo esattamente $1$. Dunque la nostra equazione ha esattamente $n$ soluzioni, contate con molteplicità, in $\mathbb{C}$.
  Per l'Osservazione \ref{dom}, se $z \not\in \mathbb{D}$ e $z\not=1/\bar{a}_j$ per $j=1,\dots,n$, si ha $|F(z)| \ge 1$; se $z=1/\bar{a}_j$ per qualche $j$, il membro sinistro di \eqref{pol-eq} è $0$ ma il membro destro no, quindi non ci sono soluzioni in quel caso. Perciò, per $|c|<1$, tutte le soluzioni trovate sono in $\mathbb{D}$ come voluto.

  Per definizione di quoziente iperbolico, $[f(z),f(w)]=[z,w]f^*(z,w)$. Il membro di sinistra è della forma $S(f(z))$, dove $S \in \text{Aut}(\mathbb{D})$ è $S(z)=[z,f(w)]$; scriviamo anche $T(z)=[z,w]$.
  Se $f^*(z,w) \in \mathcal{B}_n$, allora $T(z)f^*(z,w) \in \mathcal{B}_{n+1}$; dunque $S\circ f \in \mathcal{B}_{n+1}$ e per il punto (i) abbiamo $f \in \mathcal{B}_{n+1}$. Viceversa, se $f \in \mathcal{B}_{n+1}$ si ha $S\circ f \in \mathcal{B}_{n+1}$.
  Sappiamo anche che $S\bigl(f(w)\bigr)=0$, dunque nel prodotto ci dev'essere il fattore $[z,w]$; segue dunque che $f^*(z,w) \in \mathcal{B}_n$.
\end{proof}

\begin{prop} \label{24}
  Siano $f \in \text{\normalfont{Hol}}(\mathbb{D},\mathbb{D})\setminus\text{\normalfont{Aut}}(\mathbb{D})$ e $v \in \mathbb{D}$. Allora per ogni $z \in \mathbb{D}$ si ha che $f^*(z,v) \in \mathbb{D}$ e la funzione $z \longmapsto f^*(z,v)$ è olomorfa.
\end{prop}

\begin{proof}
  Per quanto riguarda l'olomorfia, dalla definizione sappiamo che l'unico punto che potrebbe dar problemi è $v$; abbiamo però visto che la funzione ammette limite finito per $z \longrightarrow v$, perciò $v$ è una singolarità rimovibile. Per il lemma di Schwarz-Pick, $|f^*(z,v)| \le 1$; inoltre, vale l'uguaglianza in qualche punto solo se $f$ è un automorfismo. Dunque le ipotesi su $f$ assicurano che vale la disuguaglianza stretta sempre, cioè $f^*(z,v) \in \mathbb{D}$ per ogni $z \in \mathbb{D}$.
\end{proof}

\begin{thm} \label{31}
  (Beardon-Minda, 2004) Sia $f \in \text{\normalfont{Hol}}(\mathbb{D},\mathbb{D})\setminus\text{\normalfont{Aut}}(\mathbb{D})$. Allora per ogni $z, w, v \in \mathbb{D}$ vale
  \begin{equation} \label{3.1}
    \omega\bigl(f^*(z,v),f^*(w,v)\bigr) \le \omega(z,w).
  \end{equation}
  Si ha l'uguaglianza se e solo se $f \in \mathcal{B}_2$.
\end{thm}

\begin{proof}
  Poiché $f$ non è un automorfismo, per la Proposizione \ref{24} la funzione $z \longmapsto f^*(z,v)$ è in $\text{Hol}(\mathbb{D}, \mathbb{D})$; perciò il membro sinistro della disuguaglianza \eqref{3.1} è ben definito e la tesi segue dal lemma di Schwarz-Pick e dall'osservazione \ref{oss1}, punto (ii).

  Sempre dal lemma di Schwarz-Pick, si ha l'uguaglianza se e solo se abbiamo $f^*(z,v) \in \text{Aut}(\mathbb{D})=\mathcal{B}_1$. Per il punto (ii) della Proposizione \ref{blaschke-prop}, questo è equivalente a $f \in \mathcal{B}_2$.
\end{proof}

\begin{defn}
  Una \textit{geodetica} per $\omega$ è una curva $\sigma: \mathbb{R} \longrightarrow \mathbb{D}$ tale che per ogni $t_1,t_2 \in \mathbb{R}$ si ha $\omega\bigl(\sigma(t_1),\sigma(t_2)\bigr)=|t_1-t_2|$.
\end{defn}
\marginpar{forse bisognerebbe aggiungere la caratterizzazione di appartenenza a una stessa geodetica?}

\begin{cor} \label{32}
  Sia $f \in \text{\normalfont{Hol}}(\mathbb{D},\mathbb{D})\setminus\text{\normalfont{Aut}}(\mathbb{D})$. Allora per ogni $z, w, v \in \mathbb{D}$ vale
  \begin{equation}
    \omega\bigl(0, f^*(z,v)\bigr) \le \omega\bigl(0,f^*(w,v)\bigr)+\omega(z,w).
  \end{equation}
  Si ha l'uguaglianza se e solo se $f \in \mathcal{B}_2$ e $R_f(v)$, $w$ e $z$ giacciono sulla stessa geodetica, in quest'ordine.
\end{cor}

\begin{proof}
  Applicando la disuguaglianza triangolare per $\omega$ e il Teorema \ref{31}, si ha
  \begin{align*}
    \omega\bigl(0,f^*(z,v)\bigr) & \le \omega\bigl(0,f^*(w,v)\bigr)+\omega\bigl(f^*(w,v),f^*(z,v)\bigr) \\
    & \le \omega\bigl(0,f^*(w,v)\bigr)+\omega(z,w).
  \end{align*}

  Si ha l'uguaglianza se e solo se vale in entrambe le disuguaglianze appena viste. La seconda è esattamente il caso di uguaglianza del Teorema \ref{31}, che è equivalente a $f \in \mathcal{B}_2$. Sia $T_v(z)=f^*(z,v)$; per il Teorema \ref{blaschke-prop}, $f \in \mathcal{B}_2$ è equivalente a $T_v \in \text{Aut}(\mathbb{D})$. Ricordiamo che $p$, dunque anche $\omega$, è invariante sotto l'azione di $\text{Aut}(\mathbb{D})$.
  Allora il caso di uguaglianza nella prima delle due disuguaglianze si riscrive come $\omega\bigl(T_v^{-1}(0),z\bigr)=\omega\bigl(T_v^{-1}(0),w\bigr)+\omega(w,z)$, che caratterizza l'appartenenza, nell'ordine, alla stessa geodetica. Poiché $T_v$ è un automorfismo, esiste un solo valore per cui va in $0$; ma per il Corollario \ref{rotazioni}, questo valore è proprio $R_f(v)$.
\end{proof}

\begin{cor} \label{33}
  Sia $f \in \text{\normalfont{Hol}}(\mathbb{D},\mathbb{D})\setminus\text{\normalfont{Aut}}(\mathbb{D})$. Allora per ogni $z, w, v, u \in \mathbb{D}$ vale
  \begin{equation} \label{eq33}
    \omega\bigl(0, f^*(z,v)\bigr) \le \omega\bigl(0, f^*(u,w)\bigr)+\omega(z,w)+\omega(v,u).
  \end{equation}
  Si ha l'uguaglianza se e solo se $f \in \mathcal{B}_2$ e $R_f(v), R_f(u), w$ e $z$ giacciono sulla stessa geodetica, in quest'ordine.
\end{cor}
\begin{proof}
  Applicando il Corollario \ref{32} si ha
  \begin{align*}
    \omega\bigl(0,f^*(z,v)\bigr) & \le \omega\bigl(0,f^*(w,v)\bigr)+\omega(z,w) =\omega\bigl(0,|f^*(w,v)|\bigr)+\omega(z,w) \\
    & =\omega\bigl(0,|f^*(v,w)|\bigr)+\omega(z,w)=\omega\bigl(0,f^*(v,w)\bigr)+\omega(z,w).
  \end{align*}
  Sempre per il Corollario \ref{32} abbiamo
  $$\omega\bigl(0,f^*(v,w)\bigr) \le \omega\bigl(0,f^*(u,w)\bigr)+\omega(z,w)+\omega(v,u).$$
  Mettendo assieme le due disuguaglianze otteniamo la \eqref{eq33}.

  Se si ha l'uguaglianza, dobbiamo studiarla nelle due applicazioni del Corollario \ref{32}. In entrambi i casi ci dice che $f \in \mathcal{B}_2$. La prima ci dice anche che $R_f(v), w$ e $z$ appartengono, nell'ordine, alla stessa geodetica. Dalla seconda deduciamo la stessa cosa per $R_f(w), u$ e $v$. Poiché $R_f$ è un automorfismo e $\omega$ non cambia sotto l'azione di $\text{Aut}(\mathbb{D})$, abbiamo che lascia invariate le geodetiche; allora anche $w, R_f(u)$ e $R_f(v)$ stanno sulla stessa geodetica. Segue il secondo enunciato della tesi. Viceversa, se valgono tutte queste condizioni è facile vedere che si ha l'uguaglianza.
\end{proof}

Il risultato seguente non ci servirà nel seguito, ma viene riportato per completezza.

\begin{cor} \label{35}
  Sia $f \in \normalfont{\text{Hol}}(\mathbb{D},\mathbb{D})$ e siano $z, w \in \mathbb{D}$. Sia $\sigma$ una geodetica con $\sigma(t_1)=z, \sigma(t_2)=v$ e sia $w=\sigma(t)$ con $t_1<t<t_2$. Allora
  \begin{equation} \label{geod}
    2\omega\bigl(f(z),f(v)\bigr) \le \log\Bigl(\cosh\bigl(2\omega(z,v)\bigr)+|f^h(w)|\sinh\bigl(2\omega(z,v)\bigr)\Bigr).
  \end{equation}
\end{cor}
\marginpar{da rivedere}
\begin{proof}
  Osserviamo che se $f \in \text{Aut}(\mathbb{D})$, allora per il lemma di Schwarz-Pick $|f^h(w)|=1$ e il membro destro della disuguaglianza \eqref{geod} è esattamente $2\omega(z,v)$. In questo caso, per il lemma di Schwarz-Pick si ha l'uguaglianza.

  Supponiamo ora $f \not\in \text{Aut}(\mathbb{D})$, allora possiamo applicare il Corollario \ref{33} con $u=v$ per ottenere
  \begin{gather*}
    \omega\bigl(0,f^*(z,v)\bigr) \le \omega\bigl(0,f^h(w)\bigr)+\omega(z,v) \\
    p\bigl(0,f^*(z,v)\bigr) \le \tanh\Bigl(\omega\bigl(0,f^h(w)\bigr)+\omega(z,v)\Bigr) \\
    \frac{p\bigl(f(z),f(v)\bigr)}{p(z,v)} \le \frac{|f^h(w)|+p(z,v)}{1+|f^h(w)|p(z,v)},
  \end{gather*}
  dove abbiamo usato $p=\tanh\omega$, $f^*(z,v)=\frac{[f(z),f(v)]}{[z,v]}$, $\tanh(a+b)=\frac{\tanh{a}+\tanh{b}}{1+\tanh{a}\tanh{b}}$ e $p(0,\zeta)=|\zeta|$. Riscriviamo come
  \begin{gather*}
    \frac{p\bigl(f(z),f(v)\bigr)}{p(z,v)}-p(z,v) \le |f^h(w)|\Bigl(1-p\bigl(f(z),f(v)\bigr)\Bigr) \\
    \frac{p\bigl(f(z),f(v)\bigr)-p^2(z,v)}{p(z,v)\Bigl(1-p\bigl(f(z),f(v)\bigr)\Bigr)} \le |f^h(w)| \\
    \frac{2\Bigl(p\bigl(f(z),f(v)\bigr)-p^2(z,v)\Bigr)}{\bigl(1-p^2(z,v)\bigr)\Bigl(1-p\bigl(f(z),f(v)\bigr)\Bigr)} \le |f^h(w)|\cdot \frac{2p(z,v)}{1-p^2(z,v)}.
  \end{gather*}
  Adesso usiamo le seguenti uguaglianze:
  $$\frac{1+p^2}{1-p^2}=\cosh(2\omega) \text{ e } \frac{2p}{1-p^2}=\sinh(2\omega).$$
  Sommando appunto la quantità $\dfrac{1+p^2(z,v)}{1-p^2(z,v)}$ all'ultima disuguaglianza ottenuta, il membro destro diventa $\cosh\bigl(2\omega(z,v)\bigr)+|f^h(w)|\sinh\bigl(2\omega(z,v)\bigr)$. Ci basta dunque mostrare che il membro sinistro è uguale a $\exp\Bigl(2\omega\bigl(f(z),f(v)\bigr)\Bigr)$, cioè $\dfrac{1+p\bigl(f(z),f(v)\bigr)}{1-p\bigl(f(z),f(v)\bigr)}$. Si ha infatti
  \[
    \displaylines{\quad \frac{2\Bigl(p\bigl(f(z),f(v)\bigr)-p^2(z,v)\Bigr)}{\bigl(1-p^2(z,v)\bigr)\Bigl(1-p\bigl(f(z),f(v)\bigr)\Bigr)}+\frac{1+p^2(z,v)}{1-p^2(z,v)}\hfill \cr
    \hfill = \frac{p\bigl(f(z),f(v)\bigr)-p^2(z,v)+1-p^2(z,v)p\bigl(f(z),f(v)\bigr)}{\bigl(1-p^2(z,v)\bigr)\Bigl(1-p\bigl(f(z),f(v)\bigr)\Bigr)}\hskip27pt\cr
    \hfill = \frac{\bigl(1-p^2(z,v)\bigr)\Bigl(1+p\bigl(f(z),f(v)\bigr)\Bigr)}{\bigl(1-p^2(z,v)\bigr)\Bigl(1-p\bigl(f(z),f(v)\bigr)\Bigr)}=\frac{1+p\bigl(f(z),f(v)\bigr)}{1-p\bigl(f(z),f(v)\bigr)}.\quad \cr}
  \]
\end{proof}

\begin{cor} \label{36}
  Sia $f \in \text{\normalfont{Hol}}(\mathbb{D},\mathbb{D})\setminus\text{\normalfont{Aut}}(\mathbb{D})$ tale che $f(0)=0$. Allora
  \begin{equation}
    \omega\bigl(f^h(0),f^h(z)\bigr) \le 2\omega(0,z).
  \end{equation}
  Inoltre, $2$ è la migliore costante possibile.
\end{cor}

\begin{proof}
  Da $f(0)=0$ si ha $f^*(z,0)=f^*(0,z)$. Dalla disuguaglianza triangolare per $\omega$ abbiamo che
  \begin{align*}
    \omega\bigl(f^h(0),f^h(z)\bigr) & = \omega\bigl(f^*(0,0),f^*(z,z)\bigr) \\
    & \le \omega\bigl(f^*(0,0),f^*(z,0)\bigr)+\omega\bigl(f^*(0,z),f^*(z,z)\bigr),
  \end{align*}
  inoltre per il Teorema \ref{31} si ha
  $$\omega\bigl(f^*(0,0),f^*(z,0)\bigr)+\omega\bigl(f^*(0,z),f^*(z,z)\bigr)\le 2\omega(0,z).$$
  Mettendo assieme troviamo la disuguaglianza della tesi.

  Per dire che $2$ è la migliore costante possibile, basta prendere $f(z)=z^2$ e $z \in \mathbb{D}$ con $|z|=1/3$ per ottenere l'uguaglianza.
\end{proof}

Il prossimo risultato è quello che ci permetterà di dimostrare la disuguaglianza di Golusin.

\begin{cor} \label{quasigolusin}
  Sia $f \in \text{\normalfont{Hol}}(\mathbb{D},\mathbb{D})\setminus\text{\normalfont{Aut}}(\mathbb{D})$. Allora per ogni $z, w \in \mathbb{D}$ vale
  \begin{equation} \label{quasigol}
    \omega\bigl(|f^h(z)|, |f^h(w)|\bigr) \le 2\omega(z,w).
  \end{equation}
  Si ha l'uguaglianza se e solo se $f \in \mathcal{B}_2$ e $z$ e $w$ giacciono sulla stessa geodetica, passante per il centro di rotazione di $R_f$.
\end{cor}

\begin{proof}
  Siano $z, w \in \mathbb{D}$; senza perdita di generalità possiamo supporre $|f^h(z)| \ge |f^h(w)|$. Allora dalla definizione di $\omega$ abbiamo
  \begin{align*}
    \omega\bigl(|f^h(z)|, |f^h(w)|\bigr) & =\frac{1}{2}\log\left(\frac{1+\frac{|f^h(z)|-|f^h(w)|}{1-|f^h(w)||f^h(z)|}}{1-\frac{|f^h(z)|-|f^h(w)|}{1-|f^h(w)||f^h(z)|}}\right) \\
    & =\frac{1}{2}\log\left(\frac{1-|f^h(w)||f^h(z)|+|f^h(z)|-|f^h(w)|}{1-|f^h(w)||f^h(z)|+|f^h(w)|-|f^h(z)|}\right) \\
    & =\frac{1}{2}\log\left(\frac{1+|f^h(z)|}{1-|f^h(z)|}\cdot\frac{1-|f^h(w)|}{1+|f^h(w)|}\right) \\
    & =\frac{1}{2}\log\left(\frac{1+|f^h(z)|}{1-|f^h(z)|}\right)-\frac{1}{2}\log\left(\frac{1+|f^h(w)|}{1-|f^h(w)|}\right)
  \end{align*}
  Usando di nuovo la definizione di $\omega$ otteniamo dunque
  \begin{align*}
    \omega\bigl(|f^h(z)|, |f^h(w)|\bigr)&=\omega\bigl(0,|f^h(z)|\bigr)-\omega\bigl(0,|f^h(w)|\bigr) \\
    & =\omega\bigl(0,f^h(z)\bigr)-\omega\bigl(0,f^h(w)\bigr) \le 2\omega(z,w),
  \end{align*}
  dove l'ultima disuguaglianza segue dal Corollario \ref{33} prendendo $u=w$ e $v=z$. Il caso di uguaglianza segue facilmente.
\end{proof}

Concludiamo la sezione con il lemma di Dieudonné \cite[Chapter III, ???]{D}, per il quale l'approccio dell'articolo di Beardon e Minda semplifica le dimostrazioni.
\marginpar{Chapter o Chapitre? Poi: quale dei tanti? Non sono certo di quale sia quello più pertinente alla nostra versione}

\begin{lm}
  (lemma di Dieudonné) Sia $f \in \text{Hol}(\mathbb{D},\mathbb{D})$ tale che $f(0)=0$ e sia $z_0 \in \mathbb{D}$ con $|f(z_0)| \le |z_0|$. Allora
  \begin{equation}
    |f'(z_0)-f(z_0)/z_0| \le \frac{|z_0|^2-|f(z_0)|^2}{|z_0|(1-|z_0|^2)}.
  \end{equation}
\end{lm}

\begin{proof}
  Per il Teorema \ref{31} con $z=v=z_0$ e $w=0$ abbiamo
  \begin{align*}
    \omega\bigl(f^h(z_0),f^*(0,z_0)\bigr) & \le \omega(0,z_0) \\
    \iff p\bigl(f^h(z_0),f^*(0,z_0)\bigr) & \le p(0,z_0)=|z_0|,
  \end{align*}
  dove l'equivalenza fra le due disuguaglianze segue dal fatto che $\text{arctanh}$ è strettamente crescente. Per semplificare, scriviamo $f^h(z_0)=a, f^*(0,z_0)=b$ e $|z_0|=r$. Vogliamo portare la disuguaglianza in forma euclidea. Abbiamo
  $$\left|\frac{a-b}{1-\bar{b}a}\right|=p(a,b) \le r,$$
  che si riscrive come
  \marginpar{trova un modo più leggibile di fare tutta 'sta roba}
  \begin{align*}
    (a-b)(\bar{a}-\bar{b}) & \le r^2(1-\bar{b}a)(1-b\bar{a}) \\
    & \iff |a|^2-a\bar{b}-\bar{a}b+|b|^2 \le r^2-r^2a\bar{b}-r^2\bar{a}b+r^2|b|^2|a|^2 \\
    & \iff |a|^2(1-r^2|b|^2)-a\bar{b}(1-r^2)-\bar{a}b(1-r^2) \le r^2-|b|^2 \\
    & \iff |a|^2-a\cdot\frac{\bar{b}(1-r^2)}{1-r^2|b|^2}-\bar{a}\cdot\frac{b(1-r^2)}{1-r^2|b|^2} \le \frac{r^2-|b|^2}{1-r^2|b|^2};
  \end{align*}
  ponendo $\alpha=\dfrac{b(1-r^2)}{1-r^2|b|^2}$ e $R^2=\dfrac{r^2-|b|^2}{1-r^2|b|^2}+|b|^2\left(\dfrac{1-r^2}{1-r^2|b|^2}\right)^2$, si ha
  $$|a|^2-a\bar{\alpha}-\bar{a}\alpha \le R^2-|\alpha|^2 \iff (a-\alpha)(\bar{a}-\bar{\alpha}) \le R^2\iff |a-\alpha| \le R.$$
  Ricordando che $r=|z_0|$ e osservando che $b=f^*(0,z_0)=\frac{[f(0),f(z_0)]}{[0,z_0]}=\frac{f(z_0)}{z_0}$, troviamo $\alpha=\dfrac{f(z_0)(1-|z_0|^2)}{z_0\bigl(1-|f(z_0)|^2\bigr)}$ e $R=\dfrac{|z_0|^2-|f(z_0)|^2}{|z_0|\bigl(1-|f(z_0)|^2\bigr)}$.
  Riprendendo infine la definizione di $a$, cioè $a=f^h(z_0)=\frac{f'(z_0)(1-|z_0|^2)}{1-|f(z_0)|^2}$, otteniamo che
  $$\left|\frac{f'(z_0)(1-|z_0|^2)}{1-|f(z_0)|^2}-\frac{f(z_0)(1-|z_0|^2)}{z_0\bigl(1-|f(z_0)|^2\bigr)}\right| \le \frac{|z_0|^2-|f(z_0)|^2}{|z_0|\bigl(1-|f(z_0)|^2\bigr)},$$
  che è equivalente alla tesi moltiplicando entrambi i membri per $\frac{1-|f(z_0)|^2}{1-|z_0|^2}$.
\end{proof}
