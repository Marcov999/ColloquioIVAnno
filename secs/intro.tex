L'obiettivo di questo scritto è dimostrare un teorema del 1994, il teorema di Burns-Krantz (Theorem 2.1 di \cite{BK}), attraverso risultati elementari. L'enunciato del teorema riguarda le funzioni olomorfe sul disco unitario con un certo andamento vicino al bordo: se la funzione dista dall'identità al più per un $o\bigl((z-\sigma)^3\bigr)$, allora è proprio l'identità.

La dimostrazione originale del teorema non è lunga, ma un po' tecnica. In un recente articolo di Bracci, Kraus e Roth (\cite{BKR}) si trova una dimostrazione alternativa del teorema di Burns-Krantz. Come spiegato nel Remark 2.2 dell'articolo, è possibile passare dalle ipotesi del teorema di Burns-Krantz a quelle del Theorem 2.1 di \cite{BKR} (come dimostrato nella Proposition 8.1 dello stesso articolo), dal quale poi è facile concludere. Il Theorem 2.1 è sostanzialmente una versione al bordo del lemma di Schwarz-Pick. \\

Bracci, Kraus e Roth dimostrano il Theorem 2.1 usando risultati più generali visti nell'articolo, ma complicati. Tuttavia, nel Remark 5.6 danno una traccia per una dimostrazione più elementare. L'idea è sfruttare una disuguaglianza dovuta a Golusin e vengono indicati vari articoli in cui è stata ridimostrata.

In particolare, l'articolo di Beardon e Minda del 2004 (\cite{BM}) contiene una serie di disuguaglianze di facile dimostrazione, delle quali il Corollary 3.7 ha a sua volta come corollario la disuguaglianza di Golusin. Queste disuguaglianze coinvolgono la distanza di Poincaré sul disco unitario e possono essere applicate per ottenere diversi altri risultati per funzioni olomorfe sul disco, come mostrato nell'articolo. \\

In questo scritto sviluppiamo la traccia data nel Remark 5.6 di \cite{BKR}. Dimostreremo le disuguaglianze in \cite{BM} e vedremo alcune applicazioni, concludendo con la disuguaglianza di Golusin. Grazie ad essa, e alla Proposition 8.1 di \cite{BKR}, otterremo una dimostrazione elementare del Theorem 2.1 di Burns-Krantz e di un risultato più generale dovuto a Bracci-Kraus-Roth.
