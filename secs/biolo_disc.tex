Lo scopo di questo paragrafo è mostrare che quasi tutti i domini semplicemente connessi di $\mathbb{C}$ sono biolomorfi al disco. Il teorema di uniformizzazione di Riemann, di cui riporteremo solo l'enunciato, caratterizza i biolomorfismi delle superfici di Riemann, in particolare caratterizza completamente i biolomorfismi di quelle semplicemente connesse.

\begin{lm} \label{esistefinF}
  Sia $\Omega \subset \mathbb{D}$ dominio limitato con $\Omega \not=\mathbb{D}$ e $0 \in \Omega$.
  Allora esiste $f \in \text{Hol}(\mathbb{D}, \mathbb{D})$ t.c. $f(0)=0, f'(0) \in \mathbb{R}, f'(0)>0$, $\Omega \subseteq f(\mathbb{D})$ e, se $\Omega_f$ è la componente connessa di $f^{-1}(\Omega)$ contentente $0$, $f\restrict{\Omega_f}:\Omega_f \longrightarrow \Omega$ è un rivestimento.
  Inoltre, $\displaystyle d_1=\inf_{z \not\in \Omega_f} |z|>\inf_{z \not\in \Omega} |z|=d$.
\end{lm}

\begin{proof}
  Sia $a \in \mathbb{D}\setminus\Omega$, $b \in \mathbb{D}$ t.c. $b^2=-a$. Siano $\varphi, \psi \in \text{Aut}(\mathbb{D})$, $\varphi(z)=\dfrac{z+a}{1+\bar{a}z}, \psi(z)=\dfrac{z+b}{1+\bar{b}z}$.
  Poniamo $f \in \text{Hol}(\mathbb{D}, \mathbb{D})$ t.c. $f(z)=\dfrac{\bar{b}}{|b|}\varphi(\psi(z)^2)$. $f(\mathbb{D})=\mathbb{D}, f(0)=0$. $f'(0)=2|b|\dfrac{1-|b|^2}{1-|a|^2}>0$.
  Siccome $w \longmapsto w^2$ è un rivestimento di $\mathbb{D}^*=\mathbb{D}\setminus\{0\}$ e $\varphi^{-1}(\Omega) \subseteq \mathbb{D}^*$, $f$ è un rivestimento da $\Omega_f$ a $\Omega$. Se $d_1=1$ abbiamo finito in quanto $d \le |a|<1$.
  Se invece $d_1<1$, esiste $z_1 \in \partial\Omega_f \cap \mathbb{D}$ con $|z_1|=d_1 \implies f(z_1) \not\in \Omega \implies |f(z_1)| \ge d$. Allora per il lemma di Schwarz abbiamo $d \le |f(z_1)|<|z_1|=d_1$.
\end{proof}

\begin{thm}
  (Osgood, Koebe) Sia $\Omega \subset \subset \mathbb{C}$ un dominio limitato, $z_0 \in \Omega$. Allora esiste un unico rivestimento olomorfo $f_0:\mathbb{D} \longrightarrow \Omega$ t.c. $f_0(0)=z_0$ e $f_0'(0) \in \mathbb{R}, f'(0)>0$.
\end{thm}

\begin{proof}
  Possiamo supporre $z_0=0 \in \Omega$ e $\Omega \subset \subset \mathbb{D}$. Sia $\mathcal{F} \subset \text{Hol}(\mathbb{D}, \mathbb{D})$ t.c.
  $\mathcal{F}=\{f \in \text{Hol}(\mathbb{D}, \mathbb{D}) \mid f(0)=0, f'(0) \in \mathbb{R}, f'(0)>0; \Omega \subseteq f(\mathbb{D});$ se $\Omega_f$ è la componente connessa di $f^{-1}(\Omega)$ contenente $0$ allora $f\restrict{\Omega_f}:\Omega_f \longrightarrow \Omega$ è un rivestimento$\}$.
  Se esiste $f_0 \in \mathcal{F}$ con $\Omega_{f_0}=\mathbb{D}$ ci resta da dimostrare solo l'unicità. Poniamo per ogni $f \in \mathcal{F}$ $\displaystyle d_f=\inf_{z \not\in \Omega_f} |z|=\min_{z \in \partial\Omega_f} |z| \le 1$. Abbiamo $d_f=1 \iff \Omega_f=\mathbb{D}$.
  Dobbiamo trovare $f_0 \in \mathcal{F}$ con $d_{f_0}=1$. Sia $\displaystyle d=\sup_{f \in \mathcal{F}} d_f \le 1$. Sia $\{f_n\} \subset \mathcal{F}$ t.c. $d_{f_n} \longrightarrow d$.
  Per il teorema di Montel possiamo supporre che $f_n \longrightarrow f_0 \in \text{Hol}(\mathbb{D}, \mathbb{C})$ con immaggine in $\overline {\mathbb{D}}$. Vogliamo $f_0 \in \mathcal{F}$. Chiaramente $f_0(0)=0, f_0'(0) \in \mathbb{R}, f'(0) \ge 0$.
  \begin{enumerate}
    \item $f_0$ non è costante e $f'(0)>0$: sia $r>0$ t.c. $\mathbb{D}_r \subset \subset \Omega$. Siccome $f_n:\Omega_{f_n} \longrightarrow \Omega$ è un rivestimento e $0 \in \Omega_{f_n}$, esiste un unico $h_n:\mathbb{D}_r \longrightarrow \Omega_{f_n}$ olomorfa t.c. $f_n \circ h_n=\id_{\mathbb{D}_r}$ e $h_r(0)=0$.
    Sempre per il teorema di Montel, a meno di sottosuccessioni possiamo supporre $h_n \longrightarrow h_0 \in \text{Hol}(\mathbb{D}_r, \overline{\mathbb{D}})$ t.c. $f_0 \circ h_0=\id_{\mathbb{D}_r}$.
    Dunque $h_0$ è iniettiva, quindi per il teorema dell'applicazione aperta è aperta, perciò $h_0(\mathbb{D}_r) \subseteq \mathbb{D}$, e $f_0$ non è costante, dunque per il primo teorema di Hurwitz $f_0(\mathbb{D}) \subseteq \mathbb{D}$. $1=\id_{\mathbb{D}_r}'(0)=(f_0 \circ h_0)'(0)=f_0'(h_0(0))\cdot h_0'(0)=f_0'(0) \cdot h_0'(0) \implies f_0'(0) \not=0 \implies f_0'(0)>0$.
    \item $f_0(\Omega_{f_0})=\Omega$ dove $\Omega_{f_0}$ è la componente connessa di $f_0^{-1}(\Omega)$ contenente $0$. Sia infatti $z_0 \in \Omega$ e sia $\gamma$ una curva in $\Omega$ da $0$ a $z_0$.
    Ricopriamo $\gamma$ con dischi $D_0=\mathbb{D}_r, D_1, \dots, D_k$ con $D_j \cap D_{j+1} \not=\emptyset$ e $z_0 \in D_k$. Sia $h_{n, 0}$ l'inversa di $f_n$ su $D_0$ t.c. $h_{n,0}=0$.
    Per ogni $j$ sia $h_{n, j}$ l'inversa di $f_n$ su $D_j$ che coincide con $h_{n, j-1}$ su $D_{j-1} \cap D_j$. Per Montel, a meno di sottosuccessioni $h_{n, 0} \longrightarrow h_{0, 0} \in \text{Hol}(D_0, \mathbb{D})$.
    Per il teorema di Vitali, $h_{n, 1} \longrightarrow h_{0, 1} \in \text{Hol}(D_1, \mathbb{D})$ e, per induzione, $h_{n, k} \longrightarrow h_{0, k} \in \text{Hol}(D_k, \mathbb{D})$ con $f_0 \circ h_{0, k}=\id_{D_k} \implies f_0(h_{0, k}(z_0))=z_0 \implies z_0 \in f_0(\mathbb{D})$.
    In realtà $h_{0, k}(z_0) \in \Omega_{f_0}$ perché è immagine della curva ottenuta con $h_{0, j} \circ \gamma$ che parte da $0$ e quindi è contenuta nella componente connessa di $\Omega$ contenete $0$.
    \item $f_0:\Omega_{f_0} \longrightarrow \Omega$ è un rivestimento. Sia $z_0 \in \Omega$, $D \subseteq \Omega$ un disco di centro $z_0$.
    Per ogni $w_0 \in f_0^{-1}(z_0) \cap \Omega_{f_0}$ vogliamo un intorno $U_{w_0} \subseteq \Omega_{f_0}$ t.c. $f_0\restrict{U_{w_0}}:U_{w_0} \longrightarrow D$ è un biolomorfismo e $w_0 \not=w_0' \implies U_{w_0}\cap U_{w_0'}=\emptyset$.
    Per il primo teorema di Hurwitz, fissato $w_0$ esiste $n_1 \ge 1$ t.c. per ogni $n \ge n_1$ esiste $w_n \in \Omega_{f_n}$ t.c. $g_n(w_n)=z_0$ e $w_n \longrightarrow w_0$. Sia $h_n \in \text{Hol}(D, \mathbb{D})$ l'inversa di $f_n$ su $D$ con $h_n(z_0)=w_n$ (possiamo usare $D$ per tutte le $f_n$ perché, dalla teoria dei rivestimenti, dato un rivestimento un aperto semplicemente connesso dell'insieme di arrivo è sempre ben rivestito).
    Per Montel, a meno di sottosuccessioni $h_n \longrightarrow h_0 \in \text{Hol}(D, \mathbb{D})$ t.c. $f_0 \circ h_0=\id_D$ (perché $f_n \circ h_n=\id_D$ per ogni $n$), $h_0(z_0)=w_0$.
    Poniamo $U_{w_0}=h_0(D)$. È aperto perché $h_0$ non costante $\implies$ $h_0$ aperta, inoltre $U_{w_0} \subseteq \Omega_{f_0}$ perché, essendo immagine di un connesso, è connesso, e contiene $z_0=h_0(z_0)$, dunque deve stare nella componente connessa di $z_0$. Sia $w_0' \in f_0^{-1}(z_0) \cap \Omega_{f_0}$, costruiamo $h_0', U_{w_0'}$ come prima, senza perdita di generalità con la stessa sottosuccessione.
    Per assurdo, esiste $w \in U_{w_0} \cap U_{w_0'}$. Allora esistono $z_1, z_1' \in D$ t.c. $w=h_0(z_1)=h_0'(z_1')$.
    Applichiamo $f_0$ a entrambi i membri: $z_1=f_0(h_0(z_1))=f_0(w)=f_0(h_0(z_1'))=z_1' \implies z_1=z_1'$ è uno zero di $h_0-h_0'$, dunque per il primo teorema di Hurwitz o $h_0-h_0' \equiv 0 \implies w_0=w_0' \implies U_{w_0}=U_{w_0'}$ oppure per ogni $n>>1$ $h_n-h_n'$ ha uno zero, ma $h_n$ è l'inversa di $f_n$ su $D$ con $w_n \in h_n(D)$, $h_n'$ è l'inversa di $f_n$ su $D$ con $w_n' \in h_n'(D)$.
    $w_0 \not=w_0' \implies w_n \not=w_n'$ per ogni $n>>1 \implies h_n(D)\cap h_n'(D)=\emptyset$ perché inverse di un rivestimento che mandano lo stesso punto in due punti diversi, assurdo.
  \end{enumerate}

  Per il lemma \ref{esistefinF}, $\mathcal{F}\not=\emptyset$, quindi la costruzione che abbiamo fatto di $f_0$ ha senso. Per assurdo, $\displaystyle d_{f_0}=\sup_{f \in \mathcal{F}} d_f<1 \implies \Omega_{f_0} \subset \mathbb{D}$ con $\Omega_{f_0} \not= \mathbb{D}$.
  Sempre per il lemma \ref{esistefinF} esiste $f_1 \in \text{Hol}(\mathbb{D}, \mathbb{D})$, $f_1(0)=0, f_1'(0) \in \mathbb{R}, f_1'(0)>0$, $\Omega_{f_0} \subset f_1(\mathbb{D})$, $f_1\restrict{\Omega_{f_1}}:\Omega_{f_1} \longrightarrow \Omega_{f_0}$ rivestimento con $\Omega_{f_1}$ la componente connessa di $f_1^{-1}(\Omega_{f_0})$ contentente $0$ e $d_{f_1}>d_{f_0}$.
  Ma allora, se $\tilde{f}=f_0 \circ f_1:\Omega_{f_1} \longrightarrow \Omega$, abbiamo $\tilde{f} \in \mathcal{F}$ con $d_{\tilde{f}}=d_{f_1}>d_{f_0}$, assurdo.

  Per l'unicità si veda l'esercizio \ref{un_biolo}.
\end{proof}

\begin{exc} \label{un_biolo}
  Se $f_1, f_2:\mathbb{D} \longrightarrow \Omega$ sono rivestimenti con $f_1(0)=f_2(0)$ e $f_1'(0), f_2'(0)>0$, allora $f_1 \equiv f_2$. Hint: dato che sono rivestimenti, si sfruttano $h$ e $h^{-1}$ date dal seguente diagramma commutativo:
  \begin{center}
    \begin{tikzcd}
      & \mathbb{D} \arrow[dl, "h"', shift right] \arrow[d, "f_1"]\\
      \mathbb{D} \arrow[ru, "h^{-1}" right, shift right] \arrow[r, "f_2"'] & \Omega
    \end{tikzcd}
  \end{center}
\end{exc}

\begin{thm}
  (Riemann) Se $\Omega \subset \mathbb{C}$ è un dominio semplicemente connesso con $\Omega \not=\mathbb{C}$, allora $\Omega$ è biolomorfo a $\mathbb{D}$.
\end{thm}

\begin{proof}
  Basta far vedere che $\Omega$ è biolomorfo a un dominio limitato: quest'ultimo sarebbe semplicemente connesso e rivestito da $\mathbb{D}$ che è connesso, dunque per la teoria generale dei rivestimenti il rivestimento in questione sarebbe un omeomorfismo, ma dato che era anche un olomorfismo, allora è un biolomorfismo. Prendiamo $a \in \mathbb{C}\setminus \Omega$, $h \in \mathcal{O}(\Omega)$ t.c. $h(z)^2=z-a$ per ogni $z \in \Omega$. $h$ è iniettiva, ma vale di più: $h(z_1)=\pm h(z_2) \implies z_1-a=h(z_1)^2=h(z_2)^2=z_2-a \implies z_1=z_2$.
  Dunque $h$ è iniettiva e $h(\Omega) \cap (-h(\Omega))=\emptyset$. Fissato $z_0 \in \Omega$, sia $r>0$ t.c. $D=D(h(z_0), r) \subset h(\Omega) \implies D \cap(-h(\Omega))=\emptyset \implies |h(z)+h(z_0)| \ge r$ per ogni $z \in \Omega \implies 2|h(z_0)| \ge r$.
  Sia $f \in \mathcal{O}(\Omega)$ data da $f(z)=\dfrac{r}{4}\dfrac{1}{|h(z_0)|}\dfrac{h(z)-h(z_0)}{h(z)+h(z_0)}$. $f(z_0)=0$ e $f$ è iniettiva: $f(z_1)=f(z_2) \implies \dfrac{h(z_2)-h(z_0)}{h(z_1)+h(z_0)}=\dfrac{h(z_2)-h(z_0)}{h(z_2)+h(z_0)} \implies h(z_1)=h(z_2) \implies z_1=z_2$
  $\implies$ $f$ è un biolomorfismo tra $\Omega$ e $f(\Omega)$ e $f(\Omega) \subseteq \mathbb{D}$ in quanto $\left|\dfrac{h(z)-h(z_0)}{h(z)+h(z_0)}\right|=|h(z_0)|\left|\dfrac{1}{h(z_0)}-\dfrac{2}{h(z)+h(z_0)}\right| \le |h(z_0)|\left|\dfrac{1}{|h(z_0)|}+\dfrac{2}{|h(z)+h(z_0)|}\right| \le \dfrac{4|h(z_0)|}{r}$.
\end{proof}

Per il teorema di Liouville abbiamo che $\mathbb{C}$ non è biolomorfo a $\mathbb{D}$. Dato che $\widehat{\mathbb{C}}$ è compatta, abbiamo che non è biolomorfa né a $\mathbb{D}$ né a $\mathbb{C}$.

\begin{thm}
  (Uniformizzazione di Riemann) Se $X$ è una superficie di Riemann semplicemente connessa, allora $X$ è biolomorfo a $\widehat{\mathbb{C}}$, $\mathbb{C}$ o $\mathbb{D}$. Più in generale, se $X$ è una superficie di Riemann qualsiasi e $\pi:\widetilde{X} \longrightarrow X$ è un rivestimento universale, allora $\widetilde{X}$ è una superficie di Riemann e
  \begin{nlist}
    \item se $\widetilde{X}$ è biolomorfo a $\widehat{\mathbb{C}}$, allora anche $X$ è biolomorfo a $\widehat{\mathbb{C}}$ (caso ellittico);
    \item se $\widetilde{X}$ è biolomorfo a $\mathbb{C}$, allora $X$ è biolomorfo a $\mathbb{C}$, $\mathbb{C}^*$, oppure un toro $T_{\tau}=\faktor{\mathbb{C}}{(\mathbb{Z}+\tau \mathbb{Z})}$ con $\mathfrak{Im}\tau>0$ (caso parabolico);
    \item in tutti gli altri casi $\widetilde{X}$ è biolomorfo a $\mathbb{D}$ (caso iperbolico).
  \end{nlist}
\end{thm}

Quindi, se $\Omega \subset\subset \mathbb{C}$ è limitato e semplicemente connesso, abbiamo per il teorema di Riemann che esiste $f:\mathbb{D} \longrightarrow \Omega$ biolomorfismo. Domanda: possiamo estendere $f$ a un omeomorfismo da $\overline{\mathbb{D}}$ a $\overline{\Omega}$?

\begin{thm}
  (Carathéodory) Un biolomorfismo $\mathbb{D} \longrightarrow \Omega$ si estende continuo da $\overline{\mathbb{D}}$ a $\overline{\Omega}$ se e solo se $\partial \Omega$ è localmente connesso.
\end{thm}

\begin{cor}
  Si estende a un omeomorfismo se e solo se $\partial\Omega$ è una curva di Jordan (cioè immagine omeomorfa di $S^1$).
\end{cor}

Esiste una condizione su $\partial\Omega$ diversa che è equivalente all'estendibilità di $f^{-1}:\Omega \longrightarrow \mathbb{D}$ al bordo.
