\begin{defn}
  Sia $\Omega \subset \mathbb{C}$ un aperto. Una funzione $f:\Omega \longrightarrow \mathbb{C}$ si dice \textit{olomorfa} in $\Omega$ se è derivabile in senso complesso per ogni $z \in \Omega$.
\end{defn}

\begin{defn}
  Se $f: \Omega \longrightarrow \Omega$ olomorfa è biettiva, allora si può dimostrare che anche $f^{-1}$ è olomorfa. In tal caso $f$ è detta \textit{automorfismo} (in senso olomorfo di $\Omega$).
\end{defn}

Com'è noto, la condizione di olomorfia per funzioni a valori complessi è molto più forte della derivabilità in senso reale (in particolare, è equivalente all'analiticità). Fra i vari risultati che si possono dimostrare per le funzioni olomorfe, ci interessa studiare il lemma di Schwarz-Pick. \\

Notazione: indichiamo il disco unitario con $\mathbb{D}=\{z \in \mathbb{C} \mid |z|<1\}$.

\begin{lm}
  (Schwarz) Sia $f:\mathbb{D} \longrightarrow \mathbb{D}$ una funzione olomorfa t.c. $f(0)=0$. Allora per ogni $z \in \mathbb{D}$ $|f(z)| \le |z|$ e $|f'(0)| \le 1$; inoltre, se vale l'uguale nella prima per $z \not=0$ oppure nella seconda allora $f(z)=e^{i\theta}z, \theta \in \mathbb{R}$.
\end{lm}

\begin{lm}
  (Schwarz-Pick) Sia $f:\mathbb{D} \longrightarrow \mathbb{D}$ una funzione olomorfa.
  Allora per ogni $z, w \in \mathbb{D}$
  $$\left|\frac{f(z)-f(w)}{1-\overline{f(w)}f(z)}\right| \le \left|\frac{z-w}{1-\bar{w}z}\right|, \qquad \frac{|f'(z)|}{1-|f(z)|^2} \le \frac{1}{1-|z|^2}.$$
  Inoltre se vale l'uguale nella prima per $z_0, w_0$ con $z_0 \not=w_0$ o nella seconda per $z_0$ allora $f$ è un automorfismo e vale l'uguale sempre.
\end{lm}

Il lemma di Schwarz-Pick può essere riformulato usando due funzioni di due variabili sul disco, una delle quali è nota come distanza iperbolica (in realtà, anche l'altra è una distanza, ma questo non lo useremo). Con queste funzioni dimostreremo una serie di disuguaglianze che ci permetteranno di dimostrare la disuguaglianza di Golusin, dalla quale seguirà una versione al bordo del lemma.

\begin{defn}
  Dati $z, w \in \mathbb{D}$ poniamo
  $$[z,w]:=\frac{z-w}{1-\bar{w}z}, \qquad p(z,w):=|[z,w]|, \qquad d(z,w):=\log\left(\frac{1+p(z,w)}{1-p(z,w)}\right).$$
\end{defn}

$d$ è ben definita, in quanto $p(z,w)<1$. Infatti, dobbiamo verificare che
  $$\frac{|z-w|}{|1-\bar{w}z|} < 1$$
  $$|z-w|^2 < |1-\bar{w}z|^2$$
  $$|z|^2+|w|^2-\bar{w}-w\bar{z} < 1+|wz|^2-\bar{w}z-w\bar{z}$$
  $$1+|wz|^2-|z|^2-|w|^2 > 0$$
  $$(1-|w|^2)(1-|z|^2) > 0,$$
che è vera perché $z, w \in \mathbb{D}$.

\begin{prop}
  $d$ è una distanza (la sopracitata distanza iperbolica).
\end{prop}

\begin{proof}
  Mostriamo preliminarmente che $p$ è una distanza. In entrambi i casi, l'unica cosa non ovvia da controllare è la disuguaglianza triangolare. Perciò, dati $z_0, z_1, z_2 \in \mathbb{D}$, vogliamo $p(z_1,z_2) \le p(z_1,z_0)+p(z_0,z_2)$. Osserviamo che, per il lemma di Schwarz-Pick, $p$ è invariante applicando automorfismi, perciò supponiamo senza perdita di generalità $z_1=0$ (possiamo farlo perché il gruppo degli automorfismidi $\mathbb{D}$ è transitivo). A questo punto la disuguaglianza da dimostrare diventa
  \marginpar{Come si dimostra? Qui c'è una dim, ma ponendo $z_0=0$: https://mathoverflow.net/questions/21604/nice-proof-of-the-triangle-inequality-for-the-metric-of-the-hyperbolic-plane}
  \begin{align*}
    |z_2| \le |z_0|+\frac{|z_0-z_2|}{|1-\bar{z}_2z_0|}.
  \end{align*}
  (c'è da fare la dimostrazione) \\
  A questo punto, possiamo osservare che $d(z,w) =2\,\text{arctanh}\,(p(z,w))$, perciò... ACHTUNG: LA DIM DEGLI APPUNTI DI ECA SEMBRA ESSERE FALLACE, ARCTANH NON È SUBADDITIVA SUI POSITIVI
  %Altrimenti, si può brutalmente espandere il conto e ricondursi a una disuguaglianza la cui dimostrazione si trova qui: https://www.math3ma.com/blog/the-pseudo-hyperbolic-metric-and-lindelofs-inequality
\end{proof}

\begin{defn}
  Data una funzione $f: \mathbb{D} \longrightarrow \mathbb{D}$, poniamo
  $$f^*(z,w):=\frac{[f(z),f(w)]}{[z,w]}$$
  e
  $$f^h(z):=|f^*(z,z)|:=\left|\lim_{w \longrightarrow z} f^*(z,w)\right|=\left|\lim_{w \longrightarrow z} \frac{[f(z),f(w)]}{[z,w]}\right|=\frac{|f'(z)|(1-|z|^2)}{1-|f(z)|^2}.$$
\end{defn}

\begin{oss}
  \begin{nlist}
    \item la disuguaglianza del lemma di Schwarz-Pick può essere riscritta come $|f^*(z,w)| \le 1$;
    \item  un altro modo di scrivere la disuguaglianza del lemma di Schwarz-Pick è $p(f(z),f(w)) \le p(z,w)$;
    \item $p(0,z)=|z| \implies d(0,z)=d(0,|z|)$;
    \item per definizione, $|f^*(z,w)|=|f^*(w,z)|$ e $f^h(z)$ è reale non negativo.
  \end{nlist}
  Questi risultati verranno usati nelle varie dimostrazioni e verranno esplicitati solo quando ciò che ne segue non è immediato.
\end{oss}

Seguono alcuni risultati noti di analisi complessa noti che verranno usati nelle dimostrazioni.

\begin{thm}
  (formula integrale di Cauchy) Sia $f$ olomorfa sull'aperto $\Omega$ e $D$ un disco chiuso di centro $a$ contenuto in $\Omega$. Allora
  \begin{equation}
    f^{(n)}(a)=\frac{n!}{2\pi i} \int_{\partial D} \frac{f(\zeta)}{(\zeta-a)^{n+1}}\diff\zeta.
  \end{equation}
\end{thm}

\begin{prop}
  Sia $f$ olomorfa sull'aperto $\Omega \setminus\{z_0\}$, con $z_0 \in \Omega$. Allora $f$ si estende a una funzione olomorfa $g$ definita su tutto $\Omega$ se e solo se è limitata in un intorno di $z_0$. In tal caso, $z_0$ è detta \textit{singolarità rimovibile}.
\end{prop}
