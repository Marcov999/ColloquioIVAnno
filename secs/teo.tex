\begin{frame}
  \frametitle{Strada per la dimostrazione del teorema di BB}
  \begin{itemize}
    \item Dato $(Z,d)$ spazio metrico completo e limitato, si può costruire uno spazio iperbolico $\big(\text{Con}(Z),r\big)$ tale che $Z$ è identificato con il bordo.
    \pause
    \item Per una metrica che soddisfa certe ipotesi, la distanza indotta differisce per una costante da una funzione $g$ simile alla $r$ della suddetta costruzione; questo ci permette di dire che $\Omega$ con tale distanza è iperbolico.
    \pause
    \item La metrica di Kobayashi soddisfa le suddette ipotesi.
  \end{itemize}
\end{frame}

\begin{frame}[t]
  \frametitle{Lo spazio iperbolico $\text{Con}(Z)$}
  \only<1-3>{
  \begin{thm}
    Sia $(Z,d)$ uno spazio metrico completo e limitato e sia $\text{Con}(Z)=Z\times(0\times D(Z)]$, dove $D(Z)$ è il diametro di $Z$.
    La funzione $r:\text{Con}(Z)\times\text{Con}(Z) \longrightarrow [0,+\infty)$ data da
    $$r\big((z,h),(z',h')\big)=2\log\left(\frac{d(z,z')+h\lor h'}{\sqrt{hh'}}\right)$$
    è una distanza su $\text{Con}(Z)$ che lo rende uno spazio iperbolico, il cui bordo può essere identificato con $Z$, e inoltre per ogni $x,y \in Z$ si ha
    $$d(x,y) \asymp \exp\big(-(x,y)_w\big).$$
  \end{thm}
  }
  \only<2-3>{
  \textit{Traccia della dimostrazione:} è facile verificare che $r$ è una distanza.

  }
  \only<3>{Dati $r_{ij} \ge 0$ tali che $r_{ij}=r_{ji}$ e $r_{ij} \le r_{ik}+r_{kj}$, allora $r_{12}r_{34} \le 4\big((r_{13}r_{24})\lor (r_{14}r_{23})\big)$.
  }
  \only<4-6>{Siano $x_i=(z_i,h_i) \in \text{Con}(Z)$ per $i \in \{1,2,3,4\}$, poniamo $d_{ij}=d(z_i,z_j)$ e $r_{ij}=d_{ij}+h_i\lor h_j$.} \only<5-6>{Segue che
  \begin{gather*}
    (d_{12}+h_1\lor h_2)(d_{34}+h_3\lor h_4) \\
    \le 4\Big(\big((d_{13}+h_1\lor h_3)(d_{24}+h_2\lor h_4)\big)\big((d_{14}+h_1\lor h_4)(d_{23}+h_2\lor h_3)\big)\Big),
  \end{gather*}}
  \only<6>{che ci dà
  \begin{gather*}
    r(x_1,x_2)+r(x_3,x_4) \le \big(r(x_1,x_3)+r(x_2,x_4)\big)\big(r(x_1,x_4)+r(x_2,x_3)\big)+C,
  \end{gather*}
  da cui segue l'iperbolicità di $\big(\textit{Con}(Z), r\big)$.}
  \only<7->{
  Fissiamo $w=\big(z_0,D(Z)\big) \in \text{Con}(Z)$; usando le definizioni, troviamo che dati $x=(z,h),x'=(z,h')\in \text{Con}(Z)$ vale
  $$(x,x')_w=-\log\big(d(z,z')+h\lor h'\big)+O_{D(Z)}(1).$$}
  \only<8->{Segue che una sequenza $(x_i)$ in $\big(\text{Con}(Z),r\big)$ converge a infinito se e solo se la sequenza $(z_i)$ è di Cauchy e $h_i \longrightarrow 0$; inoltre, due successioni convergenti a infinito sono equivalenti se e solo se il loro limite è lo stesso, e ogni punto del bordo è limite di una successione che converge a infinito.}\only<9->{ Essendo $Z$ completo, questo ci dà un'identificazione, come insiemi, di $Z$ e $\partial_G\text{Con}(Z)$.

  }
  \only<10->{
  La disuguaglianza finale segue da quanto visto finora e delle semplici verifiche.\qed %in realtà no, bisogna sistemarla
  }
\end{frame}

\begin{frame}[t]
  \frametitle{Disuguaglianze per metriche di Finsler}
  \only<1-2>{
  \begin{defn}
    Una \textit{metrica di Finsler} su $\Omega$ è una funzione continua $F:\Omega\times\mathbb{C}^n \longrightarrow [0,+\infty)$ tale che $F(x;tZ)=|t|F(x;Z)$ per ogni $x \in \Omega, Z \in \mathbb{C}^n, t \in \mathbb{C}$.
  \end{defn}
  }
  \only<2>{
  \begin{thm}
    Sia $F$ una metrica di Finsler su $\Omega$ tale che esistono delle costanti $\epsilon_0>0, s>0,C_1>0,C_2 \ge 0$ tali che per ogni $x \in \Omega$ con $\delta(x)<\epsilon_0$ e $Z \in \mathbb{C}^n$ si ha
    \begin{multline}\label{stimametricafinsler}
      \big(1-C_1\delta^s(x)\big)\left(\frac{|Z_N|^2}{4\delta^2(x)}+(1/C_2)\frac{L_\rho\big(\pi(x);Z_H\big)}{\delta(x)}\right)^{1/2} \le F(x;Z) \\
      \le \big(1+C_1\delta^s(x)\big)\left(\frac{|Z_N|^2}{4\delta^2(x)}+C_2\frac{L_\rho\big(\pi(x);Z_H\big)}{\delta(x)}\right)^{1/2}.
    \end{multline}
  \end{thm}
  }
  \only<3->{
  \begin{thm}
    Allora esiste $C \ge 0$ tale che per ogni $x, y \in \Omega$ vale
    \begin{equation} \label{stimadistanzafinsler}
      g(x,y)-C \le d_F(x,y) \le g(x,y)+C,
    \end{equation}
    dove $g(x,y)=2\log\left(\frac{d_H\big(\pi(x),\pi(y)\big)+\delta(x)^{1/2 }\lor\delta(y)^{1/2}}{\sqrt{\delta(x)^{1/2 }\delta(y)^{1/2}}}\right)$.
  \end{thm}
  }
  \only<4->{
  \textit{Idea della dimostrazione:} per la maggiorazione, si cercano delle curve che siano quasi geodetiche, cioè danno la distanza a meno di una costante additiva, e si integra lungo quelle curve.
  }
  \only<5>{

  Per la minorazione, bisogna mostrare che le curve trovate sono ottimali.
  }
\end{frame}
