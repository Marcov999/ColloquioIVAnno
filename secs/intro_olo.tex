Notazioni: $z=x+iy \in \mathbb{C}$ indica un numero complesso, $\bar{z}=x-iy$ il suo complesso coniugato. Con il termine \textit{dominio} si intende un aperto connesso. $\mathcal{O}(\Omega)=\{f: \Omega \rightarrow \mathbb{C} | f \text{ è olomorfa} \}$.
$\text{Hol}(\Omega_1, \Omega_2)=\{f: \Omega_1 \rightarrow \Omega_2 | f \text{ è olomorfa} \}$.
$\dfrac{\partial}{\partial z}=\dfrac{1}{2}\left(\dfrac{\partial}{\partial x}-i\dfrac{\partial}{\partial y}\right), \dfrac{\partial}{\partial \bar{z}}=\dfrac{1}{2}\left(\dfrac{\partial}{\partial x}+i\dfrac{\partial}{\partial y}\right)$. \\
$\diff z=\diff x+i\diff y, \diff\bar{z}=\diff x-i\diff y, \diff z\left(\dfrac{\partial}{\partial z}\right)=1, \diff z\left(\dfrac{\partial}{\partial \bar{z}}\right)=0$. \\

Daremo ora una definizione di funzione olomorfa basata su quattro definizioni, l'equivalenza delle quali è un prerequisito del corso e dovrebbe essere quindi nota agli studenti.

\begin{defn}
  Sia $\Omega \subseteq \mathbb{C}$ un aperto, $f: \Omega \in \mathbb{C}$ si dice \textsc{olomorfa} se vale una delle seguenti condizioni equivalenti:
  \begin{nlist}
      \item $f$ è \textit{$\mathbb{C}$-differenziabile}, cioè per ogni $a \in \Omega$ esiste $\displaystyle f'(a)=\lim_{z \rightarrow a}=\frac{f(z)-f(a)}{z-a}$;
      \item $f$ è \textit{analitica}, cioè per ogni $a \in \Omega$ esiste $U \subseteq \Omega$ aperto e intorno di $a$ e $\{c_n\}\subset \mathbb{C}$ t.c. per ogni $z \in U$ $\displaystyle f(z)=\sum_{n=0}^{+\infty} c_n(z-a)^n$;
      \item $f$ è \textit{olomorfa}, cioè $f$ è continua, $\partial f/\partial x$ e $\partial f/\partial y$ esistono su $\Omega$ e $\dfrac{\partial f}{\partial x}+i\dfrac{\partial f}{\partial y} \equiv 0$ (equazione di Cauchy-Riemann).
      Si noti che la condizione è $\dfrac{\partial f}{\partial \bar{z}} \equiv 0$, da cui si ricava $\dfrac{\partial f}{\partial z}=f'$;
      \item $f$ è continua e per ogni rettangolo (o disco) chiuso $D \subseteq \Omega$ si ha $\displaystyle \int_{\partial D} f\diff z=0$ (teorema di Cauchy-Goursat+Morera).
  \end{nlist}
\end{defn}

La seguente proposizione è anch'essa un risultato che dovrebbe essere noto agli studenti che seguono il corso.

\begin{prop}
  Sia $\{c_n\} \in \mathbb{C}$. Allora:
  \begin{nlist}
    \item esiste $R \in [0, +\infty]$ t.c. $\displaystyle \sum_{n=0}^{+\infty} c_nz^n$ converge per $|z|<R$ e diverge per $|z|>R$. $R$ è detto \textit{raggio di convergenza}. La convergenza +è uniforme su $\Delta_r=\{|z| \le r\}, r<R$. $\displaystyle \limsup_{n \rightarrow +\infty} |c_n|^{1/n}=\frac{1}{R}$;
    \item $\displaystyle \sum_{n=0}^{+\infty} nc_nz^{n-1}$ ha lo stesso raggio di convergenza;
    \item se $\displaystyle f(z)=\sum_{n=0}^{+\infty} c_n(z-a)^n$ allora $f'(z)=\sum_{n=0}^{+\infty} nc_n(z-a)^{n-1}$;
    \item se $f \in \mathcal{O}(\Omega)$ e $a \in \Omega$, allora $\displaystyle f(z)=\sum_{n=0}^{+\infty} \frac{1}{n!}f^{(n)}(a)(z-a)^n$. Questa formula è valida nel più grande disco aperto centrato in $a$ e contenuto in $\Omega$, cioè di raggio minore o uguale di $d(a, \partial \Omega)$.
  \end{nlist}
\end{prop}

\begin{thm}
  (Formula di Cauchy) Sia $\Omega$ aperto, $f \in \mathcal{O}(\Omega), D \subseteq \Omega$ disco/rettangolo chiuso. Per ogni $\displaystyle a \in D, f(a)=\frac{1}{2\pi i} \int_{\partial D} \frac{\zeta}{\zeta-a} \diff\zeta$.
  Si ha che $\displaystyle f^{(n)}(a)=\frac{n!}{2\pi i} \int_{\partial D} \frac{f(\zeta)}{(\zeta-a)^{n+1}}\diff \zeta$.
\end{thm}

\begin{cor}
  (Disuguaglianze di Cauchy) $f \in \mathcal{O}(\Omega), D=D(a, r) \subseteq \Omega$ disco di centro $a \in \Omega$ e raggio $r>0$. Sia $\displaystyle M=\max_{\zeta \in \partial D} |f(\zeta)|$. Allora per ogni $n \ge 1, |f^{(n)}(a)| \le \dfrac{n!}{r^n}M$.
\end{cor}

\begin{cor}
  (Teorema di Liouville) Sia $f \in \mathcal{O}(\mathbb{C})$ limitata. Allora $f$ è costante.
\end{cor}

\begin{proof}
  Le disuguaglianze di Cauchy danno, per ogni $r>0$, $|f'(a)| \le \dfrac{M}{r}$ dove $\displaystyle M=\sup_{z \in \mathbb{C}} |f(z)|<+\infty \implies
  f' \equiv 0$.
\end{proof}

\begin{thm}
  (Principio di identità o del prolungamento analitico) $\Omega \subseteq \mathbb{C}$ dominio, $f, g \in \mathcal{O}(\Omega)$. Se $\{z \in \Omega | f(z)=g(z)\}$ ha un punto di accumulazione in $\Omega$, allora $f \equiv g$.
\end{thm}

\begin{cor}
  $\Omega \subseteq \mathbb{C}$ dominio, $f \in \mathcal{O}(\Omega)$ non identicamente nulla, allora $\{z \in \Omega | f(z)=0\}$ è discreto in $\Omega$.
\end{cor}

\begin{thm} \label{pr_max}
  (Principio del massimo) $\Omega \subseteq \mathbb{C}$ dominio, $f \in \mathcal{O}(\Omega)$. Allora:
  \begin{nlist}
    \item se $U$ è aperto e $U \subset \subset \Omega$ (si legge "$U$ relativamente compatto in $\Omega$" e si intende $\overline{U} \subset \Omega$ e $\overline{U}$ compatto) allora $\displaystyle \sup_{z \in U} |f(z)| \le \sup_{z \in \partial D} |f(z)|$. Inoltre, se $|f|$ ha un massimo locale in $U$, allora $f$ è costante in $\Omega$;
    \item la stessa affermazione vale per $\mathfrak{Re} f$ e $\mathfrak{Im} f$;
    \item se $\Omega$ è limitato poniamo $\displaystyle M=\sup_{x \in \partial D} \limsup_{z \rightarrow x} |f(z)| \in [0, +\infty]$. Allora per ogni $z \in \Omega$ $|f(z)| \le M$ con uguaglianza in un punto se e solo se $f$ è costante.
  \end{nlist}
\end{thm}

\begin{ex}
  Controesempio per vedere che serve $\Omega$ limitato per il punto (iii) del teorema \ref{pr_max}: $\Omega=\{z \in \mathbb{C} | \mathfrak{Re} z>0\}, f(z)=e^z$. $f \in \mathcal{O}(\mathbb{C}) \subset \mathcal{O}(\Omega)$.
  $z \in \partial\Omega \implies z=iy \implies |f(iy)|=|e^{iy}|=1$, ma $f$ è illimitata in $\Omega$. Per correggere questa cosa si aggiunge il punto all'infinito.
\end{ex}

\begin{thm}
  (Applicazione aperta) $f \in \mathcal{O}(\Omega)$ non costante $\implies$ $f$ è un'applicazione aperta.
\end{thm}

Siano $X, Y$ spazi topologici e indichiamo con $C^0(X, Y)$ le funzioni continue da $X$ in $Y$.

La \textit{topologia della convergenza puntuale} è la restrizione a $C^0(X, Y) \subset Y^X=\{f:X \rightarrow Y\}$ della topologia prodotto. Una prebase è data da $\mathcal{F}(x, U)=\{f \in C^0(X, Y) |f(x) \in U\}$ dove $x \in X$ e $U \subseteq Y$ è un aperto.

\begin{exc}
  $f_n \rightarrow f \in C^0(X, Y)$ per questa topologia se e solo se $f_n(x) \rightarrow f(x)$ per ogni $x \in X$.
\end{exc}

La \textit{topologia compatta aperta} ha invece come prebase $\mathcal{F}(K, U)= \\ =\{f \in C^0(X, Y) | f(K) \subseteq U\}$ dove $U$ è preso come sopra e $K \subseteq X$ è un compatto.

\begin{prop}
  \begin{nlist}
    \item La topologia compatta aperta è più fine della topologia della convergenza puntuale;
    \item $Y$ Hausdorff $\implies$ topologia compatta aperta Hausdorff.
  \end{nlist}
\end{prop}

\begin{proof}
  \begin{nlist}
    \item Ovvia (il singoletto è un compatto).
    \item Prendiamo $f \not\equiv g$ continue, allora esiste $x_0 \in X$ t.c. $f(x_0) \not= g(x_0)$, per cui, dato che $Y$ è Hausdorff,
    esistono $U, V \subset Y$ aperti disgiunti con $f(x_0) \in U, g(x_0) \in V \implies f \in \mathcal{F}(x_0, U), g \in \mathcal{F}(x_0, V), \mathcal{F}(x_0, U) \cap \mathcal{F}(x_0, V)=\emptyset$.
  \end{nlist}
\end{proof}

\begin{thm}
  (Ascoli-Arzelà) Siano $X, Y$ spazi metrici con $X$ localmente compatto, allora $\mathcal{F} \subseteq C^0(X, Y)$ è relativamente compatta rispetto alla topologia compatta aperta se e solo se:
  \begin{nlist}
    \item per ogni $x \in X$ $\{f(x) | f \in \mathcal{F}\} \subset \subset Y$;
    \item $\mathcal{F}$ è equicontinua.
  \end{nlist}
\end{thm}

La topologia compatta aperta viene detta anche topologia della \textit{convergenza uniforme sui compatti}: $\{f_n\} \subset C^0(X, \mathbb{R}^N)$. Se $K \subseteq X$ definiamo $\displaystyle \|f\|_K=\sup_{z \in K} \|f(z)\|$.
$f_n \rightarrow f$ uniformemente sui compatti se per ogni $K \subset \subset X$ compatto e per ogni $\epsilon>0$ esiste $n_0$ t.c. $n \ge n_0 \implies \|f_n-f\|_K<\epsilon$.

\begin{exc}
  $f_n \rightarrow f$ uniformemente sui compatti se e solo se $f_n \rightarrow f$ nella topologia compatta aperta.
\end{exc}

\begin{thm}
  (Weierstrass) Sia $\{f_n\} \subset \mathcal{O}(\Omega)$ t.c. $f_n \rightarrow f \in C^0(\Omega, \mathcal{C})$ uniformemente sui compatti. Allora:
  \begin{nlist}
    \item $f \in \mathcal{O}(\Omega)$;
    \item $f_n' \rightarrow f'$ uniformemente sui compatti.
  \end{nlist}
\end{thm}

\begin{proof}
  \begin{nlist}
    \item Sia $a \in \Omega$, $0<r<d(a, \partial\Omega)$ t.c. $D=D(a, r) \subset \subset \Omega$. $\displaystyle f_n(z)=\frac{1}{2\pi i} \int_{\partial D} \frac{f_n(\zeta)}{\zeta-z} \diff \zeta$ per ogni $z \in D(a, \rho)$ per ogni $0<\rho<r$.
    Allora $\dfrac{1}{|\zeta-z|} \le \dfrac{1}{r-\rho}$ per ogni $z \in D(a, \rho), \zeta \in \partial{D}$.
    Per ogni $\displaystyle z \in D(a, \rho), f(z)=\lim_{n \rightarrow +\infty} \frac{1}{2\pi i} \int_{\partial D} \frac{f_n(\zeta)}{\zeta-z} \diff \zeta$.
    Adesso, per uniforme convergenza e uniforme limitatezza si può portare il limite dentro, perciò $\displaystyle f(z)=\frac{1}{2\pi i}\int_{\partial D} \frac{f(\zeta)}{\zeta-z} \diff \zeta$, ma questo, per il teorema di Cauchy-Goursat+Morera, implica $f \in \mathcal{O}(\Omega)$.
    \item $\displaystyle f_n'(z)=\frac{1}{2\pi i} \int_{\partial D} \frac{f_n'(\zeta)}{(\zeta-z)^2}\diff\zeta \rightarrow \frac{1}{2\pi i} \int_{\partial D} \frac{f(\zeta)}{\zeta-z} \diff\zeta=f'(z)$.
    $f_n' \rightarrow f'$ uniformemente sui dischi e ogni compatto è coperto da un numero finito di dischi $\implies$ $f_n' \rightarrow f'$ uniformemente sui compatti.
  \end{nlist}
\end{proof}

\begin{thm}
  (Montel) $\Omega \subseteq \mathbb{C}$ aperto, $\mathcal{F} \subseteq \mathcal{O}(\Omega)$ t.c. per ogni $K \subset \subset C$ compatto esiste $M_K>0$ t.c. $\|f\|_K \le M_K$ per ogni $f \in \mathcal{F}$ (si diche che $\mathcal{F}$ è \textit{uniformemente limitata sui compatti}).
  Allora $\mathcal{F}$ è relativamente compatta in $\mathcal{O}(\Omega)$.
\end{thm}

\begin{proof}
  Basta vedere che ogni successione $\{f_n\} \subseteq \mathcal{F}$ ha una sottosuccessione convergente (segue dal fatto che, nelle ipotesi del teorema di Montel, la topologia compatta aperta è metrizzabile). \\
  Dati $a \in \Omega, 0<r<d(a, \partial \Omega), f \in \mathcal{O}(\Omega)$, sia $c_n(f)=\dfrac{f^{(n)}(a)}{n!}$, allora $\displaystyle f(z)=\sum_{n=0}^{+\infty} c_n(f)(z-a)^n$ in $\overline{D(a, r)}$.
  Inoltre, se $\|f\|_{\overline{D(a, r)}} \le M$, allora per le disuguaglianze di Cauchy $|c_n(f)| \le \dfrac{M}{r^n}$ per ogni $n \ge 0$.
  Sia $\{f_n\} \subseteq \mathcal{F}$. Per ipotesi, esiste $M$ t.c. $\|f_n\|_{\overline{D(a, r)}} \le M$ per ogni $n$ $\implies$ $|c_0(f_n)| \le M$ per ogni $n$ $\implies$ esiste una sottosuccessione $c_0(f_{n_j^{(0)}})$ che tende a $c_0 \in \mathbb{C}$.
  Per  induzione, da $\{f_{n_j^{(k-1)}}\}$ possiamo estrarre una sottosuccessione $\{f_{n_j^{(k)}}\}$ t.c. $c_k(f_{n_j^{(k)}}) \rightarrow c_k \in \mathbb{C}$. Consideriamo $\{f_{n_j^{(j)}}\}$, allora $c_k(f_{n_j^{(j)}}) \rightarrow c_k \in \mathbb{C}$ per ogni $k$.
  Sia $f_{\nu_j}=f_{n_j^{(j)}}$. Poniamo $D_a=\overline{D(a, r/2)}$ e sia $z \in D_a$. Vogliamo $\displaystyle f_{\nu_j} \rightarrow f(z)=\sum_{n=0}^{+\infty} c_n(z-a)^n$ in $D_a$. Basta vedere che $f_{\nu_j}$ è di Cauchy uniformemente in $D_a$.
  $\displaystyle |f_{\nu_h}(z)-f_{\nu_k}(z)| \le \sum_{n=0}^{+\infty} |c_n(f_{\nu_h})-c_n(f_{\nu_k})||z-a|^n=\sum_{n=0}^N |c_n(f_{\nu_h})-c_n(f_{\nu_k})||z-a|^n+\sum_{n>N} |c_n(f_{\nu_h})-c_n(f_{\nu_k})||z-a|^n$.
  Sappiamo che $|c_n(f_{\nu_k})| \le \dfrac{M}{r^n}$ e $z \in D_a \implies |z-a| \le \dfrac{r}{2}$.
  Allora $\displaystyle \sum_{n>N} |c_n(f_{\nu_h})-c_n(f_{\nu_k})||z-a|^n \le \sum_{n>N} \frac{2M}{r^n}\left(\frac{r}{2}\right)^n=\frac{M}{2^{N-1}}$.
  $\displaystyle \sum_{n=0}^{+\infty} |c_n(f_{\nu_h})-c_n(f_{\nu_k})||z-a|^n \le \sum_{n=0}^{+\infty} |c_n(f_{\nu_h})-c_n(f_{\nu_k})|\left(\dfrac{r}{2}\right)^n$.
  Dato $\epsilon>0$, scegliamo $N>>1$ t.c. $\dfrac{M}{2^{N-1}}<\epsilon/2$ e $n_0$ t.c. per ogni $h, k \ge n_0, |c_n(f_{\nu_h})-c_n(f_{\nu_k})|\left(\dfrac{r}{2}\right)^n<\dfrac{\epsilon}{2(N+1)}$
  (possiamo farlo, una volta fissato $N$, perché gli $n$ tra $0$ e $N$ sono in numero finito e le successioni $c_n(f_{\nu_j})$ convergono, dunque si sceglie un indice per ogni successione e si prende come $n_0$ il massimo di questi indici).
  Mettendo insieme le disuguaglianze si ha che per ogni $\epsilon>0$ esiste $n_0$ t.c. per ogni $h, k \ge n_0$ e per ogni $z \in D_a$, $|f_{\nu_k}(z)-f_{\nu_k}(z)|<\epsilon$, dunque la sottosuccessione $f_{\nu_j}$ è di Cauchy e converge uniformemente su $D_a$.
  Deve convergere a $f$ perché, per il teorema di Weierstrass, le derivate convergono al valore della derivata limite, e questo ci dice che i coefficienti della serie della funzione limite sono proprio quelli di $f$. \\
  $\Omega$ è a base numerabile, dunque possiamo estrarre un sottoricoprimento numerabile da $\{D_a | a \in \Omega\}$. Sia dunque $\{a_j\} \subseteq \Omega$ t.c. $\displaystyle \bigcup_j D_{a_j}=\Omega$. Per quanto dimostrato finora, possiamo estrarre da $\{f_n\}$ una sottosuccessione $\{f_{n_j^{(0)}}\}$ convergente uniformemente in $D_{a_0}$.
  Per induzione, da $\{f_{n_j^{(k-1)}}\}$ estraiamo una sottosuccessione $\{f_{n_j^{(k)}}\}$ convergente uniformemente in $D_{a_0} \cup \dots \cup D_{a_k}$. Prendiamo $\{f_{n_j^{(j)}}\}$ che converge uniformemente in ogni $D_{a_k}$.
  Adesso, ogni compatto è coperto da un numero finito di $D_{a_k}$, quindi (scegliendo per ogni $\epsilon$ il massimo degli indici t.c. le cose che vogliamo valgono in quei $D_{a_k}$) $\{f_{n_j^{(j)}}\}$ converge uniformemente sui compatti.
\end{proof}
