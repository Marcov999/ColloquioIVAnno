\begin{frame}
  \frametitle{Lemma di Schwarz come risultato di rigidità}
  \begin{block}{Lemma di Schwarz}
    \begin{itshape}
      Sia $f \in \text{\normalfont{Hol}}(\mathbb{D},\mathbb{D})$ tale che $f(0)=0$. Allora per ogni $z \in \mathbb{D}$ si ha $|f(z)| \le |z|$ e $|f'(0)| \le 1$; inoltre, se vale l'uguaglianza nella prima per $z_0 \in \mathbb{D}\setminus\{0\}$ oppure nella seconda allora $f(z)=e^{i\theta}z$ per qualche $\theta \in \mathbb{R}$.
    \end{itshape}
  \end{block}
  \pause
  \begin{oss}
    Dai casi di uguaglianza seguono i seguenti risultati di rigidità: se $|f'(z)|=1+o(1)$ per $z$ che tende a $0$, allora $f \in \text{Aut}(\mathbb{D})$; se $f(z)=z+o(z)$ per $z$ che tende a $0$, allora $f$ è proprio l'identità.
  \end{oss}
\end{frame}

\begin{frame}[t]
  \frametitle{Teorema di Burns-Krantz}
  \only<1>{\begin{thm}
    (Burns-Krantz, 1994) Siano $f \in \text{\normalfont{Hol}}(\mathbb{D},\mathbb{D})$ e $\sigma \in \partial\mathbb{D}$ tali che
    \begin{equation}
      f(z)=\sigma+(z-\sigma)+o\bigl((z-\sigma)^3\bigr)
    \end{equation}
    per $z \longrightarrow \sigma$ non tangenzialmente. Allora $f$ è l'identità del disco.
  \end{thm}}
  \only<2-5>{\begin{defn}
    Dati $\sigma \in \partial \mathbb{D}$ e $M>1$, chiamiamo \textit{regione di Stolz} l'insieme
    $$K(\sigma,M)=\left\{z \in \mathbb{D} \mid \frac{|\sigma-z|}{1-|z|} < M\right\}.$$
  \end{defn}}
  \only<3>{
  \begin{center}
  \begin{tikzpicture}[line cap=round,line join=round,>=triangle 45,x=1.9cm,y=1.9cm]
    \draw[->,color=black] (-1.16,0) -- (1.18,0);
    \foreach \x in {-1,1}
    \draw[shift={(\x,0)},color=black] (0pt,2pt) -- (0pt,-2pt);
    \draw[->,color=black] (0,-1.13) -- (0,1.15);
    \foreach \y in {-1,1}
    \draw[shift={(0,\y)},color=black] (2pt,0pt) -- (-2pt,0pt);
    \clip(-1.16,-1.13) rectangle (1.18,1.15);
    \draw(0,0) circle (1.9cm);
    \draw[name path=A, smooth,samples=100,domain=-0.1715705740864879:1] plot(\x,{sqrt(7-4*sqrt(3-2*(\x))-2*(\x)-(\x)^2)});
    \draw[name path=B, smooth,samples=100,domain=-0.1715705740864879:1] plot(\x,{0-sqrt(7-4*sqrt(3-2*(\x))-2*(\x)-(\x)^2)});
    \tikzfillbetween[of=A and B]{red, opacity=0.25};
  \end{tikzpicture}

  La regione di Stolz $K(1,2)$.\end{center}}
  \only<4-5>{\begin{defn}
    Diciamo che una funzione $f:\mathbb{D} \longrightarrow \mathbb{C}$ ha \textit{limite non tangenziale} $L \in \mathbb{C}$ in $\sigma \in \partial\mathbb{D}$ e scriviamo
    $$\substack{\text{nt--lim} \\ z \longrightarrow \sigma} \, f(z)=L$$
    se  per ogni $M>1$ si ha $\displaystyle \lim_{\substack{z \longrightarrow \sigma, \\ z \in K(\sigma,M)}} f(z)=L$.
  \end{defn}}
  \only<5>{Notiamo che la definizione di limite non tangenziale è più debole di quella di limite classico; nel nostro caso rende il risultato più forte.}
  \only<6-7>{\begin{thm}
    (Burns-Krantz, 1994) Siano $f \in \text{\normalfont{Hol}}(\mathbb{D},\mathbb{D})$ e $\sigma \in \partial\mathbb{D}$ tali che
    \begin{equation}
      f(z)=\sigma+(z-\sigma)+o\bigl((z-\sigma)^3\bigr)
    \end{equation}
    per $z \longrightarrow \sigma$ non tangenzialmente. Allora $f$ è l'identità del disco.
  \end{thm}}
  \only<7>{\begin{ex}
    Se $f(z)=\dfrac{1+3z^2}{3+z^2}$, si ha $\displaystyle \lim_{z \longrightarrow 1}\frac{f(z)-z}{(z-1)^3}=-\frac{1}{4}$; dunque il termine $o\bigl((z-\sigma)^3\big)$ nel teorema di Burns-Krantz non è migliorabile.
\end{ex}}
\end{frame}

\begin{frame}[t]
  \frametitle{Lemma di Schwarz-Pick e derivata iperbolica}
  \only<1-3>{\begin{block}{Lemma di Schwarz-Pick}
    \begin{itshape}
      Sia $f \in \text{\normalfont{Hol}}(\mathbb{D},\mathbb{D})$.
      Allora per ogni $z, w \in \mathbb{D}$ si ha
      $$\left|\frac{f(z)-f(w)}{1-\overline{f(w)}f(z)}\right| \le \left|\frac{z-w}{1-\bar{w}z}\right| \text{ e } \frac{|f'(z)|}{1-|f(z)|^2} \le \frac{1}{1-|z|^2}.$$
      Inoltre se vale l'uguaglianza nella prima per $z_0, w_0$ con $z_0 \not=w_0$ o nella seconda per $z_0$ allora $f \in \text{\normalfont{Aut}}(\mathbb{D})$ e vale sempre l'uguaglianza.
    \end{itshape}
  \end{block}}
  \only<2>{\begin{defn}
    Data $f \in \text{Hol}(\mathbb{D},\mathbb{D})$, la \textit{derivata iperbolica} è definita come
    $$f^h(w):=\lim_{z \longrightarrow w} \frac{\frac{f(z)-f(w)}{1-\overline{f(w)}f(z)}}{\frac{z-w}{1-\bar{w}z}}=\frac{f'(w)(1-|w|^2)}{1-|f(w)|^2}.$$
  \end{defn}}
  \only<3>{\begin{oss}
    Il lemma di Schwarz-Pick può essere visto come un risultato di rigidità per la derivata iperbolica: se $|f^h(z)|=1+o(1)$ per $z$ che tende a $z_0 \in \mathbb{D}$, allora $f \in \text{Aut}(\mathbb{D})$.
\end{oss}}
\end{frame}
