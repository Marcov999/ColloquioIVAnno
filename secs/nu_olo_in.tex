Vediamo ora qualche risultato interessante.

\begin{thm}
  (Primo teorema di Hurwitz) $\Omega \subseteq \mathbb{C}$ aperto, $\{f_n\} \subset \mathcal{O}(\Omega)$ convergente a $f\in \mathcal{O}(\Omega)$ uniformemente sui compatti. Supponiamo che $f$ non sia costante sulle componenti connesse di $\Omega$.
  Allora per ogni $z_0 \in \Omega$ esistono $n_1=n_1(z_0) \in \mathbb{N}$ e $z_n \in \Omega$ per ogni $n \ge n_1$ t.c. $f_n(z_n)=f(z_0)$ e $\displaystyle \lim_{n \longrightarrow +\infty} z_n=z_0$.
  Senza la tesi sul limite di $z_n$, si può dire che per ogni $w=f(z_0) \in f(\Omega)$ esiste $n_1=n_1(w)$ t.c. $w \in f_n(\Omega)$ per ogni $n \ge n_1$.
\end{thm}

\begin{proof}
  Vogliamo applicare Rocuhé a $f_n-w$ e $f-w$, $w=f(z_0)$ in dischetti centrati in $z_0$ di raggio arbitrariamente piccolo. $f$ non costante sulle componenti connesse $\implies$ $f^{-1}(w)$ è discreto $\implies$ esiste $\delta>0$ t.c. $0<|z-z_0| \le \delta$ $\implies$ $z \in \Omega$ e $f(z) \not=w$.
  Se $D=D(z_0, \delta)$ allora $\overline{D} \cap f^{-1}(w)=\{z_0\}$. Per ogni $k>0$, $\gamma_k=\partial D(z_0, \delta/k)$. Poniamo $\delta_k=\min\{|f(\zeta)-w| \mid \zeta \in \gamma_k\}>0$.
  Esiste $n_k \ge 1$ t.c. per ogni $n \ge n_k$ $\displaystyle \max_{\zeta \in \gamma_k} |f_n(\zeta)-f(\zeta)|<\frac{\delta_k}{2}$ ($f_n$ converge a $f$ uniformemente sui compatti). Possiamo supporre $n_1<n_2<n_3<\dots$.
  Fissato $k \ge 1$, se $n \ge n_k$ e $\zeta \in \gamma_k$, $|(f_n(\zeta)-w)-(f(\zeta)-w)|=|f_n(\zeta)-f(\zeta)|<\dfrac{\delta_k}{2}<\delta_k \le |f(\zeta)-w|$.
  Per il teorema di Rocuhé applicato a $f_n-w$ e $f-w$ in $\overline{D(z_0, \delta/k)}$, per ogni $n \ge n_k$ $f_n-w$ ha almeno uno zero in $D(z_0, \delta_k)$ $\implies$ esiste $z_n \in D(z_0, \delta/k)$ t.c. $f_n(z_n)=w$. $z_n \longrightarrow z_0$ per $n \longrightarrow +\infty$.
\end{proof}

\begin{cor}
  (Secondo teorema di Hurwitz) $\Omega \subseteq \mathbb{C}$ dominio, $\{f_n\} \subset \mathcal{O}(\Omega)$ t.c. $f_n \longrightarrow f \in \mathcal{O}(\Omega)$. Supponiamo che le $f_n$ non si annullino mai (o, in generale, esiste $w_0 \in \mathbb{C}$ t.c. $w_0 \not\in f_n(\Omega)$ per ogni $n$),
  allora o $f \equiv 0$ o $f$ non si annulla mai (in generale, o $f \equiv w_0$ o $w_0 \not\in f(\Omega)$).
\end{cor}

\begin{proof}
  Per assurdo, $w_0 \in f(\Omega)$. Allora o $f$ è costante ($f \equiv w_0$) oppure, per il primo teorema di Hurwitz, $w_0 \in f_n(\Omega)$ per ogni $n>>1$, assurdo.
\end{proof}

\begin{cor}
  (Terzo teorema di Hurwitz) $\Omega \subseteq \mathbb{C}$ dominio, $\{f_n\} \subset \mathcal{O}(\Omega)$ t.c. $f_n \longrightarrow f \in \mathcal{O}(\Omega)$. Supponiamo che le $f_n$ siano iniettive. Allora $f$ è costante o iniettiva.
\end{cor}

\begin{proof}
  Per assurdo, sia $f$ né costante né iniettiva. Allora esistono $z_1 \not=z_2$ t.c. $f(z_1)=f(z_2)$. Poniamo $h_n(z)=f_n(z)-f_n(z_2)$ e $h(z)=f(z)-f(z_2)$. $h_n \longrightarrow h$ e le $h_n$ non si annullano mai in $\Omega \setminus \{z_2\}$ (perché le $f_n$ sono iniettive). Dato che per ipotesi $f$ non è costante, pure $h$ non è costante, dunque per il secondo teorema di Hurwitz non si annulla mai in $\Omega \setminus \{z_2\}$, ma $h(z_1)=0$, assurdo.
\end{proof}
