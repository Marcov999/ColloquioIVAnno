\begin{defn}
  Sia $\Omega \subseteq \mathbb{C}^n$ aperto, $u:\Omega \longrightarrow \mathbb{R}\cup\{-\infty\}$ scs è \textsc{plurisubarmonica} se per ogni $z \in \Omega$ e per ogni $v \in \mathbb{C}^n$ l'applicazione $\zeta \longmapsto u(z+\zeta v)$ è subarmonica dove definita. Scriviamo $u \in PSH(\Omega)$.
\end{defn}

\begin{prop}
  $u \in C^2(\Omega)$ è plurisubarmonica $\iff$ per ogni $z \in \Omega$ e per ogni $v \in \mathbb{C}^n$ vale $\displaystyle \sum_{j,k=1}^n \frac{\partial^2 u}{\partial z_j\partial\bar{z}_k}(z)v_j\bar{v}_k \ge 0$.
\end{prop}

\begin{proof}
  Poniamo $v(\zeta)=u(z+\zeta v)$. $\displaystyle \frac{\partial v}{\partial\zeta}(\zeta)=\sum_{j=1}^n \frac{\partial u}{\partial z_j}(z+\zeta v) \cdot \frac{\partial}{\partial\zeta}(z_j+\zeta v_j)+\sum_{j=1}^n\frac{\partial u}{\partial\bar{z}_j}(z+\zeta v)\cdot \frac{\partial}{\partial\zeta}(\bar{z_j}+\bar{\zeta}\bar{v}_j)$.
  Osserviamo che $\dfrac{\partial}{\partial\zeta}(\bar{z_j}+\bar{\zeta}\bar{v}_j)=0$, perciò otteniamo $\displaystyle \Delta v(\zeta)=4\frac{\partial^2}{\partial\zeta\partial\bar{\zeta}}(\zeta)=4\sum_{j,k=1}^n \frac{\partial^2 u}{\partial z_j\partial\bar{z}_k}(z+\zeta v)v_j\bar{v}_k$.
\end{proof}

\begin{defn}
  Sia $u \in C^2(\Omega)$. La \textsc{forma di Levi di $u$ in $z \in \Omega$} è $L_{u,z}=\left(\frac{\partial^2 u}{\partial z_j\partial\bar{z}_k}(z)\right)$ (è una matrice hermitiana).
\end{defn}

\begin{oss}
  $u \in PSH(\Omega) \cap C^2(\Omega) \iff L_{u,z} \ge 0$ (cioè è semidefinita positiva) per ogni $z \in \Omega$.
\end{oss}

\begin{defn}
  $u \in C^2(\Omega)$ è \textit{strettamente plurisubarmonica} se $L_{u,z}>0$ per ogni $z \in \Omega$.
\end{defn}

\begin{oss}
  Se $\rho \in C^0(\mathbb{C}^n,\mathbb{R})$ allora $\Omega=\{z \in \mathbb{C}^n \mid \rho(z)<0\}$ è un aperto.
\end{oss}

\begin{defn}
  Un \textsc{dominio di classe $C^k$} (o \textit{con bordo di classe $C^k$}), $k \in \mathbb{N}^*\cup\{\infty, \omega\}$ è $\Omega=\{z \in \mathbb{C}^n \mid \rho(z)<0\}$ con $\rho \in C^k(\mathbb{C}^n)$ (con $C^{\omega}$ si intendono le funzioni analitiche), detta \textit{funzione di definizione},
  t.c. $\vec{\nabla}\rho$ non si annulla mai su $\partial\Omega=\{z \in \mathbb{C}^n \mid \rho(z)=0\}$ ($\implies$ $\partial\Omega$ è una ipersuperficie reale di classe $C^k$).
\end{defn}

\begin{ex}
  $\mathbb{B}^n=\{\|z\|^2-1<0\}$, $\rho(z)=\|z\|^2-1$.
\end{ex}

\begin{oss}
  Sia $\Omega$ di classe $C^k$, $x_0 \in \partial\Omega$. Lo spazio tangente reale a $\partial\Omega$ in $x_0$, detto $T_{x_0}^{\mathbb{R}}\partial\Omega$, è l'ortogonale di $\vec{\nabla}\rho(x_0)$ rispetto al prodotto scalare canonico di $\mathbb{R}^{2n}$.
\end{oss}

\begin{exc}
  $\displaystyle T_{x_0}^{\mathbb{R}}\partial\Omega=\left\{v \in \mathbb{C}^n \mid \mathfrak{Re}\left(\sum_{j=1}^n \frac{\partial\rho}{\partial z_j}(x_0)v_j\right)=0\right\}$.
\end{exc}

\begin{defn}
  Sia $\Omega$ di classe $C^k$, $x_0 \in \partial\Omega$. Lo \textit{spazio tangente complesso a $\partial\Omega$ in $x_0$} è $\displaystyle T_{x_0}^{\mathbb{C}}\partial\Omega=\left\{v \in \mathbb{C}^n \mid \sum_{j=1}^n \frac{\partial\rho}{\partial z_j}(x_0)v_j=0\right\}$ sottospazio complesso di $\mathbb{C}^n$ di dimensione $\dim=n-1$.
\end{defn}

\begin{defn}
  Sia $\Omega=\{\rho<0\}$ un dominio di classe $C^2$. Diremo che $\Omega$ è \textsc{(Levi) pseudoconvesso} se $L_{\rho,x} \ge 0$ su $T_x^{\mathbb{C}}\partial\Omega$ per ogni $x \in \partial\Omega$; diremo che è \textit{strettamente pseudoconvesso} se $L_{\rho,x} > 0$ su $T_x^{\mathbb{C}}\partial\Omega$ per ogni $x \in \partial\Omega$.
\end{defn}

\begin{oss}
  Se $\rho_1$ e $\rho_2$ sono funzioni di definizione di $\Omega=\{\rho_1<0\}=\{\rho_2<0\}$ allora $\{\rho_1=0\}=\{\rho_2=0\}$ e questo implica (esercizio) che esiste $h>0$ t.c. $\rho_2=h\rho_1$ vicino a $\partial\Omega$ $\implies$
  $\dfrac{\partial\rho_2}{\partial z_j}=\dfrac{\partial h}{\partial z_j}\rho_1+h\dfrac{\partial\rho_1}{\partial z_j} \implies \dfrac{\partial^2\rho_2}{\partial z_j\partial\bar{z}_k}=\dfrac{\partial^2h}{\partial z_j\partial\bar{z}_k}\rho_1+\dfrac{\partial h}{\partial z_j}\dfrac{\partial\rho_1}{\partial\bar{z}_k}+\dfrac{\partial h}{\partial\bar{z}_k}\dfrac{\partial\rho_1}{\partial z_j}+h\dfrac{\partial^2\rho_1}{\partial z_j\partial\bar{z}_k}$.
  Se $x_0 \in \partial\Omega$, $\rho_1(x_0)=0$. Se $v \in T_{x_0}^{\mathbb{C}}\partial\Omega$ allora $\displaystyle \sum_j \frac{\partial\rho_j}{\partial z_j}(x_0)v_j=0=\overline{\sum_k\frac{\partial\rho_j}{\partial z_h}(x_0)v_k}=\sum_k\frac{\partial\rho_j}{\partial\bar{z}_k}(x_0)\bar{v}_k$.
  Se $x_0 \in \partial\Omega, v \in T_{x_0}^{\mathbb{C}}\partial\Omega$, allora $\displaystyle \sum_{j,k=1}^n \frac{\partial^2\rho_2}{\partial z_j\partial\bar{z}_k}(x_0)v_j\bar{v}_k=h(x_0)\sum_{j,k=1}^n \frac{\partial^2\rho_1}{\partial z_j\partial\bar{z}_k}(x_0)v_j\bar{v}_k$.
  Quindi la definizione di Levi pseudoconvesso non dipende dalla funzione di definizione.
\end{oss}

\begin{ftt}
  Se $\Omega$ è di classe $C^2$ allora $\delta_{\Omega}(z)=\begin{cases} -d(z,\partial\Omega) & \mbox{se }z \in \Omega \\ d(z, \partial\Omega) & \mbox{se }z \not\in \Omega \end{cases}$ è di classe $C^2$ in un intorno di $\partial\Omega$ con $\vec{\nabla}\not=0$ su $\partial\Omega$.
\end{ftt}

\begin{ftt}
  Se $\Omega$ è strettamente pseudoconvesso allora esiste una funzione di definizione $\rho$ di $\Omega$ t.c. $L_{\rho,x}>0$ su tutto $\mathbb{C}^n$ per ogni $x \in \partial\Omega$.
\end{ftt}

\begin{ftt}
  (Narasimhan) Sia $\Omega$ un dominio $C^2$ strettamente pseudoconvesso e $x_0 \in \partial\Omega$. Allora esistono $U \ni x_0$ intorno e $\varphi:U \longrightarrow \mathbb{C}^n$ biolomorfismo con l'immagine t.c. $\varphi(U \cap \Omega)$ è strettamente convesso.
\end{ftt}

\begin{ex}
  $\Omega=\{(z,w) \in \mathbb{C}^2 \mid |z|^2+|w|^2+|w|^{-2}<3\}$ è strettamente pseudoconvesso ma topologicamente è isomorfo a $\mathbb{D} \times \text{Anello}$.
\end{ex}

\begin{defn}
  $\Omega \subseteq \mathbb{C}^n$ dominio è \textsc{Hartogs pseudoconvesso} se $-\log{\mu_{\Omega}} \in PSH(\Omega)$ per qualche $\mu$ funzionale di Minkowski.
\end{defn}

\begin{thm}
  Sia $\Omega \subset\subset \mathbb{C}^n$ di classe $C^2$. Allora sono equivalenti:
  \begin{nlist}
    \item $\Omega$ è Levi pseudoconvesso;
    \item $\Omega$ è Hartogs pseudoconvesso;
    \item $\Omega$ è $PSH(\Omega)$-convesso (cioè l'inviluppo olomorfo di un compatto in $\Omega$ rispetto a $PSH(\Omega)$ è ancora compatto in $\Omega$).
  \end{nlist}
  Inoltre, l'equivalenze tra (ii) e (iii) vale per $\Omega$ dominio qualsiasi.
\end{thm}

\begin{cor} \label{domolo->pscvx}
  Dominio di olomorfia $\implies$ pseudoconvesso.
\end{cor}

\begin{proof}
  (Del corollario) Se $f \in \mathcal{O}(\Omega)$, allora $|f| \in PSH(\Omega) \implies \widehat{K}_{PSH(\Omega)} \subseteq \widehat{K}_{\Omega}$.
\end{proof}

Il corollario \ref{domolo->pscvx} spiega l'importanza dei domini pseudoconvessi e il motivo per cui li stiamo studiando, inoltre fa sorgere un dubbio abbastanza importante da avere un nome suo, il cosiddetto \textit{problema di Levi}: pseudoconvesso $\implies$ dominio di olomorfia?

\begin{defn}
  Un \textsc{disco analitico} è $\varphi:\mathbb{D} \longrightarrow \mathbb{C}^n$ olomorfa. Un disco analitico è \textit{chiuso} se $\varphi$ si estende con continuità a $\partial\mathbb{D}$. Se $\varphi:\overline{\mathbb{D}} \longrightarrow \mathbb{C}^n$ è un disco analitico chiuso porremo $d=\varphi(\overline{\mathbb{D}}), \partial d=\varphi(\partial\mathbb{D})$ e $\mathop {d}\limits^ \circ=\varphi(\mathbb{D})$.
\end{defn}

\begin{lm}
  Se $d \subset \Omega$ è un disco analitico chiuso allora $d \subseteq \widehat{\partial d}_{\Omega}$.
\end{lm}

\begin{proof}
  Se $f \in \mathcal{O}(\Omega)$ e $d=\varphi(\overline{\mathbb{D}})$ allora $f \circ \varphi \in \mathcal{O}(\mathbb{D}) \cap C^0(\overline{\mathbb{D}}) \implies$ (principio del massimo) per ogni $\zeta \in \overline{\mathbb{D}}$ vale $\displaystyle |(f\circ \varphi)(\zeta)| \le \max_{\zeta \in \partial\mathbb{D}} |f(\varphi(\zeta))| \implies$
  per ogni $\zeta \in \overline{\mathbb{D}}$ vale $\varphi(\zeta) \in \widehat{\partial d}_{\Omega}$.
\end{proof}

\begin{lm} \label{uphish}
  Se $u \in PSH(\Omega)$ e $\varphi:\mathbb{D} \longrightarrow \Omega$ è olomorfa, allora $u \circ \varphi \in SH(\mathbb{D})$.
\end{lm}

\begin{proof}
  (Idea) Supponiamo $u \in PSH(\Omega)\cap C^2(\Omega)$.
  Allora $\displaystyle \frac{\partial(u\circ\varphi)}{\partial\zeta}=\sum_h \left(\frac{\partial u}{\partial z_h}\right)\cdot\frac{\partial\varphi_h}{\partial\zeta}, \dfrac{\partial^2(u\circ\varphi)}{\partial\zeta\partial\bar{\zeta}}=\sum_{h,k} \left(\frac{\partial^2 u}{\partial z_h\partial\bar{z}_k}\circ\varphi\right)\frac{\partial\varphi_h}{\partial\zeta}\overline{\left(\frac{\partial\varphi_h}{\partial\zeta}\right)} \ge 0$ perché $u \in PSH(\Omega)$.
  \begin{ftt}
    $u \in PSH(\Omega)$ $\implies$ esistono $u_j \in PSH(\Omega) \cap C^{\infty}(\Omega)$ t.c. $u_j \downarrow u$ per $j \longrightarrow+\infty$.
  \end{ftt}

  Se $u \in PSH(\Omega)$, possiamo usare questo fatto e la prima parte della dimostrazione per dire che $u_j \circ \varphi \in SH(\mathbb{D})$ con $u_j \circ \varphi \downarrow u \circ \varphi \implies u \circ \varphi \in SH(\mathbb{D})$.
\end{proof}

\begin{thm}
  Sia $\Omega \subseteq \mathbb{C}^n$ un dominio. Sono equivalenti:
  \begin{nlist}
    \item per ogni famiglia $\{d_{\alpha}\}$ di dischi analitici chiusi contenuti in $\Omega$ se $\displaystyle \bigcup_{\alpha} \partial d_{\alpha} \subset\subset \Omega$ allora $\displaystyle \bigcup_{\alpha} d_{\alpha} \subset\subset \Omega$ (\textit{Kontinuitätsatz});
    \item per ogni $\mu$ funzionale di Minkowski e per ogni $d$ disco analitico chiuso in $\Omega$ vale che $\mu_{\Omega}(\partial d)=\mu_{\Omega}(d)$;
    \item esiste $\mu$ funzionale di Minkowski t.c. per ogni $d$ disco analitico chiuso in $\Omega$ vale che $\mu_{\Omega}(\partial d)=\mu_{\Omega}(d)$;
    \item esiste $\Phi \in PSH(\Omega) \cap C^0(\Omega)$ t.c. per ogni $c \in \mathbb{R}$ si ha che $\Omega_c=\{z \in \Omega \mid \Phi(z)<c\} \subset\subset \Omega$
    ($\Phi \in C^0(\Omega)$ t.c. $\Omega_c \subset\subset \Omega$ per ogni $c \in \mathbb{R}$ è detta \textit{esaustione}; quella che abbiamo appena definito al punto (iv) è detta \textit{esaustione plurisubarmonica});
    \item esiste un'esaustione $C^{\infty}$ strettamente plurisubarmonica;
    \item $\Omega$ è Hartogs pseudoconvesso: esiste $\mu$ funzionale di Minkowski t.c. $-\log{\mu_{\Omega}} \in PSH(\Omega)$;
    \item per ogni $\mu$ funzionale di Minkowski $-\log{\mu_{\Omega}} \in PSH(\Omega)$;
    \item (solo se $\Omega \subset\subset \mathbb{C}^n$ è di classe $C^2$) $\Omega$ è Levi pseudoconvesso;
    \item $\Omega$ ammette un'esaustione con sottodomini Hartogs pseudoconvessi, cioè esistono $\{\Omega_j\}$ dominio Hartogs pseudoconvessi t.c. $\Omega_j \subset \Omega_{j+1}$, $\displaystyle \Omega=\bigcup_j \Omega_j$ e $\Omega_j \subset\subset \mathbb{C}^n$;
    \item $\Omega$ ammette un'esaustione con sottodomini di classe $C^{\infty}$ strettamente pseudoconvessi;
    \item $\Omega$ è $PSH(\Omega)$-convesso.
  \end{nlist}
\end{thm}

\begin{proof}
  (v) $\implies$ (10) Sia $\Phi$ un'esaustione $C^{\infty}$ strettamente plurisubarmonica e $\Omega_j=\{\Phi<j\}, j \in \mathbb{N}$, allora gli $\Omega_j$ sono strettamente pseudoconvessi e sono un'esaustione di $\Omega$. Per averli $C^{\infty}$ si sfrutta il teorema di Sard: se $\Phi \in C^{\infty}(\Omega)$ con $\Omega \subseteq \mathbb{R}^n$ dominio, allora l'insieme dei valori critici di $\Phi$ ha misura zero in $\mathbb{R}$;
  a questo punto basta prendere dei valori che si scostano leggermente da $j$.

  (x) $\implies$ (ix) Segue da strettamente pseudoconvesso $\implies$ Hartogs pseudoconvesso che seguirà da (viii) $\implies$ (iv) $\implies$ (v) $\implies$ (xi) $\implies$ (i) $\implies$ (ii) $\implies$ (vii) $\implies$ (vi).

  (ix) $\implies$ (vi) Sia $\mu=\|\cdot\|$. Da (vi) $\implies$ (iv) $\implies$ (v) $\implies$ (xi) $\implies$ (i) $\implies$ (ii) $\implies$ (vii) abbiamo che $-\log{\mu_{\Omega_j}} \in PSH(\Omega_j)$ per ogni $j$.
  $\Omega_j \subset \Omega_{j+1} \implies \mu_{\Omega_j} \le \mu_{\Omega_{j+1}} \implies -\log{\mu_{\Omega_j}} \ge -\log{\mu_{\Omega_{j+1}}} \ge \dots \ge -\log{\mu_{\Omega}} \implies -\log{\mu_{\Omega_j}}$
  è una successione di funzioni plurisubarmoniche che decresce verso $-\log{\mu_{\Omega}}$ (che il limite sia proprio quello segue dal fatto che $\Omega_j$ è un'esaustione) $\implies -\log{\mu_{\Omega}} \in PSH$ per il teorema \ref{car_subarmo}.

  (vi) $\implies$ (iv) Sia $\Phi(z)=-\log{\mu_{\Omega}(z)}+\|z\|^2$. $\Phi \in PSH(\Omega)\cap C^0(\Omega)$ ed è un'esaustione perché $\Phi \ge -\log{\mu_{\Omega}(z)} \implies \mu_{\Omega} \ge e^{-\Phi} \implies \overline{\{\Phi<j\}}=\overline{\{\mu_{\Omega} \ge e^{-j}>0\}} \subset \Omega$
  e il termine $\|z\|^2$ ci dà la limitatezza, perciò $\{\Phi<j\} \subset\subset \mathbb{C}^n$.

  (iv) $\implies$ (v) Non svolta in quanto tecnica.

  (i) $\implies$ (ii) Per assurdo esiste $d$ t.c. $\mu_{\Omega}\left(\mathop {d}\limits^ \circ\right)<\mu_{\Omega}(\partial d)$. Sia $p_0 \in \mathop {d}\limits^ \circ$ t.c. $\mu_{\Omega}\left(\mathop {d}\limits^ \circ\right)=\mu_{\omega}(d)=\mu_{\Omega}(p_0)$.
  Sia $x_0 \in \partial\Omega$ t.c. $\mu_{\Omega}(p_0)=\mu(p_0-x_0)$. Poniamo $d_j=d+(1-1/j)(x_0-p_0)$ (sono traslazioni del disco analitico $d$).
  Abbiamo $\displaystyle \bigcup_j \partial d_j \subset\subset \Omega$ perché $\mu_{\Omega}(\partial d)>\mu(x_0-p_0)$, ma $\displaystyle \bigcup_j d_j \supseteq \{p_0+(1-1/j)(x_0-p_0)\}$ e $p_0+(1-1/j)(x_0-p_0) \longrightarrow x_0 \in \partial\Omega$, dunque $\displaystyle \bigcup_j d_j$ non è relativamente compatto dentro $\Omega$ contro (i), assurdo.

  (ii) $\implies$ (iii) è ovvio.

  (iii) $\implies$ (i) Per assurdo, se (i) fosse falsa esisterebbe $d_j$ con $\mu_{\Omega}(\partial d_j) \ge \delta_0>0$ ma $\mu_{\Omega}(d_j) \longrightarrow 0$ contro (iii), assurdo.

  (ii) $\implies$ (vii) Fissiamo $\mu$ funzionale di Minkowski, $z_0 \in \Omega$, $v_0 \in \mathbb{C}^n$. Vogliamo $\psi(\zeta)=-\log{\mu_{\Omega}(z_0+\zeta v_0)}$ subarmonica. Prendendo $\|v_0\| << 1$, possiamo supporre $\psi \in C^0(\overline{\mathbb{D}})$.
  Per l'arbitratietà di $z_0$ e $v_0$ basta dimostrare che $\displaystyle \psi(0) \le \frac{1}{2\pi} \int_0^{2\pi} \psi(e^{i\theta})\diff\theta$ e poi concludere con il teorema \ref{car_subarmo}.
  Sia $h \in \mathcal{H}(\mathbb{D}) \cap C^0(\overline{D})$ l'estensione armonica di $\psi$ e sia $f \in \mathcal{O}(\Omega) \cap C^0(\overline{D})$ t.c. $h=\mathfrak{Re}f$.
  Fissiamo $\epsilon>0$ e poniamo $f_{\epsilon}=f+\epsilon/2, h_{\epsilon}=\mathfrak{Re}f_{\epsilon} \implies \psi<h_{\epsilon}<\psi+\epsilon$ su $\partial\mathbb{D}$. Sia $v \in \mathbb{C}^n$ con $\mu(v)=1$ e sia $d$ il disco analitico dato dall'applicazione $\zeta \mapsto z_0+\zeta v_0+e^{-f_{\epsilon}(\zeta)}v$.
  Vogliamo $d \subset \Omega$. Su $\partial d$ abbiamo che $\mu((z_0+\zeta v_0)-(z_0+\zeta v_0+e^{-f_{\epsilon}(\zeta)}v))=\mu(e^{-f_{\epsilon}(\zeta)}v)=|e^{-f_{\epsilon}(\zeta)}|\mu(v)=e^{-h_{\epsilon}(\zeta)}$.
  Dato che siamo su $\partial d$, si ha $\zeta \in \partial\mathbb{D}$, perciò la quantità appena trovata è strettamente minore di $e^{-\psi(\zeta)}=\mu_{\Omega}(z_0+\zeta v_0) \implies z_0+\zeta v_0+e^{-f_{\epsilon}(\zeta)}v \in \Omega \implies \partial d \subset \Omega$ e questo per (ii) ci dà $d \subset \Omega$ (di questo non sono sicuro perché mi pare che serva $d \subset \Omega$ per usare (ii); purtroppo a quella lezione non c'ero, cercherò di rimediare).
  In particolare $z_0+e^{-f_{\epsilon}(0)}v \in \Omega$ per ogni $v$ t.c. $\mu(v)=1$ $\implies$ $\mu_{\Omega}(z_0) \ge |e^{-f_{\epsilon}(0)}|=e^{-h_{\epsilon}(0)} \implies \psi(0)=-\log{\mu_{\Omega}(z_0)} \le h_{\epsilon}(0)=\displaystyle \frac{1}{2\pi} \int_0^{2\pi} h_{\epsilon}(e^{i\theta})\diff\theta \le \frac{1}{2\pi} \int_0^{2\pi} \psi(e^{i\theta})\diff\theta+\epsilon$.
  A questo punto basta mandare $\epsilon$ a $0$ per ottenere quanto voluto.

  (vii) $\implies$ (vi) è ovvio.

  (v) $\implies$ (xi) Sia $K \subset\subset \Omega$ compatto, $\Phi \in C^{\infty}$ esaustione strettamente plurisubarmonica. Esiste $c>0$ t.c. $K \subset \Omega_c=\{\Phi<c\} \implies \widehat{K}_{PSH(\Omega)} \subset \Omega_c \subset\subset \Omega$.

  (xi) $\implies$ (i) Sia $d=\varphi(\overline{\mathbb{D}}) \subset \Omega$ disco analitico chiuso, $u \in PSH(\Omega)$.
  Per il lemma \ref{uphish} $u\circ\varphi \in SH(\mathbb{D})$ $\implies$ per ogni $\zeta \in \overline{\mathbb{D}}$ si ha $\displaystyle |u\circ\varphi(\zeta)| \le \max_{\zeta \in \partial\mathbb{D}}|(u\circ\varphi)(\zeta)| \implies d \subseteq \widehat{\partial d}_{PSH(\Omega)} \implies \bigcup_{\alpha} d_{\alpha} \subseteq \bigcup_{\alpha} \widehat{(\partial d_{\alpha})}_{PSH(\Omega)} \subseteq \widehat{\left(\bigcup_{\alpha} \partial d_{\alpha}\right)}_{PSH(\Omega)} \subset\subset \Omega$ dove l'ultimo contenimento segue da (xi).

  (vii) $\implies$ (viii) Sia $\mu=\|\cdot\|$. $\mu_{\Omega}=d(\cdot,\partial\Omega)$ è di classe $C^2$ vicino a $\partial\Omega$. Inoltre $\rho(z)=\begin{cases}
    -\mu_{\Omega}(z) & \mbox{se }z \in \overline{\Omega} \\
    d(z,\partial\Omega) & \mbox{se } z \not\in \Omega
  \end{cases}$ è una funzione di definizione per $\Omega$. Per (vii) $\varphi=-\log{\mu_{\Omega}} \in PSH(\Omega)$.
  $\displaystyle L_{\varphi,z}(v)=\sum_{j,k} \left(-\frac{1}{\mu_{\Omega}(z)}\frac{\partial^2 \mu_{\Omega}}{\partial z_j\partial\bar{z}_k}(z)+\frac{1}{\mu_{\Omega}(z)^2}\frac{\partial\mu_{\Omega}}{\partial z_j}(z)\frac{\partial\mu_{\Omega}}{\partial\bar{z}_k}(z)\right)v_j\bar{v}_k \ge 0$ se $v \in \mathbb{C}^n$ e $z \in \Omega$ abbastanza vicino a $\partial\Omega$ in modo che $\mu_{\Omega}$ sia $C^2$ in $z$.
  Moltiplichiamo per $\mu_{\Omega}$ e restringiamoci a $\displaystyle \left\{v \mid \sum_j \frac{\partial \mu_{\Omega}}{\partial z_j}(z)v_j=0\right\}$. Per questi $v$ abbiamo $\displaystyle \sum_{j,k} \frac{\partial^2(-\mu_{\Omega})}{\partial z_j\partial\bar{z}_k}(z)v_j\bar{v}_k \ge 0$.
  Mandando $z \longrightarrow x \in \partial\Omega$ allora $T_z^{\mathbb{C}}(\partial\Omega) \longrightarrow T_x^{\mathbb{C}}(\partial\Omega)$ e $\displaystyle \sum_{j,k} \frac{\partial^2\rho}{\partial z_j\partial\bar{z}_k}(z)v_j\bar{v}_k \ge 0$ per ogni $v \in T_x^{\mathbb{C}}(\partial\Omega)$.

  (viii) $\implies$ (iv) Sia $\mu=\|\cdot\|$, $\rho=-\mu_{\Omega}$ la funzione di definizione e $u=-\log{\mu_{\Omega}}$. Essendo $\Omega$ limitato, $u$ è un'esaustione. Per assurdo, supponiamo che $u$ non sia plurisubarmonica. Possiamo assumere che non lo sia vicino a $\partial\Omega$, dove è $C^2$. Deve esistere $z_0 \in \Omega$ t.c. $L_{u,z}$ non sia semidefinita positiva, cioè esiste $v_0 \in \mathbb{C}^n$ t.c.
  $\displaystyle 0>L_{u,z}(v)=\sum_{j,k} \frac{\partial^2 u}{\partial z_j\partial\bar{z}_k}(z)v_{0j}\bar{v}_{0k}=\frac{\partial^2}{\partial\zeta\partial\bar{\zeta}}(-\log{\mu_{\Omega}(z_0+\zeta v_0)})\restrict{\zeta=0}=-\lambda, \lambda>0$.
  Poniamo $\varphi(\zeta)=\log{\mu_{\Omega}}(z_0+\zeta v_0)$. Lo sviluppo di Taylor in $\zeta=0$ è il seguente:
  $\displaystyle \log{\mu_{\Omega}(z_0+\zeta v_0)}=\varphi(\zeta)=\varphi(0)+\mathfrak{Re}\left(\frac{\partial\varphi}{\partial\zeta}(0)\zeta+\frac{1}{2}\frac{\partial^2\varphi}{\partial\zeta^2}(0)\zeta^2\right)+\frac{\partial^2\varphi}{\partial\zeta\partial\bar{\zeta}}(0)|\zeta|^2+o(|\zeta|^2)=\log{\mu_{\Omega}(z_0)}+\mathfrak{Re}(A\zeta+B\zeta^2)+\lambda|\zeta|^2+o(|\zeta|^2)$.
  Sia $x_0 \in \partial\Omega$ t.c. $\mu_{\Omega}(z_0)=\|x_0-z_0\|$ e poniamo $w_0=x_0-z_0$; $z_0+w_0 \in \partial\Omega$ e $\|w_0\|=\mu_{\Omega}(z_0)$. Poniamo $\psi)\zeta)=z_0+\zeta v_0+\exp(A\zeta+B\zeta^2)$.
  $\psi(0)=z_0+w_0 \in \partial\Omega$, ma $\mu_{\Omega}(\psi(\zeta)) \ge \mu_{\Omega}(z_0+\zeta v_0)-\|w_0\||\exp(A\zeta+B\zeta^2)|=\mu_{\Omega}(z_0)|\exp(A\zeta+B\zeta^2)|\exp(\lambda|\zeta|^2+o(|\zeta|^2))-\|w_0\||\exp(A\zeta+B\zeta^2)|=\mu_{\Omega}(z_0)|\exp(A\zeta+B\zeta^2)|(\exp(\lambda|\zeta|^2+o(|\zeta|^2))-1)$.
  Se $0\not=|\zeta|<<1$, di modo che $\psi(\zeta) \in \Omega$, abbiamo che la quantità appena trovata è maggiore o ugale di $\mu_{\Omega}(z_0)|\exp(A\zeta+B\zeta^2)|(\exp(\lambda/2|\zeta|^2)-1)>0$.
  In particolare, $\mu_{\Omega}\circ\psi$ ha un minimo locale in $\zeta=0 \implies \displaystyle 0=\frac{\partial(\mu_{\Omega}\circ\psi)}{\partial\zeta}(0)=\sum_j \frac{\partial\mu_{\Omega}}{\partial z_j}(\psi(0))\frac{\partial\psi_j}{\partial\zeta}(0)$
  e dato che $\psi(0)=x_0$ questo ci dice che $\dfrac{\partial\psi}{\partial\zeta}(0) \in T_{x_0}^\mathbb{C}(\partial\Omega)$. Guardiamo ora lo sviluppo di Taylor di $\mu_{\Omega}\circ\psi$:
  $\displaystyle (\mu_{\Omega}\circ\psi)(\zeta)=\mathfrak{Re}\left(\frac{\partial^2(\mu_{\Omega}\circ\psi)}{\partial\zeta^2}(0)\zeta^2\right)+\frac{\partial^2(\mu_{\Omega}\circ\psi)}{\partial\zeta\partial\bar{\zeta}}(0)|\zeta|^2+o(|\zeta|^2)>0$ per $0<|\zeta|<<1$.
  Il termine tra parentesi tonde non ha segno costante e il termine $o$ piccolo è trascurabile per $\zeta$ piccolo, dunque dev'essere $\displaystyle 0<\frac{\partial^2(\mu_{\Omega}\circ\psi)}{\partial\zeta\partial\bar{\zeta}}(0)=\sum_{j,k} \frac{\partial^2 \mu_{\Omega}}{\partial z_j\partial\bar{z}_k}(x_0)\frac{\partial\psi_j}{\partial\zeta}(0)\overline{\left(\frac{\partial\psi_k}{\partial\zeta}\right)}(0)=-L_{-\mu_{\Omega},x_0}\left(\frac{\partial\psi}{\partial\zeta}(0)\right)$
  contro l'ipotesi che $L_{-\mu_{\Omega}, x_0}$ fosse $ \ge 0$ su $T_{x_0}^{\mathbb{C}}(\partial\Omega)$.
\end{proof}

\begin{oss}
  In generale, per vedere se $\Omega$ è pseudoconvesso basta controllare cosa succede ``vicino'' a $\partial\Omega$, cioè in $\Omega \setminus K$ dove $K \subset\subset \Omega$ è un compatto qualsiasi.
\end{oss}

Adesso torniamo ad occuparci del problema di Levi iniziando da un teorema di cui non è riportata la dimostrazione.

\begin{thm}
  (Hörmander) Sia $\Omega \subseteq \mathbb{C}^n$ pseudoconvesso e siano $f_1,\dots,f_n \in C^{\infty}(\Omega)$ con la condizione di compatibilità $\dfrac{\partial f_j}{\partial\bar{z}_k}=\dfrac{\partial f_k}{\partial\bar{z}_j}$ per ogni $j,k$. Poniamo $\displaystyle f=\sum_j f_j\diff\bar{z}_j$ (valgono le condizioni di compatibilità $\iff$ $\bar{\partial}f=0$).
  Allora esiste $u \in C^{\infty}(\Omega)$ unica a meno di funzioni olomorfe t.c. $\bar{\partial}u=f$ ($\iff$ $\dfrac{\partial u}{\partial\bar{z}_j}=f_j$ per ogni $j$).
\end{thm}

\begin{thm} \label{hormanderino}
  Sia $\Omega \subseteq \mathbb{C}^n$ pseudoconvesso. Sia $\Omega_n=\Omega \cap \{z_n=0\}$ e $\tilde{\Omega}_n=\{z' \in \mathbb{C}^{n-1} \mid (z',0) \in \Omega\}$. Sia $f \in \mathcal{O}(\tilde{\Omega}_n)$.
  Allora esiste $F \in \mathcal{O}(\Omega)$ t.c. $F(z',0)=f(z')$ per ogni $z' \in \tilde{\Omega}_n$.
\end{thm}

\begin{proof}
  Sia $\pi:\mathbb{C}^n \longrightarrow \mathbb{C}^{n-1}$ la proiezione $\pi(z',z)=z'$, $z'=(z_1,\dots,z_{n-1})$. Sia $B=\Omega\setminus \pi^{-1}(\tilde{\Omega}_n)=\{z \in \Omega \mid \pi(z) \in \tilde{\Omega}_n\}$ chiuso in $\Omega$, inoltre anche $\Omega_n$ è chiuso in $\Omega$ e $B\cap\Omega_n=\emptyset$.
  Quindi hanno intorni aperti (in $\Omega$) disgiunti; ci basta un intorno di $\Omega_n$ contenuto in $\pi^{-1}(\tilde{\Omega}_n)$ disgiunto da un intorno di $B$. Sia $\Psi \in C^{\infty}(\Omega)$ con $\Psi \equiv 1$ in un intorno di $\Omega_n$ e $\Psi \equiv 0$ su $B$ e $0 \le \Psi \le 1$.
  Poniamo $F(z)=\Psi(z)f(\pi(z))+z_nu(z)$ per qualche $u \in C^{\infty}(\Omega)$. Se $z' \in \tilde{\Omega}_n$, $F(z',0)=1\cdot f(z')+0=f(z')$ $\implies$ $F$ è un'estensione di $f$.
  Vogliamo $u$ in modo che $F$ sia olomorfa $\iff 0=\bar{\partial}F=f(\pi(z))\bar{\partial}\Psi+z_n\bar{\partial}u \iff \bar{\partial}u=-\dfrac{f(\pi(z))\bar{\partial}\Psi}{z_n}$.
  Ma $\bar{\partial}\Psi \equiv 0$ in un intorno di $\Omega_n=\{z_n=0\} \cap \Omega \implies -\dfrac{(f\circ\pi)\bar{\partial}\Psi}{z_n} \in C^{\infty}(\Omega)$ e soddisfa $\bar{\partial}\left(-\dfrac{(f\circ\pi)\bar{\partial}\Psi}{z_n}\right) \equiv 0$ (per Schwarz).
  Hörmander $\implies$ l'esistenza di una $u$ siffatta.
\end{proof}

\begin{cor}
  $\Omega \subseteq \mathbb{C}^n$ pseudoconvesso è un dominio di olomorfia.
\end{cor}

\begin{proof}
  Per induzione su $n$. Per $n=1$ ok (tutti i domini di $\mathbb{C}$ sono di olomorfia). Sia vero per $n-1$, prendiamo $x \in \partial\Omega$; vogliamo $F \in \mathcal{O}(\Omega)$ che non si estende oltre $x$. A meno di traslazione $x=0$. Sia $H \subset \mathbb{C}^n$ iperpiano con $0 \in H$ e t.c. $0 \in \partial(H\cap\Omega)$; a meno di rotazione $H=\{z_n=0\}$.
  Poniamo $\Omega_n=H\cap\Omega$ e $\tilde{\Omega}_n=\{z' \in \mathbb{C}^{n-1} \mid (z',0) \in \Omega\}$. $\tilde{\Omega}_n$ è pseudoconvesso (per esempio perché un'esaustione plurisubarmonica di $\Omega$ fornisce un'esaustione plurisubarmonica di $\tilde{\Omega}_n$). Per ipotesi induttiva è dominio di olomorfia $\implies$ esiste $f \in \mathcal{O}(\tilde{\Omega}_n)$ che non si estende oltre $0' \in \partial\tilde{\Omega}_n$.
  Per il teorema \ref{hormanderino} esiste $F \in \mathcal{O}(\Omega)$ t.c. $F(z',0)=f(z')$ per ogni $z' \in \tilde{\Omega}_n$ e quindi $F$ non si estende olte $0=x$.
\end{proof}
