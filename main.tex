\documentclass{article}
\usepackage{mstyle}
\usepackage{pgfplots}
\usetikzlibrary{intersections, pgfplots.fillbetween}

\title{Teoremi di rigidità per funzioni olomorfe nel disco}
\date{}
\author{Candidato: Marco Vergamini \qquad Relatore: Prof. Marco Abate}

\begin{document}
\maketitle
\newpage
\tableofcontents
\newpage


\section*{Introduzione}
\addcontentsline{toc}{section}{Introduzione}
Da scrivere alla fine


\newpage

\section{Prerequisiti}

\subsection{Lemma di Schwarz-Pick e distanza di Poincaré}
\begin{defn}
  Sia $\Omega \subset \mathbb{C}$ un aperto. Una funzione $f:\Omega \longrightarrow \mathbb{C}$ si dice \textit{olomorfa} in $\Omega$ se è derivabile in senso complesso per ogni $z \in \Omega$.
\end{defn}

\begin{defn}
  Se $f: \Omega \longrightarrow \Omega$ olomorfa è biettiva, allora si può dimostrare che anche $f^{-1}$ è olomorfa. In tal caso $f$ è detta \textit{automorfismo} (in senso olomorfo di $\Omega$).
\end{defn}

Com'è noto, la condizione di olomorfia per funzioni a valori complessi è molto più forte della derivabilità in senso reale (in particolare, è equivalente all'analiticità). Fra i vari risultati che si possono dimostrare per le funzioni olomorfe, ci interessa studiare il lemma di Schwarz-Pick. \\

Notazione: indichiamo il disco unitario con $\mathbb{D}=\{z \in \mathbb{C} \mid |z|<1\}$.

\begin{lm}
  (Schwarz) Sia $f:\mathbb{D} \longrightarrow \mathbb{D}$ una funzione olomorfa t.c. $f(0)=0$. Allora per ogni $z \in \mathbb{D}$ $|f(z)| \le |z|$ e $|f'(0)| \le 1$; inoltre, se vale l'uguale nella prima per $z \not=0$ oppure nella seconda allora $f(z)=e^{i\theta}z, \theta \in \mathbb{R}$.
\end{lm}

\begin{lm}
  (Schwarz-Pick) Sia $f:\mathbb{D} \longrightarrow \mathbb{D}$ una funzione olomorfa.
  Allora per ogni $z, w \in \mathbb{D}$
  $$\left|\frac{f(z)-f(w)}{1-\overline{f(w)}f(z)}\right| \le \left|\frac{z-w}{1-\bar{w}z}\right|, \qquad \frac{|f'(z)|}{1-|f(z)|^2} \le \frac{1}{1-|z|^2}.$$
  Inoltre se vale l'uguale nella prima per $z_0, w_0$ con $z_0 \not=w_0$ o nella seconda per $z_0$ allora $f$ è un automorfismo e vale l'uguale sempre.
\end{lm}

Il lemma di Schwarz-Pick può essere riformulato usando due funzioni di due variabili sul disco, una delle quali è nota come distanza iperbolica (in realtà, anche l'altra è una distanza, ma questo non lo useremo). Con queste funzioni dimostreremo una serie di disuguaglianze che ci permetteranno di dimostrare la disuguaglianza di Golusin, dalla quale seguirà una versione al bordo del lemma.

\begin{defn}
  Dati $z, w \in \mathbb{D}$ poniamo
  $$[z,w]:=\frac{z-w}{1-\bar{w}z}, \qquad p(z,w):=|[z,w]|, \qquad d(z,w):=\log\left(\frac{1+p(z,w)}{1-p(z,w)}\right).$$
\end{defn}

$d$ è ben definita, in quanto $p(z,w)<1$. Infatti, dobbiamo verificare che
  $$\frac{|z-w|}{|1-\bar{w}z|} < 1$$
  $$|z-w|^2 < |1-\bar{w}z|^2$$
  $$|z|^2+|w|^2-\bar{w}-w\bar{z} < 1+|wz|^2-\bar{w}z-w\bar{z}$$
  $$1+|wz|^2-|z|^2-|w|^2 > 0$$
  $$(1-|w|^2)(1-|z|^2) > 0,$$
che è vera perché $z, w \in \mathbb{D}$.

\begin{prop}
  $d$ è una distanza (la sopracitata distanza iperbolica).
\end{prop}

\begin{proof}
  Mostriamo preliminarmente che $p$ è una distanza. In entrambi i casi, l'unica cosa non ovvia da controllare è la disuguaglianza triangolare. Perciò, dati $z_0, z_1, z_2 \in \mathbb{D}$, vogliamo $p(z_1,z_2) \le p(z_1,z_0)+p(z_0,z_2)$. Osserviamo che, per il lemma di Schwarz-Pick, $p$ è invariante applicando automorfismi, perciò supponiamo senza perdita di generalità $z_1=0$ (possiamo farlo perché il gruppo degli automorfismidi $\mathbb{D}$ è transitivo). A questo punto la disuguaglianza da dimostrare diventa
  \marginpar{Come si dimostra? Qui c'è una dim, ma ponendo $z_0=0$: https://mathoverflow.net/questions/21604/nice-proof-of-the-triangle-inequality-for-the-metric-of-the-hyperbolic-plane}
  \begin{align*}
    |z_2| \le |z_0|+\frac{|z_0-z_2|}{|1-\bar{z}_2z_0|}.
  \end{align*}
  (c'è da fare la dimostrazione) \\
  A questo punto, possiamo osservare che $d(z,w) =2\,\text{arctanh}\,(p(z,w))$, perciò... ACHTUNG: LA DIM DEGLI APPUNTI DI ECA SEMBRA ESSERE FALLACE, ARCTANH NON È SUBADDITIVA SUI POSITIVI
  %Altrimenti, si può brutalmente espandere il conto e ricondursi a una disuguaglianza la cui dimostrazione si trova qui: https://www.math3ma.com/blog/the-pseudo-hyperbolic-metric-and-lindelofs-inequality
\end{proof}

\begin{defn}
  Data una funzione $f: \mathbb{D} \longrightarrow \mathbb{D}$, poniamo
  $$f^*(z,w):=\frac{[f(z),f(w)]}{[z,w]}$$
  e
  $$f^h(z):=|f^*(z,z)|:=\left|\lim_{w \longrightarrow z} f^*(z,w)\right|=\left|\lim_{w \longrightarrow z} \frac{[f(z),f(w)]}{[z,w]}\right|=\frac{|f'(z)|(1-|z|^2)}{1-|f(z)|^2}.$$
\end{defn}

\begin{oss}
  \begin{nlist}
    \item la disuguaglianza del lemma di Schwarz-Pick può essere riscritta come $|f^*(z,w)| \le 1$;
    \item  un altro modo di scrivere la disuguaglianza del lemma di Schwarz-Pick è $p(f(z),f(w)) \le p(z,w)$;
    \item $p(0,z)=|z| \implies d(0,z)=d(0,|z|)$;
    \item per definizione, $|f^*(z,w)|=|f^*(w,z)|$ e $f^h(z)$ è reale non negativo.
  \end{nlist}
  Questi risultati verranno usati nelle varie dimostrazioni e verranno esplicitati solo quando ciò che ne segue non è immediato.
\end{oss}

Seguono alcuni risultati noti di analisi complessa noti che verranno usati nelle dimostrazioni.

\begin{thm}
  (formula integrale di Cauchy) Sia $f$ olomorfa sull'aperto $\Omega$ e $D$ un disco chiuso di centro $a$ contenuto in $\Omega$. Allora
  \begin{equation}
    f^{(n)}(a)=\frac{n!}{2\pi i} \int_{\partial D} \frac{f(\zeta)}{(\zeta-a)^{n+1}}\diff\zeta.
  \end{equation}
\end{thm}

\begin{prop}
  Sia $f$ olomorfa sull'aperto $\Omega \setminus\{z_0\}$, con $z_0 \in \Omega$. Allora $f$ si estende a una funzione olomorfa $g$ definita su tutto $\Omega$ se e solo se è limitata in un intorno di $z_0$. In tal caso, $z_0$ è detta \textit{singolarità rimovibile}.
\end{prop}


\subsection{Regioni di Stolz e limiti non tangenziali}
\begin{defn}
  Dati $\alpha \in (0,\pi/2)$ e $\sigma \in \partial\mathbb{D}$, chiamiamo \textit{settore di vertice $\sigma$ e angolo $2\alpha$} l'insieme $S(\sigma,\alpha) \subset \mathbb{D}$ tale che per ogni $z \in S(\sigma,\alpha)$ l'angolo compreso tra la retta congiungente $\sigma$ e $0$ e la retta congiungente $\sigma$ e $z$ è minore di $\alpha$.
  \begin{center}
    \begin{tikzpicture}[line cap=round,line join=round,>=triangle 45,x=2.5cm,y=2.5cm]
      \draw[->,color=black] (-1.13,0) -- (1.15,0);
      \foreach \x in {-1,1}
      \draw[shift={(\x,0)},color=black] (0pt,2pt) -- (0pt,-2pt);
      \draw[->,color=black] (0,-1.11) -- (0,1.12);
      \foreach \y in {-1,1}
      \draw[shift={(0,\y)},color=black] (2pt,0pt) -- (-2pt,0pt);
      \clip(-1.13,-1.11) rectangle (1.15,1.12);
      \draw(0,0) circle (2.5cm);
      \draw[name path=A] (0.49,0.87)-- (1,0);
      \draw[name path=B] (0.49,-0.87)-- (1,0);
      \tikzfillbetween[of=A and B]{blue, opacity=0.25};
      \draw[name path=C,smooth,samples=100,domain=-1:0.491] plot(\x,{sqrt(1-(\x)^2)});
      \draw[name path=D,smooth,samples=100,domain=-1:0.491] plot(\x,{0-sqrt(1-(\x)^2)});
      \tikzfillbetween[of=C and D]{blue, opacity=0.25};
    \end{tikzpicture}

    In blu, $S(1,2\pi/3)$
  \end{center}
\end{defn}

\begin{defn}
  INSERIRE DEFINIZIONE DI LIMITE NON-TANGENZIALE
\end{defn}

La seguente proposizione asserisce che, per $z \longrightarrow 1$ non tangenzialmente, un certo andamento di $f$ può essere tradotto nell'andamento di $f^h$. È questo che ci permetterà di dimostrare il teorema 2.1 di \cite{BK} passando per la versione al bordo del lemma di Schwarz-Pick.

\begin{prop} \label{o^3->o^2}
  Sia $f:\mathbb{D} \longrightarrow \mathbb{D}$ una funzione olomorfa tale che
  \begin{equation} \label{o^3}
    f(z)=1+(z-1)+o((z-1)^3)
  \end{equation}
  per $z \longrightarrow 1$ non tangenzialmente. Allora
  \begin{equation} \label{o^2}
    f^h(z)=1+o((z-1)^2)
  \end{equation}
  per $z \longrightarrow 1$ non tangenzialmente.
\end{prop}

\begin{proof}

  Sia $S$ un settore di vertice $1$ e angolo d'apertura $2\alpha$, e $S'$ uno un po' più grande di vertice $1$ e angolo $2\beta$, $\beta>\alpha$. Per $z \in S$, sia $C(z)$ il cerchio di centro $z$ e raggio $r(z)=\text{dist}(z, \partial S')$ (la distanza di $z$ dal bordo di $S'$). Allora per la formula integrale di Cauchy
  \begin{align*}
    f'(z) & =\frac{1}{2\pi i} \int_{C(z)} \frac{f(w)}{(w-z)^2}\diff w \\
    & =\frac{1}{2\pi i} \int_{C(z)} \frac{w-1+(f(w)-w)}{(w-z)^2}\diff w \\
    & =\frac{1}{2\pi i} \int_{C(z)} \frac{1}{w-z}\diff w+\frac{1}{2\pi i} \int_{C(z)} \frac{z-1+f(w)-w}{(w-z)^2}\diff w \\
    & =1+\frac{1}{2\pi i} \int_{C(z)} \frac{f(w)-w}{(w-z)^2}\diff w=:1+I(z).
  \end{align*}

  \begin{center}
    \definecolor{qqffqq}{rgb}{0,1,0}
    \definecolor{qqqqff}{rgb}{0,0,1}
    \definecolor{uququq}{rgb}{0.25,0.25,0.25}
    \definecolor{ffqqqq}{rgb}{1,0,0}
    \begin{tikzpicture}[line cap=round,line join=round,>=triangle 45,x=3.0cm,y=3.0cm]
      \draw[->,color=black] (-1.12,0) -- (1.2,0);
      \foreach \x in {-1,1}
      \draw[shift={(\x,0)},color=black] (0pt,2pt) -- (0pt,-2pt);
      \draw[->,color=black] (0,-1.11) -- (0,1.11);
      \foreach \y in {-1,1}
      \draw[shift={(0,\y)},color=black] (2pt,0pt) -- (-2pt,0pt);
      \clip(-1.12,-1.11) rectangle (1.2,1.11);
      \draw [shift={(1,0)},color=ffqqqq,fill=ffqqqq,fill opacity=0.1] (0,0) -- (180:0.26) arc (180:244.77:0.26) -- cycle;
      \draw [shift={(1,0)},color=qqqqff,fill=qqqqff,fill opacity=0.1] (0,0) -- (-180:0.14) arc (-180:-127.09:0.14) -- cycle;
      \draw [shift={(1,0)},color=qqffqq,fill=qqffqq,fill opacity=0.1] (0,0) -- (115.23:0.33) arc (115.23:143.3:0.33) -- cycle;
      \draw(0,0) circle (3cm);
      \draw (1,0)-- (0.27,0.96);
      \draw (0.64,0.77)-- (1,0);
      \draw (0.27,-0.96)-- (1,0);
      \draw (1,0)-- (0.64,-0.77);
      \draw [shift={(1,0)},color=ffqqqq] (180:0.26) arc (180:244.77:0.26);
      \draw [shift={(1,0)},color=ffqqqq] (180:0.23) arc (180:244.77:0.23);
      \draw(0.5,0.37) circle (0.888cm);
      \draw (0.5,0.37)-- (0.77,0.49);
      \draw (0.27,0.54)-- (1,0);
      \draw [shift={(1,0)},color=qqffqq] (115.23:0.33) arc (115.23:143.3:0.33);
      \draw [shift={(1,0)},color=qqffqq] (115.23:0.3) arc (115.23:143.3:0.3);
      \draw [shift={(1,0)},color=qqffqq] (115.23:0.28) arc (115.23:143.3:0.28);
      \begin{scriptsize}
        \draw[color=ffqqqq] (0.84,-0.09) node {$\beta$};
        \fill [color=black] (0.5,0.37) circle (1.5pt);
        \draw[color=black] (0.48,0.32) node {$z$};
        \draw[color=black] (0.18,0.14) node {$C(z)$};
        \draw[color=black] (0.54,0.465) node {$r(z)$};
        \fill [color=black] (1,0) circle (1.5pt);
        \draw[color=black] (1.04,0.05) node {$1$};
        \fill [color=black] (0.26,0.545) circle (1.5pt);
        \draw[color=black] (0.20,0.58) node {$A$};
        \fill [color=black] (1,0) circle (1.5pt);
        \draw[color=qqqqff] (0.92,-0.03) node {$\alpha$};
        \draw[color=qqffqq] (0.83,0.16) node {$\gamma$};
      \end{scriptsize}
    \end{tikzpicture}
  \end{center}

  Dato $\epsilon>0$ fissato, per ipotesi esiste $\delta>0$ tale che $|f(w)-w|<\epsilon|1-w|^3$ per ogni $w \in S'$ con $|w-1|<\delta$. $B(z,r(z)) \subset \mathbb{D} \implies r(z) \le 1-|z|$. Se $|z-1|<\delta/2$, $r(z) \le 1-|z|=|z-1-z|-|z| \le |z-1|+|z|-|z|=|z-1|<\delta/2$, dunque per ogni $w \in C(z)$ abbiamo $|w-1| \le |w-z|+|z-1|=r(z)+|z-1|<\delta$. Per questi $z$ vale che
  \begin{align*}
    |I(z)| & \le \frac{\epsilon}{2\pi} \int_0^{2\pi} \frac{|1-(z+r(z)e^{i\theta})|^3}{|(z+r(z)e^{i\theta})-z|^2}r(z)\diff\theta \\
    & \le \frac{\epsilon}{r(z)}\max_{\theta \in [0,2\pi]} |1-(z+r(z)e^{i\theta})|^3 \\
    & =\frac{\epsilon}{r(z)}\max_{w \in C(z)}|1-w|^3.
  \end{align*}
  Il massimo è raggiunto per l'intersezione più lontana da $1$ tra la circonferenza $C(z)$ e la retta passante per $1$ e $z$ (il punto $A$ in figura), perciò, detto $\gamma$ l'angolo tra $\partial S'$ (per essere precisi, la retta contenente il tratto affine più vicino a $z$) e la retta congiungente $1$ e $z$:
  \begin{align*}
    |I(z)| & \le \frac{\epsilon}{r(z)}(r(z)+|z-1|)^3 \\
    & =\epsilon r(z)^2\left(1+\frac{|z-1|}{r(z)}\right)^3 \\
    & =\epsilon r(z)^2(1+\csc\gamma)^3 \\
    & \le \epsilon r(z)^2(1+\csc(\beta-\alpha))^3 \\
    & \le \epsilon |z-1|^2(1+\csc(\beta-\alpha))^3.
  \end{align*}
  La penultima disuguaglianza segue da $\gamma \ge \beta-\alpha$ e dal fatto che $\csc$ è decrescente sui positivi, mentre l'ultima segue da quanto visto sopra. Otteniamo dunque $f'(z)=1+o((z-1)^2)$ per $z \longrightarrow 1$ non tangenzialmente.

  Adesso ci servirà il seguente lemma.

  \begin{lm} \label{opiccoli}
    Per $z \longrightarrow 1$ non tangenzialmente, $|z-1|$ e $1-|z|$ hanno gli stessi $o$-piccoli.
  \end{lm}

  \begin{proof}
    $1-|z| \le |z-1|$ l'abbiamo già visto. Per concludere ci basta dunque mostrare che, per $z$ appartenente a un settore $S$ di angolo $2\alpha$ fissato, vale una disuguaglianza opposta, a meno di una qualche costante.
    $$|z-1|=r(z)\frac{|z-1|}{r(z)} \le r(z)\csc(\beta-\alpha) \le (1-|z|)\csc(\beta-\alpha),$$
    dove $\beta>\alpha$ è stato scelto come sopra e le disuguaglianze le abbiamo già viste.
  \end{proof}

  Per ipotesi
  $$\frac{1-|f(z)|}{1-|z|}=\frac{1-|z|+o((z-1)^3)}{1-|z|}=1+o((z-1)^2)$$
  per $z \longrightarrow 1$ non tangenzialmente (abbiamo usato il lemma \ref{opiccoli} per poter usare indipendentemente $z-1$ o $1-|z|$ negli $o$-piccoli). \\
  Possiamo quindi concludere che
  $$f^h(z)=|f'(z)|\frac{1-|z|^2}{1-|f(z)|^2}=1+o((z-1)^2)$$
  per $z \longrightarrow 1$ non tangenzialmente.
\end{proof}


\newpage

\section{Lemmi di Schwarz-Pick multi-punto}

\subsection{Teorema di Beardon-Minda e corollari}
Adesso possiamo procedere a dimostrare la serie di disuguaglianze di \cite{BM}, che coinvolgono la distanza di Poincaré $\omega$ e le funzioni olomorfe dal disco in sé che non sono automorfismi.

\begin{defn}
  La \textit{rotazione iperbolica} di ordine due attorno a un punto $a \in \mathbb{D}$ è la funzione $r_a \in \text{Aut}(\mathbb{D})$ data da
  $$r_a(z)=-\frac{z-\frac{2a}{1+|a|^2}}{1-\frac{2\bar{a}}{1+|a|^2}z}.$$
  Chiamiamo $a$ il \textit{centro di rotazione} di $r_a$. Con semplici passaggi algebrici, si può mostrare che $r_a$ è caratterizzata dall'equazione $[r_a(z),a]=-[z,a]$, da cui $r_a\circ r_a=\id$.
\end{defn}

\begin{defn}
  Dati $a_1,\dots,a_n \in \mathbb{D}$ e $\theta \in \mathbb{R}$, chiamiamo \textit{prodotto di Blaschke} di grado $n$ la funzione
  $$e^{i\theta}\prod_{j=1}^n \frac{z-a_j}{1-\bar{a}_jz}.$$
  Indichiamo con $\mathcal{B}_n$ i prodotti di Blaschke di grado $n$.
\end{defn}

\begin{oss}
  I prodotti di Blaschke sono funzioni olomorfe sul disco unitario, con zeri assegnati. Quelli di grado $1$ sono $\text{Aut}(\mathbb{D})$.
\end{oss}

\begin{lm} \label{Blaschke-car}
  Tra le funzioni $f$ continue in $\overline{\mathbb{D}}$ e olomorfe in $\mathbb{D}$, i prodotti di Blaschke di grado $n$ sono caratterizzati dalle seguenti proprietà:
  \begin{nlist}
    \item se $|z|=1$ allora $|f(z)|=1$;
    \item $f$ ha esattamente $n$ zeri in $\mathbb{D}$ contati con molteplicità.
  \end{nlist}
\end{lm}

\begin{proof}
  Se $f$ è un prodotto di Blaschke di grado $n$, che soddisfi (ii) è ovvio e che soddisfi (i) segue dall'Osservazione \ref{dom}.

  Fissiamo ora $f$ che soddisfi (i) e (ii); consideriamo $B$ il prodotto di Blaschke definito con $\theta=0$ e $a_n$ gli zeri in $\mathbb{D}$ di $f$, contati con molteplicità. Allora $f/B$ e $B/f$ sono due funzioni olomorfe su $\mathbb{D}$ e continue in $\overline{\mathbb{D}}$, di modulo $1$ sul bordo. Per il principio del massimo per funzioni olomorfe, deve essere $|f/B| \le 1$ e $|B/f| \le 1$ sul disco unitario, da cui $|f/B|=1$ e $f/B$ è costante in $\mathbb{D}$.
\end{proof}

\begin{lm} \label{z^2}
  Sia $f \in \mathcal{B}_2$, esistono $S, T \in \text{Aut}(\mathbb{D})$ tali che $S\circ f\circ T(z)=z^2$.
\end{lm}

\begin{proof}
  Per transitività di $\text{Aut}(\mathbb{D})$ su $\mathbb{D}$, ci basta dimostrare che esiste un punto $c \in \mathbb{D}$ tale che $f(z)-c$ abbia uno zero doppio nel disco. È sufficiente trovare $z_0 \in \mathbb{D}$ con $f'(z_0)=0$, dato che sappiamo già $f(z_0) \in \mathbb{D}$. Scriviamo $f(z)=e^{i\theta}\cdot\dfrac{z-a_1}{1-\bar{a}_1z}\cdot\dfrac{z-a_2}{1-\bar{a}_2z}$.
  Sempre per transitività, usando $T$ possiamo supporre $a_1=0$ e poniamo $a_2=a$. Si ha
  $$f'(z)=e^{i\theta}\left(\frac{z-a}{1-\bar{a}z}+z\cdot\frac{1-|a|^2}{1-\bar{a}z}\right).$$
  Con qualche passaggio algebrico, l'equazione $f'(z)=0$ diventa
  $$\bar{a}z^2-2z+a=0.$$
  Le soluzioni sono $(1 \pm \sqrt{1-|a|^2})/\bar{a}$; da $|a|<1$, abbiamo che quella con il più sta fuori da $\mathbb{D}$ mentre quella con il meno sta dentro, dunque è la soluzione cercata.
\end{proof}

\begin{defn}
  Sia $f \in \mathcal{B}_2$. Indichiamo con $R_f$ la rotazione iperbolica di ordine due attorno al punto $z_0$, trovato nella dimostrazione del Lemma \ref{z^2}.
\end{defn}

\begin{cor} \label{rotazioni}
  Sia $f \in \mathcal{B}_2$. Allora $f^*\bigl(R_f(w),w\bigr)=0$.
\end{cor}

\begin{proof}
  Siano $S, T \in \text{Aut}(\mathbb{D})$ date dal Lemma \ref{z^2} tali che $S\circ f\circ T=F$ con $F(z)=z^2$. Abbiamo $[R_F(z),0]=-[z,0]$, dunque per l'Osservazione \ref{muu} troviamo $\mu[T\bigl(R_F(z)\bigr),T(0)]=-\mu[T(z),T(0)]$.
  Per costruzione, $T(0)$ dev'essere il punto attorno al quale avviene la rotazione $R_f$, che quindi è caratterizzata dall'equazione $[R_f(z),T(0)]=-[z,T(0)]$. Ne segue che $T\bigl(R_F(z)\bigr)=R_f\bigl(T(z)\bigr)$. Perciò abbiamo $f=S^{-1}\circ F\circ T^{-1}$ e $R_f=T\circ R_F\circ T^{-1}$, inoltre $F\circ R_F=F$. Ne deduciamo che
  $$f\circ R_f=S^{-1}\circ F\circ R_F\circ T^{-1}=S^{-1}\circ F\circ T^{-1}=f,$$
  che implica la tesi.
\end{proof}

\begin{prop} \label{blaschke-prop}
  Valgono le seguenti affermazioni:
  \begin{nlist}
    \item date $F \in \text{Hol}(\mathbb{D},\mathbb{D})$ e $S \in \text{Aut}(\mathbb{D})$, si ha che $F \in \mathcal{B}_n$ se e solo se $S\circ F \in \mathcal{B}_n$;
    \item $f \in \mathcal{B}_{n+1}$ se e solo se $f^*(z,w) \in \mathcal{B}_n$, con $w$ un qualsiasi elemento di $\mathbb{D}$ fissato.
  \end{nlist}
\end{prop}

\begin{proof}
  Per dimostrare (i), basta mostrare che $S$ conserva le proprietà del Lemma \ref{Blaschke-car}. La prima segue dall'Osservazione \ref{dom}. Per transitività di $\text{Aut}(\mathbb{D})$, la seconda corrisponde a dover dimostrare che, dato $c \in \mathbb{D}$, l'equazione $F(z)=c$ ha esattamente $n$ zeri contati con molteplicità in $\mathbb{D}$, dove $F \in B_n$. Scriviamo $F(z)=\displaystyle e^{i\theta}\prod_{j=1}^n \frac{z-a_j}{1-\bar{a}_jz}$. Allora $F(z)=c$ si riscrive come
  \begin{equation} \label{pol-eq}
    c\prod_{j=1}^n (1-\bar{a}_jz)=e^{i\theta}\prod_{j=1}^n(z-a_j).
  \end{equation}
  Stiamo uguagliando due polinomi di grado $n$ con coefficienti direttivi diversi: quello al membro sinistro è di modulo minore di $1$, mentre quello al membro destro ha modulo esattamente $1$. Dunque la nostra equazione ha esattamente $n$ soluzioni, contate con molteplicità, in $\mathbb{C}$.
  Per l'Osservazione \ref{dom}, se $z \not\in \mathbb{D}$ e $z\not=1/\bar{a}_j$ per $j=1,\dots,n$, si ha $|F(z)| \ge 1$; se $z=1/\bar{a}_j$ per qualche $j$, il membro sinistro di \eqref{pol-eq} è $0$ ma il membro destro no, quindi non ci sono soluzioni in quel caso. Perciò, per $|c|<1$, tutte le soluzioni trovate sono in $\mathbb{D}$ come voluto.

  Per definizione di quoziente iperbolico, $[f(z),f(w)]=[z,w]f^*(z,w)$. Il membro di sinistra è della forma $S(f(z))$, dove $S \in \text{Aut}(\mathbb{D})$ è $S(z)=[z,f(w)]$; scriviamo anche $T(z)=[z,w]$.
  Se $f^*(z,w) \in \mathcal{B}_n$, allora $T(z)f^*(z,w) \in \mathcal{B}_{n+1}$; dunque $S\circ f \in \mathcal{B}_{n+1}$ e per il punto (i) abbiamo $f \in \mathcal{B}_{n+1}$. Viceversa, se $f \in \mathcal{B}_{n+1}$ si ha $S\circ f \in \mathcal{B}_{n+1}$.
  Sappiamo anche che $S\bigl(f(w)\bigr)=0$, dunque nel prodotto ci dev'essere il fattore $[z,w]$; segue dunque che $f^*(z,w) \in \mathcal{B}_n$.
\end{proof}

\begin{prop} \label{24}
  Siano $f \in \text{\normalfont{Hol}}(\mathbb{D},\mathbb{D})\setminus\text{\normalfont{Aut}}(\mathbb{D})$ e $v \in \mathbb{D}$. Allora per ogni $z \in \mathbb{D}$ si ha che $f^*(z,v) \in \mathbb{D}$ e la funzione $z \longmapsto f^*(z,v)$ è olomorfa.
\end{prop}

\begin{proof}
  Per quanto riguarda l'olomorfia, dalla definizione sappiamo che l'unico punto che potrebbe dar problemi è $v$; abbiamo però visto che la funzione ammette limite finito per $z \longrightarrow v$, perciò $v$ è una singolarità rimovibile. Per il lemma di Schwarz-Pick, $|f^*(z,v)| \le 1$; inoltre, vale l'uguaglianza in qualche punto solo se $f$ è un automorfismo. Dunque le ipotesi su $f$ assicurano che vale la disuguaglianza stretta sempre, cioè $f^*(z,v) \in \mathbb{D}$ per ogni $z \in \mathbb{D}$.
\end{proof}

\begin{thm} \label{31}
  (Beardon-Minda, 2004) Sia $f \in \text{\normalfont{Hol}}(\mathbb{D},\mathbb{D})\setminus\text{\normalfont{Aut}}(\mathbb{D})$. Allora per ogni $z, w, v \in \mathbb{D}$ vale
  \begin{equation} \label{3.1}
    \omega\bigl(f^*(z,v),f^*(w,v)\bigr) \le \omega(z,w).
  \end{equation}
  Si ha l'uguaglianza se e solo se $f \in \mathcal{B}_2$.
\end{thm}

\begin{proof}
  Poiché $f$ non è un automorfismo, per la Proposizione \ref{24} la funzione $z \longmapsto f^*(z,v)$ è in $\text{Hol}(\mathbb{D}, \mathbb{D})$; perciò il membro sinistro della disuguaglianza \eqref{3.1} è ben definito e la tesi segue dal lemma di Schwarz-Pick e dall'osservazione \ref{oss1}, punto (ii).

  Sempre dal lemma di Schwarz-Pick, si ha l'uguaglianza se e solo se abbiamo $f^*(z,v) \in \text{Aut}(\mathbb{D})=\mathcal{B}_1$. Per il punto (ii) della Proposizione \ref{blaschke-prop}, questo è equivalente a $f \in \mathcal{B}_2$.
\end{proof}

\begin{defn}
  Una \textit{geodetica} per $\omega$ è una curva $\sigma: \mathbb{R} \longrightarrow \mathbb{D}$ tale che per ogni $t_1,t_2 \in \mathbb{R}$ si ha $\omega\bigl(\sigma(t_1),\sigma(t_2)\bigr)=|t_1-t_2|$.
\end{defn}
\marginpar{forse bisognerebbe aggiungere la caratterizzazione di appartenenza a una stessa geodetica?}

\begin{cor} \label{32}
  Sia $f \in \text{\normalfont{Hol}}(\mathbb{D},\mathbb{D})\setminus\text{\normalfont{Aut}}(\mathbb{D})$. Allora per ogni $z, w, v \in \mathbb{D}$ vale
  \begin{equation}
    \omega\bigl(0, f^*(z,v)\bigr) \le \omega\bigl(0,f^*(w,v)\bigr)+\omega(z,w).
  \end{equation}
  Si ha l'uguaglianza se e solo se $f \in \mathcal{B}_2$ e $R_f(v)$, $w$ e $z$ giacciono sulla stessa geodetica, in quest'ordine.
\end{cor}

\begin{proof}
  Applicando la disuguaglianza triangolare per $\omega$ e il Teorema \ref{31}, si ha
  \begin{align*}
    \omega\bigl(0,f^*(z,v)\bigr) & \le \omega\bigl(0,f^*(w,v)\bigr)+\omega\bigl(f^*(w,v),f^*(z,v)\bigr) \\
    & \le \omega\bigl(0,f^*(w,v)\bigr)+\omega(z,w).
  \end{align*}

  Si ha l'uguaglianza se e solo se vale in entrambe le disuguaglianze appena viste. La seconda è esattamente il caso di uguaglianza del Teorema \ref{31}, che è equivalente a $f \in \mathcal{B}_2$. Sia $T_v(z)=f^*(z,v)$; per il Teorema \ref{blaschke-prop}, $f \in \mathcal{B}_2$ è equivalente a $T_v \in \text{Aut}(\mathbb{D})$. Ricordiamo che $p$, dunque anche $\omega$, è invariante sotto l'azione di $\text{Aut}(\mathbb{D})$.
  Allora il caso di uguaglianza nella prima delle due disuguaglianze si riscrive come $\omega\bigl(T_v^{-1}(0),z\bigr)=\omega\bigl(T_v^{-1}(0),w\bigr)+\omega(w,z)$, che caratterizza l'appartenenza, nell'ordine, alla stessa geodetica. Poiché $T_v$ è un automorfismo, esiste un solo valore per cui va in $0$; ma per il Corollario \ref{rotazioni}, questo valore è proprio $R_f(v)$.
\end{proof}

\begin{cor} \label{33}
  Sia $f \in \text{\normalfont{Hol}}(\mathbb{D},\mathbb{D})\setminus\text{\normalfont{Aut}}(\mathbb{D})$. Allora per ogni $z, w, v, u \in \mathbb{D}$ vale
  \begin{equation} \label{eq33}
    \omega\bigl(0, f^*(z,v)\bigr) \le \omega\bigl(0, f^*(u,w)\bigr)+\omega(z,w)+\omega(v,u).
  \end{equation}
  Si ha l'uguaglianza se e solo se $f \in \mathcal{B}_2$ e $R_f(v), R_f(u), w$ e $z$ giacciono sulla stessa geodetica, in quest'ordine.
\end{cor}
\begin{proof}
  Applicando il Corollario \ref{32} si ha
  \begin{align*}
    \omega\bigl(0,f^*(z,v)\bigr) & \le \omega\bigl(0,f^*(w,v)\bigr)+\omega(z,w) =\omega\bigl(0,|f^*(w,v)|\bigr)+\omega(z,w) \\
    & =\omega\bigl(0,|f^*(v,w)|\bigr)+\omega(z,w)=\omega\bigl(0,f^*(v,w)\bigr)+\omega(z,w).
  \end{align*}
  Sempre per il Corollario \ref{32} abbiamo
  $$\omega\bigl(0,f^*(v,w)\bigr) \le \omega\bigl(0,f^*(u,w)\bigr)+\omega(z,w)+\omega(v,u).$$
  Mettendo assieme le due disuguaglianze otteniamo la \eqref{eq33}.

  Se si ha l'uguaglianza, dobbiamo studiarla nelle due applicazioni del Corollario \ref{32}. In entrambi i casi ci dice che $f \in \mathcal{B}_2$. La prima ci dice anche che $R_f(v), w$ e $z$ appartengono, nell'ordine, alla stessa geodetica. Dalla seconda deduciamo la stessa cosa per $R_f(w), u$ e $v$. Poiché $R_f$ è un automorfismo e $\omega$ non cambia sotto l'azione di $\text{Aut}(\mathbb{D})$, abbiamo che lascia invariate le geodetiche; allora anche $w, R_f(u)$ e $R_f(v)$ stanno sulla stessa geodetica. Segue il secondo enunciato della tesi. Viceversa, se valgono tutte queste condizioni è facile vedere che si ha l'uguaglianza.
\end{proof}

Il risultato seguente non ci servirà nel seguito, ma viene riportato per completezza.

\begin{cor} \label{35}
  Sia $f \in \normalfont{\text{Hol}}(\mathbb{D},\mathbb{D})$ e siano $z, w \in \mathbb{D}$. Sia $\sigma$ una geodetica con $\sigma(t_1)=z, \sigma(t_2)=v$ e sia $w=\sigma(t)$ con $t_1<t<t_2$. Allora
  \begin{equation} \label{geod}
    2\omega\bigl(f(z),f(v)\bigr) \le \log\Bigl(\cosh\bigl(2\omega(z,v)\bigr)+|f^h(w)|\sinh\bigl(2\omega(z,v)\bigr)\Bigr).
  \end{equation}
\end{cor}
\marginpar{da rivedere}
\begin{proof}
  Osserviamo che se $f \in \text{Aut}(\mathbb{D})$, allora per il lemma di Schwarz-Pick $|f^h(w)|=1$ e il membro destro della disuguaglianza \eqref{geod} è esattamente $2\omega(z,v)$. In questo caso, per il lemma di Schwarz-Pick si ha l'uguaglianza.

  Supponiamo ora $f \not\in \text{Aut}(\mathbb{D})$, allora possiamo applicare il Corollario \ref{33} con $u=v$ per ottenere
  \begin{gather*}
    \omega\bigl(0,f^*(z,v)\bigr) \le \omega\bigl(0,f^h(w)\bigr)+\omega(z,v) \\
    p\bigl(0,f^*(z,v)\bigr) \le \tanh\Bigl(\omega\bigl(0,f^h(w)\bigr)+\omega(z,v)\Bigr) \\
    \frac{p\bigl(f(z),f(v)\bigr)}{p(z,v)} \le \frac{|f^h(w)|+p(z,v)}{1+|f^h(w)|p(z,v)},
  \end{gather*}
  dove abbiamo usato $p=\tanh\omega$, $f^*(z,v)=\frac{[f(z),f(v)]}{[z,v]}$, $\tanh(a+b)=\frac{\tanh{a}+\tanh{b}}{1+\tanh{a}\tanh{b}}$ e $p(0,\zeta)=|\zeta|$. Riscriviamo come
  \begin{gather*}
    \frac{p\bigl(f(z),f(v)\bigr)}{p(z,v)}-p(z,v) \le |f^h(w)|\Bigl(1-p\bigl(f(z),f(v)\bigr)\Bigr) \\
    \frac{p\bigl(f(z),f(v)\bigr)-p^2(z,v)}{p(z,v)\Bigl(1-p\bigl(f(z),f(v)\bigr)\Bigr)} \le |f^h(w)| \\
    \frac{2\Bigl(p\bigl(f(z),f(v)\bigr)-p^2(z,v)\Bigr)}{\bigl(1-p^2(z,v)\bigr)\Bigl(1-p\bigl(f(z),f(v)\bigr)\Bigr)} \le |f^h(w)|\cdot \frac{2p(z,v)}{1-p^2(z,v)}.
  \end{gather*}
  Adesso usiamo le seguenti uguaglianze:
  $$\frac{1+p^2}{1-p^2}=\cosh(2\omega) \text{ e } \frac{2p}{1-p^2}=\sinh(2\omega).$$
  Sommando appunto la quantità $\dfrac{1+p^2(z,v)}{1-p^2(z,v)}$ all'ultima disuguaglianza ottenuta, il membro destro diventa $\cosh\bigl(2\omega(z,v)\bigr)+|f^h(w)|\sinh\bigl(2\omega(z,v)\bigr)$. Ci basta dunque mostrare che il membro sinistro è uguale a $\exp\Bigl(2\omega\bigl(f(z),f(v)\bigr)\Bigr)$, cioè $\dfrac{1+p\bigl(f(z),f(v)\bigr)}{1-p\bigl(f(z),f(v)\bigr)}$. Si ha infatti
  \[
    \displaylines{\quad \frac{2\Bigl(p\bigl(f(z),f(v)\bigr)-p^2(z,v)\Bigr)}{\bigl(1-p^2(z,v)\bigr)\Bigl(1-p\bigl(f(z),f(v)\bigr)\Bigr)}+\frac{1+p^2(z,v)}{1-p^2(z,v)}\hfill \cr
    \hfill = \frac{p\bigl(f(z),f(v)\bigr)-p^2(z,v)+1-p^2(z,v)p\bigl(f(z),f(v)\bigr)}{\bigl(1-p^2(z,v)\bigr)\Bigl(1-p\bigl(f(z),f(v)\bigr)\Bigr)}\hskip27pt\cr
    \hfill = \frac{\bigl(1-p^2(z,v)\bigr)\Bigl(1+p\bigl(f(z),f(v)\bigr)\Bigr)}{\bigl(1-p^2(z,v)\bigr)\Bigl(1-p\bigl(f(z),f(v)\bigr)\Bigr)}=\frac{1+p\bigl(f(z),f(v)\bigr)}{1-p\bigl(f(z),f(v)\bigr)}.\quad \cr}
  \]
\end{proof}

\begin{cor} \label{36}
  Sia $f \in \text{\normalfont{Hol}}(\mathbb{D},\mathbb{D})\setminus\text{\normalfont{Aut}}(\mathbb{D})$ tale che $f(0)=0$. Allora
  \begin{equation}
    \omega\bigl(f^h(0),f^h(z)\bigr) \le 2\omega(0,z).
  \end{equation}
  Inoltre, $2$ è la migliore costante possibile.
\end{cor}

\begin{proof}
  Da $f(0)=0$ si ha $f^*(z,0)=f^*(0,z)$. Dalla disuguaglianza triangolare per $\omega$ abbiamo che
  \begin{align*}
    \omega\bigl(f^h(0),f^h(z)\bigr) & = \omega\bigl(f^*(0,0),f^*(z,z)\bigr) \\
    & \le \omega\bigl(f^*(0,0),f^*(z,0)\bigr)+\omega\bigl(f^*(0,z),f^*(z,z)\bigr),
  \end{align*}
  inoltre per il Teorema \ref{31} si ha
  $$\omega\bigl(f^*(0,0),f^*(z,0)\bigr)+\omega\bigl(f^*(0,z),f^*(z,z)\bigr)\le 2\omega(0,z).$$
  Mettendo assieme troviamo la disuguaglianza della tesi.

  Per dire che $2$ è la migliore costante possibile, basta prendere $f(z)=z^2$ e $z \in \mathbb{D}$ con $|z|=1/3$ per ottenere l'uguaglianza.
\end{proof}

Il prossimo risultato è quello che ci permetterà di dimostrare la disuguaglianza di Golusin.

\begin{cor} \label{quasigolusin}
  Sia $f \in \text{\normalfont{Hol}}(\mathbb{D},\mathbb{D})\setminus\text{\normalfont{Aut}}(\mathbb{D})$. Allora per ogni $z, w \in \mathbb{D}$ vale
  \begin{equation} \label{quasigol}
    \omega\bigl(|f^h(z)|, |f^h(w)|\bigr) \le 2\omega(z,w).
  \end{equation}
  Si ha l'uguaglianza se e solo se $f \in \mathcal{B}_2$ e $z$ e $w$ giacciono sulla stessa geodetica, passante per il centro di rotazione di $R_f$.
\end{cor}

\begin{proof}
  Siano $z, w \in \mathbb{D}$; senza perdita di generalità possiamo supporre $|f^h(z)| \ge |f^h(w)|$. Allora dalla definizione di $\omega$ abbiamo
  \begin{align*}
    \omega\bigl(|f^h(z)|, |f^h(w)|\bigr) & =\frac{1}{2}\log\left(\frac{1+\frac{|f^h(z)|-|f^h(w)|}{1-|f^h(w)||f^h(z)|}}{1-\frac{|f^h(z)|-|f^h(w)|}{1-|f^h(w)||f^h(z)|}}\right) \\
    & =\frac{1}{2}\log\left(\frac{1-|f^h(w)||f^h(z)|+|f^h(z)|-|f^h(w)|}{1-|f^h(w)||f^h(z)|+|f^h(w)|-|f^h(z)|}\right) \\
    & =\frac{1}{2}\log\left(\frac{1+|f^h(z)|}{1-|f^h(z)|}\cdot\frac{1-|f^h(w)|}{1+|f^h(w)|}\right) \\
    & =\frac{1}{2}\log\left(\frac{1+|f^h(z)|}{1-|f^h(z)|}\right)-\frac{1}{2}\log\left(\frac{1+|f^h(w)|}{1-|f^h(w)|}\right)
  \end{align*}
  Usando di nuovo la definizione di $\omega$ otteniamo dunque
  \begin{align*}
    \omega\bigl(|f^h(z)|, |f^h(w)|\bigr)&=\omega\bigl(0,|f^h(z)|\bigr)-\omega\bigl(0,|f^h(w)|\bigr) \\
    & =\omega\bigl(0,f^h(z)\bigr)-\omega\bigl(0,f^h(w)\bigr) \le 2\omega(z,w),
  \end{align*}
  dove l'ultima disuguaglianza segue dal Corollario \ref{33} prendendo $u=w$ e $v=z$. Il caso di uguaglianza segue facilmente.
\end{proof}

Concludiamo la sezione con il lemma di Dieudonné \cite[Chapter III, ???]{D}, per il quale l'approccio dell'articolo di Beardon e Minda semplifica le dimostrazioni.
\marginpar{Chapter o Chapitre? Poi: quale dei tanti? Non sono certo di quale sia quello più pertinente alla nostra versione}

\begin{lm}
  (lemma di Dieudonné) Sia $f \in \text{Hol}(\mathbb{D},\mathbb{D})$ tale che $f(0)=0$ e sia $z_0 \in \mathbb{D}$ con $|f(z_0)| \le |z_0|$. Allora
  \begin{equation}
    |f'(z_0)-f(z_0)/z_0| \le \frac{|z_0|^2-|f(z_0)|^2}{|z_0|(1-|z_0|^2)}.
  \end{equation}
\end{lm}

\begin{proof}
  Per il Teorema \ref{31} con $z=v=z_0$ e $w=0$ abbiamo
  \begin{align*}
    \omega\bigl(f^h(z_0),f^*(0,z_0)\bigr) & \le \omega(0,z_0) \\
    \iff p\bigl(f^h(z_0),f^*(0,z_0)\bigr) & \le p(0,z_0)=|z_0|,
  \end{align*}
  dove l'equivalenza fra le due disuguaglianze segue dal fatto che $\text{arctanh}$ è strettamente crescente. Per semplificare, scriviamo $f^h(z_0)=a, f^*(0,z_0)=b$ e $|z_0|=r$. Vogliamo portare la disuguaglianza in forma euclidea. Abbiamo
  $$\left|\frac{a-b}{1-\bar{b}a}\right|=p(a,b) \le r,$$
  che si riscrive come
  \marginpar{trova un modo più leggibile di fare tutta 'sta roba}
  \begin{align*}
    (a-b)(\bar{a}-\bar{b}) & \le r^2(1-\bar{b}a)(1-b\bar{a}) \\
    & \iff |a|^2-a\bar{b}-\bar{a}b+|b|^2 \le r^2-r^2a\bar{b}-r^2\bar{a}b+r^2|b|^2|a|^2 \\
    & \iff |a|^2(1-r^2|b|^2)-a\bar{b}(1-r^2)-\bar{a}b(1-r^2) \le r^2-|b|^2 \\
    & \iff |a|^2-a\cdot\frac{\bar{b}(1-r^2)}{1-r^2|b|^2}-\bar{a}\cdot\frac{b(1-r^2)}{1-r^2|b|^2} \le \frac{r^2-|b|^2}{1-r^2|b|^2};
  \end{align*}
  ponendo $\alpha=\dfrac{b(1-r^2)}{1-r^2|b|^2}$ e $R^2=\dfrac{r^2-|b|^2}{1-r^2|b|^2}+|b|^2\left(\dfrac{1-r^2}{1-r^2|b|^2}\right)^2$, si ha
  $$|a|^2-a\bar{\alpha}-\bar{a}\alpha \le R^2-|\alpha|^2 \iff (a-\alpha)(\bar{a}-\bar{\alpha}) \le R^2\iff |a-\alpha| \le R.$$
  Ricordando che $r=|z_0|$ e osservando che $b=f^*(0,z_0)=\frac{[f(0),f(z_0)]}{[0,z_0]}=\frac{f(z_0)}{z_0}$, troviamo $\alpha=\dfrac{f(z_0)(1-|z_0|^2)}{z_0\bigl(1-|f(z_0)|^2\bigr)}$ e $R=\dfrac{|z_0|^2-|f(z_0)|^2}{|z_0|\bigl(1-|f(z_0)|^2\bigr)}$.
  Riprendendo infine la definizione di $a$, cioè $a=f^h(z_0)=\frac{f'(z_0)(1-|z_0|^2)}{1-|f(z_0)|^2}$, otteniamo che
  $$\left|\frac{f'(z_0)(1-|z_0|^2)}{1-|f(z_0)|^2}-\frac{f(z_0)(1-|z_0|^2)}{z_0\bigl(1-|f(z_0)|^2\bigr)}\right| \le \frac{|z_0|^2-|f(z_0)|^2}{|z_0|\bigl(1-|f(z_0)|^2\bigr)},$$
  che è equivalente alla tesi moltiplicando entrambi i membri per $\frac{1-|f(z_0)|^2}{1-|z_0|^2}$.
\end{proof}


\subsection{Applicazioni dei lemmi di Schwarz-Pick multi-punto}
Vediamo ora alcune applicazioni dei risultati visti nella sezione precedente.

\begin{thm} \label{distortion}
  Dato $b \in [0,1)$, scriviamo $F_b(z)=\dfrac{z(z+b)}{1+b z}$. Consideriamo $f \in \normalfont{\text{Hol}}(\mathbb{D},\mathbb{D})$ tale che $f(0)=0$. Se $f'(0)=b$, allora per ogni $z \in \mathbb{D}$ si ha
  \begin{equation}
    \left|\frac{b-f^h(z)}{1-b f^h(z)}\right| \le \frac{2|z|}{1+|z|^2}
  \end{equation}
  e
  \begin{equation}
    F_b^h(-|z|) \le \mathfrak{Re}f^h(z) \le |f^h(z)| \le F_b^h(|z|).
  \end{equation}
\end{thm}

\begin{proof}
  Poiché $|f'(0)|<1$, per il lemma di Schwarz si ha $f \not\in \text{Aut}(\mathbb{D})$. Inoltre $f(0)=0$, perciò possiamo applicare il Corollario \ref{36}; si ha dunque
  \begin{align*}
    \omega\bigl(f^h(0),f^h(z)\bigr) \le 2\omega(0,z) \\
    \omega\bigl(b,f^h(z)\bigr) \le 2\omega(0,z) \\
    p\bigl(b,f^h(z)\bigr) \le \frac{2p(0,z)}{1+p^2(0,z)} \\
    \left|\frac{b-f^h(z)}{1-b f^h(z)}\right| \le \frac{2|z|}{1+|z|^2},
  \end{align*}
  dove abbiamo usato il fatto che $\tanh$ è strettamente crescente e l'uguaglianza $\tanh(2x)=\frac{2\tanh{x}}{1+\tanh^2{x}}$.

  Per dimostrare la seconda disuguaglianza, ripetiamo i passaggi svolti nella dimostrazione del lemma di Dieudonné prendendo $a=f^h(z)$ e $r=\frac{2|z|}{1+|z|^2}$. Otteniamo la disuguaglianza $|f^h(z)-\alpha| \le R$, dove si ha $\alpha=\dfrac{b(1-r^2)}{1-r^2b^2}$ e $R^2=\dfrac{r^2-b^2}{1-r^2b^2}+b^2\left(\dfrac{1-r^2}{1-r^2b^2}\right)^2$. Sostituendo troviamo
  $$\alpha=\frac{b(1-|z|^2)^2}{(1+2b|z|+|z|^2)(1-2b|z|+|z|^2)},$$
  $$R=\frac{2|z|(|z|^2+1)(1-b^2)}{(1+2b|z|+|z|^2)(1-2b|z|+|z|^2)}.$$
  Consideriamo adesso $F_b^h(z)=\dfrac{bz^2+2z+b}{|z|^2+2b\mathfrak{Re}z+1}\left(\dfrac{|1+b z|}{1+b z}\right)^2$. Si ha
  $$F_b^h(|z|)=\dfrac{b|z|^2+2|z|+b}{|z|^2+2|z|+1}, \quad F_b^h(-|z|)=\dfrac{b|z|^2-2|z|+b}{|z|^2-2|z|+1}.$$
  Notiamo che $\alpha=\bigl(F_b^h(|z|)+F_b^h(-|z|)\bigr)/2$ e $R=\bigl(F_b^h(|z|)-F_b^h(-|z|)\bigr)/2$, perciò la disuguaglianza $|f^h(z)-\alpha| \le R$ ci dice che $f^h(z)$ appartiene al cerchio con diametro sull'asse reale passante per i punti $F_b^h(|z|)$ e $F_b^h(-|z|)$. La seconda disuguaglianza segue allora da semplici considerazioni geometriche.
\end{proof}

\begin{oss}
  Sapendo solo che $|f'(0)|=b$, si può dimostrare che
  $$F_b^h(-|z|) \le |f^h(z)| \le F_b^h(|z|).$$
  Basta infatti considerare la funzione $bf(z)/f'(0)$.
\end{oss}

\begin{cor} \label{distorto}
  Sia $f$ come nel Teorema \ref{distortion}. Allora $\mathfrak{Re}f'(z)>0$ per $|z|<b/(1+\sqrt{1-b^2})$.
\end{cor}

\begin{proof}
  Per $0 \le b<1$ e $z \in \mathbb{D}$ si ha $|z|^2-2b|z|+1>|z|^2-2|z|+1>0$, dunque il segno di $F_b^h(-|z|)$ coincide con quello di $b|z|^2-2|z|+b$. Quest'ultima quantità è minore di $0$ per $|z| \in \bigl((1-\sqrt{1-b^2})/b, (1+\sqrt{1-b^2})/b\bigr)$, zero agli estremi e maggiore di $0$ altrove. Poiché l'estremo destro è maggiore di $1$, è da scartare.
  Per il teorema \ref{distortion} abbiamo dunque che $\mathfrak{Re}f'(z) \ge F_b^h(-|z|)>0$ per gli $z$ tali che $|z|<(1-\sqrt{1-b^2})/b=b/(1+\sqrt{1-b^2})$.
\end{proof}

Del prossimo enunciato, dimostrato indipendentemente da Pick nel 1916 e Nevanlinna nel 1919, vedremo nel dettaglio solo un paio di casi particolari.

\begin{thm}
  (Pick-Nevanlinna, Theorem 2.2, Chapter 1 \cite{JBG}) Siano dati $n$ punti distinti $z_1, \dots, z_n \in \mathbb{D}$ e altri $n$ punti distinti (non necessariamente diversi dai primi) $w_1, \dots, w_n \in \mathbb{D}$. Per $k=1, \dots, n$, sia $A_k$ la matrice $k \times k$ data da $A_k(i,j)=\dfrac{1-w_i\bar{w}_j}{1-z_i\bar{z}_j}$.
  Allora esiste una funzione $f \in \normalfont{\text{Hol}}(\mathbb{D},\mathbb{D})$ tale che $f(z_i)=w_i$ per $j=1, \dots, n$ se e solo se $\det{A_k} \ge 0$ per ogni $k=1, \dots, n$.
\end{thm}

Vediamo il caso $n=2$.

\begin{proof}
  La condizione è sempre verificata per $k=1$, mentre per $k=2$ si riscrive come
  \begin{gather*}
    \frac{1-|w_1|^2}{1-|z_1|^2}\cdot\frac{1-|w_2|^2}{1-|z_2|^2}-\frac{1-w_1\bar{w}_2}{1-z_1\bar{z}_2}\cdot\frac{1-\bar{w}_1w_2}{1-\bar{z}_1z_2} \ge 0 \\
    \frac{(1-|w_1|^2)(1-|w_2|^2)}{(1-|z_1|^2)(1-|z_2|^2)} \ge \frac{|1-w_1\bar{w}_2|^2}{|1-z_1\bar{z}_2|^2} \\
    \frac{|1-z_1\bar{z}_2|^2}{(1-|z_1|^2)(1-|z_2|^2)} \ge \frac{|1-w_1\bar{w}_2|^2}{(1-|w_1|^2)(1-|w_2|^2)} \\
    \frac{|1-z_1\bar{z}_2|^2}{1-|z_1|^2-|z_2|^2+|z_1|^2|z_2|^2} \ge \frac{|1-w_1\bar{w}_2|^2}{1-|w_1|^2-|w_2|^2+|w_1|^2|w_2|^2} \\
    \frac{|1-z_1\bar{z}_2|^2}{|1-\bar{z}_2z_1|^2-|z_1-z_2|^2} \ge \frac{|1-w_1\bar{w}_2|^2}{|1-\bar{w}_2w_1|^2-|w_1-w_2|^2} \\
    \frac{1}{1-\left|\frac{z_1-z_2}{1-\bar{z}_2z_1}\right|^2} \ge \frac{1}{1-\left|\frac{w_1-w_2}{1-\bar{w}_2w_1}\right|^2} \\
    \frac{1}{1-p^2(w_1,w_2)} \le \frac{1}{1-p^2(z_1,z_2)} \\
    p(w_1,w_2) \le p(z_1,z_2).
  \end{gather*}
  Ricordiamo adesso che $p$ è invariante per azione di $\text{Aut}(\mathbb{D})$; quindi, a meno di comporre a sinistra e a destra con opportuni automorfismi olomorfi di $\mathbb{D}$, possiamo supporre senza perdita di generalità $z_1=w_1=0$. La condizione diventa dunque $p(0,w_2) \le p(0,z_2) \implies |w_2| \le |z_2|$, perciò basta prendere la funzione $f(z)=w_2z/z_2$.
\end{proof}

Andiamo adesso a dimostrare il Teorema di Pick-Nevanlinna nel caso $n=3$, con una formulazione differente.

\begin{thm}
  Siano $z_1, z_2, z_3$ e $w_1, w_2, w_3$ due triple di punti distinti in $\mathbb{D}$. Allora esiste $f \in \normalfont{\text{Hol}}(\mathbb{D},\mathbb{D}) \setminus \normalfont{\text{Aut}}(\mathbb{D})$ tale che $f(z_i)=w_i$ per $i=1,2,3$ se e solo se valgono le seguenti condizioni:
  \begin{nlist}
    \item $\omega(w_i,w_j)<\omega(z_i,z_j)$ per $i,j=1,2,3$ e $i\not=j$;
    \item $\omega\left(\dfrac{[w_2,w_1]}{[z_2,z_1]},\dfrac{[w_3,w_1]}{[z_3,z_1]}\right) \le \omega(z_2,z_3)$.
  \end{nlist}
\end{thm}

\begin{proof}
  Supponiamo che esista siffatta $f$. Allora la condizione (i) segue dal lemma di Schwarz-Pick. La condizione (ii) invece si riscrive come $\omega\bigl(f^*(z_2,z_1),f^*(z_3,z_1)\bigr) \le \omega(z_2,z_3)$, che è l'enunciato del Teorema \ref{31}.

  Adesso dimostriamo l'altra freccia. Vediamo prima nel caso $z_1=w_1=0$. Allora per la condizione (i) abbiamo che $\omega(0,w_i) < \omega(0,z_i) \implies |w_i/z_i|<1$ per $i=2,3$. La condizione (ii) si riscrive invece come $\omega(w_2/z_2,w_3/z_3) \le \omega(z_2,z_3)$, cioè $p(w_2/z_2,w_3/z_3) \le p(z_2,z_3)$.
  Dunque, per il caso $n=2$ del Teorema di Pick-Nevanlinna, esiste $g \in \text{Hol}(\mathbb{D},\mathbb{D})$ tale che $g(z_2)=w_2/z_2$ e $g(z_3)=w_3/z_3$. Allora basta prendere $f(z)=zg(z)$.

   Mostriamo che ci si può ridurre a questo caso. Consideriamo $h, g \in \text{Aut}(\mathbb{D})$ date da
   $$g(z)=\frac{z-z_1}{1-\bar{z}_1z}, \quad h(z)=\frac{z-w_1}{1-\bar{w}_1z}.$$
   Allora esiste $f$ come quella richiesta dal Teorema se e solo se esiste $F \in \text{Hol}(\mathbb{D},\mathbb{D})$, con $F=h \circ f \circ g^{-1}$, tale che $F(0)=0$, $F\bigl(g(z_2)\bigr)=h(w_2)$ e $F\bigl(g(z_3)\bigr)=h(w_3)$.
   Questo corrisponde proprio al caso precedente, quindi tale $F$ esiste se e solo se
   $$\omega\bigl(h(w_i),h(w_j)\bigr) \le \omega\bigl(g(z_i),g(z_j)\bigr)$$
   per $i,j=1,2,3$ con $i\not=j$ e
   $$\omega\left(\frac{h(w_2)}{g(z_2)},\frac{h(w_3)}{g(z_3)}\right) \le \omega\bigl(g(z_2),g(z_3)\bigr).$$
   Poiché $p$, e di conseguenza $\omega$, è invariante per azione di $\text{Aut}(\mathbb{D})$, la prima disuguaglianza è equivalente alla condizione (i). Sempre per questo motivo e sostituendo $h(z)=[z,w_1], g(z)=[z,z_1]$ otteniamo che la seconda è equivalente alla condizione (ii).
\end{proof}

Concludiamo la sezione con il risultato che, come già anticipato, ci permetterà di dimostrare i teoremi successivi. L'enunciato originale si trova in \cite{GMG}, ma vedremo una formulazione che ci tornerà più utile in seguito, in particolare perché coinvolge la funzione $f^h$.

\begin{thm} \label{golusin}
  (disuguaglianza di Golusin, 1945) Sia $f \in \text{\normalfont{Hol}}(\mathbb{D},\mathbb{D})\setminus\text{\normalfont{Aut}}(\mathbb{D})$. Allora per ogni $z \in \mathbb{D}$ vale
  \begin{equation} \label{gol}
    |f^h(z)| \le \frac{|f^h(0)|+\frac{2|z|}{1+|z|^2}}{1+|f^h(0)|\frac{2|z|}{1+|z|^2}}.
  \end{equation}
\end{thm}

\begin{proof}
  Con passaggi analoghi a quelli della dimostrazione del Corollario \ref{quasigolusin} abbiamo che valgono le seguenti uguaglianze:
  \begin{gather*}
    \omega\bigl(|f^h(z)|,|f^h(0)|\bigr)=\frac{1}{2}\log\left(\frac{1+|f^h(z)|}{1-|f^h(z)|}\cdot\frac{1-|f^h(0)|}{1+|f^h(0)|}\right)\\
    \omega(z, 0)=\omega(|z|,0)=\frac{1}{2}\log\left(\frac{1+|z|}{1-|z|}\right).
  \end{gather*}
  Prendendo $w=0$ nella disuguaglianza \eqref{quasigol} otteniamo
  \begin{align}
    \nonumber \frac{1}{2}\log\left(\frac{1+|f^h(z)|}{1-|f^h(z)|}\cdot\frac{1-|f^h(0)|}{1+|f^h(0)|}\right) \le \log\left(\frac{1+|z|}{1-|z|}\right) \\
    \frac{1+|f^h(z)|}{1-|f^h(z)|} \le \frac{1+|f^h(0)|}{1-|f^h(0)|}\left(\frac{1+|z|}{1-|z|}\right)^2. \label{golprimo}
  \end{align}
  Adesso, dalla Proposizione \ref{24} sappiamo che $f^h(z),f^h(0) \in \mathbb{D}$, in particolare $|f^h(z)|,|f^h(0)|<1$, perciò è giustificato il seguente passaggio:
  \begin{align*}
    |f^h(z)| & \le \frac{\frac{1+|f^h(0)|}{1-|f^h(0)|}\left(\frac{1+|z|}{1-|z|}\right)^2-1}{\frac{1+|f^h(0)|}{1-|f^h(0)|}\left(\frac{1+|z|}{1-|z|}\right)^2+1} \\
    & =\frac{(1+|f^h(0)|)(1+2|z|+|z|^2)-(1-|f^h(0)|)(1-2|z|+|z|^2)}{(1+|f^h(0)|)(1+2|z|+|z|^2)+(1-|f^h(0)|)(1-2|z|+|z|^2)} \\
    & =\frac{2|f^h(0)|+2|f^h(0)||z|^2+4|z|}{2+2|z|^2+4|f^h(0)||z|}=\frac{|f^h(0)|+\frac{2|z|}{1+|z|^2}}{1+|f^h(0)|\frac{2|z|}{1+|z|^2}}.
  \end{align*}
\end{proof}


\newpage

\section{Dalla disuguaglianza di Golusin al teorema di Burns-Krantz}

\subsection{Rigidità al bordo}
Dalla disuguaglianza di Golusin possiamo dimostrare una versione al bordo del lemma di Schwarz-Pick, seguendo la traccia data nel remark 5.6 di \cite{BKR}.

\begin{thm} \label{boundary_schwarz_pick}
  (lemma di Schwarz-Pick al bordo) Sia $f:\mathbb{D} \longrightarrow \mathbb{D}$ una funzione olomorfa dal disco in sé tale che
  \begin{equation} \label{n_o^2}
    f^h(z_n)=1+o((|z_n|-1)^2)
  \end{equation}
  per qualche successione $\{z_n\}_{n \in \mathbb{N}} \subset \mathbb{D}$ con $|z_n| \longrightarrow 1$. Allora $f$ è un automorfismo.
\end{thm}

\begin{proof}
  Supponiamo per assurdo che $f$ non sia un automorfismo. Allora possiamo applicare il corollario \ref{quasigolusin}, che per $w=0$ ci dà
  \begin{align*}
    d(f^h(z), f^h(0)) \le 2d(z,0) \\
    \log{\left(\frac{1+\left|\frac{f^h(z)-f^h(0)}{1-f^h(z)f^h(0)}\right|}{1-\left|\frac{f^h(z)-f^h(0)}{1-f^h(z)f^h(0)}\right|}\right)} \le 2\log{\left(\frac{1+|z|}{1-|z|}\right)} \\
    \frac{|1-f^h(z)f^h(0)|+|f^h(z)-f^h(0)|}{|1-f^h(z)f^h(0)|-|f^h(z)-f^h(0)|} \le \frac{(1+|z|)^2}{(1-|z|)^2}.
  \end{align*}
  Ricordiamo che per definizione $f^h(z) \ge 0$ e inoltre per il lemma di Schwarz-Pick $f^h(z) \le 1$, per ogni $z \in \mathbb{D}$. Sempre per il lemma originale, se valesse $f^h(0)=1$ avremmo che $f$ è un automorfismo, contraddizione. Perciò dev'essere $f^h(0)<1$, ma $\displaystyle \lim_{n \longrightarrow +\infty} f^h(z_n)=1$, quindi definitivamente $f^h(z_n)-f^h(0)>0$ e $1-f^h(z_n)f^h(0)>0$, da cui
  \begin{align*}
    \frac{(1-f^h(0))(1+f^h(z_n))}{(1-f^h(z_n))(1+f^h(0))} \le \frac{(1+|z_n|)^2}{(1-|z_n|)^2} \\
    \frac{1+f^h(0)}{(1-f^h(0))(1+f^h(z_n))}(1-f^h(z_n)) \ge \frac{(1-|z_n|)^2}{(1+|z_n|)^2}.
  \end{align*}
  Per ipotesi vale \eqref{n_o^2}, dunque
  \begin{align*}
    \frac{1+f^h(0)}{(1-f^h(0))(1+f^h(z_n))}o((|z_n|-1)^2) \ge \frac{(1-|z_n|)^2}{(1+|z_n|)^2} \\
    \frac{(1+f^h(0))(1+|z_n|)^2}{(1-f^h(0))(1+f^h(z_n))}o((|z_n|-1)^2) \ge 1.
  \end{align*}
  Poiché $\displaystyle \lim_{n \longrightarrow +\infty} \frac{(1+f^h(0))(1+|z_n|)^2}{(1-f^h(0))(1+f^h(z_n))}=\frac{2(1+f^h(0))}{1-f^h(0)} < +\infty$, otteniamo di nuovo una contraddizione.
\end{proof}

Siamo ora pronti a dimostrare il teorema 2.1 di \cite{BK}.


\subsection{Teorema di Burns-Krantz}
\marginpar{Magari dire qualcosa sull'utilità di questa proposizione, che forse verrà spostata nei prerequisiti}

\begin{prop} \label{o^3->o^2}
  Sia $f:\mathbb{D} \longrightarrow \mathbb{D}$ una funzione tale che
  \begin{equation} \label{o^3}
    f(z)=1+(z-1)+o(|z-1|^3)
  \end{equation}
  per $z \longrightarrow 1$ non tangenzialmente. Allora
  \begin{equation} \label{o^2}
    f^h(z)=1+o(|z-1|^2)
  \end{equation}
  per $z \longrightarrow 1$ non tangenzialmente.
\end{prop}

\marginpar{ricordati di definire $f^h$, con la notazione di BKR, quindi occhio quando scrivi tutti i risultati e le dimostrazioni in BM}

\begin{proof}
  Da scrivere, praticamente va copiata.
\end{proof}

Siamo ora pronti a dimostrare il teorema 2.1 di \cite{BK}.

\begin{thm} \label{burns_krantz}
  (Burns-Krantz, 1994) Sia $f:\mathbb{D} \longrightarrow \mathbb{D}$ una funzione olomorfa dal disco in sé tale che
  \begin{equation} \label{O^4}
    f(z)=1+(z-1)+\mathcal{O}(|z-1|^4)
  \end{equation}
  per $z \longrightarrow 1$. Allora $f$ è l'identità sul disco.
\end{thm}

\marginpar{Servono i risultati visti in BKR; poi: è comprensibile? Da rivedere in seguito}

\begin{proof}
  Se vale l'ipotesi \eqref{O^4} per $z \longrightarrow 1$ vale anche \eqref{o^3}, in particolare per $z \longrightarrow 1$ non tangenzialmente. Dalla proposizione \ref{o^3->o^2} segue che anche \eqref{o^2} vale per $z \longrightarrow 1$ non tangenzialmente, quindi esiste una successione $z_n$ che soddisfa le ipotesi del teorema \ref{boundary_schwarz_pick}, da cui la tesi.
\end{proof}

Aggiungere controesempio


\newpage

\begin{thebibliography}{widest entry}
  \bibitem[BK]{BK} D. M. Burns, S. G. Krantz: Rigidity of holomorphic mappings and a new Schwarz lemma at the boundary. \textit{Journal of the American Mathematical Society}, \textbf{Volume 7} (1994), no. 3, 661--676
  \bibitem[BKR]{BKR} F. Bracci, D. Kraus, O. Roth: A new Schwarz-Pick Lemma at the boundary and rigidity of holomorphic maps. Preprint, ArXiv:2003.02019v1 (2020)
  \bibitem[BM]{BM} A. F. Beardon, D. Minda: A multi-point Schwarz-Pick lemma. \textit{Journal d'Analyse Mathématique}, \textbf{Volume 92} (2004), 81--104
  \bibitem[NN]{NN} N. Narasimhan, Y. Nievergelt: \textbf{Complex analysis in one variable (2nd edition)}. Springer, INSERIRE CITTÀ, 2001
\end{thebibliography}


\section*{Ringraziamenti}
\addcontentsline{toc}{section}{Ringraziamenti}
Un sentito grazie al professor Marco Abate, che è riuscito a tenere il suo interessantissimo corso ugualmente appassionante anche con la didattica a distanza.


\end{document}
