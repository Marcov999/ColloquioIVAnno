\begin{frame}[t]
  \frametitle{Teorema di Bracci-Kraus-Roth}
  \begin{thm}
    (Bracci-Kraus-Roth, 2020) Sia $f \in \text{\normalfont{Hol}}(\mathbb{D},\mathbb{D})$ tale che
    \begin{equation}
      |f^h(z_n)|=1+o\bigl((|z_n|-1)^2\bigr)
    \end{equation}
    per qualche successione $\{z_n\}_{n \in \mathbb{N}} \subset \mathbb{D}$ con $|z_n| \longrightarrow 1$. Allora $f \in \text{\normalfont{Aut}}(\mathbb{D})$.
  \end{thm}
  \pause
  \begin{ex}
    Prendiamo di nuovo $f(z)=\dfrac{1+3z^2}{3+z^2}$. Si ha $|f^h(z)|=\dfrac{2|z|}{1+|z|^2}$, perciò $\displaystyle \lim_{|z| \longrightarrow 1} \frac{|f^h(z)|-1}{(|z|-1)^2}=-\frac{1}{2}$.
  \end{ex}
\end{frame}

\begin{frame}[t]
  \frametitle{Risultato sui limiti non tangenziali}
  \only<1-3>{\begin{prop}
    Siano $f \in \text{\normalfont{Hol}}(\mathbb{D},\mathbb{D})$ e $\sigma \in \partial\mathbb{D}$ tali che
    \begin{equation*}
      f(z)=\sigma+(z-\sigma)+o\bigl((z-\sigma)^3\bigr)
    \end{equation*}
    per $z \longrightarrow \sigma$ non tangenzialmente. Allora
    \begin{equation*}
      |f^h(z)|=1+o\bigl((z-\sigma)^2\bigr)
    \end{equation*}
    per $z \longrightarrow \sigma$ non tangenzialmente.
  \end{prop}}
  \only<2>{\textit{Traccia della dimostrazione:} il punto è riuscire a stimare $f'$. Senza perdita di generalità $\sigma=1$. Dalla formula di Cauchy troviamo
  $$f'(z)=1+\frac{1}{2\pi i} \int_{C(z)} \frac{f(w)-w}{(w-z)^2}\diff w=:1+I(z).$$}
  \only<3>{All'interno delle regioni di Stolz, con ragionamenti geometrici si possono fare stime per dire che $I(z)=o\bigl((z-1)^2\bigr)$. \qed}
\end{frame}

\begin{frame}[t]
  \frametitle{Teorema di Burns-Krantz}
  \only<1->{\begin{thm}
    (Burns-Krantz, 1994) Siano $f \in \text{\normalfont{Hol}}(\mathbb{D},\mathbb{D})$ e $\sigma \in \partial\mathbb{D}$ tali che
    \begin{equation}
      f(z)=\sigma+(z-\sigma)+o\bigl((z-\sigma)^3\bigr)
    \end{equation}
    per $z \longrightarrow \sigma$ non tangenzialmente. Allora $f$ è l'identità del disco.
  \end{thm}}
  \only<2-3>{\textit{Traccia della dimostrazione:} senza perdita di generalità $\sigma=1$. Dalla Proposizione sui limiti non tangenziali segue che
  $$|f^h(z)|=1+o\bigl((z-1)^2\bigr)$$
  per $z \longrightarrow 1$ non tangenzialmente.}

  \only<3>{Per il teorema di Bracci-Kraus-Roth, $f \in \text{Aut}(\mathbb{D})$; per ipotesi dev'essere $f(1)=1$ e $f''(1)=0$, perciò $f(z)=z$. \qed}
\end{frame}
