\documentclass{article}
\usepackage{mstyle}

\title{Appunti di Elementi di Analisi Complessa}
\date{}
\author{Marco Vergamini}

\begin{document}
\maketitle
\newpage
\tableofcontents
\newpage


\section{Introduzione}
Da scrivere alla fine


\newpage

\section{Funzioni olomorfe in una variabile}

\subsection{Notazioni e prerequisiti}
Notazioni: $z=x+iy \in \mathbb{C}$ indica un numero complesso, $\bar{z}=x-iy$ il suo complesso coniugato. Con il termine \textit{dominio} si intende un aperto connesso. $\mathcal{O}(\Omega)=\{f: \Omega \rightarrow \mathbb{C} | f \text{ è olomorfa} \}$.
$\text{Hol}(\Omega_1, \Omega_2)=\{f: \Omega_1 \rightarrow \Omega_2 | f \text{ è olomorfa} \}$.
$\dfrac{\partial}{\partial z}=\dfrac{1}{2}\left(\dfrac{\partial}{\partial x}-i\dfrac{\partial}{\partial y}\right), \dfrac{\partial}{\partial \bar{z}}=\dfrac{1}{2}\left(\dfrac{\partial}{\partial x}+i\dfrac{\partial}{\partial y}\right)$. \\
$\diff z=\diff x+i\diff y, \diff\bar{z}=\diff x-i\diff y, \diff z\left(\dfrac{\partial}{\partial z}\right)=1, \diff z\left(\dfrac{\partial}{\partial \bar{z}}\right)=0$. \\

Daremo ora una definizione di funzione olomorfa basata su quattro definizioni, l'equivalenza delle quali è un prerequisito del corso e dovrebbe essere quindi nota agli studenti.

\begin{defn}
  Sia $\Omega \subseteq \mathbb{C}$ un aperto, $f: \Omega \in \mathbb{C}$ si dice \textsc{olomorfa} se vale una delle seguenti condizioni equivalenti:
  \begin{nlist}
      \item $f$ è \textit{$\mathbb{C}$-differenziabile}, cioè per ogni $a \in \Omega$ esiste $\displaystyle f'(a)=\lim_{z \rightarrow a}=\frac{f(z)-f(a)}{z-a}$;
      \item $f$ è \textit{analitica}, cioè per ogni $a \in \Omega$ esiste $U \subseteq \Omega$ aperto e intorno di $a$ e $\{c_n\}\subset \mathbb{C}$ t.c. per ogni $z \in U$ $\displaystyle f(z)=\sum_{n=0}^{+\infty} c_n(z-a)^n$;
      \item $f$ è \textit{olomorfa}, cioè $f$ è continua, $\partial f/\partial x$ e $\partial f/\partial y$ esistono su $\Omega$ e $\dfrac{\partial f}{\partial x}+i\dfrac{\partial f}{\partial y} \equiv 0$ (equazione di Cauchy-Riemann).
      Si noti che la condizione è $\dfrac{\partial f}{\partial \bar{z}} \equiv 0$, da cui si ricava $\dfrac{\partial f}{\partial z}=f'$;
      \item $f$ è continua e per ogni rettangolo (o disco) chiuso $D \subseteq \Omega$ si ha $\displaystyle \int_{\partial D} f\diff z=0$ (teorema di Cauchy-Goursat+Morera).
  \end{nlist}
\end{defn}

La seguente proposizione è anch'essa un risultato che dovrebbe essere noto agli studenti che seguono il corso.

\begin{prop}
  Sia $\{c_n\} \in \mathbb{C}$. Allora:
  \begin{nlist}
    \item esiste $R \in [0, +\infty]$ t.c. $\displaystyle \sum_{n=0}^{+\infty} c_nz^n$ converge per $|z|<R$ e diverge per $|z|>R$. $R$ è detto \textit{raggio di convergenza}. La convergenza +è uniforme su $\Delta_r=\{|z| \le r\}, r<R$. $\displaystyle \limsup_{n \rightarrow +\infty} |c_n|^{1/n}=\frac{1}{R}$;
    \item $\displaystyle \sum_{n=0}^{+\infty} nc_nz^{n-1}$ ha lo stesso raggio di convergenza;
    \item se $\displaystyle f(z)=\sum_{n=0}^{+\infty} c_n(z-a)^n$ allora $f'(z)=\sum_{n=0}^{+\infty} nc_n(z-a)^{n-1}$;
    \item se $f \in \mathcal{O}(\Omega)$ e $a \in \Omega$, allora $\displaystyle f(z)=\sum_{n=0}^{+\infty} \frac{1}{n!}f^{(n)}(a)(z-a)^n$. Questa formula è valida nel più grande disco aperto centrato in $a$ e contenuto in $\Omega$, cioè di raggio minore o uguale di $d(a, \partial \Omega)$.
  \end{nlist}
\end{prop}

\begin{thm}
  (Formula di Cauchy) Sia $\Omega$ aperto, $f \in \mathcal{O}(\Omega), D \subseteq \Omega$ disco/rettangolo chiuso. Per ogni $\displaystyle a \in D, f(a)=\frac{1}{2\pi i} \int_{\partial D} \frac{\zeta}{\zeta-a} \diff\zeta$.
  Si ha che $\displaystyle f^{(n)}(a)=\frac{n!}{2\pi i} \int_{\partial D} \frac{f(\zeta)}{(\zeta-a)^{n+1}}\diff \zeta$.
\end{thm}

\begin{cor}
  (Disuguaglianze di Cauchy) $f \in \mathcal{O}(\Omega), D=D(a, r) \subseteq \Omega$ disco di centro $a \in \Omega$ e raggio $r>0$. Sia $\displaystyle M=\max_{\zeta \in \partial D} |f(\zeta)|$. Allora per ogni $n \ge 1, |f^{(n)}(a)| \le \dfrac{n!}{r^n}M$.
\end{cor}

\begin{cor}
  (Teorema di Liouville) Sia $f \in \mathcal{O}(\mathbb{C})$ limitata. Allora $f$ è costante.
\end{cor}

\begin{proof}
  Le disuguaglianze di Cauchy danno, per ogni $r>0$, $|f'(a)| \le \dfrac{M}{r}$ dove $\displaystyle M=\sup_{z \in \mathbb{C}} |f(z)|<+\infty \implies
  f' \equiv 0$.
\end{proof}

\begin{thm}
  (Principio di identità o del prolungamento analitico) $\Omega \subseteq \mathbb{C}$ dominio, $f, g \in \mathcal{O}(\Omega)$. Se $\{z \in \Omega | f(z)=g(z)\}$ ha un punto di accumulazione in $\Omega$, allora $f \equiv g$.
\end{thm}

\begin{cor} \label{olo_discr}
  $\Omega \subseteq \mathbb{C}$ dominio, $f \in \mathcal{O}(\Omega)$ non identicamente nulla, allora $\{z \in \Omega | f(z)=0\}$ è discreto in $\Omega$.
\end{cor}

\begin{thm} \label{pr_max}
  (Principio del massimo) $\Omega \subseteq \mathbb{C}$ dominio, $f \in \mathcal{O}(\Omega)$. Allora:
  \begin{nlist}
    \item se $U$ è aperto e $U \subset \subset \Omega$ (si legge "$U$ relativamente compatto in $\Omega$" e si intende $\overline{U} \subset \Omega$ e $\overline{U}$ compatto) allora $\displaystyle \sup_{z \in U} |f(z)| \le \sup_{z \in \partial D} |f(z)|$. Inoltre, se $|f|$ ha un massimo locale in $U$, allora $f$ è costante in $\Omega$;
    \item la stessa affermazione vale per $\mathfrak{Re} f$ e $\mathfrak{Im} f$;
    \item se $\Omega$ è limitato poniamo $\displaystyle M=\sup_{x \in \partial D} \limsup_{z \rightarrow x} |f(z)| \in [0, +\infty]$. Allora per ogni $z \in \Omega$ $|f(z)| \le M$ con uguaglianza in un punto se e solo se $f$ è costante.
  \end{nlist}
\end{thm}

\begin{ex}
  Controesempio per vedere che serve $\Omega$ limitato per il punto (iii) del teorema \ref{pr_max}: $\Omega=\{z \in \mathbb{C} | \mathfrak{Re} z>0\}, f(z)=e^z$. $f \in \mathcal{O}(\mathbb{C}) \subset \mathcal{O}(\Omega)$.
  $z \in \partial\Omega \implies z=iy \implies |f(iy)|=|e^{iy}|=1$, ma $f$ è illimitata in $\Omega$. Per correggere questa cosa si aggiunge il punto all'infinito.
\end{ex}

\begin{thm}
  (Applicazione aperta) $f \in \mathcal{O}(\Omega)$ non costante $\implies$ $f$ è un'applicazione aperta.
\end{thm}

Siano $X, Y$ spazi topologici e indichiamo con $C^0(X, Y)$ le funzioni continue da $X$ in $Y$.

La \textit{topologia della convergenza puntuale} è la restrizione a $C^0(X, Y) \subset Y^X=\{f:X \rightarrow Y\}$ della topologia prodotto. Una prebase è data da $\mathcal{F}(x, U)=\{f \in C^0(X, Y) |f(x) \in U\}$ dove $x \in X$ e $U \subseteq Y$ è un aperto.

\begin{exc}
  $f_n \rightarrow f \in C^0(X, Y)$ per questa topologia se e solo se $f_n(x) \rightarrow f(x)$ per ogni $x \in X$.
\end{exc}

La \textit{topologia compatta-aperta} ha invece come prebase $\mathcal{F}(K, U)= \\ =\{f \in C^0(X, Y) | f(K) \subseteq U\}$ dove $U$ è preso come sopra e $K \subseteq X$ è un compatto.

\begin{prop}
  \begin{nlist}
    \item La topologia compatta-aperta è più fine della topologia della convergenza puntuale;
    \item $Y$ Hausdorff $\implies$ topologia compatta aperta Hausdorff.
  \end{nlist}
\end{prop}

\begin{proof}
  \begin{nlist}
    \item Ovvia (il singoletto è un compatto).
    \item Prendiamo $f \not\equiv g$ continue, allora esiste $x_0 \in X$ t.c. $f(x_0) \not= g(x_0)$, per cui, dato che $Y$ è Hausdorff,
    esistono $U, V \subset Y$ aperti disgiunti con $f(x_0) \in U, g(x_0) \in V \implies f \in \mathcal{F}(x_0, U), g \in \mathcal{F}(x_0, V), \mathcal{F}(x_0, U) \cap \mathcal{F}(x_0, V)=\emptyset$.
  \end{nlist}
\end{proof}

\begin{thm}
  (Ascoli-Arzelà) Siano $X, Y$ spazi metrici con $X$ localmente compatto, allora $\mathcal{F} \subseteq C^0(X, Y)$ è relativamente compatta rispetto alla topologia compatta-aperta se e solo se:
  \begin{nlist}
    \item per ogni $x \in X$ $\{f(x) | f \in \mathcal{F}\} \subset \subset Y$;
    \item $\mathcal{F}$ è equicontinua.
  \end{nlist}
\end{thm}

La topologia compatta-aperta viene detta anche topologia della \textit{convergenza uniforme sui compatti}: $\{f_n\} \subset C^0(X, \mathbb{R}^N)$. Se $K \subseteq X$ definiamo $\displaystyle \|f\|_K=\sup_{z \in K} \|f(z)\|$.
$f_n \rightarrow f$ uniformemente sui compatti se per ogni $K \subset \subset X$ compatto e per ogni $\epsilon>0$ esiste $n_0$ t.c. $n \ge n_0 \implies \|f_n-f\|_K<\epsilon$.

\begin{exc}
  $f_n \rightarrow f$ uniformemente sui compatti se e solo se $f_n \rightarrow f$ nella topologia compatta-aperta.
\end{exc}


\subsection{Risultati preliminari}
Vediamo ora alcuni risultati e definizioni preliminari, da considerarsi comunque come prerequisiti per altri risultati più interessanti che vedremo più avanti nel corso.

\begin{thm}
  (Weierstrass) Sia $\{f_n\} \subset \mathcal{O}(\Omega)$ t.c. $f_n \rightarrow f \in C^0(\Omega, \mathcal{C})$ uniformemente sui compatti. Allora:
  \begin{nlist}
    \item $f \in \mathcal{O}(\Omega)$;
    \item $f_n' \rightarrow f'$ uniformemente sui compatti.
  \end{nlist}
\end{thm}

\begin{proof}
  \begin{nlist}
    \item Sia $a \in \Omega$, $0<r<d(a, \partial\Omega)$ t.c. $D=D(a, r) \subset \subset \Omega$. $\displaystyle f_n(z)=\frac{1}{2\pi i} \int_{\partial D} \frac{f_n(\zeta)}{\zeta-z} \diff \zeta$ per ogni $z \in D(a, \rho)$ per ogni $0<\rho<r$.
    Allora $\dfrac{1}{|\zeta-z|} \le \dfrac{1}{r-\rho}$ per ogni $z \in D(a, \rho), \zeta \in \partial{D}$.
    Per ogni $\displaystyle z \in D(a, \rho), f(z)=\lim_{n \rightarrow +\infty} \frac{1}{2\pi i} \int_{\partial D} \frac{f_n(\zeta)}{\zeta-z} \diff \zeta$.
    Adesso, per uniforme convergenza e uniforme limitatezza si può portare il limite dentro, perciò $\displaystyle f(z)=\frac{1}{2\pi i}\int_{\partial D} \frac{f(\zeta)}{\zeta-z} \diff \zeta$, ma questo, per il teorema di Cauchy-Goursat+Morera, implica $f \in \mathcal{O}(\Omega)$.
    \item $\displaystyle f_n'(z)=\frac{1}{2\pi i} \int_{\partial D} \frac{f_n'(\zeta)}{(\zeta-z)^2}\diff\zeta \rightarrow \frac{1}{2\pi i} \int_{\partial D} \frac{f(\zeta)}{\zeta-z} \diff\zeta=f'(z)$.
    $f_n' \rightarrow f'$ uniformemente sui dischi e ogni compatto è coperto da un numero finito di dischi $\implies$ $f_n' \rightarrow f'$ uniformemente sui compatti.
  \end{nlist}
\end{proof}

\begin{thm}
  (Montel) $\Omega \subseteq \mathbb{C}$ aperto, $\mathcal{F} \subseteq \mathcal{O}(\Omega)$ t.c. per ogni $K \subset \subset C$ compatto esiste $M_K>0$ t.c. $\|f\|_K \le M_K$ per ogni $f \in \mathcal{F}$ (si diche che $\mathcal{F}$ è \textit{uniformemente limitata sui compatti}).
  Allora $\mathcal{F}$ è relativamente compatta in $\mathcal{O}(\Omega)$.
\end{thm}

\begin{proof}
  Basta vedere che ogni successione $\{f_n\} \subseteq \mathcal{F}$ ha una sottosuccessione convergente (segue dal fatto che, nelle ipotesi del teorema di Montel, la topologia compatta aperta è metrizzabile). \\
  Dati $a \in \Omega, 0<r<d(a, \partial \Omega), f \in \mathcal{O}(\Omega)$, sia $c_n(f)=\dfrac{f^{(n)}(a)}{n!}$, allora $\displaystyle f(z)=\sum_{n=0}^{+\infty} c_n(f)(z-a)^n$ in $\overline{D(a, r)}$.
  Inoltre, se $\|f\|_{\overline{D(a, r)}} \le M$, allora per le disuguaglianze di Cauchy $|c_n(f)| \le \dfrac{M}{r^n}$ per ogni $n \ge 0$.
  Sia $\{f_n\} \subseteq \mathcal{F}$. Per ipotesi, esiste $M$ t.c. $\|f_n\|_{\overline{D(a, r)}} \le M$ per ogni $n$ $\implies$ $|c_0(f_n)| \le M$ per ogni $n$ $\implies$ esiste una sottosuccessione $c_0(f_{n_j^{(0)}})$ che tende a $c_0 \in \mathbb{C}$.
  Per  induzione, da $\{f_{n_j^{(k-1)}}\}$ possiamo estrarre una sottosuccessione $\{f_{n_j^{(k)}}\}$ t.c. $c_k(f_{n_j^{(k)}}) \rightarrow c_k \in \mathbb{C}$. Consideriamo $\{f_{n_j^{(j)}}\}$, allora $c_k(f_{n_j^{(j)}}) \rightarrow c_k \in \mathbb{C}$ per ogni $k$.
  Sia $f_{\nu_j}=f_{n_j^{(j)}}$. Poniamo $D_a=\overline{D(a, r/2)}$ e sia $z \in D_a$. Vogliamo $\displaystyle f_{\nu_j} \rightarrow f(z)=\sum_{n=0}^{+\infty} c_n(z-a)^n$ in $D_a$. Basta vedere che $f_{\nu_j}$ è di Cauchy uniformemente in $D_a$.
  $\displaystyle |f_{\nu_h}(z)-f_{\nu_k}(z)| \le \sum_{n=0}^{+\infty} |c_n(f_{\nu_h})-c_n(f_{\nu_k})||z-a|^n=\sum_{n=0}^N |c_n(f_{\nu_h})-c_n(f_{\nu_k})||z-a|^n+\sum_{n>N} |c_n(f_{\nu_h})-c_n(f_{\nu_k})||z-a|^n$.
  Sappiamo che $|c_n(f_{\nu_k})| \le \dfrac{M}{r^n}$ e $z \in D_a \implies |z-a| \le \dfrac{r}{2}$.
  Allora $\displaystyle \sum_{n>N} |c_n(f_{\nu_h})-c_n(f_{\nu_k})||z-a|^n \le \sum_{n>N} \frac{2M}{r^n}\left(\frac{r}{2}\right)^n=\frac{M}{2^{N-1}}$.
  $\displaystyle \sum_{n=0}^{+\infty} |c_n(f_{\nu_h})-c_n(f_{\nu_k})||z-a|^n \le \sum_{n=0}^{+\infty} |c_n(f_{\nu_h})-c_n(f_{\nu_k})|\left(\dfrac{r}{2}\right)^n$.
  Dato $\epsilon>0$, scegliamo $N>>1$ t.c. $\dfrac{M}{2^{N-1}}<\epsilon/2$ e $n_0$ t.c. per ogni $h, k \ge n_0, |c_n(f_{\nu_h})-c_n(f_{\nu_k})|\left(\dfrac{r}{2}\right)^n<\dfrac{\epsilon}{2(N+1)}$
  (possiamo farlo, una volta fissato $N$, perché gli $n$ tra $0$ e $N$ sono in numero finito e le successioni $c_n(f_{\nu_j})$ convergono, dunque si sceglie un indice per ogni successione e si prende come $n_0$ il massimo di questi indici).
  Mettendo insieme le disuguaglianze si ha che per ogni $\epsilon>0$ esiste $n_0$ t.c. per ogni $h, k \ge n_0$ e per ogni $z \in D_a$, $|f_{\nu_k}(z)-f_{\nu_k}(z)|<\epsilon$, dunque la sottosuccessione $f_{\nu_j}$ è di Cauchy e converge uniformemente su $D_a$.
  Deve convergere a $f$ perché, per il teorema di Weierstrass, le derivate convergono al valore della derivata limite, e questo ci dice che i coefficienti della serie della funzione limite sono proprio quelli di $f$. \\
  $\Omega$ è a base numerabile, dunque possiamo estrarre un sottoricoprimento numerabile da $\{D_a | a \in \Omega\}$. Sia dunque $\{a_j\} \subseteq \Omega$ t.c. $\displaystyle \bigcup_j D_{a_j}=\Omega$. Per quanto dimostrato finora, possiamo estrarre da $\{f_n\}$ una sottosuccessione $\{f_{n_j^{(0)}}\}$ convergente uniformemente in $D_{a_0}$.
  Per induzione, da $\{f_{n_j^{(k-1)}}\}$ estraiamo una sottosuccessione $\{f_{n_j^{(k)}}\}$ convergente uniformemente in $D_{a_0} \cup \dots \cup D_{a_k}$. Prendiamo $\{f_{n_j^{(j)}}\}$ che converge uniformemente in ogni $D_{a_k}$.
  Adesso, ogni compatto è coperto da un numero finito di $D_{a_k}$, quindi (scegliendo per ogni $\epsilon$ il massimo degli indici t.c. le cose che vogliamo valgono in quei $D_{a_k}$) $\{f_{n_j^{(j)}}\}$ converge uniformemente sui compatti.
\end{proof}

\begin{thm}
  (Vitali) $\Omega \subseteq \mathbb{C}$ dominio, $A \subseteq \Omega$ con almeno un punto di accumulazione in $\Omega$. Sia $\{f_n\} \subset \mathcal{O}(\Omega)$ uniformemente limitata sui compatti. Supponiamo che, per ogni $a \in A$, $\{f_n(a)\}$ converge (cioe $f_n$ converge puntualmente). Allora esiste $f \in \mathcal{O}(\Omega)$ t.c. $f_n \rightarrow f$ uniformemente sui compatti di $\Omega$.
\end{thm}

\begin{proof}
  Facciamola per assurdo. Supponiamo che esistono $K \subset\subset \Omega, \\ \{n_k\}, \{m_k\} \subset \mathbb{N}, \{z_k\} \subset K, \delta>0$ t.c. $|f_{n_k}(z_k)-f_{m_k}(z_k)| \ge \delta$. A meno di sottosuccessioni, $z_k \rightarrow z_0 \in K$.
  Per il teorema di Montel, a meno di sottosuccessioni $f_{n_k} \rightarrow g_1 \in \Omega$ e $f_{m_k} \rightarrow g_2 \in \Omega$ con $|g_1(z_0)-g_2(z_0)| \ge \delta$ (passando al limite). Per ipotesi,  $g_1(a)=g_2(a)$ per ogni $a \in A$. Per il principio di identità, $g_1 \equiv g_2$, assurdo.
\end{proof}

\begin{thm}
  (Sviluppo di Laurent) Siano $0 \le r_1 < r_2 \le +\infty, A(r_1, r_2):=\{z \in \mathbb{C} | r_1 < |z| < r_2 \}$. Sia $f \in \mathcal{O}(A(r_1, r_2))$, allora $\displaystyle f(z)=\sum_{n=-\infty}^{+\infty} c_nz^n$ e converge uniformemente e assolutamente sui compatti di $A(r_1, r_2)$.
  In particolare, se $\Omega \subseteq \mathbb{C}$ aperto, $a \in \Omega$ e $\displaystyle f \in \mathcal{O}(\Omega \setminus \{a\}), f(z)=\sum_{n=-\infty}^{+\infty} c_n(z-a)^n$ in $\{0<|z-a|<r\} \subset \Omega$.
\end{thm}

\begin{cor}
  (Teorema di estensione di Riemann) $f \in \mathcal{O}(\Omega \setminus \{a\})$ si estende olomorficamente ad $a$ $\Leftrightarrow$ $\displaystyle \lim_{z \rightarrow a} (z-a)f(z)=0$.
\end{cor}

\begin{proof}
  Per lo sviluppo di Laurent, $\displaystyle (z-a)f(z)=\sum_{n=-\infty}^{+\infty} c_n(z-a)^{n+1} \rightarrow 0 \Leftrightarrow c_n=0$ per ogni $n \le -1$.
\end{proof}

\begin{thm} \label{biolo}
  \begin{nlist}
    \item $f \in \text{Hol}(\Omega, \Omega_1)$ biettiva $\implies$ $f^{-1}$ è olomorfa e $f'$ non si annulla mai;
    \item $f \in \mathcal{O}(\Omega)$ t.c. $f'(z_0) \not=0$ $\implies$ $f$ è iniettiva vicino a $z_0$.
  \end{nlist}
\end{thm}

\begin{proof}
  \begin{nlist}
    \item Per il teorema dell'applicazione aperta, $f$ è aperta $\implies$ $f$ omeomorfismo. $g=f^{-1}$. Sia $w_0 \in \Omega_1$ t.c. $f'(g(w_0)) \not=0$. Allora
    $$ \frac{g(w)-g(w_0)}{w-w_0}=\frac{1}{\frac{w-w_0}{g(w)-g(w_0)}}=\frac{1}{\frac{f(g(w))-f(g(w_0))}{g(w)-g(w_0)}}=\frac{1}{f'(g(w_0))}. $$
    Quindi $g$ è olomorfa in $\Omega_1 \setminus f(\{f'=0\})$. Per il corollario \ref{olo_discr} $\{f'=0\}$ è discreto in $\Omega$. $f$ omeomorfismo $\implies$ $f(\{f'=0\})$ discreto in $\Omega_1$.
    Ma $g$ è continua (quindi localmente limitata) in $\Omega_1$, dunque per il teorema di estensione di Riemann $g \in \mathcal{O}(\Omega_1)$. $(f' \circ g)g' \equiv 1$ su $\Omega_1 \setminus f(\{f'=0\})$ $\implies$ vale su $\Omega_1$ $\implies$ $f'\circ g \not=0$ sempre.
    \item Possiamo supporre $z_0=0$. $\displaystyle f(z)=\sum_{n=0}^{+\infty} c_nz^n$. Per ipotesi, $c_1 \not=0$. \\
    $\displaystyle f(z)-f(w)=c_1(z-w)+(z-w)\sum_{n=2}^{+\infty} c_n\sum_{k=1}^n w^{k-1}z^{n-k}$. \\
    $\displaystyle |f(z)-f(w)| \ge |c_1||z-w|-|z-w|\sum_{n=2}^{+\infty} |c_n|\sum_{k=1}^n |w|^{k-1}|z|^{n-k}$. \\
    Prendiamo $z, w \in D(0, r)$, allora \\
    $\displaystyle |c_1||z-w|-|z-w|\sum_{n=2}^{+\infty} |c_n|\sum_{k=1}^n |w|^{k-1}|z|^{n-k} \ge \\
    \ge |c_1||z-w|-|z-w|\sum_{n=2}^{+\infty} |c_n|nr^{n-1}=(|c_1|-\sum_{n=2}^{+\infty} |c_n|nr^{n-1})|z-w|$.
    Scegliamo $r$ t.c. $\displaystyle \sum_{n=2}^{+\infty} |c_n|nr^{n-1} \le \frac{|c_1|}{2}$, allora $\displaystyle (|c_1|-\sum_{n=2}^{+\infty} |c_n|nr^{n-1})|z-w| \ge \frac{|c_1|}{2}|z-w|$.
    Dato che $c_1 \not=0$, si ha quindi (concatenando le disuguaglianze) che $z \not=w \implies |f(z)-f(w)| \ge \dfrac{|c_1|}{2}|z-w|>0 \implies f(z) \not= f(w)$.
  \end{nlist}
\end{proof}

\begin{defn}
  Se $f: \Omega_1 \rightarrow \Omega_2$ è olomorfa e biettiva (quindi con inversa olomorfa per il teorema \ref{biolo}) si chiama \textsc{biolomorfismo}.
\end{defn}

\begin{defn}
  $f:\Omega_1 \rightarrow \mathbb{C}$ è un \textsc{biolomorfismo locale} se ogni $a \in \Omega_1$ ha un intorno $U \ni a$ t.c. $f\restrict{U}:U \rightarrow f(U)$ è un biolomorfismo.
\end{defn}

Per il teorema \ref{biolo}, $f$ è un biolomorfismo locale se e solo se $f'$ non si annulla mai.

\begin{defn}
  Sia $\displaystyle f(z)=\sum_{n=-\infty}^{+\infty} c_n(z-a)^n$ in $D^*=D(a, r) \setminus \{a\}$. $ord_a(f):=\inf\{n \in \mathbb{Z} | c_n \not=0\}$ è detto \textsc{ordine di $f$ in $a$}. $ord_a(f) \ge 0 \Leftrightarrow f$ è olomorfa in $a$.
  Se $0>ord_a(f)>-\infty$ diremo che $a$ è un \textsc{polo} di $f$. Se $ord_a(f)=-\infty$ $a$ è una \textsc{singolarità essenziale}.
\end{defn}

\begin{thm}
  (Casorati-Weierstrass) Se $a$ è una singolarità essenziale, $f(D^*)$ è denso in $\mathbb{C}$.
\end{thm}

\begin{defn}
  $c_{-1}=:res_f(a)$ è detto \textsc{residuo di $f$ in $a$}.
\end{defn}

\begin{oss} \label{int_res}
  $\gamma(t)=a+\rho e^{2\pi i t}, 0<\rho<r$. \\
  $\displaystyle \frac{1}{2\pi i} \int_{\gamma} f \diff z=\frac{1}{2\pi i} \sum_{n=-\infty}^{+\infty} c_n \int_{\gamma} (z-a)^n \diff z= \\ \frac{1}{2\pi i} \sum_{n=-\infty}^{+\infty} c_n \int_0^1 \rho^ne^{2\pi i n t} \rho 2 \pi i e^{2\pi i t} \diff t=\sum_{n=-\infty}^{+\infty} c_n \rho^{n+1} \int_0^1 e^{2\pi i (n+1)t} \diff t=c_{-1}$.
\end{oss}

\begin{prop}
  $\Omega \subseteq \mathbb{C}$ aperto, $E \subset \Omega$ discreto e chiuso in $\Omega$, $D \subset \subset \Omega$ disco chiuso t.c. $E \cap \partial D=\emptyset$, $f\in \mathcal{O}(\Omega \setminus E)$.
  Allora $\displaystyle \frac{1}{2\pi i}\int_{\partial D} f \diff z=\sum_{a \in D \cap E} res_f(a)$.
\end{prop}

\begin{proof}
  Traccia: si dimostra che $E \cap D$ è finito e si applica una versione leggermente più forte del teorema di Cauchy-Goursat+Morera, prendendo per ogni punto di $E$ un dischetto tutto contenuto in $D$ che lo isoli dagli altri e considerando la regione $D$ meno quei dischetti. Il bordo di questa regione è considerato il bordo di $D$ meno il bordo dei dischetti. Questo bordo, a meno di aggiungere dei tratti lineari che uniscono una circonferenza all'altra (che quindi verranno percorsi in entrambi i sensi nell'integrale e non daranno contributo), è percorribile con un solo cammino omotopo al cammino costante in $\Omega \setminus E$, il cui integrale fa $0$ per la versione forte del teorema di C-G+M, dunque l'integrale sul bordo di $D$ meno l'integrale sul bordo dei dischetti (occhio al verso di percorrenza di uno e degli altri!) deve essere uguale a $0$. Per l'osservazione \ref{int_res} si ha la tesi.
\end{proof}

\begin{oss} \label{wn}
  $\gamma:[0, 1] \rightarrow \mathbb{C}$ chiusa ($\gamma(0)=\gamma(1))$, $a \not\in \gamma([0, 1])$. $p_s: \mathbb{C} \rightarrow \mathbb{C} \setminus \{a\}$, $p_a(z)=a+e^z$ è un rivestimento.
  \begin{center}
    \begin{tikzcd}
            & \mathbb{C} \arrow[d, "p_a"]\\
            \left[0,1\right] \arrow[ru, "\tilde{\gamma}"] \arrow[r, "\gamma"'] & \mathbb{C} \setminus \{a\}
     \end{tikzcd}
  \end{center}
  Sia $\tilde{\gamma}$ un sollevamento di $\gamma$ rispetto a $p_a$, $p_a(\tilde{\gamma}(1))=\gamma(1)=\gamma(0)=p_a(\tilde{\gamma}(0)) \iff e^{\tilde{\gamma}(1)}=e^{\tilde{\gamma}(0)} \iff \tilde{\gamma}(1)-\tilde{\gamma}(0) \in 2\pi i \mathbb{Z}$.
\end{oss}

\begin{defn}
  L'\textsc{indice di avvolgimento $\gamma$ rispetto ad $a$} (\textit{winding number} in inglese) è dato dall'osservazione \ref{wn}: $n(\gamma, a):=\dfrac{1}{2\pi i}(\tilde{\gamma}(1)-\tilde{\gamma}(0)) \in \mathbb{Z}$.
\end{defn}

\begin{thm}
  \begin{nlist}
    \item $n(\gamma, a)$ dipende solo da $a$ e da $\gamma$ e non dal sollevamento scelto;
    \item $n(\gamma, a) \in \mathbb{Z}$;
    \item $\displaystyle n(\gamma, a)=\frac{1}{2\pi i}\int_{\gamma} \frac{1}{z-a} \diff z$;
    \item $a \mapsto n(\gamma, a)$ è costante sulle componente connesse di $\mathbb{C} \setminus \gamma([0, 1])$. In particolare $n(\gamma, a)=0$ sulla componente connessa illimitata di $\mathbb{C} \setminus \gamma([0, 1])$;
    \item $\gamma(t)=a_0+re^{2\pi i t} \implies n(a, \gamma)=1$ per ogni $a \in D(a_0, r)$;
    \item $\gamma_1$ e $\gamma_2$ chiuse con $\gamma_1(0)=\gamma_2(0)=p_0$ omotope (tramite omotopia che fissa il punto base $p_0$) e $a \not\in \gamma_1([0, 1]) \cup \gamma_2([0, 1])$, se l'omotopia è in $\mathbb{C} \setminus \{a\}$ allora $n(\gamma_1, a)=n(\gamma_2, a)$.
  \end{nlist}
\end{thm}

\begin{thm}
  (Teorema dei residui) $\Omega \subseteq \mathbb{C}$ aperto, $E \subset \Omega$ discreto e chiuso in $\Omega$, $\gamma$ curva chiusa in $\Omega \setminus E$ omotopa a una costante in $\Omega$.
  Allora per ogni $f \in \mathcal{O}(\Omega \setminus E)$ $\displaystyle \frac{1}{2\pi i} \int_{\gamma} f \diff z=\sum_{a \in E} res_f(a) \cdot n(\gamma, a)$.
\end{thm}

\begin{defn}
  $\Omega \subseteq \mathbb{C}$, $f$ è \textsc{meromorfa} in $\Omega$ se esiste $E \subset \Omega$ discreto e chiuso in $\Omega$ t.c. $f \in \mathcal{O}(\Omega \setminus E)$ e nessun punto di $E$ è una singolarità essenziale. Scriveremo che $f \in \mathcal{M}(\Omega)$.
\end{defn}

\begin{prop}
  \begin{nlist}
    \item $f \in \mathcal{O}(\Omega \setminus E)$ è meromorfa $\iff$ localmente è quoziente di due funzioni olomorfe;
    \item $f \in \mathcal{O}(\Omega \setminus E)$ è meromorfa $\iff$ per ogni $a \in E$ o $|f|$ è limitato vicino ad $a$ o $\displaystyle \lim_{z \rightarrow a} |f(z)|=+\infty$.
  \end{nlist}
\end{prop}

\begin{proof}
  \begin{nlist}
    \item ($\implies$) Se $a \in \Omega \setminus E$ banalmente $f=\dfrac{f}{1}$ vicino ad $a$. \\
    Se $a \in E$, $\displaystyle f(z)=\sum_{n \ge n_0} c_n(z-a)^n=(z-a)^{n_0}(c_{n_0}+h(z))$, $h$ olomorfa vicino ad $a$. Se $n_0<0$, $f(z)=\dfrac{c_{n_0}+h(z)}{(z-a)^{-n_0}}$.

    ($\Leftarrow$) Se $\displaystyle f(z)=\frac{h_1(z)}{h_2(z)}=\frac{\sum_{n \ge n_1} b_n(z-a)^n}{\sum_{m \ge n_2} c_m(z-a)^m}=(z-a)^{n_1-n_2}k(z)$, $k$ olomorfa vicino ad $a$.
    \item Per Casorati-Weierstrass, $a \in E$ è singolarità essenziale $\iff$ $\displaystyle \lim_{z \rightarrow a} |f(z)|$ non esiste. Per lo stesso motivo, è un polo $\iff$ $\displaystyle \lim_{z \rightarrow a} |f(z)|=+\infty$.
  \end{nlist}
\end{proof}

\begin{thm}
  (Principio dell'argomento) $\Omega \subseteq \mathbb{C}$, $f \in \mathcal{M}(\Omega)$. $Z_f:=\{\text{zeri di } f\}, P_f:=\{\text{poli di } f\}$. $\gamma$ curva chiusa in $\Omega \setminus (Z_f \cup P_f)$ omotopa a una costante in $\Omega$.
  Allora $\displaystyle \sum_{a \in Z_f \cup P_f} n(\gamma, a) \cdot ord_a(f)=\frac{1}{2\pi i}\int_{\gamma} \frac{f'}{f} \diff z$.
\end{thm}

\begin{proof}
  $ord_a(f)=res_{f'/f}(a)$. Infatti $f(z)=(z-a)^mh(z)$ con $m=ord_a(f)$, $h(a) \not=0$ e $h$ olomorfa. $f'(z)=m(z-a)^{m-1}h(z)+(z-a)^mh'(z)$. Allora $\dfrac{f'}{f}=\dfrac{m}{z-a}+\dfrac{h'(z)}{h(z)}$ e $\dfrac{h'}{h}$ è olomorfa $\implies$ $res_{f'/f}(a)=m=ord_a(f)$. La tesi segue allora dal teorema dei residui.
\end{proof}

\begin{prop}
  (Versione semplice del teorema di Rouché) $\Omega \subseteq \mathbb{C}$, $f, g \in \mathcal{O}(\Omega)$, $D$ disco con $\overline{D} \subset \Omega$. Supponiamo che $|f-g|<|g|$ su $\partial D$ (questo implica anche che non si annullano mai su $\partial D$). Allora $f$ e $g$ hanno lo stesso numero di zeri (contati con molteplicità) su $D$.
\end{prop}

\begin{proof}
  Per $t \in [0, 1]$ poniamo $f_t=g+t(f-g)$ ($f_0=g$, $f_1=f$). Se $z \in \partial D$, $0<|g(z)|-|f(z)-g(z)| \le |g(z)|-t|f(z)-g(z)| \le |f_t(z)|$. Sia $\displaystyle a_t=\sum_{a \in \overline{D}} ord_a(f_t)=\text{numero di zeri di } f_t \text{ in } \overline{D}$.
  Non ci sono poli, dunque $a_t \in \mathbb{N}$, quindi per il principio dell'argomento $\displaystyle a_t=\frac{1}{2\pi i} \int_{\partial D} \frac{f_t'}{f_t} \diff z=\frac{1}{2\pi i} \int_{\partial D} \frac{g'+t(f'-g')}{g+t(f-g)} \diff z$,
  che dipende con continuità da $t$ $\implies$ $a_t$ è costante (è a valori in $\mathbb{N}$) $\implies$ $a_0=a_1$ come voluto.
\end{proof}

\begin{cor}
  (Teorema di Ritt) Sia $h \in \mathcal{O}(\mathbb{D})$ ($\mathbb{D}:=\{|z|<1\}$) t.c. $h(\mathbb{D}) \subset \subset \mathbb{D}$. Allora $h$ ha un punto fisso.
\end{cor}

\begin{proof}
  Esiste $0<r<1$ t.c. $|h(z)|<r$ per ogni $z \in \mathbb{D}$. Sia $\mathbb{D}_r:=\{|z|<r\}$. Su $\partial \mathbb{D}_r$, $|z-(z-h(z))|=|h(z)|<r=|z|$.
  Per il teorema di Rouché su $g(z)=z, f(z)=z-h(z)$, $g$ e $f$ hanno lo stesso numero di zeri in $\mathbb{D}$, ma $g$ ha un unico zero $\implies$ $f=\id_{\mathbb{D}}-h$ ha un unico zero $z_0$ $\implies$ $h(z_0)=z_0$.
\end{proof}


\subsection{Teoremi di Hurwitz}
Vediamo ora qualche risultato interessante.

\begin{thm}
  (Primo teorema di Hurwitz) $\Omega \subseteq \mathbb{C}$ aperto, $\{f_n\} \subset \mathcal{O}(\Omega)$ convergente a $f\in \mathcal{O}(\Omega)$ uniformemente sui compatti. Supponiamo che $f$ non sia costante sulle componenti connesse di $\Omega$.
  Allora per ogni $z_0 \in \Omega$ esistono $n_1=n_1(z_0) \in \mathbb{N}$ e $z_n \in \Omega$ per ogni $n \ge n_1$ t.c. $f_n(z_n)=f(z_0)$ e $\displaystyle \lim_{n \longrightarrow +\infty} z_n=z_0$.
  Senza la tesi sul limite di $z_n$, si può dire che per ogni $w=f(z_0) \in f(\Omega)$ esiste $n_1=n_1(w)$ t.c. $w \in f_n(\Omega)$ per ogni $n \ge n_1$.
\end{thm}

\begin{proof}
  Vogliamo applicare Rocuhé a $f_n-w$ e $f-w$, $w=f(z_0)$ in dischetti centrati in $z_0$ di raggio arbitrariamente piccolo. $f$ non costante sulle componenti connesse $\implies$ $f^{-1}(w)$ è discreto $\implies$ esiste $\delta>0$ t.c. $0<|z-z_0| \le \delta$ $\implies$ $z \in \Omega$ e $f(z) \not=w$.
  Se $D=D(z_0, \delta)$ allora $\overline{D} \cap f^{-1}(w)=\{z_0\}$. Per ogni $k>0$, $\gamma_k=\partial D(z_0, \delta/k)$. Poniamo $\delta_k=\min\{|f(\zeta)-w| \mid \zeta \in \gamma_k\}>0$.
  Esiste $n_k \ge 1$ t.c. per ogni $n \ge n_k$ $\displaystyle \max_{\zeta \in \gamma_k} |f_n(\zeta)-f(\zeta)|<\frac{\delta_k}{2}$ ($f_n$ converge a $f$ uniformemente sui compatti). Possiamo supporre $n_1<n_2<n_3<\dots$.
  Fissato $k \ge 1$, se $n \ge n_k$ e $\zeta \in \gamma_k$, $|(f_n(\zeta)-w)-(f(\zeta)-w)|=|f_n(\zeta)-f(\zeta)|<\dfrac{\delta_k}{2}<\delta_k \le |f(\zeta)-w|$.
  Per il teorema di Rocuhé applicato a $f_n-w$ e $f-w$ in $\overline{D(z_0, \delta/k)}$, per ogni $n \ge n_k$ $f_n-w$ ha almeno uno zero in $D(z_0, \delta_k)$ $\implies$ esiste $z_n \in D(z_0, \delta/k)$ t.c. $f_n(z_n)=w$. $z_n \longrightarrow z_0$ per $n \longrightarrow +\infty$.
\end{proof}

\begin{cor}
  (Secondo teorema di Hurwitz) $\Omega \subseteq \mathbb{C}$ dominio, $\{f_n\} \subset \mathcal{O}(\Omega)$ t.c. $f_n \longrightarrow f \in \mathcal{O}(\Omega)$. Supponiamo che le $f_n$ non si annullino mai (o, in generale, esiste $w_0 \in \mathbb{C}$ t.c. $w_0 \not\in f_n(\Omega)$ per ogni $n$),
  allora o $f \equiv 0$ o $f$ non si annulla mai (in generale, o $f \equiv w_0$ o $w_0 \not\in f(\Omega)$).
\end{cor}

\begin{proof}
  Per assurdo, $w_0 \in f(\Omega)$. Allora o $f$ è costante ($f \equiv w_0$) oppure, per il primo teorema di Hurwitz, $w_0 \in f_n(\Omega)$ per ogni $n>>1$, assurdo.
\end{proof}

\begin{cor}
  (Terzo teorema di Hurwitz) $\Omega \subseteq \mathbb{C}$ dominio, $\{f_n\} \subset \mathcal{O}(\Omega)$ t.c. $f_n \longrightarrow f \in \mathcal{O}(\Omega)$. Supponiamo che le $f_n$ siano iniettive. Allora $f$ è costante o iniettiva.
\end{cor}

\begin{proof}
  Per assurdo, sia $f$ né costante né iniettiva. Allora esistono $z_1 \not=z_2$ t.c. $f(z_1)=f(z_2)$. Poniamo $h_n(z)=f_n(z)-f_n(z_2)$ e $h(z)=f(z)-f(z_2)$. $h_n \longrightarrow h$ e le $h_n$ non si annullano mai in $\Omega \setminus \{z_2\}$ (perché le $f_n$ sono iniettive). Dato che per ipotesi $f$ non è costante, pure $h$ non è costante, dunque per il secondo teorema di Hurwitz non si annulla mai in $\Omega \setminus \{z_2\}$, ma $h(z_1)=0$, assurdo.
\end{proof}


\subsection{La sfera di Riemann}
\begin{defn}
  La \textsc{sfera di Riemann} è l'insieme $\hat{\mathbb{C}}=\overline{\mathbb{C}}=\mathbb{C}_{\infty}=\mathbb{C} \cup \{\infty\}=\mathbb{P}^1(\mathbb{C})$ (l'ultimo è la retta proiettiva complessa).
  Per noi sarà $\hat{\mathbb{C}}=\mathbb{C} \cup \{\infty\}$ con la seguente topologia: ristretta a $\mathbb{C}$ è la topologia usuale, mentre gli intorni aperti di $\infty$ sono della forma $(\mathbb{C} \setminus K) \cup \{\infty\}$ con $K \subset \subset \mathbb{C}$ compatto.
\end{defn}

Siano $U_0=\mathbb{C}, U_1=\mathbb{C}^*\cup\{\infty\}$ (si noti che $U_1$ è un intorno aperto di $\infty$). Sia $\varphi_1:U_1 \longrightarrow \mathbb{C}$ definita come
$$\varphi_1(w)=\begin{cases} 1/w & \mbox{se }w\not=\infty \\ 0 & \mbox{se }w=\infty. \end{cases}$$
$\varphi_1$ è un omeomorfismo fra $U_1$ e $\mathbb{C}$.
Sia $\varphi_0:U_0 \longrightarrow \mathbb{C}$, $\varphi_0(z)=z$ (l'identità); è un omeomorfismo fra $U_0$ e $\mathbb{C}$. \\
$U_0 \cap U_1=\mathbb{C}^*$, $\varphi_1(U_0 \cap U_1)=\mathbb{C}^*=\varphi_0(U_0 \cap U_1)$. \\
$\varphi_0 \circ \varphi_1^{-1}, \varphi_1 \circ \varphi_0^{-1}:\mathbb{C}^* \longrightarrow \mathbb{C}^*$,
$(\varphi_0 \circ \varphi_1^{-1})(w)=\dfrac{1}{w}, (\varphi_1 \circ \varphi_0^{-1})(z)=\dfrac{1}{z}$ sono olomorfe. \\
$\varphi_0$ e $\varphi_1$ si chiamano \textit{carte}. Una funzione definita a valori in $\hat{\mathbb{C}}$ è olomorfa se lo è letta tramite carte. Vediamo nello specifico cosa significa.

Sia $\Omega \subseteq \hat{\mathbb{C}}$ aperto, $f:\Omega \longrightarrow \mathbb{C}$ continua; quando è olomorfa? \\
Risposta:
\begin{nlist}
  \item $f \restrict{\Omega \cap \mathbb{C}}$ è olomorfa in senso classico (notiamo che $\Omega \cap \mathbb{C}=\Omega \setminus \{\infty\}$);
  \item $f \circ \varphi_1^{-1}:\varphi_1(\Omega) \longrightarrow \mathbb{C}$ è olomorfa vicino a $0=\varphi_1(\infty)$. \\
  $(f \circ \varphi_1^{-1})(w)=f\left(\dfrac{1}{w}\right)$.
\end{nlist}

\begin{ex}
  $\displaystyle f(z)=\sum_{n=-\infty}^{+\infty} c_nz^n$. Quando è olomorfa in $\infty$? Se e solo se $f\left(\dfrac{1}{w}\right)$ è olomorfa in $0$.
  $\displaystyle f\left(\frac{1}{w}\right)=\sum_{n=-\infty}^{+\infty} c_nw^{-n}$ è olomorfa in $0$ $\iff$ $c_n=0$ per ogni $n>0$.
\end{ex}

\begin{oss}
  $f: \hat{\mathbb{C}} \longrightarrow \mathbb{C}$ è olomorfa se e solo se è costante. Infatti, $\hat{\mathbb{C}}$ compatto $\implies$ $f(\hat{\mathbb{C}})$ compatto, cioè chiuso e limitato in $\mathbb{C}$ $\implies$ $|f|$ ha max in $x_0 \in \hat{\mathbb{C}}$.
  Se $z_0 \in \mathbb{C}$, allora per il teorema di Liouville \ref{liou} $f \restrict{\mathbb{C}}$ è costante $\implies$ $f$ costante. Se $z_0=\infty$, $f(1/w)$ ha massimo in $0$, dunque ragionando come prima è costante.
\end{oss}

Sia $\Omega \subseteq \mathbb{C}$ aperto, $f:\Omega \longrightarrow \hat{\mathbb{C}}$ continua; quando è olomorfa? \\
Risposta:
\begin{nlist}
  \item $f$ è olomorfa in $\Omega \setminus f^{-1}(\infty)$ in senso classico;
  \item se $f(z_0)=\infty$, $\varphi_1 \circ f=\dfrac{1}{f}$ è olomorfa vicino a $z_0$.
\end{nlist}

\begin{ex}
  $\displaystyle f(z)=\sum_{n=-\infty}^{+\infty} c_nz^n$ è a valori in $\hat{\mathbb{C}}$ se $0$ è un polo (ci interessa il caso in cui $\infty$ sia effettivamente nell'immagine, altrimenti è una comune funzione olomorfa a valori in $\mathbb{C}$), cioè consideriamo $f(0)=\infty$.
  Supponiamo allora $\displaystyle f(z)=\sum_{n=k}^{+\infty} c_nz^n=z^{-k}\sum_{n=k}^{+\infty} c_nz^{n+k}=z^{-k}h(z)$, $h(0)=c_{-k}\not=0$, $h$ olomorfa. $\dfrac{1}{f}(z)=\dfrac{z^k}{h(z)}$ è olomorfa in $0$.
  Viceversa, se $f$ è olomorfa, $\dfrac{1}{f}$ è olomorfa in $0$ $\implies$ $\displaystyle \frac{1}{f}(z)=z^k\sum_{n=0}^{+\infty}c_nz^n$, $c_0 \not=0, k \ge 1$ (la condizion $k \ge 1$ segue dal fatto che siamo nell'ipotesi $f(0)=\infty \implies (1/f)(0)=0$).
  Allora $\dfrac{1}{f}(z)=z^kh(z)$ $\implies$ $f(z)=z^{-k}\frac{1}{h(z)}$ e quindi ha un polo in $0$.
\end{ex}

\begin{cor}
  $\Omega \subseteq \mathbb{C}$, $f:\Omega \longrightarrow \hat{\mathbb{C}}$ è olomorfa se e solo se è meromorfa.
\end{cor}

Possiamo ora dare una definizione generale.

\begin{defn}
  $f:\hat{\mathbb{C}} \longrightarrow \hat{\mathbb{C}}$ continua è \textit{olomorfa} se e solo se $f\left(\dfrac{1}{w}\right)$ è olomorfa vicino a $0$ e $\dfrac{1}{f}$ è olomorfa vicino a $f^{-1}(\infty)$ (e ovviamente dev'essere normalmente olomorfa in tutti gli altri punti). \\
  Se $f(\infty)=\infty$, la condizione è che $\dfrac{1}{f(1/w)}$ sia olomorfa in $0$.
\end{defn}

\begin{ex}
  $p(z)=a_0+a_1z+\dots+a_dz^d, a_d \not=0$ (un polinomio). $p(\infty)=\infty$. È olomorfo in $\infty$? Sì: $\displaystyle \frac{1}{p(1/z)}=\frac{1}{a_0+a_1z^{-1}+\dots+a_dz^{-d}}=\frac{1}{z^{-d}(a_0z^d+\dots+a_d)}=\frac{z^d}{a_d+\dots+a_0z^d}$ è olomorfo in $0$.
  $\dfrac{1}{p(1/z)}$ ha uno zero di ordine $d$ in $0$ $\iff$ $p$ ha un polo di ordine di $-d$ in $\infty$ (vedremo più avanti come è definito $ord_f(\infty)$).
\end{ex}

\begin{prop}
  $f \in \text{Hol}(\hat{\mathbb{C}}, \hat{\mathbb{C}})$ $\iff$ $f=\dfrac{P}{Q}$ con $P, Q \in \mathbb{C}[z]$ senza fattori comuni, cioè $f$ è una funzione razionale.
\end{prop}

\begin{proof}
  ($\Leftarrow$) Sappiamo che $\mathbb{C}[z] \subset \text{Hol}(\hat{\mathbb{C}}, \hat{\mathbb{C}})$ e quozienti di funzioni olomorfe sono olomorfi.

  ($\implies$) Sia $f: \hat{\mathbb{C}} \longrightarrow \hat{\mathbb{C}}$ olomorfa non costante. $Z_f=f^{-1}(0)$ è chiuso e discreto in $\hat{\mathbb{C}}$ che è compatto, dunque è finito, perciò $Z_f \cap \mathbb{C}=\{z_1, \dots, z_k\} \subset \mathbb{C}$.
  Analogamente $P_f=f^{-1}(\infty)=Z_{1/f}$, $P_f \cap \mathbb{C}=\{w_1, \dots, w_h\} \subset \mathbb{C}$. Sia $g(z)=\dfrac{(z-w_1)\cdots(z-w_h)}{(z-z_1) \cdots (z-z_k)}f(z)$, $g \in \text{Hol}(\hat{\mathbb{C}}, \hat{\mathbb{C}})$ (gli zeri e i poli compaiono con molteplicità nei prodotti al numeratore e al denominatore).
  In questo modo $g$ non ha né zeri né poli in $\mathbb{C}$. Se $g(\infty) \in \mathbb{C}$, $g \in \text{Hol}(\hat{\mathbb{C}}, \mathbb{C})$ $\implies$ $g$ costante, diciamo $g \equiv c$ $\implies$ $f(z)=c\dfrac{(z-z_1) \cdots (z-z_k)}{(z-w_1)\cdots(z-w_h)}$, come voluto.
  Se $g(\infty)=\infty$ $\implies$ $\dfrac{1}{g}(\infty)=0 \in \mathbb{C}$ $\implies$ $\dfrac{1}{g} \in \text{Hol}(\hat{\mathbb{C}}, \mathbb{C})$ $\implies$ $\dfrac{1}{g}$ costante e si conclude come sopra.
\end{proof}

\begin{defn}
  Sia $f=\dfrac{P}{Q} \in \text{Hol}(\hat{\mathbb{C}}, \hat{\mathbb{C}})$. Il \textsc{grado di $f$} è $\deg{f}=\max{\{\deg{P}, \deg{Q}\}}$. \\
  La definizione dell'ordine di zeri e poli in $\mathbb{C}$ ce l'abbiamo. \\
  $\displaystyle f(\infty)=\lim_{w \longrightarrow 0} \frac{P(1/w)}{Q(1/w)}=\lim_{w \longrightarrow 0} \frac{a_m\left(\frac{1}{w}\right)^m+\dots+a_0}{b_n\left(\frac{1}{w}\right)^n+\dots+b_0}=$ \\
  $$\lim_{w \longrightarrow 0} w^{n-m} \frac{a_m+\dots+a_0w^m}{b_n+\dots+b_0w^n}=\begin{cases} 0 & \mbox{se }n>m \\ \frac{a_m}{b_n} & \mbox{se }n=m \\ \infty & \mbox{se } n<m. \end{cases}$$
  Definiamo allora $ord_f(\infty))=n-m=\deg{Q}-\deg{P}$.
\end{defn}

\begin{defn}
  Siano $f \in \text{Hol}(\hat{\mathbb{C}}, \hat{\mathbb{C}})$, $z_0 \in \hat{\mathbb{C}}$.
  La \textsc{molteplicità di $f$ in $z_0$} è $\delta_f(z_0)$ definita come segue: se $f(z_0)=w_0 \in \mathbb{C}$, $z_0$ è uno zero di $f-w_0$ e poniamo $\delta_f(z_0)=ord_{f-w_0}(z_0)$; se $f(z_0)=\infty$, $z_0$ è un polo di $f$ e poniamo $\delta_f(z_0)=-ord_f(z_0)$. Si ha che $\delta_f(z_0) \in \mathbb{N}$.
\end{defn}

\begin{prop}
  Sia $f \in \text{Hol}(\hat{\mathbb{C}}, \hat{\mathbb{C}})$ non costante. Allora per ogni $q \in \hat{\mathbb{C}}$ $\displaystyle \sum_{f(p)=q} \delta_f(p)=\deg{f}$.
\end{prop}

\begin{proof}
  Sia $f=\dfrac{P}{Q}$. Se $q=0$, $\displaystyle \sum_{f(p)=0} \delta_f(p)=\sum_{\substack{f(p)=0 \\ p \in \mathbb{C}}} \delta_f(p)+c \cdot \delta_f(\infty)$ dove $c=1$ se $f(\infty)=0$ e $c=0$ altrimenti.
  Si noti che per il teorema fondamentale dell'algebra $\displaystyle \sum_{\substack{f(p)=0 \\ p \in \mathbb{C}}} \delta_f(p)=\deg{P}$. Per com'è definito $c$, $c \cdot \delta_f(\infty)=\max{\{0, \deg{Q}-\deg{P}\}}$.
  Allora $\displaystyle \displaystyle \sum_{f(p)=0} \delta_f(p)=\deg{P}+\max{\{0, \deg{Q}-\deg{P}\}}=\max{\{\deg{P}, \deg{Q}\}}=\deg{f}$.
  Se $q=\infty$, $\displaystyle \sum_{f(p)=\infty} \delta_f(p)=\sum_{\substack{f(p)=\infty \\ p \in \mathbb{C}}} \delta_f(p)+c \cdot \delta_f(\infty)$ dove stavolta $c=1$ se $f(\infty)=\infty$ e $c=0$ altrimenti.
  Dunque in questo caso la sommatoria vale, per il teorema fondamentale dell'algebra, $\deg{Q}$, mentre $c \cdot \delta_f(\infty)=\max{\{0, \deg{P}-\deg{Q}\}}$,
  per cui $\displaystyle \sum_{f(p)=\infty} \delta_f(p)=\deg{Q}+\max{\{0, \deg{P}-\deg{Q}\}}=\max{\{\deg{Q}, \deg{P}\}}=\deg{f}$.
  Se $q \in \mathbb{C}^*$, $\displaystyle \sum_{f(p)=q} \delta_f(p)=\sum_{f(p)=q} ord_{f-q}(p)=\sum_{f(p)-q} ord_{f-q}(p)=\deg{(f-q)}$. $f(z)-q=\dfrac{P(z)-qQ(z)}{Q(z)}$.
  $$\deg{(P-qQ)}\begin{cases} =\max{\{\deg{P}, \deg{Q}\}} & \mbox{se sono diversi} \\ \le \max{\{\deg{P}, \deg{Q}\}} & \mbox{se sono uguali} \end{cases}$$ $\implies$ $\deg{(f-q)}=\deg{f}$.
\end{proof}

\begin{cor}
  Siano $f \in \text{Hol}(\hat{\mathbb{C}}, \hat{\mathbb{C}})$ non costante, $w_0 \in \hat{\mathbb{C}}$. Allora $1 \le card(f^{-1}(w_0)) \le \deg{f}$.
\end{cor}

\begin{proof}
  $\ge 1$: se $w_0 \not=\infty$, $f(z)=w_0$ $\iff$ $f(z)-w_0=0$ $\iff$ $P(z)-w_0Q(z)=0$ e per il teorema fondamentale dell'algebra esiste $z$ che soddisfa; se $w_0=\infty$, si considera $1/f$. \\
  $\displaystyle card(f^{-1}(w_0)) \le \sum_{f(p)=w_0} \delta_f(p)=\deg{f}$.
\end{proof}

\begin{oss}
  $\delta_f(p)>1$ $\implies$ $f'(p)=0 \lor \left(\dfrac{1}{f}\right)'(p)=0$. Infatti, senza perdita di generalità $p=0$ e $f(p)=0$, allora se $\delta_f(p)=k>1$ si ha che $f(z)=z^kh(z)$ con $h$ olomorfa e $h(0) \not=0$ $\implies$ $f'(z)=[kz^{k-1}h(z)+z^kh'(z)]$. Ricordando che $k>1$, si ha che $f'(0)=0$.
\end{oss}

\begin{cor}
  $f \in \text{Aut}(\hat{\mathbb{C}})$ $\iff$ $\deg{f}=1$ $\iff$ $f(z)=\dfrac{az+b}{cz+d}$ con $ad-bc=1$.
\end{cor}

\begin{proof}
  ($\Leftarrow$) Ogni $f \in \text{Hol}(\hat{\mathbb{C}}, \hat{\mathbb{C}})$ è suriettiva per quanto appena dimostrato. Se $\deg{f}=1$, allora $f$ è iniettiva, quindi biettiva, per cui per il teorema \ref{biolo} $f \in \text{Aut}(\hat{\mathbb{C}})$. \\
  ($\implies$) $f$ automorfismo $\implies$ $f$ iniettiva $\implies$ $\displaystyle \sum_{f(p)=w_0} \delta_f(p)$ contiene un unico addento con molteplicità uno (da cui la tesi). Infatti, da $f$ non costante si ha che $f'$ ha un insieme di zeri discreto $C_f$ e $(1/f)'$ ha un insieme di zeri discreto $C_{1/f}$. Allora basta prendere $z_0 \not\in C_f \cup C_{1/f}$ per ottenere, dall'osservazione precedente, che $\delta_f(z_0)=1$. \\
  Il secondo se e solo se è un banale esercizio lasciato al lettore.
\end{proof}

\begin{oss}
  Siccome numeratore e denominatore sono definiti a meno di una costante moltiplicativa, possiamo suppore $ad-bc=1$.
\end{oss}

\begin{exc}
  $\text{Aut}(\hat{\mathbb{C}})$ è isomorfo a $\faktor{SL(2, \mathbb{C})}{\{\pm I_2\}}$.
\end{exc}

\begin{cor}
  $f \in \text{Aut}(\mathbb{C})$ $\iff$ $f(z)=az+b, a, b \in \mathbb{C}, a\not=0$.
\end{cor}

\begin{proof}
  ($\Leftarrow$) Ovvia.

  ($\implies$) $f \in \text{Aut}(\mathbb{C})$ $\implies$ $f$ è iniettiva, dunque per Casorati-Weierstrass $\infty$ è un polo di $f$ $\implies$ $f$ si estende a un automorfismo di $\hat{\mathbb{C}}$ con $f(\infty)=\infty$ $\implies$ $f(z)=\frac{a}{d}z+\frac{b}{d}$.
\end{proof}


\subsection{Il disco unitario}
Come abbiamo già visto, il disco unitario (aperto) è definito come $\mathbb{D}=\{z \in \mathbb{C} \mid |z|<1\}$.

\begin{lm}
  (Lemma di Schwarz) Sia $f \in \text{Hol}(\mathbb{D}, \mathbb{D})$ t.c. $f(0)=0$. Allora per ogni $z \in \mathbb{D}$ $|f(z)| \le |z|$ e $|f'(0)| \le 1$; inoltre, se vale l'uguale nella prima per $z \not=0$ oppure nella seconda allora $f(z)=e^{i\theta}z, \theta \in \mathbb{R}$, cioè $f$ è una rotazione.
\end{lm}

\begin{proof}
  $f(0)=0$ $\implies$ possiamo costruire $g \in \text{Hol}(\mathbb{D}, \mathbb{C})$ con $g(z)=\dfrac{f(z)}{z}$ estendendola per continuità in $0$ a $g(0)=f'(0)$. Fissiamo $0<r<1$.
  Per ogni $|z| \le r$, per il principio del massimo $\displaystyle |g(z)| \le \max_{|w|=r} |g(w)|=\max_{|w|=r} \frac{|f(w)|}{r} \le \frac{1}{r}$. Mandando $r$ a $1$ otteniamo che per ogni $z \in \mathbb{D}$ si ha $|g(z)| \le 1$, da cui $|f(z)|\le |z|$ e $|f'(0)| \le 1$. \\
  Se vale uno dei due uguali sopra, allora esiste $z_0 \in \mathbb{D}$ t.c. $|g(z_0)|=1$, per cui sempre per il principio del massimo $g$ è costantemente uguale a un valore di modulo $1$, cioè $g(z)=e^{i\theta}$ con $\theta \in \mathbb{R}$ da cui $f(z)=e^{i\theta}z$.
\end{proof}

\begin{cor}
  Se $f \in \text{Aut}(\mathbb{D})$ è t.c. $f(0)=0$, allora $f(z)=e^{i\theta}z$.
\end{cor}

\begin{proof}
  $f^{-1} \in \text{Aut}(\mathbb{D})$. $(f^{-1})'(0)=\dfrac{1}{f'(0)}$. Per il lemma di Schwarz, $|f'(0)| \le 1$ e $|(f^{-1})'(0)| \le 1$ $\implies$ $|f'(0)|=1$, da cui la tesi sempre per il lemma di Schwarz.
\end{proof}

\begin{lm} \label{az_gr}
  Sia $G$ un gruppo che agisce fedelmente su uno spazio $X$, cioè per ogni $g \in G$ è data una biezione $\gamma_g:X \rightarrow X$ t.c. $\gamma_{e}=\id$ e $\gamma_{g_1} \circ \gamma_{g_2} =\gamma_{g_1g_2}$, inoltre $\gamma_{g_1}=\gamma_{g_2} \iff g_1=g_2$.
  Sia $G_{x_0}$ il gruppo di isotropia di $x_0 \in X$, cioè $G_{x_0}=\{g \in G \mid \gamma_g(x_0)=x_0\}$. Supponiamo che per ogni $x \in X$ esiste $g_x \in G$ t.c. $\gamma_{g_x}(x)=x_0$ e sia $\Gamma=\{g_x \mid x \in X\}$.
  Allora $G$ è generato da $\Gamma$ e $G_{x_0}$, cioè ogni $g \in G$ è della forma $g=hg_x$ con $x \in X$ e $h \in G_{x_0}$.
\end{lm}

\begin{proof}
  Sia $g \in G$ e $x=\gamma_g(x_0)$. Allora $(\gamma_{g_x}\circ \gamma_g)(x_0)=x_0$ $\implies$ $\gamma_{g_x}\circ \gamma_g=\gamma_{g_xg}=\gamma_h$ con $h \in G_{x_0}$ $\implies$ $g_xg=h$ $\implies$ $g=g_x^{-1}h$.
  Partendo da $g^{-1}$ avremmo ottenuto $g^{-1}=g_x^{-1}h$ $\implies$ $g=h^{-1}g_x$ con $h \in G_{x_0}$.
\end{proof}

\begin{prop}
  $f \in \text{Aut}(\mathbb{D})$ $\iff$ esistono $\theta \in \mathbb{R}$ e $a \in \mathbb{D}$ t.c. $f(z)=e^{i\theta}\dfrac{z-a}{1-\bar{a}z}$.
\end{prop}

\begin{proof}
  ($\Leftarrow$) $1-\left|\dfrac{z-a}{1-\bar{a}z}\right|^2=\dfrac{(1-|a|^2)(1-|z|^2)}{|1-\bar{a}z|^2}$. Se $a, z \in \mathbb{D}$, $f(z) \in \mathbb{D}$.
  Se $a \in \mathbb{D}, z \in \partial\mathbb{D}$, $f(z) \in \partial\mathbb{D}$. L'inversa è $f^{-1}(z)=e^{-i\theta}\dfrac{z+ae^{i\theta}}{z+\bar{a}e^{-i\theta}z}$ ed è della stessa forma. Si noti che $f(a)=0$.

  ($\implies$) Scriviamo per semplicità $f_{a, \theta}=e^{i\theta}\dfrac{z-a}{1-\bar{a}z}$. Vediamo $\text{Aut}(\mathbb{D})$ come gruppo che agisce su $\mathbb{D}$. $\text{Aut}(\mathbb{D})_0$ è, per il corollario del lemma di Schwarz, $\{f_{0, \theta} \mid \theta \in \mathbb{R}\}$.
  $\Gamma=\{f_{a, 0} \mid a \in \mathbb{D}\}$ ($f_{a, 0}(a)=0$).
  Per il lemma \ref{az_gr}, $\text{Aut}(\mathbb{D})$ è generato da $\text{Aut}(\mathbb{D})$ e $\Gamma$, cioè ogni $\gamma \in \text{Aut}(\mathbb{D})$ è della forma $\gamma=f_{0, \theta} \circ f_{a, 0}=f_{a, \theta}$.
\end{proof}

\begin{cor}
  $\text{Aut}(\mathbb{D})$ agisce in modo transitivo su $\mathbb{D}$, cioè per ogni $z_0, z_1 \in \mathbb{D}$ esiste $\gamma \in \text{Aut}(\mathbb{D})$ t.c. $\gamma(z_0)=z_1$.
\end{cor}

\begin{proof}
  $\gamma=f_{z_1, 0}^{-1} \circ f_{z_0, 0}$.
\end{proof}

\begin{oss}
  Dati $z_0, z_1, w_0, w_1 \in \mathbb{D}$ ($z_0 \not=z_1, w_0 \not=w_1$), in generale non esiste $\gamma \in \text{Aut}(\mathbb{D})$ t.c. $\gamma(z_0)=w_0$ e $\gamma_(z_1)=w_1$.
  Infatti, se poniamo $z_0=w_0=0, z_1, w_1 \not=0$, abbiamo che $\gamma(0)=0$ $\implies$ $\gamma(z)=e^{i\theta}z$ $\implies$ $|\gamma(z_1)|=|z_1|$, per cui se $|w_1|\not=|z_1|$ non è possibile trovare un siffatto $\gamma$.
\end{oss}

\begin{exc}
  Per ogni $\sigma_0, \sigma_1, \tau_0, \tau_1 \in \partial\mathbb{D}$ con $\sigma_0\not=\sigma_1, \tau_0\not=\tau_1$ esiste $\gamma \in \text{Aut}(\mathbb{D})$ t.c. $\gamma(\sigma_0)=\tau_0$ e $\gamma(\sigma_1)=\tau_1$.
\end{exc}


\subsection{Dinamica del disco e del semipiano superiore}
Vogliamo adesso cercare di studiare qual è la "dinamica" delle funzioni olomorfe. Lo faremo nei casi del disco e del semipiano superiore.

\begin{prop}
  Sia $\gamma \in \text{Aut}(\mathbb{D}), \gamma \not=\id_{\mathbb{D}}$. Allora o
  \begin{nlist}
    \item $\gamma$ ha un unico punto fisso in $\mathbb{D}$ (si parla in questo caso di automorfismo \textit{ellittico}) o
    \item $\gamma$ non ha punti fissi in $\mathbb{D}$ e ha un unico punto fisso in $\partial\mathbb{D}$ (\textit{parabolico}) o
    \item $\gamma$ non ha punti fissi in $\mathbb{D}$ e ha due punti fissi distinti in $\partial\mathbb{D}$ (\textit{iperbolico}).
  \end{nlist}
\end{prop}

\begin{proof}
  $\gamma(z_0)=z_0 \iff e^{i\theta}(z_0-a)=(1-\bar{a}z_0)z_0 \iff \bar{a}z_0^2+(e^{i\theta}-1)z_0-e^{i\theta}a=0$, equazione di secondo grado con radici $z_1, z_2$ (può essere che $z_1=z_2$) t.c.
  $z_1 \cdot z_2=-e^{i\theta}\dfrac{a}{\bar{a}} \in \partial\mathbb{D} \implies |z_1||z_2|=1$.
  Se $z_1 \not=z_2$, o $z_1 \in \mathbb{D}$ e $z_2 \in \mathbb{C} \setminus \{\overline{\mathbb{D}}\}$ (caso ellittico) e $z_1, z_2 \in \partial\mathbb{D}$ (caso iperbolico). Se $z_1=z_2$, $|z_1|=|z_2|=1$ (caso iperbolico).
\end{proof}

\begin{oss}
  Se $f \in \text{Hol}(\mathbb{D}, \mathbb{D})$ è t.c. $f(z_1)=z_1$ e $f(z_2)=z_2$ con $z_1, z_2 \in \mathbb{D}, z_1 \not=z_2$, allora $f=\id_{\mathbb{D}}$.
  Infatti, possiamo supporre $z_1=0$ $\implies$ $f(0)=0$ e $f(z_2)=z_2$, quindi siamo nel caso del lemma di Schwarza in cui vale l'uguaglianza, per cui $f(z)=e^{i\theta}z$, ma $f(z_2)=z_2$ $\implies$ $e^{i\theta}=1$.
\end{oss}

\begin{ex}
  Esempio di automorfismo ellittico: la rotazione intorno a $0$ $\gamma_{0, \theta}(z)=e^{i\theta}z$. Più in generale, se $a \in \mathbb{D}$, $\gamma_{a, 0}(z)=\dfrac{z-a}{1-\bar{a}z}$, allora $\gamma_{a, 0}^{-1} \circ \gamma_{0, \theta} \circ \gamma_{a, 0}$ è ellittico con punto fisso $a$.
  Queste sono dette \textit{rotazioni non euclidee} e caratterizzano tutti gli automorfismi ellittici (lo si può vedere coniugando opportunamente con $\gamma_{a, 0}$ o $\gamma_{a, 0}^{-1}$).
\end{ex}

\begin{defn}
  Il \textsc{semipiano superiore} è $\mathbb{H}^+=\{w \in \mathbb{C} \mid \mathfrak{Im}w>0\}$. La \textsc{trasformata di Cayley} è $\Psi:\mathbb{D} \longrightarrow \mathbb{H}^+$ t.c. $\Psi(z)=i\dfrac{1+z}{1-z}$.
\end{defn}

Notiamo che possiamo vedere $\mathbb{H}^+ \subset \hat{\mathbb{C}}$ e in questo caso $\partial\mathbb{H}^+=\mathbb{R}\cup\{\infty\}$. $\Psi^{-1}(w)=\dfrac{w-i}{w+i}$. $\Psi(0)=1, \Psi(1)=\infty$. \\
$\mathfrak{Im}\Psi(z)=\mathfrak{Im}\left(i\dfrac{1+z}{1-z}\right)=\mathfrak{Re}\left(\dfrac{1+z}{1-z}\right)=\dfrac{1}{|1-z|^2}\mathfrak{Re}((1+z)(1-\bar{z}))=\dfrac{1-|z|^2}{|1-z|^2}$ che è $>0$ $\iff$ $z \in \mathbb{D}$ e $=0$ $\iff$ $z \in \partial\mathbb{D}\setminus\{1\}$.

$\Psi$ è un biolomorfismo fra $\mathbb{D}$ e $\mathbb{H}^+$ che si estende continua a $\partial\mathbb{D} \longrightarrow \partial\mathbb{H}^+$. Se abbiamo $f: \mathbb{D} \longrightarrow \mathbb{D}$, abbiamo anche $F=\Psi \circ f \circ \Psi^{-1}:\mathbb{H}^+ \longrightarrow \mathbb{H}^+$ e viceversa.

\begin{cor}
  $\gamma \in \text{Aut}(\mathbb{H}^+) \iff \gamma(w)=\dfrac{aw+b}{cw+s}$ con $ad-bc=1$ e $a, b, c, d \in \mathbb{R}$. Si ha allora che $\text{Aut}(\mathbb{H}^+) \cong \faktor{SL(2, \mathbb{R})}{\{\pm I_2\}}=PSL(2, \mathbb{R})$ (questo è detto \textit{gruppo speciale lineare proiettivo}).
\end{cor}

\begin{proof}
  $\gamma \in \text{Aut}(\mathbb{H}^+) \iff \Psi^{-1} \circ \gamma \circ \Psi \in \text{Aut}(\mathbb{D}) \iff (\Psi^{-1} \circ \gamma \circ \Psi)(z)=e^{i\theta}\dfrac{z-a}{1-\bar{a}z}$.
  Ponendo $\Psi(z)=w$, l'uguaglianza sopra equivale a $\gamma(w)=\Psi\left(e^{i\theta}\dfrac{z-a}{1-\bar{a}z}\right)=\Psi\left(e^{i\theta}\dfrac{\Psi^{-1}(w)-a}{1-\bar{a}\Psi^{-1}(w)}\right)$. Facendo il conto si trova l'enunciato.
\end{proof}

\begin{exc}
  $\gamma \in \text{Aut}(\mathbb{H}^+)$ è t.c. $\gamma(i)=i \iff \gamma(w)=\dfrac{w\cos{\theta}-\sin{\theta}}{w\sin{\theta}+\cos{\theta}}$.
\end{exc}

\begin{ex}
  Sia $\gamma \in \text{Aut}(\mathbb{H}^+)$, $\gamma(\infty)=\infty \iff \gamma(w)=\alpha w+\beta$ con $\alpha, \beta \in \mathbb{R}, \alpha>0$.
  Se lo vogliamo parabolico non deve avere altri punti fissi in $\mathbb{C}$ e questo è possibile se e solo se $\alpha w+\beta=w$ non ha altre soluzioni $\iff$ $\alpha=1, \beta \not=0$, cioè $\gamma(w)=w+\beta$.
  È una traslazione di $\mathbb{H}^+$ parallela al suo bordo.
\end{ex}

\begin{exc}
  Sia $\tau \in \partial\mathbb{D}$. Dimostrare che tutti gli automorfismi $\gamma$ parabolici di $\mathbb{D}$ con $\gamma(\tau)=\tau$ sono della forma $\gamma(z)=\sigma_0\dfrac{z+z_0}{1+\bar{z}_0z}$ con $z_0=\dfrac{ic}{2-ic}\tau$ e $\sigma_0=\dfrac{2-ic}{2+ic}$ con $c \in \mathbb{R}$.
  Hint: a meno di una rotazione, $\tau=1$.
\end{exc}

\begin{ex}
  $\gamma \in \text{Aut}(\mathbb{H}^+)$ è iperbolico con $\gamma(\infty)=\infty$ e $\gamma(0)=0$ $\iff$ $\gamma(w)=\alpha w$ con $\alpha>0$.
\end{ex}

Passiamo ora alla \textsc{dinamica di funzioni iterate}. Abbiamo uno spazio generico $X$ e una funzione $f:X \longrightarrow X$. Le sue \textit{iterate} sono $f^2=f \circ f$ e, induttivamente, $f^{k+1}=f \circ f^{k-1}$. Vogliamo capire il comportamento asintotico di $\{f^k\}$ (in relazione alla struttura presente su $X$), per esempio, capire cosa succede all'\textit{orbita} $O^+(x)=\{f^k(x)\}$ con $x \in X$.

\begin{ex}
  $\gamma(w)=\alpha w \implies \gamma^2(w)=\alpha(\alpha w)=\alpha^2 w \implies \gamma^k(w)=\alpha^k w$.
  Quindi: se $0<\alpha<1$, $\gamma^k(w) \longrightarrow 0$ per $k \longrightarrow +\infty$ $\implies$ $\gamma^k \longrightarrow 0$ (costante) uniformemente sui compatti; se $\alpha>1$, $\gamma^k(w) \longrightarrow \infty$ per $k \longrightarrow +\infty$ $\implies$ $\gamma^k \longrightarrow \infty$ (costante) uniformemente sui compatti.
\end{ex}

\begin{ex}
  $\gamma(w)=w+\beta \implies \gamma^k(w)=w+k\beta \implies \gamma^k(w) \longrightarrow \infty$ per $k \longrightarrow +\infty$ $\implies$ $\gamma^k \longrightarrow \infty$ (costante) uniformemente sui compatti.
\end{ex}

\begin{oss}
  \begin{center}
    \begin{tikzcd}
      X \arrow[r, "f"] & X\\
      Y \arrow[u, "\Psi", "\cong" right] \arrow[r, "F"] & Y \arrow[u, "\cong", "\Psi" right]
    \end{tikzcd}
  \end{center}
  $\Psi$ bigezione (omeomorfismo/biolomorfismo/eccetera), $F=\Psi^{-1} \circ f \circ \Psi$, cioè \textit{$F$ è coniugata a $f$}. Allora $f^k=\Psi^{-1} \circ f^k \circ \Psi$, cioè $F^k$ è coniugata a $f^k$ per ogni $k$. In particolare, la "dinamica di $F$" è "uguale" alla "dinamica di $f$".
\end{oss}

\begin{cor}
  Sia $\gamma \in \text{Aut}(\mathbb{D})$ parabolico o iperbolico, allora $\gamma^k$ converge uniformemente sui compatti a una funzione costantemente uguale a un punto fisso di $\gamma$ sul bordo.
\end{cor}

\begin{proof}
  A meno di coniugio possiamo supporre $\text{Fix}(\gamma)=\{1\}$ nel caso parabolico e $\{1, -1\}$ nel caso iperbolico. Coniughiamo con $\Psi$ e usiamo gli esempi.
\end{proof}

Sia $\gamma \in \text{Aut}(\mathbb{D})$ ellittico, a meno di coniugio $\gamma(0)=0 \implies \gamma(z)=e^{2\pi i \theta}z \implies \gamma^k(z)=e^{2k\pi i \theta}z$.
Se $\theta \in \mathbb{Q}$, esiste $k_0$ t.c. $k_0\theta \in \mathbb{Z} \implies \gamma^{k_0}(z) \equiv z \iff \gamma^{k_0}=\id_{\mathbb{D}}$.

\begin{exc}
  Se $\theta \not\in \mathbb{Q}$, $\gamma^k(z) \not=z$ per ogni $z \not=0$ e $k \in \mathbb{N}^*$, da cui si ha anche che $\gamma^k(z)\not=\gamma^h(z)$ per ogni $z \not=0$ e $h \not=k$.
\end{exc}

\begin{exc}
  Se $\theta \not\in \mathbb{Q}$, $\{\gamma^k(z_0) \mid k \in \mathbb{N}\}$ è densa nella circonferenza $\{|z|=|z_0|\}$.
\end{exc}

Vogliamo adesso studiare la dinamica di una $f \in \text{Hol}(\mathbb{D}, \mathbb{D})$ qualunque.

\begin{defn}
  Sia $f \in \text{Hol}(\Omega, \Omega)$, un \textit{punto limite} di $\{f^k\}$ è $g \in \text{Hol}(\Omega, \mathbb{C})$ t.c. è il limite di una sottosuccessione $\{f^{k_{\nu}}\}$, cioè $f^{k_{\nu}} \longrightarrow g$.
\end{defn}

\begin{lm} \label{pli}
  Sia $\Omega \subseteq \mathbb{C}$ dominio, $f \in \text{Hol}(\Omega, \Omega)$. Se $\id_{\Omega}$ è un punto limite di $\{f^k\}$, allora $f \in \text{Aut}(\Omega)$.
\end{lm}

\begin{proof}
  $f^{k_{\nu}} \longrightarrow \id_{\Omega}$ $\implies$ $f$ è iniettiva (se $z_1 \not=z_2$ sono t.c. $f(z_1)=f(z_2)$, allora $f^{k_{\nu}}(z_1)=f^{k_{\nu}}(z_2)$, ma la prima tende a $z_1$ e la seconda a $z_2$, che sono diversi, assurdo).
  Se $z_0 \in \Omega$, $z_0=\id_{\Omega}(z_0)$. Per il primo teorema di Hurwitz, $\id_{\Omega}(z_0) \in f^{k_{\nu}}(\Omega)$ per $\nu \gg 1$, ma $f^{k_{\nu}}(\Omega) \subseteq f(\Omega)$ $\implies$ $f$ è suriettiva.
\end{proof}

\begin{prop} \label{lim_aut}
  Sia $\Omega \subset \subset \mathbb{C}$ un dominio limitato, $f \in \text{Hol}(\Omega, \Omega)$. Sia $h \in \text{Hol}(\Omega, \mathbb{C})$ un punto limite di $\{f^k\}$ (che esiste per il teorema di Montel). Allora o
  \begin{nlist}
    \item $h \equiv c \in \overline{\Omega}$ oppure
    \item $h \in \text{Aut}(\Omega)$ e in questo caso $f \in \text{Aut}(\Omega)$.
  \end{nlist}
\end{prop}

\begin{proof}
  Sia $\displaystyle h=\lim_{\nu \longrightarrow +\infty} f^{k_{\nu}}$. Poniamo $m_{\nu}=k_{\nu+1}-k_{\nu}$. Possiamo supporre $m_{\nu} \longrightarrow +\infty$. Per Montel, a meno di una sottosuccessione possiamo supporre $f^{m_{\nu}} \xrightarrow{\nu \longrightarrow +\infty} g \in \text{Hol}(\Omega, \mathbb{C})$.
  Se $h$ è costante abbiamo finito. Se $h$ non è costante, per il teorema dell'applicazione aperta $h$ è aperta $\implies$ $h(\Omega)$ è aperto e per il primo teorema di Hurwitz è contenuto in $\Omega$.
  Se $z \in \Omega$, $\displaystyle g(h(z))=\lim_{\nu \longrightarrow +\infty} f^{m_{\nu}}(f^{k_{\nu}}(z))=\lim_{\nu \longrightarrow +\infty} f^{k_{\nu+1}}(z)=h(z) \implies g\restrict{h(\Omega)}=\id_{\Omega}$,
  ma per il principio di identità questo ci dà $g \equiv \id_{\Omega}$, dunque per il lemma \ref{pli} abbiamo che $f \in \text{Aut}(\Omega)$. A meno di sottosuccessioni è facile vedere che $f^{-k_{\nu}}=(f^{-1})^{k_{\nu}}$ converge a $h^{-1}$.
\end{proof}

\begin{prop}
  Sia $f \in \text{Hol}(\mathbb{D}, \mathbb{D})$, $f(z_0)=z_0, z_0 \in \mathbb{D}$, $f\not\in \text{Aut}(\mathbb{D})$. Allora $f^k \longrightarrow z_0$ (costante) uniformemente sui compatti.
\end{prop}

\begin{proof}
  A meno di coniugio possiamo supporre $z_0=0$. Per il lemma di Schwarz, $|f(z)|<|z|$ per ogni $z \in \mathbb{D}\setminus\{0\}$. Fissiamo $0<r<1$. In $\overline{\mathbb{D}}_r$, $\left|\dfrac{f(z)}{z}\right|$ ha un massimo $\lambda_r<1$.
  Per ogni $z \in \overline{\mathbb{D}}_r$, $|f(z)| \le \lambda_r|z| \implies |f^2(z)| \le \lambda_r|f(z)| \le \lambda_r^2|z| \implies |f^k(z)| \le \lambda_r^k|z| \le \lambda_r^kr \longrightarrow 0$ per $k \longrightarrow +\infty$ $\implies$ $f^k \longrightarrow 0$ (costante) uniformemente sui compatti.
\end{proof}

\begin{defn}
  Chiamiamo \textit{orociclo} di centro $\tau \in \partial\mathbb{D}$ e raggio $R>0$ l'insieme $E(\tau, R)=\left\{z \in \mathbb{D} \, \bigg| \, \dfrac{|\tau-z|^2}{1-|z|^2}<R \right\}$. Geometricamente, è un disco di raggio $\dfrac{R}{R+1}$ tangente a $\partial \mathbb{D}$ in $\tau$.
\end{defn}

\begin{exc}
  $\displaystyle E(\tau, R)=\left\{z \in \mathbb{D} \, \bigg| \, \lim_{w \longrightarrow \tau} [\omega(z, w)-\omega(0, w)]<\frac{1}{2}\log{R}\right\}$.
\end{exc}

\begin{lm}
  (Lemma di Wolff) Sia $f \in \text{Hol}(\mathbb{D}, \mathbb{D})$ senza punti fissi. Allora esiste un unico $\tau \in \partial\mathbb{D}$ t.c. per ogni $z \in \mathbb{D}$ $\dfrac{|\tau-f(z)|^2}{1-|f(z)|^2} \le \dfrac{|\tau-z|^2}{1-|z|^2}$ $(\star)$.
  In altre parole, per ogni $R>0$ $f(E(\tau, R)) \subseteq E(\tau, R)$.
\end{lm}

\begin{proof}
  Unicità: se ce ne fossero due, $\tau$ e $\tau_1$, prendiamo un orociclo centrato in $\tau$ e uno centrato in $\tau_1$ tangenti, allora il punto di tangenza verrebbe mandato in sé e sarebbe dunque un punto fisso in $\mathbb{D}$, assurdo.

  Esistenza: prendiamo $\{r_{\nu}\} \subset (0, 1)$ t.c. $r_{\nu} \nearrow 1^{-}$ e poniamo $f_{\nu}=r_{\nu}f$ $\implies$ $f_{\nu}(\mathbb{D}) \subseteq \mathbb{D}_{r_{\nu}} \subset\subset \mathbb{D}$,
  allora per il teorema di Ritt esiste $w_{\nu} \in \mathbb{D}$ t.c. $f_{\nu}(w_{\nu})=w_{\nu}$. A meno di sottosuccessioni, possiamo suppore $w_{\nu} \longrightarrow \in \overline{\mathbb{D}}$.
  Se $\tau \in \mathbb{D}$, $\displaystyle f(\tau)=\lim_{\nu \longrightarrow +\infty} f_{\nu}(w_{\nu})=\lim_{\nu \longrightarrow +\infty} w_{\nu}=\tau$, assurdo $\implies$ $\tau \in \partial\mathbb{D}$.
  Per Schwarz-Pick, $\left|\dfrac{f_{\nu}(z)-w_{\nu}}{1-\bar{w}_{\nu}f_{\nu}(z)}\right|^2 \le \left|\dfrac{z-w_{\nu}}{1-\bar{w}_{\nu}z}\right|^2 \implies 1-\left|\dfrac{f_{\nu}(z)-w_{\nu}}{1-\bar{w}_{\nu}f_{\nu}(z)}\right|^2 \ge 1-\left|\dfrac{z-w_{\nu}}{1-\bar{w}_{\nu}z}\right|^2 \implies \dfrac{|1-\bar{w}_{\nu}f_{\nu}(z)|^2}{1-|f_{\nu}(z)|^2} \le \dfrac{|1-\bar{w}_{\nu}z|^2}{1-|z|^2}$.
  Mandando $\nu \longrightarrow +\infty$ otteniamo $\dfrac{|1-\bar{\tau}f(z)|^2}{1-|f(z)|^2} \le \dfrac{|1-\bar{\tau}z|^2}{1-|z|^2}$ che moltiplicando per $\tau$ ($\tau\bar{\tau}=1$) dà la tesi.
\end{proof}

\begin{exc}
  Si ha l'uguaglianza in $(\star)$ nel lemma di Wolff $\iff$ $f$ è un automorfismo parabolico con punto fisso $\tau$ $\iff$ vale l'uguaglianza in $(\star)$ per ogni $z \in \mathbb{D}$.
\end{exc}

\begin{thm}
  (Wolff-Denjoy) Sia $f \in \text{Hol}(\mathbb{D}, \mathbb{D})$ senza punti fissi in $\mathbb{D}$. Allora esiste un unico $\tau \in \partial\mathbb{D}$ t.c. $f^k \longrightarrow \tau$ (costante) uniformemente sui compatti.
\end{thm}

\begin{proof}
  Se $f \in \text{Aut}(\mathbb{D})$ parabolico o iperbolico l'abbiamo già visto. Supponiamo $f \not\in \text{Aut}(\mathbb{D})$. Per Montel, $\{f^k\}$ è relativamente compatta in $\text{Hol}(\mathbb{D}, \mathbb{C})$. Useremo il seguente risultato di topologia che viene lasciato come esercizio.

  \begin{exc} \label{sct}
    Sia $X$ spazio topologico di Hausdorff. Sia $\{x_k\} \subset X$ con $\overline{\{x_k\}}$ compatta in $X$. Supponiamo che esista un unico $\bar{x} \in X$ t.c. ogni sottosuccessione convergente di $\{x_k\}$ converge a $\bar{x}$. Allora $x_k \longrightarrow \bar{x}$.
  \end{exc}

  Sia $\tau \in \partial\mathbb{D}$ dato dal lemma di Wolff. Sia $\displaystyle h=\lim_{\nu \longrightarrow +\infty} f^{k_{\nu}}$ un punto limite di $\{f^k\}$ (che esiste per Montel). Per la proposizione \ref{lim_aut}, $h \equiv \sigma \in \overline{\mathbb{D}}$.
  Se $\sigma \in \mathbb{D}$, $\displaystyle f(\sigma)=\lim_{\nu \longrightarrow +\infty} f(f^{k_{\nu}}(\sigma))=\lim_{\nu \longrightarrow +\infty} f^{k_{\nu}}(f(\sigma))=\sigma$, assurdo. Quindi $h \equiv \sigma \in \partial\mathbb{D}$.
  Vogliamo $\sigma=\tau$. Per il lemma di Wolff $f^{k_{\nu}}(E(\tau, R)) \subseteq E(\tau, R)$ per ogni $R>0$ $\implies$ $\{\sigma\}=h(E(\tau, R)) \subseteq \overline{E(\tau, R)} \cap \partial\mathbb{D}=\{\tau\}$ $\implies$ $\sigma=\tau$.
  Si conclude allora per l'esercizio \ref{sct}.
\end{proof}


\subsection{Germi e prolungamenti analitici}
\begin{defn}
  Sia $\gamma:[0, 1] \longrightarrow \mathbb{C}$ un cammino continuo.
  Se esistono $0=t_0<t_1<\dots<t_r=1$, intorni $U_0, \dots, U_j, \dots, U_r$ di $\gamma(t_j)$ e $f_j:U_j \longrightarrow \mathbb{C}$ olomorfe t.c. $f_j\restrict{U_j \cap U_{j+1}} \equiv f_{j+1}\restrict{U_j \cap U_{j+1}}$ diremo che \textsc{$f_0$ si prolunga olomorficamente lungo $\gamma$}.
\end{defn}

\begin{ex}
  $\gamma(t)=e^{2\pi i t}, \gamma(0)=\gamma(1)=1$. $z=|z|e^{2\pi i \theta}, \theta \in \mathbb{R}$. $U_0=D(1, 1/2), f_0:U_0: \longrightarrow \mathbb{C}, f_0(z)=z^{1/2}=|z|^{1/2}e^{2\pi i(\theta/2)}$ ($\theta \in (-\pi, \pi)$).
  $f_0 \in \mathcal{O}(U_0)$. È possibile prolungare olomorficamente $f_0$ lungo $\gamma$ con $f(\gamma(t))=e^{2\pi i(t/2)}$ $\implies$ $f(\gamma(1))=e^{2\pi i/2}=e^{\pi i}=-1$. $f(\gamma(0))=1$.
\end{ex}

\begin{defn}
  Sia $a \in \mathbb{C}$ e consideriamo le coppie $(U, f)$ dove $U \subseteq \mathbb{C}$ è un intorno aperto di $a$ e $f \in \mathcal{O}(U)$. Definiamo la seguente relazione di equivalenza: $(U, f) \sim (V, g)$ se esiste $W \subseteq U \cap V$ intorno aperto di $a$ t.c. $f\restrict{W}=g\restrict{W}$. \\
  $\mathcal{O}_a:=\faktor{\{(U, f)\}}{\sim}$ è detta \textsc{spiga dei germi di funzioni olomorfe in $a$}. \\
  $\underline{f_a} \in \mathcal{O}_a$ si dice \textsc{germe} di funzione olomorfa. \\
  $(U,f) \in \underline{f_a}$ si dice \textsc{rappresentante} di $\underline{f_a}$. \\
  $\displaystyle \mathcal{O}:=\bigcup_{a \in \mathbb{C}} \mathcal{O}_a$ si dice \textsc{fascio dei germi} di funzioni olomorfe. \\
  Dato $\Omega \subseteq \mathbb{C}$ aperto, definiamo anche $\displaystyle \mathcal{O}_{\Omega}:=\bigcup_{a \in \Omega} \mathcal{O}_a$.
\end{defn}

\begin{exc}
  $\sim$ appena definita è una relazione di equivalenza.
\end{exc}

\begin{exc}
  $\mathcal{O}_a$ è una $\mathbb{C}$-algebra ($\underline{f_a}+\underline{g}_a$ è il germe rappresentato da $(U \cap V, (f+g)\restrict{U \cap V})$ dove $(U,f) \in \underline{f_a}$ e $(V, g) \in \underline{g}_a$).
\end{exc}

\begin{oss}
  Possiamo definire per ogni $k \ge 0$ $\underline{f_a}^{(k)}(a) \in \mathbb{C}$ ponendo $\underline{f_a}^{(k)}(a)=f^{(k)}(a)$ con $(U, f) \in \underline{f_a}$.
\end{oss}

\begin{defn}
  Definiamo $p$ come la \textit{proiezione}
  \begin{align*}
    p: \mathcal{O} &\longrightarrow \mathbb{C}\\
    \underline{f_z} &\longmapsto z
  \end{align*}
  Vale che $p(\mathcal{O}_a)=\{a\}$. Vogliamo rendere $p$ "quasi" un rivestimento (vedremo che, per i soliti esempio stupidi, non può essere un rivestimento).
\end{defn}

Vogliamo definire una topologia su $\mathcal{O}$. Definiamo un sistema fondamentale di intorni.

\begin{defn}
  Gli intorni del sistema fondamentale sono i seguenti: dati $U \subseteq \mathbb{C}$ aperto, $f \in \mathcal{O}(U)$ l'intorno associato è $N(U, f)=\{\underline{f_z} \mid z \in U, (U, f) \in \underline{f_z}\}$.
\end{defn}

\begin{exc}
  Esiste un'unica topologia su $\mathcal{O}$ t.c. $\{N(U, f)\}$ siano un sistema fondamentale di intorni.
\end{exc}

\begin{oss}
  $p\restrict{N(U, f)}:N(U, f) \longrightarrow U$ è una bigezione.
\end{oss}

\begin{prop}
  $\mathcal{O}$ è uno spazio di Hausdorff.
\end{prop}

\begin{proof}
  Siano $\underline{f_a} \not \underline{g_b}$. Se $a \not= b$, esistono $(U, f) \in \underline{f_a}, (V, g) \in \underline{g_b}$ con $U \cap V=\emptyset$ $\implies$ $N(U, f) \cap N(V, g)=\emptyset$.
  Se $a=b$, siano $(U, f) \in \underline{f_a}, (V, g) \in \underline{g_a}$, $D \subset U \cap V$ disco aperto di centro $a$. Vogliamo $N(D, f) \cap N(D, g)=\emptyset$.
  Per assurdo, sia $\underline{h}_z \in N(D, f) \cap N(D, g)$ $\implies$ $z \in D$ e $\underline{h_z}=\underline{f_z}$ e $\underline{h_z}=\underline{g_z}$ $\implies$ $\underline{f_z}=\underline{g_z}$ $\implies$ esiste un aperto $W \subseteq D$ intorno di $z$ t.c. $f\restrict{W}=g\restrict{W}$ e per il principio di identità si avrebbe $f \equiv g$ su $D$ $\implies$ $\underline{f_a}=\underline{g_a}$, assurdo.
\end{proof}

\begin{prop}
  $p: \mathcal{O} \longrightarrow \mathbb{C}$ è continua, aperta e omeomorfismo locale.
\end{prop}

\begin{proof}
  Sia $V \subseteq \mathbb{C}$, $\displaystyle p^{-1}(V)=\bigcup\{N(W, f) \mid W \subseteq V \text{ aperto}, f \in \mathcal{O}(W)\}$ è aperto. $p(N(U, f))=U$ $\implies$ $p$ è aperta.
  $p\restrict{N(U, f)}$ è invertibile: $p^{-1}(z)=\underline{f_z}$ $\implies$ $p\restrict{N(U, f)}$ è un omeomorfismo $\implies$ $p$ è un omeomorfismo locale.
\end{proof}

\begin{defn}
  Una \textit{sezione} di $\mathcal{O}$ su un $\Omega \subset \mathbb{C}$ aperto è una $\underline{f}:\Omega \longrightarrow \mathcal{O}$ continua t.c. $p \circ \underline{f}=\id_{\Omega}$, cioè $\underline{f}(z) \in \mathcal{O}_z$ per ogni $z \in \Omega$.
\end{defn}

\begin{exc}
  L'insieme delle sezioni di $\mathcal{O}$ su $\Omega$ è in corrispondenza biunivoca con lo spazio $\mathcal{O}(\Omega)$ delle funzioni olomorfe su $\Omega$.
\end{exc}

\begin{defn}
  Siano $a \in \mathbb{C}, \underline{f_a} \in \mathcal{O}_a$. Sia $\gamma:[0, 1] \longrightarrow \mathbb{C}$ una curva continua con $\gamma(0)=a$.
  Un \textsc{prolungamento analitico di $\underline{f_a}$ lungo $\gamma$} è un sollevamento $\tilde{\gamma}:[0, 1] \longrightarrow \mathcal{O}$ di $\gamma$ (cioè $p \circ \tilde{\gamma}=\gamma$) t.c. $\tilde{\gamma}(0)=\underline{f_a}$.
\end{defn}

\begin{oss}
  $p$ non è un rivestimento perché non tutte le curve possono essere sollevate. Vediamo un esempio.
\end{oss}

\begin{ex}
  $a=1, \underline{f_a}=(\mathbb{C}^*, 1/z), \gamma(t)=1-t$. Non esiste alcun sollevamento di $\gamma$ che parte da $\underline{f_a}$.
\end{ex}

\begin{defn}
  Sia $\diff:\mathcal{O} \longrightarrow \mathcal{O}$ così definita: dato $\underline{f_a} \in \mathcal{O}_a$, $\diff\underline{f_a}$ è il germe in $a$ rappresentato dalla derivata di un rappresentante di $\underline{f_a}$, cioè se $(U, f) \in \underline{f_a}$, $\diff\underline{f_a}$ è rappresentato da $(U, f')$.
\end{defn}

\begin{lm} \label{primitiva}
  Sia $D \subseteq \mathbb{C}$ un disco aperto. Allora ogni $f \in \mathcal{O}(D)$ ha una primitiva in $D$, e due primitive differiscono per una costante additiva.
\end{lm}

\begin{proof}
  Se $a \in D$ è il centro, $\displaystyle f(z)=\sum_{n=0}^{+\infty} c_n(z-a)^n$. Una primitiva è data da $\displaystyle F(z)=\sum_{n=0}^{+\infty} c_n(z-a)^n$. È chiaro che due primitive differiscono per una costante additiva.
\end{proof}

\begin{prop}
  $\diff:\mathcal{O} \longrightarrow \mathcal{O}$ è un rivestimento.
\end{prop}

\begin{proof}
  Dati $\underline{f_a} \in \mathcal{O}_a$, $(U, f) \in \underline{f_a}$, $D \subseteq U$ un disco centrato in $a$, poniamo $\mathcal{D}=N(D, f)$, intorno aperto di $\underline{f_a}$. Sia $F$ una primitiva di $f$ su $D$ che esiste per il lemma \ref{primitiva}, per ogni $c \in \mathbb{C}$ poniamo $\mathcal{D}_c=N(D, F+c)$. Vogliamo dimostrare che:
  \begin{nlist}
    \item $\displaystyle \diff^{-1}(\mathcal{D})=\bigcup_{c \in \mathbb{C}} \mathcal{D}_c$;
    \item $\diff\restrict{\mathcal{D}_c}:\mathcal{D}_c \longrightarrow \mathcal{D}$ è un omeomorfismo;
    \item $c_1\not=c_2 \implies \mathcal{D}_{c_1} \cap \mathbb{D}_{c_2} \emptyset$.
  \end{nlist}
  (i), (ii), (iii) $\implies$ $\diff$ è un rivestimento. Procediamo con la dimostrazione.
  \begin{nlist}
    \item Sia $z \in D$ e $\underline{f_z} \in \mathcal{D}$. Sia $\underline{g_z} \in \mathcal{O}_z$ t.c. $\diff\underline{g_z}=\underline{f_z}$ $\implies$ esiste $(W, g) \in \underline{g_z}$ t.c. $g'=f$;
    possiamo supporre che $W \subseteq D$, il disco, quindi sempre per il lemma \ref{primitiva} esiste $c \in \mathbb{C}$ t.c. $g\restrict{W}=F\restrict{W}+c$ $\implies$ $\underline{g_z} \in \mathcal{D}_c$. È banale vedere che $\underline{g_z} \in \mathcal{D}_c \implies \diff\underline{g_z} \in \mathcal{D}$.
    \item È ovvio che $\diff(\mathcal{D}_c)=\mathcal{D}$ (per definizione di $\diff$ e $\mathcal{D}_c$). Questo più il punto (i) ci danno che $\diff$ è continua e aperta: infatti,\ gli insiemi della forma $\mathcal{D}$ formano un sistema fondamentale di intorni e la loro preimmagine, unione di aperti, è aperta; anche gli insiemi $\mathcal{D}_c$ sono un sistema fondamentale di intorni (ogni funzione olomorfa è la primitiva della sua derivata) e la loro immagine, come abbiamo visto, è un aperto.
    $\diff\restrict{\mathcal{D}_c}: \mathcal{D}_c \longrightarrow \mathcal{D}$ è, come visto sopra, suriettiva, ma anche iniettiva perché $\displaystyle \mathcal{D}_c=\bigcup_{z \in D} \mathcal{D}_c \cap \mathcal{O}_z, \mathcal{D}=\bigcup_{z \in D} \mathcal{D} \cap \mathcal{O}_z$,
     ma per ogni $z \in D$, $\mathcal{D}_c\cap \mathcal{O}_z$ e $\mathcal{D}\cap \mathcal{O}_z$ contengono un unico germe e $\diff(\mathcal{O}_z) \subseteq \mathcal{O}_z$, da cui appunto segue l'iniettività ($z\not=z' \implies \mathcal{O}_z\cap\mathcal{O}_{z'}=\emptyset$).
     \item Se $\underline{F_z} \in \mathcal{D}_{c_1}\cap \mathcal{D}_{c_2}$ $\implies$ $\underline{F_z}$ è rappresentanto sia da $(D, F+c_1)$ che da $(D, F+c_2)$ $\implies$ $F+c_1\equiv F+c_2$ vicino a $z$ $\implies$ $c_1=c_2$.
  \end{nlist}
\end{proof}

\begin{thm}
  Sia $\Omega \subseteq \mathbb{C}$ un aperto semplicemente connesso. Allora ogni $f \in \mathcal{O}(\Omega)$ ammette una primitiva.
\end{thm}

\begin{proof}
  Sia $\varphi:\Omega \longrightarrow \mathcal{O}$ la sezione corrispondente a $f$. Sia $\Phi$ un sollevamento di $\varphi$, cioè $d \circ \Phi=\varphi$ (che esiste per la teoria generali dei rivestimenti).
  \begin{center}
    \begin{tikzcd}
      & \mathcal{O} \arrow[d, "\diff"]\\
      \Omega \arrow[ru, "\Phi"] \arrow[r, "\varphi"] & \mathcal{O}
    \end{tikzcd}
  \end{center}
  Anche $\Phi$ è una sezione di $\mathcal{O}$: infatti, siccome $p \circ \diff=p$, $p \circ \Phi=p\circ\diff\circ\Phi=p\circ\varphi=\id_{\Omega}$ $\implies$ la $F \in \mathcal{O}(\Omega)$ associata a $\Phi$ è una primitiva di $f$.
\end{proof}

Concludiamo il paragrafo definendo logaritmo e radice $n$-esima su insiemi semplicemente connessi.

\begin{cor}
  Sia $\Omega \subseteq \mathbb{C}$ aperto semplicemente connesso, $f \in \text{Hol}(\Omega, \mathbb{C}^*)$. Allora esiste $g \in \mathcal{O}(\Omega)$ t.c. $f=\exp(g)$. Inoltre $g$ è unica a meno di costanti additive della forma $2k\pi i$ con $k \in \mathbb{Z}$.
\end{cor}

\begin{proof}
  Sia $g_0$ una primitiva di $f'/f$. $\dfrac{\diff}{\diff z}(fe^{-g_0})=f'e^{-g_0}+f(-e^{-g_0}g_0')=f'e^{-g_0}+f(-e^{-g_0}f'/f)=e^{-g_0}(f'-f')=0$ $\implies$ $f \cdot e^{-g_0}=\text{costante diversa da zero}=e^{c_0}$ $\implies$ $f\equiv e^{c_0+g_0}$.
  Per l'unicità a meno di costanti additive, $\exp(g_1)=\exp(g_2) \implies \exp(g_1-g_2)\equiv 1 \implies$ $g_1-g_2$ è continua a valori in $2\pi i \mathbb{Z}$ discreto $\implies$ $g_1-g_2=2k\pi i$ con $k \in \mathbb{Z}$ costante.
\end{proof}

\begin{cor}
  Sia $\Omega \subseteq \mathbb{C}$ aperto semplicemente connesso, $f \in \text{Hol}(\Omega, \mathbb{C}^*)$, $n \in \mathbb{Z}^*$. Allora esiste $h \in \mathcal{O}(\Omega)$ t.c. $f=h^n$. Inoltre $h$ è unica a meno di costanti moltiplicative della forma $e^{2\pi ik/n}$ con $k \in \mathbb{Z}$.
\end{cor}

\begin{proof}
  Sia $g \in \mathcal{O}(\Omega)$ t.c. $f=\exp(g)$, allora $h=\exp(g/n)$ soddisfa le condizioni richieste. Poi, $h_1^n=h_2^n \iff (h_1/h_2)^n\equiv 1 \implies h_1=e^{2\pi i k/n}h_2$ con $k \in \mathbb{Z}$ costante.
\end{proof}


\subsection{Essere biolomorfi al disco}
Lo scopo di questo paragrafo è mostrare che quasi tutti i domini semplicemente connessi di $\mathbb{C}$ sono biolomorfi al disco. Il teorema di uniformizzazione di Riemann, di cui riporteremo solo l'enunciato, caratterizza i biolomorfismi delle superfici di Riemann, in particolare caratterizza completamente i biolomorfismi di quelle semplicemente connesse.

\begin{lm} \label{esistefinF}
  Sia $\Omega \subset \mathbb{D}$ dominio limitato con $\Omega \not=\mathbb{D}$ e $0 \in \Omega$.
  Allora esiste $f \in \text{Hol}(\mathbb{D}, \mathbb{D})$ t.c. $f(0)=0, f'(0) \in \mathbb{R}, f'(0)>0$, $\Omega \subseteq f(\mathbb{D})$ e, se $\Omega_f$ è la componente connessa di $f^{-1}(\Omega)$ contentente $0$, $f\restrict{\Omega_f}:\Omega_f \longrightarrow \Omega$ è un rivestimento.
  Inoltre, $\displaystyle d_1=\inf_{z \not\in \Omega_f} |z|>\inf_{z \not\in \Omega} |z|=d$.
\end{lm}

\begin{proof}
  Sia $a \in \mathbb{D}\setminus\Omega$, $b \in \mathbb{D}$ t.c. $b^2=-a$. Siano $\varphi, \psi \in \text{Aut}(\mathbb{D})$, $\varphi(z)=\dfrac{z+a}{1+\bar{a}z}, \psi(z)=\dfrac{z+b}{1+\bar{b}z}$.
  Poniamo $f \in \text{Hol}(\mathbb{D}, \mathbb{D})$ t.c. $f(z)=\dfrac{\bar{b}}{|b|}\varphi(\psi(z)^2)$. $f(\mathbb{D})=\mathbb{D}, f(0)=0$. $f'(0)=2|b|\dfrac{1-|b|^2}{1-|a|^2}>0$.
  Siccome $w \longmapsto w^2$ è un rivestimento di $\mathbb{D}^*=\mathbb{D}\setminus\{0\}$ e $\varphi^{-1}(\Omega) \subseteq \mathbb{D}^*$, $f$ è un rivestimento da $\Omega_f$ a $\Omega$. Se $d_1=1$ abbiamo finito in quanto $d \le |a|<1$.
  Se invece $d_1<1$, esiste $z_1 \in \partial\Omega_f \cap \mathbb{D}$ con $|z_1|=d_1 \implies f(z_1) \not\in \Omega \implies |f(z_1)| \ge d$. Allora per il lemma di Schwarz abbiamo $d \le |f(z_1)|<|z_1|=d_1$.
\end{proof}

\begin{thm}
  (Osgood, Koebe) Sia $\Omega \subset \subset \mathbb{C}$ un dominio limitato, $z_0 \in \Omega$. Allora esiste un unico rivestimento olomorfo $f_0:\mathbb{D} \longrightarrow \Omega$ t.c. $f_0(0)=z_0$ e $f_0'(0) \in \mathbb{R}, f'(0)>0$.
\end{thm}

\begin{proof}
  Possiamo supporre $z_0=0 \in \Omega$ e $\Omega \subset \subset \mathbb{D}$. Sia $\mathcal{F} \subset \text{Hol}(\mathbb{D}, \mathbb{D})$ t.c.
  $\mathcal{F}=\{f \in \text{Hol}(\mathbb{D}, \mathbb{D}) \mid f(0)=0, f'(0) \in \mathbb{R}, f'(0)>0; \Omega \subseteq f(\mathbb{D});$ se $\Omega_f$ è la componente connessa di $f^{-1}(\Omega)$ contenente $0$ allora $f\restrict{\Omega_f}:\Omega_f \longrightarrow \Omega$ è un rivestimento$\}$.
  Se esiste $f_0 \in \mathcal{F}$ con $\Omega_{f_0}=\mathbb{D}$ ci resta da dimostrare solo l'unicità. Poniamo per ogni $f \in \mathcal{F}$ $\displaystyle d_f=\inf_{z \not\in \Omega_f} |z|=\min_{z \in \partial\Omega_f} |z| \le 1$. Abbiamo $d_f=1 \iff \Omega_f=\mathbb{D}$.
  Dobbiamo trovare $f_0 \in \mathcal{F}$ con $d_{f_0}=1$. Sia $\displaystyle d=\sup_{f \in \mathcal{F}} d_f \le 1$. Sia $\{f_n\} \subset \mathcal{F}$ t.c. $d_{f_n} \longrightarrow d$.
  Per il teorema di Montel possiamo supporre che $f_n \longrightarrow f_0 \in \text{Hol}(\mathbb{D}, \mathbb{C})$ con immaggine in $\overline {\mathbb{D}}$. Vogliamo $f_0 \in \mathcal{F}$. Chiaramente $f_0(0)=0, f_0'(0) \in \mathbb{R}, f'(0) \ge 0$.
  \begin{enumerate}
    \item $f_0$ non è costante e $f'(0)>0$: sia $r>0$ t.c. $\mathbb{D}_r \subset \subset \Omega$. Siccome $f_n:\Omega_{f_n} \longrightarrow \Omega$ è un rivestimento e $0 \in \Omega_{f_n}$, esiste un unico $h_n:\mathbb{D}_r \longrightarrow \Omega_{f_n}$ olomorfa t.c. $f_n \circ h_n=\id_{\mathbb{D}_r}$ e $h_r(0)=0$.
    Sempre per il teorema di Montel, a meno di sottosuccessioni possiamo supporre $h_n \longrightarrow h_0 \in \text{Hol}(\mathbb{D}_r, \overline{\mathbb{D}})$ t.c. $f_0 \circ h_0=\id_{\mathbb{D}_r}$.
    Dunque $h_0$ è iniettiva, quindi per il teorema dell'applicazione aperta è aperta, perciò $h_0(\mathbb{D}_r) \subseteq \mathbb{D}$, e $f_0$ non è costante, dunque per il primo teorema di Hurwitz $f_0(\mathbb{D}) \subseteq \mathbb{D}$. $1=\id_{\mathbb{D}_r}'(0)=(f_0 \circ h_0)'(0)=f_0'(h_0(0))\cdot h_0'(0)=f_0'(0) \cdot h_0'(0) \implies f_0'(0) \not=0 \implies f_0'(0)>0$.
    \item $f_0(\Omega_{f_0})=\Omega$ dove $\Omega_{f_0}$ è la componente connessa di $f_0^{-1}(\Omega)$ contenente $0$. Sia infatti $z_0 \in \Omega$ e sia $\gamma$ una curva in $\Omega$ da $0$ a $z_0$.
    Ricopriamo $\gamma$ con dischi $D_0=\mathbb{D}_r, D_1, \dots, D_k$ con $D_j \cap D_{j+1} \not=\emptyset$ e $z_0 \in D_k$. Sia $h_{n, 0}$ l'inversa di $f_n$ su $D_0$ t.c. $h_{n,0}=0$.
    Per ogni $j$ sia $h_{n, j}$ l'inversa di $f_n$ su $D_j$ che coincide con $h_{n, j-1}$ su $D_{j-1} \cap D_j$. Per Montel, a meno di sottosuccessioni $h_{n, 0} \longrightarrow h_{0, 0} \in \text{Hol}(D_0, \mathbb{D})$.
    Per il teorema di Vitali, $h_{n, 1} \longrightarrow h_{0, 1} \in \text{Hol}(D_1, \mathbb{D})$ e, per induzione, $h_{n, k} \longrightarrow h_{0, k} \in \text{Hol}(D_k, \mathbb{D})$ con $f_0 \circ h_{0, k}=\id_{D_k} \implies f_0(h_{0, k}(z_0))=z_0 \implies z_0 \in f_0(\mathbb{D})$.
    In realtà $h_{0, k}(z_0) \in \Omega_{f_0}$ perché è immagine della curva ottenuta con $h_{0, j} \circ \gamma$ che parte da $0$ e quindi è contenuta nella componente connessa di $\Omega$ contenete $0$.
    \item $f_0:\Omega_{f_0} \longrightarrow \Omega$ è un rivestimento. Sia $z_0 \in \Omega$, $D \subseteq \Omega$ un disco di centro $z_0$.
    Per ogni $w_0 \in f_0^{-1}(z_0) \cap \Omega_{f_0}$ vogliamo un intorno $U_{w_0} \subseteq \Omega_{f_0}$ t.c. $f_0\restrict{U_{w_0}}:U_{w_0} \longrightarrow D$ è un biolomorfismo e $w_0 \not=w_0' \implies U_{w_0}\cap U_{w_0'}=\emptyset$.
    Per il primo teorema di Hurwitz, fissato $w_0$ esiste $n_1 \ge 1$ t.c. per ogni $n \ge n_1$ esiste $w_n \in \Omega_{f_n}$ t.c. $g_n(w_n)=z_0$ e $w_n \longrightarrow w_0$. Sia $h_n \in \text{Hol}(D, \mathbb{D})$ l'inversa di $f_n$ su $D$ con $h_n(z_0)=w_n$ (possiamo usare $D$ per tutte le $f_n$ perché, dalla teoria dei rivestimenti, dato un rivestimento un aperto semplicemente connesso dell'insieme di arrivo è sempre ben rivestito).
    Per Montel, a meno di sottosuccessioni $h_n \longrightarrow h_0 \in \text{Hol}(D, \mathbb{D})$ t.c. $f_0 \circ h_0=\id_D$ (perché $f_n \circ h_n=\id_D$ per ogni $n$), $h_0(z_0)=w_0$.
    Poniamo $U_{w_0}=h_0(D)$. È aperto perché $h_0$ non costante $\implies$ $h_0$ aperta, inoltre $U_{w_0} \subseteq \Omega_{f_0}$ perché, essendo immagine di un connesso, è connesso, e contiene $z_0=h_0(z_0)$, dunque deve stare nella componente connessa di $z_0$. Sia $w_0' \in f_0^{-1}(z_0) \cap \Omega_{f_0}$, costruiamo $h_0', U_{w_0'}$ come prima, senza perdita di generalità con la stessa sottosuccessione.
    Per assurdo, esiste $w \in U_{w_0} \cap U_{w_0'}$. Allora esistono $z_1, z_1' \in D$ t.c. $w=h_0(z_1)=h_0'(z_1')$.
    Applichiamo $f_0$ a entrambi i membri: $z_1=f_0(h_0(z_1))=f_0(w)=f_0(h_0(z_1'))=z_1' \implies z_1=z_1'$ è uno zero di $h_0-h_0'$, dunque per il primo teorema di Hurwitz o $h_0-h_0' \equiv 0 \implies w_0=w_0' \implies U_{w_0}=U_{w_0'}$ oppure per ogni $n>>1$ $h_n-h_n'$ ha uno zero, ma $h_n$ è l'inversa di $f_n$ su $D$ con $w_n \in h_n(D)$, $h_n'$ è l'inversa di $f_n$ su $D$ con $w_n' \in h_n'(D)$.
    $w_0 \not=w_0' \implies w_n \not=w_n'$ per ogni $n>>1 \implies h_n(D)\cap h_n'(D)=\emptyset$ perché inverse di un rivestimento che mandano lo stesso punto in due punti diversi, assurdo.
  \end{enumerate}

  Per il lemma \ref{esistefinF}, $\mathcal{F}\not=\emptyset$, quindi la costruzione che abbiamo fatto di $f_0$ ha senso. Per assurdo, $\displaystyle d_{f_0}=\sup_{f \in \mathcal{F}} d_f<1 \implies \Omega_{f_0} \subset \mathbb{D}$ con $\Omega_{f_0} \not= \mathbb{D}$.
  Sempre per il lemma \ref{esistefinF} esiste $f_1 \in \text{Hol}(\mathbb{D}, \mathbb{D})$, $f_1(0)=0, f_1'(0) \in \mathbb{R}, f_1'(0)>0$, $\Omega_{f_0} \subset f_1(\mathbb{D})$, $f_1\restrict{\Omega_{f_1}}:\Omega_{f_1} \longrightarrow \Omega_{f_0}$ rivestimento con $\Omega_{f_1}$ la componente connessa di $f_1^{-1}(\Omega_{f_0})$ contentente $0$ e $d_{f_1}>d_{f_0}$.
  Ma allora, se $\tilde{f}=f_0 \circ f_1:\Omega_{f_1} \longrightarrow \Omega$, abbiamo $\tilde{f} \in \mathcal{F}$ con $d_{\tilde{f}}=d_{f_1}>d_{f_0}$, assurdo.

  Per l'unicità si veda l'esercizio \ref{un_biolo}.
\end{proof}

\begin{exc} \label{un_biolo}
  Se $f_1, f_2:\mathbb{D} \longrightarrow \Omega$ sono rivestimenti con $f_1(0)=f_1(0)$ e $f_1'(0), f_2'(0)>0$, allora $f_1 \equiv f_2$. Hint: dato che sono rivestimenti, si sfruttano $h$ e $h^{-1}$ date dal seguente diagramma commutativo:
  \begin{center}
    \begin{tikzcd}
      & \mathbb{D} \arrow[dl, "h"', shift right] \arrow[d, "f_1"]\\
      \mathbb{D} \arrow[ru, "h^{-1}" right, shift right] \arrow[r, "f_2"'] & \Omega
    \end{tikzcd}
  \end{center}
\end{exc}

\begin{thm}
  (Riemann) Se $\Omega \subset \mathbb{C}$ è un dominio semplicemente connesso con $\Omega \not=\mathbb{C}$, allora $\Omega$ è biolomorfo a $\mathbb{D}$.
\end{thm}

\begin{proof}
  Basta far vedere che $\Omega$ è biolomorfo a un dominio limitato: quest'ultimo sarebbe semplicemente connesso e rivestito da $\mathbb{D}$ che è connesso, dunque per la teoria generale dei rivestimenti il rivestimento in questione sarebbe un omeomorfismo, ma dato che era anche un olomorfismo, allora è un biolomorfismo. Prendiamo $a \in \mathbb{C}\setminus \Omega$, $h \in \mathcal{O}(\Omega)$ t.c. $h(z)^2=z-a$ per ogni $z \in \Omega$. $h$ è iniettiva, ma vale di più: $h(z_1)=\pm h(z_2) \implies z_1-a=h(z_1)^2=h(z_2)^2=z_2-a \implies z_1=z_2$.
  Dunque $h$ è iniettiva e $h(\Omega) \cap (-h(\Omega))=\emptyset$. Fissato $z_0 \in \Omega$, sia $r>0$ t.c. $D=D(h(z_0), r) \subset h(\Omega) \implies D \cap(-h(\Omega))=\emptyset \implies |h(z)+h(z_0)| \ge r$ per ogni $z \in \Omega \implies 2|h(z_0)| \ge r$.
  Sia $f \in \mathcal{O}(\Omega)$ data da $f(z)=\dfrac{r}{4}\dfrac{1}{|h(z_0)|}\dfrac{h(z)-h(z_0)}{h(z)+h(z_0)}$. $f(z_0)=0$ e $f$ è iniettiva: $f(z_1)=f(z_2) \implies \dfrac{h(z_2)-h(z_0)}{h(z_1)+h(z_0)}=\dfrac{h(z_2)-h(z_0)}{h(z_2)+h(z_0)} \implies h(z_1)=h(z_2) \implies z_1=z_2$
  $\implies$ $f$ è un biolomorfismo tra $\Omega$ e $f(\Omega)$ e $f(\Omega) \subseteq \mathbb{D}$ in quanto $\left|\dfrac{h(z)-h(z_0)}{h(z)+h(z_0)}\right|=|h(z_0)|\left|\dfrac{1}{h(z_0)}-\dfrac{2}{h(z)+h(z_0)}\right| \le |h(z_0)|\left|\dfrac{1}{|h(z_0)|}+\dfrac{2}{|h(z)+h(z_0)|}\right| \le \dfrac{4|h(z_0)|}{r}$.
\end{proof}

Per il teorema di Liouville abbiamo che $\mathbb{C}$ non è biolomorfo a $\mathbb{D}$. Dato che $\widehat{\mathbb{C}}$ è compatta, abbiamo che non è biolomorfa né a $\mathbb{D}$ né a $\mathbb{C}$.

\begin{thm}
  (Uniformizzazione di Riemann) Se $X$ è una superficie di Riemann semplicemente connessa, allora $X$ è biolomorfo a $\widehat{\mathbb{C}}$, $\mathbb{C}$ o $\mathbb{D}$. Più in generale, se $X$ è una superficie di Riemann qualsiasi e $\pi:\widetilde{X} \longrightarrow X$ è un rivestimento universale, allora $\widetilde{X}$ è una superficie di Riemann e
  \begin{nlist}
    \item se $\widetilde{X}$ è biolomorfo a $\widehat{\mathbb{C}}$, allora anche $X$ è biolomorfo a $\widehat{\mathbb{C}}$ (caso ellittico);
    \item se $\widetilde{X}$ è biolomorfo a $\mathbb{C}$, allora $X$ è biolomorfo a $\mathbb{C}$, $\mathbb{C}^*$, oppure un toro $T_{\tau}=\faktor{\mathbb{C}}{(\mathbb{Z}+\tau \mathbb{Z})}$ con $\mathfrak{Im}\tau>0$ (caso parabolico);
    \item in tutti gli altri casi $\widetilde{X}$ è biolomorfo a $\mathbb{D}$ (caso iperbolico).
  \end{nlist}
\end{thm}

Quindi, se $\Omega \subset\subset \mathbb{C}$ è limitato e semplicemente connesso, abbiamo per il teorema di Riemann che esiste $f:\mathbb{D} \longrightarrow \Omega$ biolomorfismo. Domanda: possiamo estendere $f$ a un omeomorfismo da $\overline{\mathbb{D}}$ a $\overline{\Omega}$?

\begin{thm}
  (Carathéodory) Un biolomorfismo $\mathbb{D} \longrightarrow \Omega$ si estende continuo da $\overline{\mathbb{D}}$ a $\overline{\Omega}$ se e solo se $\partial \Omega$ è localmente connesso.
\end{thm}

\begin{cor}
  Si estende a un omeomorfismo se e solo se $\partial\Omega$ è una curva di Jordan (cioè immagine omeomorfa di $S^1$).
\end{cor}

Esiste una condizione su $\partial\Omega$ diversa che è equivalente all'estendibilità di $f^{-1}:\Omega \longrightarrow \mathbb{D}$ al bordo.


\end{document}
