\begin{thebibliography}{widest entry}
  \bibitem[BK]{BK} D. M. Burns, S. G. Krantz: Rigidity of holomorphic mappings and a new Schwarz lemma at the boundary. \textit{Journal of the American Mathematical Society}, \textbf{Volume 7} (1994), no. 3, 661--676
  \bibitem[BKR]{BKR} F. Bracci, D. Kraus, O. Roth: A new Schwarz-Pick Lemma at the boundary and rigidity of holomorphic maps. Preprint, ArXiv:2003.02019v1 (2020)
  \bibitem[BM]{BM} A. F. Beardon, D. Minda: A multi-point Schwarz-Pick lemma. \textit{Journal d'Analyse Mathématique}, \textbf{Volume 92} (2004), 81--104
  \bibitem[D]{D} J. Dieudonné: Recherches sur quelques problèmes relatifs aux polynômes et aux fonctions bornées d'une variable complexe. \textit{Annales Scientifiques de l'École Normale Supérieure}, \textbf{Volume 48} (1931), 247--358
  \bibitem[GMG]{GMG} G. M. Golusin: Some estimations of derivatives of bounded functions. \textit{Recueil Mathématique [Matematicheskiĭ Sbornik]}, \textbf{Volume 16(58)} (1945), no. 3, 295--306
  \bibitem[JBG]{JBG} J. B. Garnett: \textbf{Bounded Analytic Functions (Revised First Edition)}. Springer, New York, 2007
  \bibitem[NN]{NN} R. Narasimhan, Y. Nievergelt: \textbf{Complex analysis in one variable (2nd edition)}. Springer, New York, 2001
  \bibitem[R]{R} W. Rogosinski: Zum Schwarzschen Lemma. \textit{Jahresbericht der Deutschen Mathematiker-Vereinigung}, \textbf{Volume 44} (1934), 258--261
\end{thebibliography}
