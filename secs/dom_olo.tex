\begin{defn}
  Sia $\Omega \subset \mathbb{C}^n$ un dominio. $P \in \partial \Omega$ è \textit{essenziale} se esiste $u \in \mathcal{O}(\Omega)$ t.c. per ogni intorno aperto connesso $\Omega_2$ di $P$ e per ogni aperto connesso $\Omega_1 \subseteq \Omega \cap \Omega_2$ con $\Omega_1\not=\emptyset,\Omega$ non esiste $u_2 \in \mathcal{O}(\Omega)$ con $u_2\restrict{\Omega_1}=u\restrict{\Omega_1}$.
  Diremo che $\Omega$ è un \textsc{dominio di olomorfia} se ogni $P \in \partial D$ è essenziale.
\end{defn}

\begin{defn}
  Un \textsc{funzionale di Minkovski} è una funzione $\mu:\mathbb{C}^n \longrightarrow \mathbb{R}^+\cup\{0\}$ continua t.c.
  \begin{nlist}
    \item $\mu(z)=0 \iff z=0$;
    \item $\mu(\zeta z)=|\zeta|\mu(z)$ per ogni $z \in \mathbb{C}^n$ e per ogni $\zeta \in \mathbb{C}$.
  \end{nlist}
\end{defn}

\begin{ex}
  $\mu(z)=\|z\|_p=(|z_1|^p+\dots+|z_n|^p)^{1/p}$ con $p>0$ e $\mu(z)=\|z\|_{\infty}=\max\{|z_1|,\dots,|z_n|\}$ sono funzionali di Minkovski.
\end{ex}

\begin{exc}
  $\Omega \subseteq \mathbb{C}^n$ dominio con $0 \in \Omega$ è circolare e stellato rispetto a $0$ (cioè per ogni $\zeta \in \mathbb{D}, z \in \Omega$ anche $\zeta z \in \Omega$) $\iff$ esiste $\mu$ funzionale di Minkovski t.c. $\Omega=\{z \in \mathbb{C}^n \mid \mu(z)<1\}$. Hint: una freccia è ovvia, per l'altra si consideri $\mu(z)=\inf\{r>0 \mid z/r \in \Omega \}$.
\end{exc}

Se $\Omega \subset \mathbb{C}^n$ è un dominio poniamo $\displaystyle \mu_\Omega(z)=\inf_{w \in \mathbb{C}^n\setminus\Omega} \mu(z-w)=$"$\mu$-distanza da $\partial\Omega$". Se $X \subseteq \Omega$ poniamo $\displaystyle \mu_\Omega(X)=\inf_{z \in X} \mu_\Omega(z)$.

\begin{defn}
  Sia $\Omega \subseteq \mathbb{C}^n$ un dominio, $\mathcal{F}$ una famiglia di funzioni su $\Omega$. Sia $K \subset \Omega$. Il \textsc{$\mathcal{F}$-inviluppo di $K$ in $\Omega$} è $\hat{K}_{\mathcal{F}}=\{z \in \Omega \mid |f(z)| \le \|f\|_K \text{ per ogni } f \in \mathcal{F}\}$.
  Se $\mathcal{F}=\mathcal{O}(\Omega)$, $\hat{K}_{\mathcal{F}}=\hat{K}_{\Omega}$ è \textit{inviluppo olomorfo di $K$}, che abbiamo già incontrato in una variabile.
\end{defn}

Diremo che $\Omega$ è \textit{$\mathcal{F}$-convesso} se e solo se per ogni $K \subset \subset \Omega$ compatto anche $\hat{K}_{\mathcal{F}} \subset \subset \Omega$ compatto. Se $\mathcal{F}=\mathcal{O}(\Omega)$, diremo che $\Omega$ è \textsc{olomorficamente convesso}.

\begin{oss}
  Ogni aperto di $\mathbb{C}$ è olomorficamente convesso.
\end{oss}

\begin{exc}
  Siano $\Omega=P^2(0,1)\setminus\overline{P^2(0,1/2)}, K=\{(0,3e^{i\theta}/4) \mid \theta \in \mathbb{R}\}$. Dimostrare che $\hat{K}_\Omega=\{0, te^{i\theta} \mid \theta \in \mathbb{R}, 1/2 < t \le 3/4\}$.
\end{exc}

\begin{exc}
  Se $\Omega \subseteq \mathbb{R}^n$, $\mathcal{F}=\text{funzioni lineari su }\Omega$, allora $\hat{K}_{\mathcal{F}}=\text{inviluppo convesso di }K$, e $\Omega$ è $\mathcal{F}$-convesso $\iff$ è convesso (nella disuguaglianza che definisce gli insiemi in $\mathbb{R}$ c'è una differenza: non si prende il modulo [notare quindi che anche al posto della norma infinito c'è un $\sup$]).
\end{exc}

\begin{lm}
  Sia $\Omega \subseteq \mathbb{C}^n$ un dominio, $K \subset \Omega$ limitato $\implies$ $\hat{K}_{\Omega}$ è limitato.
\end{lm}

\begin{proof}
  $K$ è limitato $\iff$ per ogni $j=1,\dots,n$ esiste $M_j$ t.c. $|z_j| \le M_j$ per ogni $z \in K$. Dato che le proiezioni alle singole coordinate sono funzioni olomorfe, otteniamo che per ogni $z \in \hat{K}_\Omega$ $|z_j| \le M_j$ $\implies$ $\hat{K}_\Omega$ limitato.
\end{proof}

\begin{oss}
  Se $\mathcal{F} \subset C^0(\Omega)$, allora $\hat{K}_{\mathcal{F}}$ è chiuso in $\Omega$.
\end{oss}

\begin{lm}
  Sia $\Omega \subseteq \mathbb{C}^n$ un dominio, $K \subset \Omega$ $\implies$ $\hat{K}_\Omega$ è contenuto nella chiusura dell'inviluppo convesso di $K$.
\end{lm}

\begin{proof}
  $\mathcal{O}(\Omega)$ contiene le funzioni $e^{L(z)}$ con $L: \mathbb{C}^n \longrightarrow \mathbb{C}$ lineare. $|e^{L(z)}|=e^{\mathfrak{Re}L(z)}$. $|e^{L(z)}| \le \|e^L\|_K$ $\iff$ $\displaystyle \mathfrak{Re}L(z) \le \sup_{w \in K} \mathfrak{Re}L(w)$.
  Ogni $l: \mathbb{C}^n \longrightarrow \mathbb{R}$ $\mathbb{R}$-lineare è la parte reale di $L:\mathbb{C}^n \longrightarrow \mathbb{C}$ $\mathbb{C}$-lineare.
\end{proof}

\begin{thm} \label{car_dom_olo}
  Sia $\Omega \subseteq \mathbb{C}^n$ un dominio. Sono equivalenti:
  \begin{nlist}
    \item esiste $h \in \mathcal{O}(\Omega)$ che non può essere estesa olomorficamente a un qualsiasi aperto $\Omega' \supsetneq \Omega$;
    \item $\Omega$ è un dominio di olomorfia;
    \item $\Omega$ è olomorficamente convesso;
    \item per ogni $\mu$ funzionale di Minkovski, per ogni $f \in \mathcal{O}(\Omega)$ e per ogni $K \subset\subset \Omega$ compatto, $|f| \le \mu_{\Omega}$ su $K$ $\implies$ $|f| \le \mu_\Omega$ su $\hat{K}_\Omega$;
    \item per ogni $\mu$ funzionale di Minkovski e per ogni $K \subset\subset \Omega$ compatto, $\mu_{\Omega}(\hat{K}_\Omega)=\mu_\Omega(K)$;
    \item per ogni $\mu$ funzionale di Minkovski, per ogni $f \in \mathcal{O}(\Omega)$ e per ogni $K \subset\subset \Omega$ compatto, $\displaystyle \sup_{z \in K} \frac{|f(z)|}{\mu_\Omega(z)}=\sup_{z\in\hat{K}_\Omega} \frac{|f(z)|}{\mu_\Omega(z)}$;
    \item esiste $\mu$ funzionale di Minkovski, per ogni $f \in \mathcal{O}(\Omega)$ e per ogni $K \subset\subset \Omega$ compatto, $|f| \le \mu_{\Omega}$ su $K$ $\implies$ $|f| \le \mu_\Omega$ su $\hat{K}_\Omega$;
    \item esiste $\mu$ funzionale di Minkovski e per ogni $K \subset\subset \Omega$ compatto, $\mu_{\Omega}(\hat{K}_\Omega)=\mu_\Omega(K)$;
    \item esiste $\mu$ funzionale di Minkovski, per ogni $f \in \mathcal{O}(\Omega)$ e per ogni $K \subset\subset \Omega$ compatto, $\displaystyle \sup_{z \in K} \frac{|f(z)|}{\mu_\Omega(z)}=\sup_{z\in\hat{K}_\Omega} \frac{|f(z)|}{\mu_\Omega(z)}$.
  \end{nlist}
\end{thm}

\begin{proof}
  (i) $\implies$ (ii) è ovvia (basta prendere $h$ per tutti i $p \in \partial\Omega$). (iv) $\implies$ (vii) è ovvia. (vii) $\implies$ (ix) è ovvia (basta dividere $f$ per $\displaystyle \sup_{z \in K} \frac{|f(z)|}{\mu_\Omega(z)}$). (ix) $\implies$ (vii) è ovvia (basta prendere $f \equiv 1$). (iv) $\implies$ (vi) $\implies$ (v) sono analoghe a (vii) $\implies$ (xi) $\implies$ (viii).
  (v) $\implies$ (viii) è ovvia. Per concludere mostriamo che (ii) $\implies$ (viii) $\implies$ (iii) $\implies$ (i) $\implies$ (iv).

  (ii) $\implies$ (viii) Poniamo $\mu=\|\cdot\|_{\infty}$. Supponiamo, per assurdo, che (viii) non valga: allora esiste $K \subset\subset \Omega$ compatto t.c. $\mu_{\Omega}(\hat{K}_\Omega)<\mu_\Omega(K)$. Scegliamo $\mu_\Omega(\hat{K}_\Omega)< \delta_1<\delta_2<\mu_\Omega(K)$.
  Sia $z^0 \in \hat{K}_\Omega$ t.c. $\mu_\Omega(z^0)<\delta_1$.
  Poniamo $\displaystyle K_{\delta_2}=\bigcup_{z \in K} \overline{P(z, \delta_2)}=\{w \in \mathbb{C}^n \mid \min_{z \in K} \|z-w\|_\infty \le \delta_2\}$ chiuso e $\subset\subset \Omega$.
  Dalle disuguaglianze di Cauchy otteniamo che per ogni $f \in \mathcal{O}(\Omega)$ e per ogni $\alpha \in \mathbb{N}^n$ vale $\left|\dfrac{\partial^{|\alpha|}f}{\partial z^\alpha}(z)\right| \le \dfrac{\alpha!}{\delta_2^{|\alpha|}}\|f\|_{K_{\delta_2}}$ $(*)$ per ogni $z \in K$.
  Ma allora lo sviluppo in serie di $f$ in $z^0$ converge in $P(z^0, (\delta_1+\delta_2)/2)$ perché $\delta_1<(\delta_1+\delta_2)/2<\delta_2$ $\implies$ $f$ si estende olomorficamente a tutto $P(z^0, (\delta_1+\delta_2)/2)$, ma $P(z^0, (\delta_1+\delta_2)/2) \cap \partial\Omega\not=\emptyset$, assurdo.

  (viii) $\implies$ (iii) $K \subset\subset \Omega$ compatto $\iff$ $K$ è chiuso, limitato e $\mu_\Omega(K)>0$. Se $K \subset\subset \Omega$ compatto, allora $\hat{K}_\Omega$ è chiuso, limitato e, per (viii), $\mu_\Omega(\hat{K}_\Omega)=\mu_\Omega(K)>0$ $\implies$ $\hat{K}_\Omega \subset\subset \Omega$ compatto.

  (iii) $\implies$ (i) Sia $\{w_k\}_{j \in \mathbb{N}^*} \subset \Omega$ una successione ovunque densa e che ripete ogni punto infinite volte. Per ogni $j$ sia $P_j=P(w_j,r_j)$ il polidisco di centro $w_j$ più grande contenuto in $\Omega$ ($\iff$ $P(w_j,r_j) \subset \Omega$ ma $\overline{P(w_j,r_j)}\cap\partial\Omega\not=\emptyset$). In particolare, $P_j$ non è $\subset\subset\Omega$.
  Sia $\{K_j\}$ una successione di compatti che invade $\Omega$, cioè $K_j \subset\subset \Omega$ compatto, $K_j \subset \mathop {K_{j+1}}\limits^ \circ$ e $\displaystyle \bigcup_j K_j=\Omega$.
  (iii) $\implies$ $\widehat{(K_j)}_\Omega \subset\subset \Omega$ $\implies$ esiste $z_j \in P_j\setminus\widehat{(K_j)}_\Omega$ $\implies$ esiste $h_j \in \mathcal{O}(\Omega)$ t.c. $|h_j(z_j)|>\|h_j\|_{K_j}$. Possiamo suppore $h_j(z_j)=1$ e (a meno di sostituire $h_j^{M_j}$ con $M_j>>1$) possiamo supporre $\|h_j\|_{K_j}<2^{-j}$.
  Poniamo $\displaystyle h(z)=\prod_{j=1}^{+\infty} (1-h_j(z))^j$. $h \in \mathcal{O}(\Omega)$ perché $\displaystyle \sum_j \frac{j}{2^j}<+\infty$. Inoltre $h \not\equiv 0$ perché non lo è su $K_1$. Ogni $P_j$ contiene infiniti $z_l$ che si accumulano a $z_j^0 \in \overline{P_j}$.
  $h$ si annulla di ordine almeno $l$ in $z_l$ (segue dalla definizione). Se $z_j^0 \in \Omega$ allora $h$ dovrebbe annullarsi di ordine $\infty$ in $z_j^0$ $\implies$ $h \equiv 0$, assurdo $\implies$ $z_j^0 \in \partial\Omega$.
  Gli $\{z_j^0\}$ sono densi in $\partial\Omega$; se non lo fossero, esisterebbe $w_{j_0}$ t.c. $\overline{P(w_{j_0},r_{j_0})} \cap \partial\Omega$ non contiene alcun $z_j^0$, assurdo. Se $h$ si estendesse a $\Omega' \supsetneq \Omega$ allora si dovrebbe estendere a un intorno di qualche $z_j^0$ $\implies$ $h \equiv 0$, assurdo.

  (i) $\implies$ (iv) Fissiamo $\underline{r}=(r_1,\dots,r_n) \in (\mathbb{R}^+)^n$ e poniamo $\mu^{\underline{r}}(z)=\max\left\{\dfrac{|z_j|}{r_j}\right\}$.
  Vogliamo dimostrare che vale (iv) per $\mu^{\underline{r}}$. Siano $f \in \mathcal{O}(\Omega)$ e $K\subset\subset\Omega$ compatto t.c. $|f(z)| \le \mu_{\Omega}^{\underline{r}}(z)$ per ogni $z \in K$.
  \begin{ftt} \label{2.5.13}
    Data $f$ come sopra, per ogni $g \in \mathcal{O}(\Omega)$ e $w \in \hat{K}_\Omega$ $g$ ha un'espansione in serie di potenze centrata in $w$ e convergente in $P(w, |f(w)|\underline{r})=\{z \in \mathbb{C}^n \mid \mu^{\underline{r}}(z-w)<|f(w)|\}$.
  \end{ftt}
  \begin{proof}
    Fissiamo $0<t<1$, $\displaystyle W_t=\bigcup_{z \in K} P(z, |f(z)|t\underline{r})$. Siccome $t|f(z)|<u_\Omega^{\underline{r}}(z)$ per ogni $z \in K$ $\implies$ $W_t \subset\subset \Omega$. Sia $g \in \mathcal{O}(\Omega)$. Esiste $M>0$ t.c. $\|g\|_{W_t} \le M$.
    Per le disuguaglianze di Cauchy, per ogni $z \in K$ si ha $\left|\dfrac{\partial^{|\alpha|}g}{\partial z^\alpha}(z)\right| \le \dfrac{\alpha!M}{t^{|\alpha|}|f(z)|^{|\alpha|}\underline{r}^\alpha} \iff \left|f(z)^{|\alpha|}\dfrac{\partial^{|\alpha|}g}{\partial z^\alpha}(z)\right| \le \dfrac{\alpha!M}{t^{|\alpha|}\underline{r}^\alpha} \implies$
    per ogni $w \in \hat{K}_\Omega$ si ha $\left|f(w)^{|\alpha|}\dfrac{\partial^{|\alpha|}g}{\partial z^\alpha}(w)\right| \le \dfrac{\alpha!M}{t^{|\alpha|}\underline{r}^\alpha} \implies \left|\dfrac{\partial^{|\alpha|}g}{\partial z^\alpha}(w)\right| \le \dfrac{\alpha!M}{t^{|\alpha|}|f(w)|^{|\alpha|}\underline{r}^\alpha}$ $\implies$
    lo sviluppo in serie di $g$ in $w$ converge in $P(w,|f(w)|t\underline{r})$. Mandando $t$ a $1$ si ottiene quanto voluto.
  \end{proof}
  Usiamo il fatto \ref{2.5.13} per dimostrare che vale (iv) per $\mu^{\underline{r}}$.
  Per assurdo, se esiste $w \in \hat{K}_\Omega$ t.c. $|f(w)|>\mu_\Omega^{\underline{r}}(w)$ allora $P(w,|f(w)|\underline{r}) \cap \partial \Omega \not=\emptyset$, dunque per il fatto \ref{2.5.13} ogni $g \in \mathcal{O}(\Omega)$ si estende olomorficamente in $P(w,|f(w)|\underline{r})$ contro (i), assurdo.
  Sia adesso $\mu$ un funzionale di Minkovski qualsiasi. Dato $v \in \mathbb{C}^n$ poniamo $S^v_\Omega(z)=\sup\{r>0 \mid z+\zeta v \in \Omega \forall |\zeta|<r\}$. Allora $\displaystyle \mu_\Omega(z)=\inf_{\mu(v)=1} S_\Omega^v(z)$ (la dimostrazione è lasciata come esercizio per il lettore). Quindi basta mostrare che (iv) vale per $S^v_\Omega$. Chiaramente, possiamo assumere $v=e_1$.
  Dato $k \in \mathbb{N}^*$ poniamo $\underline{r}^k=(1,1/k,\dots,1/k)$. $\underline{r}^\infty=(1,0,\dots,0)=e_1$. Ora $S^{e_1}_\Omega=\mu_\Omega^{\underline{r}^\infty}$ e $\mu_\Omega^{\underline{r}^k} \uparrow S_\Omega^{e_1}$ (la dimostrazione è lasciata come esercizio per il lettore).
  Sia $K \subset\subset \Omega$ compatto. Per il lemma del Dini, $\mu_\Omega^{\underline{r}^k} \uparrow S_\Omega^{e_1}$ uniformemente su $K$. Sia $f \in \mathcal{O}(\Omega)$ t.c. $|f| \le S_\Omega^{e_1}$ su $K$.
  Fissiamo $\epsilon>0$. Per ogni $k \ge k_0(\epsilon)$ $S_\Omega^{e_1} \le (1+\epsilon)\mu_\Omega^{\underline{r}^k}$ su $K$ $\implies$ $|f| \le (1+\epsilon)\mu_\Omega^{\underline{r}^k}$ su $K$, ma (iv) vale per i funzionali di quella forma, dunque $|f| \le (1+\epsilon)\mu_\Omega^{\underline{r}^k} \le (1+\epsilon)S_\Omega^{e_1}$ su $\hat{K}_\Omega$. Mandando $\epsilon$ a $0$ otteniamo la tesi.
\end{proof}

\begin{cor} \label{conv->dom_olo}
  $\Omega$ convesso $\implies$ $\Omega$ dominio di olomorfia.
\end{cor}

\begin{proof}
  Sia $P \in \partial \Omega$. Essendo $\Omega$ convesso, esiste $L: \mathbb{C}^n \longrightarrow \mathbb{C}$ $\mathbb{C}$-lineare t.c. $\mathfrak{Re}L(z) < \mathfrak{Re}L(P)$ per ogni $z \in \Omega$. Sia $f(z)=\dfrac{1}{L(z)-L(P)}$. Allora $f \in \mathcal{O}(\Omega)$ e non si estende oltre $P$ $\implies$ $P$ è essenziale.
\end{proof}

\begin{defn}
  $P \in \partial\Omega$ è un \textit{punto di picco} se esiste $f \in \mathcal{O}(\Omega) \cap C^0(\overline{\Omega})$ t.c. $f(P)=1$ ma $\|f\|_\Omega<1$.
\end{defn}

\begin{ex}
  Nel caso del corollario \ref{conv->dom_olo}, $f(z)=e^{L(z)-L(P)}$.
\end{ex}

\begin{exc}
  Se ogni punto di $\partial\Omega$ è di picco, allora $\Omega$ è dominio di olomorfia.
\end{exc}

\begin{prop} \label{cclc->do}
  Sia $\Omega \subseteq \mathbb{C}^n$ un dominio circolare completo logaritmicamente convesso con $0 \in \Omega$. Allora $\Omega$ è dominio di olomorfia.
\end{prop}

\begin{proof}
  Sia $K \subset\subset \Omega$ compatto. Vogliamo mostrare che $\hat{K}_\Omega \subset\subset \Omega$ e allora la tesi seguirà dal teorema \ref{car_dom_olo}. Per ogni $w \in K$ esiste un intorno $w \in U_w \subset \Omega$ e $\zeta^w \in \Omega$ t.c. $|z_j| \le |\zeta_j^w|$ per ogni $z \in U_w$.
  $K$ compatto $\implies$ esistono $\zeta^1, \dots, \zeta^k \in \Omega$ t.c. $\displaystyle K \subset \bigcup_{l=1}^k \{|z_j| \le |\zeta_j^l| \forall j\}=W \subset\subset \Omega$. Possiamo asumere $\zeta_j^l \not=0$ per ogni $l,j$. Sia $z \in \hat{K}_\Omega$, vogliamo $z \in W$.
  A meno di permutare le coordinate possiamo supporre $z_1= \dots, z_m \not=0, z_{m+1}=\dots=z_n=0$ per qualche $1 \le m \le n$ (ovviamente $0 \in W$).
  Per definizione di $\hat{K}_\Omega$ si ha che $\displaystyle |z_1^{\alpha_1}\dots z_m^{\alpha_m}| \le \sup_{w \in K} |w_1^{\alpha_1}\dots w_m^{\alpha_m}| \le \max_{1 \le l \le k} |(\zeta_1^l)^{\alpha_1}\dots(\zeta_m^l)^{\alpha_m}|$ $(*)$.
  Poniamo $\nu_j=\alpha_j/|\alpha|$. $\nu_j \in \mathbb{Q}^+\cup\{0\}$ con $\displaystyle \sum_{j=1}^n \nu_j=1$. Prendendo il logaritmo e dividendo per $|\alpha|$ a entrambi i membri di $(*)$ otteniamo $\displaystyle \sum_{j=1}^m \nu_j\log{|z_j|} \le \max_l \sum_{j=1}^m \nu_j\log{|\zeta_j^l|}$ $(**)$.
  Per continuità questo vale per ogni $\nu_j \in \mathbb{R}^+\cup\{0\}$ t.c. $\displaystyle \sum_j \nu_j=1$.
  $(**)$ ci dice che $(\log{|z_1|},\dots,\log{|z_m|})$ è nell'inviluppo convesso $Z$ di $\displaystyle \bigcup_l \{(t_1,\dots,t_m) \in \mathbb{R}^n \mid -\infty<t_j \le \log{|\zeta_j^l|}\forall j\}$ che è ben contenuto in $\log{|\Omega|}$ $\implies$ $\displaystyle z \in \widehat{\bigcup_{l=1}^k \{|z_j| \le |\zeta_j^l|\}}=\hat{W}$ che è il dominio circolare completo $\subset\subset \Omega$ che ha come immagine logaritimica $Z$.
\end{proof}

\begin{cor}
  Se $\Omega \subseteq \mathbb{C}^n$ è un dominio circolare completo logaritmicamente convesso con $0 \in \Omega$, allora $\Omega$ è il dominio di convergenza di una serie di potenze.
\end{cor}

\begin{proof}
  Per la proposizione \ref{cclc->do} abbiamo che $\Omega$ è un dominio di olomorfia, dunque esiste $h \in \mathcal{O}(\Omega)$ che non si può estendere ad aperti più grandi, Prendiamo l'espansione in serie di $h$ in $0$. Abbiamo visto che tale espansione converge in $\Omega$ e non può convergere in alcunché di più grande.
\end{proof}
