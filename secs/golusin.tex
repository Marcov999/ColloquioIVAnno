Adesso possiamo procedere a dimostrare una serie di disuguaglianze, che coinvolgono la distanza iperbolica $d$ e le funzioni olomorfe dal disco in sé che non sono automorfismi. L'ultima di esse ha come corollario la disuguaglianza di Golusin.

\begin{prop} \label{24}
  Siano $f:\mathbb{D} \longrightarrow \mathbb{D}$ una funzione olomorfa che non è un automorfismo, $v \in \mathbb{D}$. Allora per ogni $z \in \mathbb{D}$ si ha che $f^*(z,v) \in \mathbb{D}$ e la funzione $z \longmapsto f^*(z,v)$ è olomorfa.
\end{prop}

\begin{proof}
  Per quanto riguarda l'olomorfia, dalla definizione sappiamo che l'unico punto che potrebbe dar problemi è $v$, ma abbiamo visto che la funzione ammette limite finito per $z \longrightarrow v$, perciò $v$ è una singolarità rimovibile. Per il lemma di Schwarz-Pick, $|f^*(z,w)| \le 1$, inoltre vale l'uguale in qualche punto solo se $f$ è un automorfismo, dunque con le ipotesi su $f$ abbiamo che vale la disuguaglianza stretta sempre, cioè $f^*(z,v) \in \mathbb{D}$ per ogni $z \in \mathbb{D}$.
\end{proof}

\begin{thm} \label{31}
  Sia $f:\mathbb{D} \longrightarrow \mathbb{D}$ una funzione olomorfa che non è un automorfismo. Allora per ogni $z, w, v \in \mathbb{D}$ vale
  \begin{equation} \label{3.1}
    d(f^*(z,v),f^*(w,v)) \le d(z,w).
  \end{equation}
\end{thm}

\begin{proof}
  Poiché $f$ non è un automorfismo, per la proposizione \ref{24} la mappa $z \longmapsto f^*(z,v)$ è olomorfa dal disco unitario in sé, perciò il membro sinistro della disuguaglianza \eqref{3.1} è ben definito. Per quanto riguarda la disuguaglianza,
  \begin{align*}
    p(f^*(z,v), f^*(w,v)) \le p(z,w) \\
    2\,\text{arctanh}\,(p(f^*(z,v), f^*(w,v))) \le 2\,\text{arctanh}\,(p(z,w)) \\
    d(f^*(z,v), f^*(w,v)) \le d(z,w),
  \end{align*}
  dove la prima riga segue dal lemma di Schwarz-Pick applicato alla funzione di cui sopra, il passaggio dalla prima alla seconda è perché $\text{arctanh}$ è crescente e dalla seconda all'ultima è la definizione di $d$.
\end{proof}

\begin{cor} \label{32}
  Sia $f:\mathbb{D} \longrightarrow \mathbb{D}$ una funzione olomorfa che non è un automorfismo. Allora per ogni $z, w, v \in \mathbb{D}$ vale
  \begin{equation}
    d(0, f^*(z,v)) \le d(0,f^*(w,v))+d(z,w).
  \end{equation}
\end{cor}

\begin{proof}
  \begin{align*}
    d(0,f^*(z,v)) & \le d(0,f^*(w,v))+d(f^*(w,v),f^*(z,v)) \\
    & \le d(0,f^*(w,v))+d(z,w),
  \end{align*}
  dove la prima è la disuguaglianza triangolare per la distanza $d$ e la seconda segue dal teorema \ref{31}.
\end{proof}

\begin{cor} \label{33}
  Sia $f:\mathbb{D} \longrightarrow \mathbb{D}$ una funzione olomorfa che non è un automorfismo. Allora per ogni $z, w, v, u \in \mathbb{D}$ vale
  \begin{equation}
    d(0, f^*(z,v)) \le d(0, f^*(u,w))+d(z,w)+d(v,u).
  \end{equation}
\end{cor}

\begin{proof}
  \begin{align*}
    d(0,f^*(z,v)) & \le d(0,f^*(w,v))+d(z,w) \\
    & =d(0,|f^*(w,v)|)+d(z,w) \\
    & =d(0,|f^*(v,w)|)+d(z,w) \\
    & =d(0,f^*(v,w))+d(z,w) \\
    & \le d(0,f^*(u,w))+d(z,w)+d(v,u),
  \end{align*}
  dove le due disuguaglianze seguono dal corollario \ref{32}.
\end{proof}

\begin{cor} \label{quasigolusin}
  Sia $f:\mathbb{D} \longrightarrow \mathbb{D}$ una funzione olomorfa che non è un automorfismo. Allora per ogni $z, w \in \mathbb{D}$ vale
  \begin{equation} \label{quasigol}
    d(f^h(z), f^h(w)) \le 2d(z,w).
  \end{equation}
\end{cor}

\begin{proof}
  Siano $z, w \in \mathbb{D}$, senza perdita di generalità possiamo supporre $f^h(z) \ge f^h(w)$. Allora
  \begin{align*}
    d(f^h(z), f^h(w)) & =\log\left(\frac{1+\frac{f^h(z)-f^h(w)}{1-f^h(w)f^h(z)}}{1-\frac{f^h(z)-f^h(w)}{1-f^h(w)f^h(z)}}\right) \\
    & =\log\left(\frac{1-f^h(w)f^h(z)+f^h(z)-f^h(w)}{1-f^h(w)f^h(z)+f^h(w)-f^h(z)}\right) \\
    & =\log\left(\frac{1+f^h(z)}{1-f^h(z)}\cdot\frac{1-f^h(w)}{1+f^h(w)}\right) \\
    & =\log\left(\frac{1+f^h(z)}{1-f^h(z)}\right)-\log\left(\frac{1+f^h(w)}{1-f^h(w)}\right) \\
    & =d(0,f^h(z))-d(0,f^h(w)) \le 2d(z,w).
  \end{align*}
  dove la disuguaglianza finale segue dal corollario \ref{33} ponendo $u=w, v=z$.
\end{proof}

Ponendo $w=0$ in \eqref{quasigol} otteniamo la disuguaglianza di Golusin, che ci servirà per dimostrare il risultato a cui puntiamo.
