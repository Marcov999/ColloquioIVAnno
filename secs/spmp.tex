\begin{frame}
  \frametitle{Prodotti di Blaschke}
  \begin{defn}
    Dati $a_1,\dots,a_n \in \mathbb{D}$ e $\theta \in \mathbb{R}$, chiamiamo \textit{prodotto di Blaschke} di grado $n$ la funzione
    $$e^{i\theta}\prod_{j=1}^n \frac{z-a_j}{1-\bar{a}_jz}.$$
    Indichiamo con $\mathcal{B}_n$ i prodotti di Blaschke di grado $n$.
  \end{defn}
  \pause
  Notiamo che $\mathcal{B}_1=\text{Aut}(\mathbb{D})$.
  \pause
  \begin{prop}
    Valgono le seguenti:
    \begin{enumerate}[(i)]
      \item si ha che $f \in \mathcal{B}_{n+1}$ se e solo se $f^*(\cdot,w) \in \mathcal{B}_n$, con $w \in \mathbb{D}$ fissato;
      \item se $f \in \mathcal{B}_2$ allora $f^*\bigl(R_f(w),w\bigr)$, dove $R_f$ è la rotazione attorno al punto in cui $f$ ha molteplicità doppia.
    \end{enumerate}
  \end{prop}
\end{frame}

\begin{frame}
  \frametitle{Lemma di Schwarz-Pick a tre punti}
  \begin{thm} \label{31}
    (Beardon-Minda, 2004) Sia $f \in \text{\normalfont{Hol}}(\mathbb{D},\mathbb{D})\setminus\text{\normalfont{Aut}}(\mathbb{D})$. Allora per ogni $z, w, v \in \mathbb{D}$ vale
    \begin{equation} \label{3.1}
      \omega\bigl(f^*(z,v),f^*(w,v)\bigr) \le \omega(z,w).
    \end{equation}
    Si ha l'uguaglianza se e solo se $f \in \mathcal{B}_2$.
  \end{thm}
  \pause
  \textit{Traccia della dimostrazione:} basta applicare il lemma di Schwarz-Pick alla funzione $f^*(\cdot,v)$. \qed
  \pause
  \begin{oss}
    Se $f(0)=0$ troviamo $\omega\bigl(f(z)/z,f'(0)\bigr) \le \omega(z,0)$. Il disco di centro $f'(0)$ e raggio $\omega(z)$ è, in generale, strettamente contenuto in $\mathbb{D}$.
  \end{oss}
\end{frame}

\begin{frame}
  \frametitle{Lemma di Schwarz-Pick a tre punti}
  \begin{cor} \label{33}
    Sia $f \in \text{\normalfont{Hol}}(\mathbb{D},\mathbb{D})\setminus\text{\normalfont{Aut}}(\mathbb{D})$. Allora per ogni $z, w, v, u \in \mathbb{D}$ vale
    \begin{equation} \label{eq33}
      \omega\bigl(0, f^*(z,v)\bigr) \le \omega\bigl(0, f^*(u,w)\bigr)+\omega(z,w)+\omega(v,u).
    \end{equation}
    Si ha l'uguaglianza se e solo se $f \in \mathcal{B}_2$ e $R_f(v), R_f(u), w$ e $z$ giacciono sulla stessa geodetica, in quest'ordine.
  \end{cor}
  \pause
  \textit{Traccia della dimostrazione:}
  \begin{align*}
    \action<+->{\omega\bigl(0,f^*(z,v)\bigr) & \le \omega\bigl(0,f^*(w,v)\bigr)+\omega\bigl(f^*(w,v),f^*(z,v)\bigr) \\}
    \action<+->{& \le \omega\bigl(0,f^*(w,v)\bigr)+\omega(w,z) \\}
    \action<+->{& = \omega\bigl(0,f^*(v,w)\bigr)+\omega(w,z) \\}
    \action<+->{& \le \omega\bigl(0,f^*(u,w)\bigr)+\omega(u,v)+\omega(w,z). & \qed}
  \end{align*}
\end{frame}
