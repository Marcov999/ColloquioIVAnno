\begin{frame}
  \frametitle{Il lemma di Schwarz-Pick}
  \begin{block}{Lemma di Schwarz-Pick} \label{SP}
    \begin{itshape}
      Sia $f \in \text{\normalfont{Hol}}(\mathbb{D},\mathbb{D})$. Allora per ogni $z, w \in \mathbb{D}$ si ha
      $$\left|\frac{f(z)-f(w)}{1-\overline{f(w)}f(z)}\right| \le \left|\frac{z-w}{1-\bar{w}z}\right| \text{ e } \frac{|f'(z)|}{1-|f(z)|^2} \le \frac{1}{1-|z|^2}.$$
      Inoltre, se vale l'uguaglianza nella prima per $z_0, w_0$ con $z_0 \not=w_0$ o nella seconda per $z_0$ allora $f \in \text{\normalfont{Aut}}(\mathbb{D})$ e vale sempre l'uguaglianza.
    \end{itshape}
  \end{block}
  \pause
  \begin{oss}
    Se $f \in \text{Aut}(\mathbb{D})$, allora $f(z)=e^{i\theta}\dfrac{z-a}{1-\bar{a}z}$ con $\theta \in \mathbb{R}$ e $a \in \mathbb{D}$.
  \end{oss}
  \pause
  Dal lemma, si ha che la quantità $\left|\dfrac{z-w}{1-\bar{w}z}\right|$ è contratta dalle funzioni in $\text{Hol}(\mathbb{D},\mathbb{D})$. A partire da essa è possibile definire una distanza sul disco.
\end{frame}

\begin{frame}
  \frametitle{La distanza di Poincaré}
  Scriviamo $[z,w]:=\dfrac{z-w}{1-\bar{w}z}$ e $p(z,w):=|[z,w]|$. \pause
  \begin{defn}
    La \textit{distanza di Poincaré} (o \textit{iperbolica}) sul disco è la funzione $\omega:\mathbb{D}\times \mathbb{D} \longrightarrow [0,+\infty)$ data da
    $$\omega(z,w):=\text{arctanh}\bigl(p(z,w)\bigr)=\frac{1}{2}\log\left(\frac{1+p(z,w)}{1-p(z,w)}\right).$$
  \end{defn}
  \pause
  Per stretta crescenza della tangente iperbolica, in termini di $\omega$ il lemma di Schwarz-Pick si riscrive come
  $$\omega\bigl(f(z),f(w)\bigr) \le \omega(z,w).$$
  Vale l'uguaglianza in qualche caso se e solo se $f \in \text{Aut}(\mathbb{D})$; in tal caso c'è sempre l'uguaglianza.
\end{frame}
