\begin{defn}
  La \textsc{sfera di Riemann} è l'insieme $\hat{\mathbb{C}}=\overline{\mathbb{C}}=\mathbb{C}_{\infty}=\mathbb{C} \cup \{\infty\}=\mathbb{P}^1(\mathbb{C})$ (l'ultimo è la retta proiettiva complessa).
  Per noi sarà $\hat{\mathbb{C}}=\mathbb{C} \cup \{\infty\}$ con la seguente topologia: ristretta a $\mathbb{C}$ è la topologia usuale, mentre gli intorni aperti di $\infty$ sono della forma $(\mathbb{C} \setminus K) \cup \{\infty\}$ con $K \subset \subset \mathbb{C}$ compatto.
\end{defn}

Siano $U_0=\mathbb{C}, U_1=\mathbb{C}^*\cup\{\infty\}$ (si noti che $U_1$ è un intorno aperto di $\infty$). Sia $\varphi_1:U_1 \rightarrow \mathbb{C}$ definita come
$$\varphi_1(w)=\begin{cases} 1/w & \mbox{se }w\not=\infty \\ 0 & \mbox{se }w=\infty. \end{cases}$$
$\varphi_1$ è un omeomorfismo fra $U_1$ e $\mathbb{C}$.
Sia $\varphi_0:U_0 \rightarrow \mathbb{C}$, $\varphi_0(z)=z$ (l'identità); è un omeomorfismo fra $U_0$ e $\mathbb{C}$. \\
$U_0 \cap U_1=\mathbb{C}^*$, $\varphi_1(U_0 \cap U_1)=\mathbb{C}^*=\varphi_0(U_0 \cap U_1)$. \\
$\varphi_0 \circ \varphi_1^{-1}, \varphi_1 \circ \varphi_0^{-1}:\mathbb{C}^* \rightarrow \mathbb{C}^*$,
$(\varphi_0 \circ \varphi_1^{-1})(w)=\dfrac{1}{w}, (\varphi_1 \circ \varphi_0^{-1})(z)=\dfrac{1}{z}$ sono olomorfe. \\
$\varphi_0$ e $\varphi_1$ si chiamano \textit{carte}. Una funzione definita a valori in $\hat{\mathbb{C}}$ è olomorfa se lo è letta tramite carte. Vediamo nello specifico cosa significa.

Sia $\Omega \subseteq \hat{\mathbb{C}}$ aperto, $f:\Omega \rightarrow \mathbb{C}$ continua; quando è olomorfa? \\
Risposta:
\begin{nlist}
  \item $f \restrict{\Omega \cap \mathbb{C}}$ è olomorfa in senso classico (notiamo che $\Omega \cap \mathbb{C}=\Omega \setminus \{\infty\}$);
  \item $f \circ \varphi_1^{-1}:\varphi_1(\Omega) \rightarrow \mathbb{C}$ è olomorfa vicino a $0=\varphi_1(\infty)$. \\
  $(f \circ \varphi_1^{-1})(w)=f\left(\dfrac{1}{w}\right)$.
\end{nlist}

\begin{ex}
  $\displaystyle f(z)=\sum_{n=-\infty}^{+\infty} c_nz^n$. Quando è olomorfa in $\infty$? Se e solo se $f\left(\dfrac{1}{w}\right)$ è olomorfa in $0$.
  $\displaystyle f\left(\frac{1}{w}\right)=\sum_{n=-\infty}^{+\infty} c_nw^{-n}$ è olomorfa in $0$ $\iff$ $c_n=0$ per ogni $n>0$.
\end{ex}

\begin{oss}
  $f: \hat{\mathbb{C}} \rightarrow \mathbb{C}$ è olomorfa se e solo se è costante. Infatti, $\hat{\mathbb{C}}$ compatto $\implies$ $f(\hat{\mathbb{C}})$ compatto, cioè chiuso e limitato in $\mathbb{C}$ $\implies$ $|f|$ ha max in $x_0 \in \hat{\mathbb{C}}$.
  Se $z_0 \in \mathbb{C}$, allora per il teorema di Liouville \ref{liou} $f \restrict{\mathbb{C}}$ è costante $\implies$ $f$ costante. Se $z_0=\infty$, $f(1/w)$ ha massimo in $0$, dunque ragionando come prima è costante.
\end{oss}

Sia $\Omega \subseteq \mathbb{C}$ aperto, $f:\Omega \rightarrow \hat{\mathbb{C}}$ continua; quando è olomorfa? \\
Risposta:
\begin{nlist}
  \item $f$ è olomorfa in $\Omega \setminus f^{-1}(\infty)$ in senso classico;
  \item se $f(z_0)=\infty$, $\varphi_1 \circ f=\dfrac{1}{f}$ è olomorfa vicino a $z_0$.
\end{nlist}

\begin{ex}
  $\displaystyle f(z)=\sum_{n=-\infty}^{+\infty} c_nz^n$ è a valori in $\hat{\mathbb{C}}$ se $0$ è un polo (ci interessa il caso in cui $\infty$ sia effettivamente nell'immagine, altrimenti è una comune funzione olomorfa a valori in $\mathbb{C}$), cioè consideriamo $f(0)=\infty$.
  Supponiamo allora $\displaystyle f(z)=\sum_{n=k}^{+\infty} c_nz^n=z^{-k}\sum_{n=k}^{+\infty} c_nz^{n+k}=z^{-k}h(z)$, $h(0)=c_{-k}\not=0$, $h$ olomorfa. $\dfrac{1}{f}(z)=\dfrac{z^k}{h(z)}$ è olomorfa in $0$.
  Viceversa, se $f$ è olomorfa, $\dfrac{1}{f}$ è olomorfa in $0$ $\implies$ $\displaystyle \frac{1}{f}(z)=z^k\sum_{n=0}^{+\infty}c_nz^n$, $c_0 \not=0, k \ge 1$ (la condizion $k \ge 1$ segue dal fatto che siamo nell'ipotesi $f(0)=\infty \implies (1/f)(0)=0$).
  Allora $\dfrac{1}{f}(z)=z^kh(z)$ $\implies$ $f(z)=z^{-k}\frac{1}{h(z)}$ e quindi ha un polo in $0$.
\end{ex}

\begin{cor}
  $\Omega \subseteq \mathbb{C}$, $f:\Omega \rightarrow \hat{\mathbb{C}}$ è olomorfa se e solo se è meromorfa.
\end{cor}

Possiamo ora dare una definizione generale.

\begin{defn}
  $f:\hat{\mathbb{C}} \rightarrow \hat{\mathbb{C}}$ continua è \textit{olomorfa} se e solo se $f\left(\dfrac{1}{w}\right)$ è olomorfa vicino a $0$ e $\dfrac{1}{f}$ è olomorfa vicino a $f^{-1}(\infty)$ (e ovviamente dev'essere normalmente olomorfa in tutti gli altri punti). \\
  Se $f(\infty)=\infty$, la condizione è che $\dfrac{1}{f(1/w)}$ sia olomorfa in $0$.
\end{defn}

\begin{ex}
  $p(z)=a_0+a_1z+\dots+a_dz^d, a_d \not=0$ (un polinomio). $p(\infty)=\infty$. È olomorfo in $\infty$? Sì: $\displaystyle \frac{1}{p(1/z)}=\frac{1}{a_0+a_1z^{-1}+\dots+a_dz^{-d}}=\frac{1}{z^{-d}(a_0z^d+\dots+a_d)}=\frac{z^d}{a_d+\dots+a_0z^d}$ è olomorfo in $0$.
  $\dfrac{1}{p(1/z)}$ ha uno zero di ordine $d$ in $0$ $\iff$ $p$ ha un polo di ordine di $-d$ in $\infty$ (vedremo più avanti come è definito $ord_f(\infty)$).
\end{ex}

\begin{prop}
  $f \in \text{Hol}(\hat{\mathbb{C}}, \hat{\mathbb{C}})$ $\iff$ $f=\dfrac{P}{Q}$ con $P, Q \in \mathbb{C}[z]$ senza fattori comuni, cioè $f$ è una funzione razionale.
\end{prop}

\begin{proof}
  ($\Leftarrow$) Sappiamo che $\mathbb{C}[z] \subset \text{Hol}(\hat{\mathbb{C}}, \hat{\mathbb{C}})$ e quozienti di funzioni olomorfe sono olomorfi.

  ($\implies$) Sia $f: \hat{\mathbb{C}} \rightarrow \hat{\mathbb{C}}$ olomorfa non costante. $Z_f=f^{-1}(0)$ è chiuso e discreto in $\hat{\mathbb{C}}$ che è compatto, dunque è finito, perciò $Z_f \cap \mathbb{C}=\{z_1, \dots, z_k\} \subset \mathbb{C}$.
  Analogamente $P_f=f^{-1}(\infty)=Z_{1/f}$, $P_f \cap \mathbb{C}=\{w_1, \dots, w_h\} \subset \mathbb{C}$. Sia $g(z)=\dfrac{(z-w_1)\cdots(z-w_h)}{(z-z_1) \cdots (z-z_k)}f(z)$, $g \in \text{Hol}(\hat{\mathbb{C}}, \hat{\mathbb{C}})$ (gli zeri e i poli compaiono con molteplicità nei prodotti al numeratore e al denominatore).
  In questo modo $g$ non ha né zeri né poli in $\mathbb{C}$. Se $g(\infty) \in \mathbb{C}$, $g \in \text{Hol}(\hat{\mathbb{C}}, \mathbb{C})$ $\implies$ $g$ costante, diciamo $g \equiv c$ $\implies$ $f(z)=c\dfrac{(z-z_1) \cdots (z-z_k)}{(z-w_1)\cdots(z-w_h)}$, come voluto.
  Se $g(\infty)=\infty$ $\implies$ $\dfrac{1}{g}(\infty)=0 \in \mathbb{C}$ $\implies$ $\dfrac{1}{g} \in \text{Hol}(\hat{\mathbb{C}}, \mathbb{C})$ $\implies$ $\dfrac{1}{g}$ costante e si conclude come sopra.
\end{proof}

\begin{defn}
  Sia $f=\dfrac{P}{Q} \in \text{Hol}(\hat{\mathbb{C}}, \hat{\mathbb{C}})$. Il \textsc{grado di $f$} è $\deg{f}=\max{\{\deg{P}, \deg{Q}\}}$. \\
  La definizione dell'ordine di zeri e poli in $\mathbb{C}$ ce l'abbiamo. \\
  $\displaystyle f(\infty)=\lim_{w \rightarrow 0} \frac{P(1/w)}{Q(1/w)}=\lim_{w \rightarrow 0} \frac{a_m\left(\frac{1}{w}\right)^m+\dots+a_0}{b_n\left(\frac{1}{w}\right)^n+\dots+b_0}=$ \\
  $$\lim_{w \rightarrow 0} w^{n-m} \frac{a_m+\dots+a_0w^m}{b_n+\dots+b_0w^n}=\begin{cases} 0 & \mbox{se }n>m \\ \frac{a_m}{b_n} & \mbox{se }n=m \\ \infty & \mbox{se } n<m. \end{cases}$$
  Definiamo allora $ord_f(\infty))=n-m=\deg{Q}-\deg{P}$.
\end{defn}

\begin{defn}
  Siano $f \in \text{Hol}(\hat{\mathbb{C}}, \hat{\mathbb{C}})$, $z_0 \in \hat{\mathbb{C}}$.
  La \textsc{molteplicità di $f$ in $z_0$} è $\delta_f(z_0)$ definita come segue: se $f(z_0)=w_0 \in \mathbb{C}$, $z_0$ è uno zero di $f-w_0$ e poniamo $\delta_f(z_0)=ord_{f-w_0}(z_0)$; se $f(z_0)=\infty$, $z_0$ è un polo di $f$ e poniamo $\delta_f(z_0)=-ord_f(z_0)$. Si ha che $\delta_f(z_0) \in \mathbb{N}$.
\end{defn}

\begin{prop}
  Sia $f \in \text{Hol}(\hat{\mathbb{C}}, \hat{\mathbb{C}})$ non costante. Allora per ogni $q \in \hat{\mathbb{C}}$ $\displaystyle \sum_{f(p)=q} \delta_f(p)=\deg{f}$.
\end{prop}

\begin{proof}
  Sia $f=\dfrac{P}{Q}$. Se $q=0$, $\displaystyle \sum_{f(p)=0} \delta_f(p)=\sum_{\substack{f(p)=0 \\ p \in \mathbb{C}}} \delta_f(p)+c \cdot \delta_f(\infty)$ dove $c=1$ se $f(\infty)=0$ e $c=0$ altrimenti.
  Si noti che per il teorema fondamentale dell'algebra $\displaystyle \sum_{\substack{f(p)=0 \\ p \in \mathbb{C}}} \delta_f(p)=\deg{P}$. Per com'è definito $c$, $c \cdot \delta_f(\infty)=\max{\{0, \deg{Q}-\deg{P}\}}$.
  Allora $\displaystyle \displaystyle \sum_{f(p)=0} \delta_f(p)=\deg{P}+\max{\{0, \deg{Q}-\deg{P}\}}=\max{\{\deg{P}, \deg{Q}\}}=\deg{f}$.
  Se $q=\infty$, $\displaystyle \sum_{f(p)=\infty} \delta_f(p)=\sum_{\substack{f(p)=\infty \\ p \in \mathbb{C}}} \delta_f(p)+c \cdot \delta_f(\infty)$ dove stavolta $c=1$ se $f(\infty)=\infty$ e $c=0$ altrimenti.
  Dunque in questo caso la sommatoria vale, per il teorema fondamentale dell'algebra, $\deg{Q}$, mentre $c \cdot \delta_f(\infty)=\max{\{0, \deg{P}-\deg{Q}\}}$,
  per cui $\displaystyle \sum_{f(p)=\infty} \delta_f(p)=\deg{Q}+\max{\{0, \deg{P}-\deg{Q}\}}=\max{\{\deg{Q}, \deg{P}\}}=\deg{f}$.
  Se $q \in \mathbb{C}^*$, $\displaystyle \sum_{f(p)=q} \delta_f(p)=\sum_{f(p)=q} ord_{f-q}(p)=\sum_{f(p)-q} ord_{f-q}(p)=\deg{(f-q)}$. $f(z)-q=\dfrac{P(z)-qQ(z)}{Q(z)}$.
  $$\deg{(P-qQ)}\begin{cases} =\max{\{\deg{P}, \deg{Q}\}} & \mbox{se sono diversi} \\ \le \max{\{\deg{P}, \deg{Q}\}} & \mbox{se sono uguali} \end{cases}$$ $\implies$ $\deg{(f-q)}=\deg{f}$.
\end{proof}

\begin{cor}
  Siano $f \in \text{Hol}(\hat{\mathbb{C}}, \hat{\mathbb{C}})$ non costante, $w_0 \in \hat{\mathbb{C}}$. Allora $1 \le card(f^{-1}(w_0)) \le \deg{f}$.
\end{cor}

\begin{proof}
  $\ge 1$: se $w_0 \not=\infty$, $f(z)=w_0$ $\iff$ $f(z)-w_0=0$ $\iff$ $P(z)-w_0Q(z)=0$ e per il teorema fondamentale dell'algebra esiste $z$ che soddisfa; se $w_0=\infty$, si considera $1/f$. \\
  $\displaystyle card(f^{-1}(w_0)) \le \sum_{f(p)=w_0} \delta_f(p)=\deg{f}$.
\end{proof}

\begin{oss}
  $\delta_f(p)>1$ $\implies$ $f'(p)=0 \lor \left(\dfrac{1}{f}\right)'(p)=0$. Infatti, senza perdita di generalità $p=0$ e $f(p)=0$, allora se $\delta_f(p)=k>1$ si ha che $f(z)=z^kh(z)$ con $h$ olomorfa e $h(0) \not=0$ $\implies$ $f'(z)=[kz^{k-1}h(z)+z^kh'(z)]$. Ricordando che $k>1$, si ha che $f'(0)=0$.
\end{oss}

\begin{cor}
  Sia $f \in \text{Hol}(\hat{\mathbb{C}}, \hat{\mathbb{C}})$, $f \in \text{Aut}(\hat{\mathbb{C}})$ $\iff$ $\deg{f}=1$ $\iff$ $f(z)=\dfrac{az+b}{cz+d}$ con $ad-bc=1$.
\end{cor}

\begin{proof}
  ($\Leftarrow$) Ogni $f \in \text{Hol}(\hat{\mathbb{C}}, \hat{\mathbb{C}})$ è suriettiva per quanto appena dimostrato. Se $\deg{f}=1$, allora $f$ è iniettiva, quindi biettiva, per cui per il teorema \ref{biolo} $f \in \text{Aut}(\hat{\mathbb{C}})$. \\
  ($\implies$) $f$ automorfismo $\implies$ $f$ iniettiva $\implies$ $\displaystyle \sum_{f(p)=w_0} \delta_f(p)$ contiene un unico addento con molteplicità uno (da cui la tesi). Infatti, da $f$ non costante si ha che $f'$ ha un insieme di zeri discreto $C_f$ e $(1/f)'$ ha un insieme di zeri discreto $C_{1/f}$. Allora basta prendere $z_0 \not\in C_f \cup C_{1/f}$ per ottenere, dall'osservazione precedente, che $\delta_f(z_0)=1$. \\
  Il secondo se e solo se è un banale esercizio lasciato al lettore.
\end{proof}

\begin{oss}
  Siccome numeratore e denominatore sono definiti a meno di una costante moltiplicativa, possiamo suppore $ad-bc=1$.
\end{oss}

\begin{exc}
  $\text{Aut}(\hat{\mathbb{C}})$ è isomorfo a $\faktor{SL(2, \mathbb{C})}{\{\pm I_2\}}$.
\end{exc}
