\begin{frame}[t]
  \only<1->{\frametitle{Teorema di Bracci-Kraus-Roth}
  \begin{thm} \label{boundary_schwarz_pick}
    (Bracci-Kraus-Roth, 2020) Sia $f \in \text{\normalfont{Hol}}(\mathbb{D},\mathbb{D})$ tale che
    \begin{equation} \label{n_o^2}
      |f^h(z_n)|=1+o\bigl((|z_n|-1)^2\bigr)
    \end{equation}
    per qualche successione $\{z_n\}_{n \in \mathbb{N}} \subset \mathbb{D}$ con $|z_n| \longrightarrow 1$. Allora $f \in \text{\normalfont{Aut}}(\mathbb{D})$.
  \end{thm}}
  \only<2->{\textit{Traccia della dimostrazione:} per assurdo $f\not\in \text{Aut}(\mathbb{D})$. La disuguaglianza di Golusin si riscrive come
  $$\frac{\bigl(1+|f^h(0)|\bigr)(1+|z_n|)^2}{\bigl(1-|f^h(0)|\bigr)\bigl(1+|f^h(z_n)|\bigr)}\bigl(1-|f^h(z_n)|\bigr) \ge (1-|z_n|)^2.$$}
  \only<3->{Poiché $f \not\in \text{Aut}(\mathbb{D})$, per Schwarz-Pick $|f^h(0)|<1$ e dunque}
  \only<4>{\begin{align*}
  \lim_{n \longrightarrow +\infty} \frac{\bigl(1+|f^h(0)|\bigr)(1+|z_n|)^2}{\bigl(1-|f^h(0)|\bigr)\bigl(1+|f^h(z_n)|\bigr)}=\frac{2\bigl(1+|f^h(0)|\bigr)}{1-|f^h(0)|} < +\infty. & \qed
\end{align*}}
\end{frame}

\begin{frame}
  \frametitle{Risultato sui limiti non tangenziali}
  Per poter dimostrare il teorema di Burns-Krantz passando dal teorema di Bracci-Kraus-Roth, dobbiamo vedere che le ipotesi di quest'ultimo siano verificate sotto le ipotesi del primo; la seguente proposizione ci garantisce che è vero.
  \begin{prop}
    Siano $f \in \text{\normalfont{Hol}}(\mathbb{D},\mathbb{D})$ e $\sigma \in \partial\mathbb{D}$ tali che
    \begin{equation}
      f(z)=\sigma+(z-\sigma)+o\bigl((z-\sigma)^3\bigr)
    \end{equation}
    per $z \longrightarrow \sigma$ non tangenzialmente. Allora
    \begin{equation}
      |f^h(z)|=1+o\bigl((z-\sigma)^2\bigr)
    \end{equation}
    per $z \longrightarrow \sigma$ non tangenzialmente.
  \end{prop}
\end{frame}

\begin{frame}[t]
  \frametitle{Teorema di Burns-Krantz}
  \only<1->{\begin{thm} \label{burns_krantz}
    (Burns-Krantz, 1994) Siano $f \in \text{\normalfont{Hol}}(\mathbb{D},\mathbb{D})$ e $\sigma \in \partial\mathbb{D}$ tali che
    \begin{equation} \label{o^3bis}
      f(z)=\sigma+(z-\sigma)+o\bigl((z-\sigma)^3\bigr)
    \end{equation}
    per $z \longrightarrow 1$ non tangenzialmente. Allora $f$ è l'identità del disco.
  \end{thm}}
  \only<2-4>{\textit{Traccia della dimostrazione:} senza perdita di generalità $\sigma=1$.} \only<3-4>{ Dalla Proposizione sui limiti non tangenziali segue che
  $$|f^h(z)|=1+o\bigl((z-1)^2\bigr).$$}
  \only<4>{Per il teorema di Bracci-Kraus-Roth, $f \in \text{Aut}(\mathbb{D})$;} \only<4>{ per ipotesi dev'essere $f(1)=1$ e $f''(1)=0$, perciò $f(z)=z$. \qed}
  \only<5>{\begin{ex}
    Se $f(z)=\dfrac{1+3z^2}{3+z^2}$, si ha $\displaystyle \lim_{z \longrightarrow 1}\frac{f(z)-z}{(z-1)^3}=-\frac{1}{4}$; dunque il termine $o\bigl((z-\sigma)^3\big)$ nel teorema di Burns-Krantz non è migliorabile.
\end{ex}}
\end{frame}
