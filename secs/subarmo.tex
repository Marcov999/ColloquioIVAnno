\begin{exc} \label{succ_uj}
  Sia $u: \Omega \longrightarrow \mathbb{R}\cup\{-\infty\}$ una funzione semicontinua superiormente (scs) limitata dall'alto. Allora esistono $\{u_j\} \subset C^0(\Omega)$ limitate dall'alto t.c. $u_j \downarrow u$ puntualmente. Hint: per $j \ge 1$ si pone $\displaystyle u_j(x)=\sup_{y \in \Omega} \{u(y)-j\|x-y\|\} \ge u(x)$.
\end{exc}

\begin{defn}
  Sia $\Omega \subseteq \mathbb{C}$ aperto. Una funzione $u: \Omega \longrightarrow \mathbb{R}\cup\{-\infty\}$ scs è \textsc{subarmonica} se per ogni $a \in \Omega$, per ogni $r>0$ t.c. $\overline{D(a,r)} \subset \Omega$ e per ogni $h \in C^0(\overline{D(a,r)})\cap \mathcal{H}(D(a,r))$, se $h\restrict{\partial D(a,r)} \ge u\restrict{\partial D(a,r)}$, allora anche $h\restrict{D(a,r)} \ge u \restrict{D(a,r)}$.
\end{defn}

\begin{oss}
  Armonica $\implies$ subarmonica (segue dalla formula di Poisson).
\end{oss}

\begin{oss}
  Se $\pm u$ sono subarmoniche, $u$ è armonica (basta prendere la sua estensione armonica $h$ e far vedere che coincide con $u$).
\end{oss}

\begin{thm}
  Sia $\Omega \subseteq \mathbb{C}$ aperto, $u:\Omega \longrightarrow \mathbb{R}\cup\{-\infty\}$ scs. Sono equivalenti:
  \begin{nlist}
    \item $u$ è subarmonica;
    \item per ogni $x \in \Omega$ e per ogni $r>0$ t.c. $\overline{D(x,r)} \subset \Omega$ vale la \textit{proprietà della sottomedia}, cioè $\displaystyle u(x) \le \frac{1}{2\pi r} \int_{\partial D(x,r)} u(\zeta)\diff\zeta=\frac{1}{2\pi}\int_0^{2\pi} u(x+re^{i\theta})\diff\theta$;
    \item per ogni $K \subset\subset \Omega$ compatto e per ogni $h \in C^0(K) \cap \mathcal{H}\left(\mathop {K}\limits^ \circ\right)$ se $h\restrict{\partial K}\ge u \restrict{\partial K}$ allora $h\restrict{K} \ge u \restrict{K}$;
    \item esiste una successione $\{u_j\}$ di funzioni subarmoniche t.c. $u_j \downarrow u$;
    \item per ogni $x \in \Omega$, per ogni $0<\delta<d(x, \partial \Omega)$ e per ogni $\mu$ misura di Borel positiva su $[0,\delta]$ si ha $\displaystyle u(x)\int_0^{\delta} \diff\mu \le \frac{1}{2\pi}\int_0^{\delta}\left[\int_0^{2\pi} u(x+se^{i\theta})\diff\theta\right]\diff\mu(s)$;
    \item per ogni $x \in \Omega$ e per ogni $0<\delta<d(x, \partial \Omega)$ esiste $\mu$ misura di Borel positiva su $[0,\delta]$ si ha $\displaystyle u(x)\int_0^{\delta} \diff\mu \le \frac{1}{2\pi}\int_0^{\delta}\left[\int_0^{2\pi} u(x+se^{i\theta})\diff\theta\right]\diff\mu(s)$;
    \item per ogni $x \in \Omega$, per ogni $r>0$ t.c. $\overline{D(x,r)}\subset \Omega$ e per ogni $y \in D(x,r)$ vale $\displaystyle u(y) \le \int_0^{2\pi} P_{x,r}(y,x+re^{i\theta})u(x+re^{i\theta})\diff\theta$;
    \item per ogni $x \in \Omega$ e per ogni $r>0$ t.c. $\overline{D(x,r)}\subset \Omega$ vale $\displaystyle u(x) \le \frac{1}{\pi r^2} \int_{D(x,r)} u(y)\diff\text{Leb}(y)$.
  \end{nlist}
\end{thm}

\begin{proof}
  (v) $\implies$ (vi) è ovvio.

  (vi) $\implies$ (iii) Siano $K \subset \subset \Omega$ e $h \in C^0(K) \cap \mathcal{H}\left(\mathop {K}\limits^ \circ\right)$ t.c. $h\restrict{\partial K} \ge u\restrict{\partial K}$; poniamo $f=u-h$. Per ipotesi $f \le 0$ su $\partial K$. Per assurdo, $f>0$ da qualche parte in $K$.
  $f$ scs $\implies$ ha massimo $M>0$ su $K$. Sia $L=\{y \in K \mid f(y)=M\} \subset \mathop {K}\limits^ \circ$. Sia $y_0 \in L$ il punto di $L$ più vicino a $\partial K$, Sia $\rho_0>0$ t.c. $\overline{D(y_0,\rho_0)} \subset \mathop {K}\limits^ \circ$.
  Se $\rho_0 \ge \rho > 0$ esiste un arco di $\partial D(y_0, \rho)$ non contenuto in $L$ $\implies$ $f<M$ almeno su un arco di $\partial D(y_0, \rho)$.
  Ma allora $\displaystyle \int_0^{\rho_0} \left[\int_0^{2\pi} f(y_0+\rho e^{i\theta})\diff\theta\right] \diff\mu(\rho)<2\pi M\int_0^{\rho_0} \diff\mu(\rho)=2\pi f(y_0)\int_0^{\rho_0} \diff\mu(\rho)$.
  Siccome $\displaystyle \int_0^{2\pi} h(y_0+\rho e^{i\theta}) \diff\theta=2\pi h(y_0)$, otteniamo $\displaystyle \int_0^{\rho_0} \left[\int_0^{2\pi} u(y_0+\rho e^{i\theta})\diff\theta\right]\diff\mu(\rho)<2\pi u(y_0) \int_0^{\rho_0} \diff \mu$, contro (vi).

  (iii) $\implies$ (i) è ovvio.

  (i) $\implies$ (vii) Sia $\{u_j\} \subset C^0(\overline{D(x,r)})$ con $u_j \downarrow u$ data dall'esercizio \ref{succ_uj}. Sia $h_j \in C^0(\overline{D(x,r)})\cap \mathcal{H}(D(x,r))$ l'estensione armonica di $u_j$ data da $\displaystyle h_j(y)=\int_0^{2\pi} P_{x,r}(y,x+re^{i\theta})u_j(x+re^{i\theta})\diff\theta$.
  Allora $u(y) \le u_j(y)$ su $\overline{D(x,r)}$; in particolare $u\restrict{\partial{D(x,r)}} \le u_j\restrict{\partial D(x,r)}=h_j\restrict{\partial D(x,r)}$, dunque per (i) abbiamo che $\displaystyle u(y) \le h_j(y)=\int_0^{2\pi} P_{x,r}(y,x+re^{i\theta})u_j(x+re^{i\theta})\diff\theta$. A questo punto basta mandare $j \longrightarrow +\infty$.

  (vii) $\implies$ (ii) Basta porre $y=x$ in (vii) perché $P_{x,r}(x,x+re^{i\theta})=\dfrac{1}{2\pi}$.

  (ii) $\implies$ (v) è ovvio (basta integrare (ii) rispetto a $\mu$).

  (i) $\implies$ (iv) Basta porre $u_j=u+\dfrac{1}{j}$.

  (iv) $\implies$ (i) Fissiamo $\epsilon>0, j \ge 1, x \in \Omega, r>0$ t.c. $\overline{D(x,r)} \subset \Omega$, $h \in C^0(\overline{D(x,r)})\cap \mathcal{H}(D(x,r))$ t.c. $h\restrict{\partial D(x,r)} \ge u\restrict{\partial D(x,r)}$.
  Sia $S_j=\{e^{i\theta} \in S^1 \mid u_j(x+re^{i\theta}) \ge h(x+re^{i\theta})+\epsilon\}$. Ogni $S_j$ è chiuso e compatto, $S_{j+1} \subseteq S_j$, $\displaystyle \bigcap_j S_j=\emptyset \implies$ esiste $j_0$ t.c. $S_j=\emptyset$ per ogni $j \ge j_0$.
  Quindi $u\restrict{\partial D(x,r)} \le u_j\restrict{D(x,r)}<h\restrict{\partial D(x,r)}+\epsilon$ per ogni $j >> 1$ $\implies$ $u\restrict{D(x,r)} \le u_j\restrict{D(x,r)} \le h\restrict{D(x,r)}+\epsilon$ per ogni $j>>1$. Con $\epsilon \longrightarrow 0$ otteniamo $u$ subarmonica.

  (ii) $\implies$ (viii) Da (ii) abbiamo che $\displaystyle \frac{1}{2}r_0^2u(x)=\int_0^{r_0} ru(x)\diff r \le \frac{1}{2\pi} \int_0^{r_0} \int_0^{2\pi} u(x+re^{i\theta}) r\diff\theta\diff r=\frac{1}{2\pi} \int_{D(x,r_0)} u(y)\diff\text{Leb}(y)$.

  (viii) $\implies$ (iii) Come (vi) $\implies$ (iii) con $\mu=r\diff r$.
\end{proof}

\begin{cor}
  (\textit{Principio del massimo}) Sia $u:\Omega \longrightarrow \mathbb{R}\cup\{-\infty\}$ subarmonica con $\Omega$ aperto connesso. Supponiamo che esista $x_0 \in \Omega$ t.c. $\displaystyle u(x_0)=\sup_{x \in \Omega} u(x)$. Allora $u \equiv u(x_0)$.
\end{cor}

\begin{proof}
  $\{x \in \Omega \mid u(x)=u(x_0)\}$ è chiuso e aperto per il punto (viii) del teorema appena dimostrato.
\end{proof}

\begin{lm}
  (Disuguaglianza di Jensen) Sia $\varphi:\mathbb{R} \longrightarrow \mathbb{R}$ convessa e $\mu$ una misura di probabilità. Allora per ogni $g \in L^1(\mu)$ si ha $\displaystyle \varphi\left(\int g(x)\diff\mu(x)\right) \le \int \varphi\circ g(x)\diff\mu(x)$.
\end{lm}

\begin{proof}
  Poniamo $\displaystyle x_0=\int g(x)\diff\mu(x) \in \mathbb{R}$. Siccome $\varphi$ è convessa, esistono $a, b \in \mathbb{R}$ t.c. $ax_0+b=\varphi(x_0)$ e $ax+b \le \varphi(x)$ per ogni $x \in \mathbb{R}$.
  Allora $ag(x)+b \le \varphi(g(x))$ per ogni $x$ $\implies$ $\displaystyle \varphi(x_0)=ax_0+b=a\int g(x)\diff\mu(x)+b=\int (ag(x)+b)\diff\mu(x) \le \int \varphi(g(x))\diff\mu(x)$.
\end{proof}

\begin{cor}
  $u:\Omega \longrightarrow \mathbb{R}\cup\{-\infty\}$ subarmonica, $\varphi:\mathbb{R}\longrightarrow \mathbb{R}$ crescente e convessa $\implies$ $\varphi\circ u$ è subarmonica.
\end{cor}

\begin{proof}
  Dal fatto che $\varphi$ è crescente e $u$ subarmonica otteniamo $\displaystyle \varphi(u(x)) \le \varphi\left(\frac{1}{\pi r^2}\int_{D(x,r)}u(y)\diff\text{Leb}\right) \le \frac{1}{\pi r^2} \int_{D(x,r)} \varphi(u(y))\diff\text{Leb}$ dove l'ultima disuguaglianza segue dal fatto che $\dfrac{1}{\pi r^2}\diff\text{Leb}$ è di probabilità e dalla disuguaglianza di Jensen.
\end{proof}

\begin{prop}
  $u \in C^2(\Omega)$ è subarmonica $\iff$ $\Delta u \ge 0$.
\end{prop}

\begin{proof}
  ($\Leftarrow$) $D=D(x,r) \subset\subset \Omega$, $h \in C^0(D)\cap \mathcal{H}(D)$ con $h\restrict{\partial D} \ge u\restrict{\partial D}$. Sappiamo che $\Delta(u-h) \ge 0$ in $D$ e $u-h \le 0$ su $\partial D$, dunque per il principio del massimo (quello con l'ipotesi sul laplaciano) abbiamo $u-h \le 0$ su $D$.

  ($\implies$) Per assurdo esiste $x_0 \in \Omega$ t.c. $\Delta u(x_0)<0 \implies \Delta u(x)<0$ per ogni $x \in D(x_0,r) \implies -u$ è subarmonica in $D(x_0,r)$ $\implies$ $u$ è armonica in $D(x_0,r)$ $\implies$ $\Delta u(x_0)=0$, assurdo.
\end{proof}

\begin{cor}
  $f \in \mathcal{O}(\Omega) \implies |f|^p$ per ogni $p \ge 0$ e $\log{|f|}$ sono subarmoniche.
\end{cor}

\begin{proof}
  Se $f(x_0)\not=0$ abbiamo $\Delta|f|^p=p^2|f|^{p-2}\left|\dfrac{\partial f}{\partial z}\right|^2 \ge 0$ (segue da $|f|^p=|f\bar{f}|^{p/2}$ e $\Delta=4\dfrac{\partial^2}{\partial z\partial\bar{z}}$) e $\log{|f|}=\mathfrak{Re}\log{f}$ vicino a $x_0$.
  Se $f(x_0)=0$, la proprietà della sottomedia in $x_0$ è ovvia.
\end{proof}

\begin{defn}
  Sia $\Omega \subseteq \mathbb{C}^n$ aperto, $u:\Omega \longrightarrow \mathbb{R}\cup\{-\infty\}$ scs è \textsc{plurisubarmonica} se per ogni $z \in \Omega$ e per ogni $v \in \mathbb{C}^n$ l'applicazione $\zeta \longmapsto u(z+\zeta v)$ è subarmonica dove definita. Scriviamo $u \in PSH(\Omega)$.
\end{defn}

\begin{prop}
  $u \in C^2(\Omega)$ è plurisubarmonica $\iff$ per ogni $z \in \Omega$ e per ogni $v \in \mathbb{C}^n$ vale $\displaystyle \sum_{j,k=1}^n \frac{\partial^2 u}{\partial z_j\partial\bar{z}_k}(z)v_j\bar{v}_k \ge 0$.
\end{prop}

\begin{proof}
  Poniamo $v(\zeta)=u(z+\zeta v)$. $\displaystyle \frac{\partial v}{\partial\zeta}(\zeta)=\sum_{j=1}^n \frac{\partial u}{\partial z_j}(z+\zeta v) \cdot \frac{\partial}{\partial\zeta}(z_j+\zeta v_j)+\sum_{j=1}^n\frac{\partial u}{\partial\bar{z}_j}(z+\zeta v)\cdot \frac{\partial}{\partial\zeta}(\bar{z_j}+\bar{\zeta}\bar{v}_j)$.
  Osserviamo che $\dfrac{\partial}{\partial\zeta}(\bar{z_j}+\bar{\zeta}\bar{v}_j)=0$, perciò otteniamo $\displaystyle \Delta v(\zeta)=4\frac{\partial^2}{\partial\zeta\partial\bar{\zeta}}(\zeta)=4\sum_{j,k=1}^n \frac{\partial^2 u}{\partial z_j\partial\bar{z}_k}(z+\zeta v)v_j\bar{v}_k$.
\end{proof}

\begin{defn}
  Sia $u \in C^2(\Omega)$. La \textsc{forma di Levi di $u$ in $z \in \Omega$} è $L_{u,z}=\left(\frac{\partial^2 u}{\partial z_j\partial\bar{z}_k}(z)\right)$ (è una matrice hermitiana).
\end{defn}

\begin{oss}
  $u \in PSH(\Omega) \cap C^2(\Omega) \iff L_{u,z} \ge 0$ (cioè è semidefinita positiva) per ogni $z \in \Omega$.
\end{oss}

\begin{defn}
  $u \in C^2(\Omega)$ è \textit{strettamente plurisubarmonica} se $L_{u,z}>0$ per ogni $z \in \Omega$.
\end{defn}

\begin{oss}
  Se $\rho \in C^0(\mathbb{C}^n,\mathbb{R})$ allora $\Omega=\{z \in \mathbb{C}^n \mid \rho(z)<0\}$ è un aperto.
\end{oss}

\begin{defn}
  Un \textsc{dominio di classe $C^k$} (o \textit{con bordo di classe $C^k$}), $k in \mathbb{N}^*\cup\{\infty, \omega\}$ è $\Omega=\{z \in \mathbb{C}^n \mid \rho(z)<0\}$ con $\rho \in C^k(\mathbb{C}^n)$ (con $C^{\omega}$ si intendono le funzioni analitiche), detta \textit{funzione di definizione},
  t.c. $\vec{\nabla}\rho$ non si annulla mai su $\partial\Omega=\{z \in \mathbb{C}^n \mid \rho(z)=0\}$ ($\implies$ $\partial\Omega$ è una ipersuperficie reale di classe $C^k$).
\end{defn}

\begin{ex}
  $\mathbb{B}^n=\{\|z\|^2-1<0\}$, $\rho(z)=\|z\|^2-1$.
\end{ex}
