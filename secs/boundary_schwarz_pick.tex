Dalla disuguaglianza di Golusin possiamo dimostrare una versione al bordo del lemma di Schwarz-Pick, seguendo la traccia data nel remark 5.6 di \cite{BKR}.

\begin{thm} \label{boundary_schwarz_pick}
  (lemma di Schwarz-Pick al bordo) Sia $f:\mathbb{D} \longrightarrow \mathbb{D}$ una funzione olomorfa dal disco in sé tale che
  \begin{equation} \label{n_o^2}
    f^h(z_n)=1+o((|z_n|-1)^2)
  \end{equation}
  per qualche successione $\{z_n\}_{n \in \mathbb{N}} \subset \mathbb{D}$ con $|z_n| \longrightarrow 1$. Allora $f$ è un automorfismo.
\end{thm}

\begin{proof}
  Supponiamo per assurdo che $f$ non sia un automorfismo. Allora possiamo applicare il corollario \ref{quasigolusin}, che per $w=0$ ci dà
  \begin{align*}
    d(f^h(z), f^h(0)) \le 2d(z,0) \\
    \log{\left(\frac{1+\left|\frac{f^h(z)-f^h(0)}{1-f^h(z)f^h(0)}\right|}{1-\left|\frac{f^h(z)-f^h(0)}{1-f^h(z)f^h(0)}\right|}\right)} \le 2\log{\left(\frac{1+|z|}{1-|z|}\right)} \\
    \frac{|1-f^h(z)f^h(0)|+|f^h(z)-f^h(0)|}{|1-f^h(z)f^h(0)|-|f^h(z)-f^h(0)|} \le \frac{(1+|z|)^2}{(1-|z|)^2}.
  \end{align*}
  Ricordiamo che per definizione $f^h(z) \ge 0$ e inoltre per il lemma di Schwarz-Pick $f^h(z) \le 1$, per ogni $z \in \mathbb{D}$. Sempre per il lemma originale, se valesse $f^h(0)=1$ avremmo che $f$ è un automorfismo, contraddizione. Perciò dev'essere $f^h(0)<1$, ma $\displaystyle \lim_{n \longrightarrow +\infty} f^h(z_n)=1$, quindi definitivamente $f^h(z_n)-f^h(0)>0$ e $1-f^h(z_n)f^h(0)>0$, da cui
  \begin{align*}
    \frac{(1-f^h(0))(1+f^h(z_n))}{(1-f^h(z_n))(1+f^h(0))} \le \frac{(1+|z_n|)^2}{(1-|z_n|)^2} \\
    \frac{1+f^h(0)}{(1-f^h(0))(1+f^h(z_n))}(1-f^h(z_n)) \ge \frac{(1-|z_n|)^2}{(1+|z_n|)^2}.
  \end{align*}
  Per ipotesi vale \eqref{n_o^2}, dunque
  \begin{align*}
    \frac{1+f^h(0)}{(1-f^h(0))(1+f^h(z_n))}o((|z_n|-1)^2) \ge \frac{(1-|z_n|)^2}{(1+|z_n|)^2} \\
    \frac{(1+f^h(0))(1+|z_n|)^2}{(1-f^h(0))(1+f^h(z_n))}o((|z_n|-1)^2) \ge 1.
  \end{align*}
  Poiché $\displaystyle \lim_{n \longrightarrow +\infty} \frac{(1+f^h(0))(1+|z_n|)^2}{(1-f^h(0))(1+f^h(z_n))}=\frac{2(1+f^h(0))}{1-f^h(0)} < +\infty$, otteniamo di nuovo una contraddizione.
\end{proof}

Siamo ora pronti a dimostrare il teorema 2.1 di \cite{BK}.
