\begin{frame}[t]
  \frametitle{Strada per la dimostrazione del teorema di BB}
  \begin{itemize}
    \item Per una metrica $F$ su $\Omega$ che soddisfa certe ipotesi, si ha che esiste $C \ge 0$ tale che per ogni $x, y \in \Omega$ vale
    $$g(x,y)-C \le d_F(x,y) \le g(x,y)+C,$$
    dove $g(x,y)=2\log\left(\dfrac{d_H\big(\pi(x),\pi(y)\big)+h(x)\lor h(y)}{\sqrt{h(x)h(y)}}\right)$.
    \pause
    \item La metrica di Kobayashi soddisfa le suddette ipotesi.
    \pause
    \item Si ha dunque che possiamo confrontare $d_K$ con la funzione $g$, e da questa disuguaglianza segue il teorema di Balogh-Bonk.
  \end{itemize}
\end{frame}

\begin{frame}
  \frametitle{Disuguaglianze per metriche di Finsler}
  \only<1>{
  \begin{defn}
    Una \textit{metrica di Finsler} su $\Omega$ è una funzione continua $F:\Omega\times\mathbb{C}^n \longrightarrow [0,+\infty)$ tale che $F(x;tZ)=|t|F(x;Z)$ per ogni $x \in \Omega, Z \in \mathbb{C}^n, t \in \mathbb{C}$.
  \end{defn}
  }
  \only<2>{
  \begin{thm}
    Sia $F$ una metrica di Finsler su $\Omega$ tale che esistono delle costanti $\epsilon_0>0, s>0,C_1>0,C_2 \ge 0$ tali che per ogni $x \in \Omega$ con $\delta(x)<\epsilon_0$ e $Z \in \mathbb{C}^n$ si ha
    \begin{multline}\label{stimametricafinsler}
      \big(1-C_1\delta^s(x)\big)\left(\frac{|Z_N|^2}{4\delta^2(x)}+(1/C_2)\frac{L_\rho\big(\pi(x);Z_H\big)}{\delta(x)}\right)^{1/2} \le F(x;Z) \\
      \le \big(1+C_1\delta^s(x)\big)\left(\frac{|Z_N|^2}{4\delta^2(x)}+C_2\frac{L_\rho\big(\pi(x);Z_H\big)}{\delta(x)}\right)^{1/2}.
    \end{multline}
    Allora esiste $C \ge 0$ tale che per ogni $x, y \in \Omega$ vale
    \begin{equation} \label{stimadistanzafinsler}
      g(x,y)-C \le d_F(x,y) \le g(x,y)+C.
    \end{equation}
  \end{thm}
  }
  \only<3>{
  \textit{Idea della dimostrazione:} capire quali sono i punti salienti e riassumerli. Richiede un po' di lavoro.
  }
\end{frame}

\begin{frame}[t]
  \frametitle{La metrica di Kobayashi soddisfa la disuguaglianza}
  \only<1->{\begin{prop}
    Per ogni $\epsilon>0$ esistono $\epsilon_0>0$ e $C \ge 0$ tali che per ogni $x \in \Omega$ con $\delta(x)<\epsilon_0$ e per ogni $Z \in \mathbb{C}^n$ si ha
    \begin{multline*}
      \big(1-C\delta^{1/2}(x)\big)\left(\frac{|Z_N|^2}{4\delta^2(x)}+(1-\epsilon)\frac{L_\rho\big(\pi(x);Z_H\big)}{\delta(x)}\right)^{1/2} \le K(x;Z) \\
      \le \big(1+C\delta^{1/2}(x)\big)\left(\frac{|Z_N|^2}{4\delta^2(x)}+(1+\epsilon)\frac{L_\rho\big(\pi(x);Z_H\big)}{\delta(x)}\right)^{1/2}.
    \end{multline*}
  \end{prop}}
  \only<2-4>{\textit{Traccia della dimostrazione:} si localizza a un intorno di un punto del bordo;

  }
  \only<3-4>{con un opportuno biolomorfismo, ci si sposta in $\mathbb{C}^n$;

  }
  \only<4>{stringendo l'immagine del biolomorfismo tra due ellissoidi complessi, uno contenuto e uno che lo contiene, seguono le stime volute. \qed}
  \only<5>{
  \begin{cor}
    Esiste $C \ge 0$ tale che per ogni $x,y \in \Omega$ si ha
    \begin{equation}\label{stimadistanzakobayashi}
      g(x,y)-C \le d_K(x,y) \le g(x,y)+C. \tag{3}
    \end{equation}
  \end{cor}
  }
\end{frame}
