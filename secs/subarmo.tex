\begin{exc}
  Sia $u: \Omega \longrightarrow \mathbb{R}\cup\{-\infty\}$ una funzione semicontinua superiormente (scs) limitata dall'alto. Allora esistono $\{u_j\} \subset C^0(\Omega)$ limitate dall'alto t.c. $u_j \downarrow u$ puntualmente. Hint: per $j \ge 1$ si pone $\displaystyle u_j(x)=\sup_{y \in \Omega} \{u(y)-j\|x-y\|\} \ge u(x)$.
\end{exc}

\begin{defn}
  Sia $\Omega \subseteq \mathbb{C}$ aperto. Una funzione $u: \Omega \longrightarrow \mathbb{R}\cup\{-\infty\}$ scs è \textsc{subarmonica} se per ogni $a \in \Omega$, per ogni $r>0$ t.c. $\overline{D(a,r)} \subset \Omega$ e per ogni $h \in C^0(\overline{D(a,r)})\cap \mathcal{H}(D(a,r))$, se $h\restrict{\partial D(a,r)} \ge u\restrict{\partial D(a,r)}$, allora anche $h\restrict{D(a,r)} \ge u \restrict{D(a,r)}$.
\end{defn}

\begin{oss}
  Armonica $\implies$ subarmonica (segue dalla formula di Poisson).
\end{oss}

\begin{oss}
  Se $\pm u$ sono subarmoniche, $u$ è armonica (basta prendere la sua estensione armonica $h$ e far vedere che coincide con $u$).
\end{oss}
