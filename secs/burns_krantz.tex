\marginpar{Magari dire qualcosa sull'utilità di questa proposizione, che forse verrà spostata nei prerequisiti}

\begin{prop} \label{o^3->o^2}
  Sia $f:\mathbb{D} \longrightarrow \mathbb{D}$ una funzione tale che
  \begin{equation} \label{o^3}
    f(z)=1+(z-1)+o((z-1)^3)
  \end{equation}
  per $z \longrightarrow 1$ non tangenzialmente. Allora
  \begin{equation} \label{o^2}
    f^h(z)=1+o((z-1)^2)
  \end{equation}
  per $z \longrightarrow 1$ non tangenzialmente.
\end{prop}

\begin{proof}
  Sia $S$ un settore (cono? non conosco il termine tecnico in italiano) di vertice $1$ e angolo d'apertura $2\alpha$, e $S'$ uno un po' più grande di vertice $1$ e angolo $2\beta$, $\beta>\alpha$. Per $z \in S$, sia $C(z)$ il cerchio di centro $z$ e raggio $r(z)=\text{dist}(z, \partial S')$ (la distanza di $z$ dal bordo di $S'$). Allora per la formula integrale di Cauchy
  \begin{align*}
    f'(z) & =\frac{1}{2\pi i} \int_{C(z)} \frac{f(w)}{(w-z)^2}\diff w \\
    & =\frac{1}{2\pi i} \int_{C(z)} \frac{w-1+(f(w)-w)}{(w-z)^2}\diff w \\
    & =\frac{1}{2\pi i} \int_{C(z)} \frac{1}{w-z}\diff w+\frac{1}{2\pi i} \int_{C(z)} \frac{z-1+f(w)-w}{(w-z)^2}\diff w \\
    & =1+\frac{1}{2\pi i} \int_{C(z)} \frac{f(w)-w}{(w-z)^2}\diff w=:1+I(z).
  \end{align*}
  Dato $\epsilon>0$ fissato, per ipotesi esiste $\delta>0$ tale che $|f(w)-w|<\epsilon|1-w|^3$ per ogni $w \in S'$ con $|w-1|<\delta$. Per questi $w$ vale che
  \marginpar{Mettere un disegno. Qualche spiegazione in più?}
  \begin{align*}
    |I(z)| \le \frac{\epsilon}{2\pi} \int_0^{2\pi} \frac{|1-(z+r(z)e^{i\theta})|^3}{|(z+r(z)e^{i\theta})-z|^2}r(z)\diff\theta \le \\
    \le \frac{\epsilon}{r(z)}\max_{\theta \in [0,2\pi]} |1-(z+r(z)e^{i\theta})|^3=\frac{\epsilon}{r(z)}\max_{w \in C(z)}|1-w|^3= \\
    =\epsilon r(z)^2\left(1+\frac{|z-1|}{r(z)}\right)^3 \le \epsilon r(z)^2(1+\csc(\beta-\alpha))^3 \le \epsilon |z-1|^2(1+\csc(\beta-\alpha))^3,
  \end{align*}
  da cui otteniamo $f'(z)=1+o((z-1)^2)$ per $z \longrightarrow 1$ non tangenzialmente. Inoltre, per ipotesi
  \marginpar{Ok, bisogna chiarire questa cosa di $z-1$ e $|z|-1$, il claim è che non tangenzialmente hanno gli stessi $o$-piccoli}
  $$\frac{1-|f(z)|}{1-|z|}=\frac{1-|z|+o((z-1)^3)}{1-|z|}=1+o((z-1)^2)$$
  per $z \longrightarrow 1$ non tangenzialmente. \\
  Possiamo quindi concludere che
  $$f^h(z)=|f'(z)|\frac{1-|z|^2}{1-|f(z)|^2}=1+o((z-1)^2)$$
  per $z \longrightarrow 1$ non tangenzialmente.
  \marginpar{gli ultimi passaggi mi sono un po' oscuri, per via degli $o$-piccoli}
\end{proof}

Siamo ora pronti a dimostrare il teorema 2.1 di \cite{BK}.

\begin{thm} \label{burns_krantz}
  (Burns-Krantz, 1994) Sia $f:\mathbb{D} \longrightarrow \mathbb{D}$ una funzione olomorfa dal disco in sé tale che
  \begin{equation} \label{O^4}
    f(z)=1+(z-1)+\mathcal{O}((z-1)^4)
  \end{equation}
  per $z \longrightarrow 1$. Allora $f$ è l'identità sul disco.
\end{thm}

\marginpar{È comprensibile? E siamo sicuri che valgono le ipotesi di \ref{boundary_schwarz_pick}? (c'era quel discorso di $(z-1)^2$ e $(|z|-1)^2$)}

\begin{proof}
  Chiaramente, se vale \eqref{O^4} per $z \longrightarrow 1$ vale anche \eqref{o^3}, in particolare per $z \longrightarrow 1$ non tangenzialmente. Dalla proposizione \ref{o^3->o^2} segue che anche \eqref{o^2} vale per $z \longrightarrow 1$ non tangenzialmente, quindi esiste una successione $z_n$ che soddisfa le ipotesi del teorema \ref{boundary_schwarz_pick}, da cui la tesi.
\end{proof}

Aggiungere controesempio
