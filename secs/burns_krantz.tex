Per poter dimostrare il risultato finale sfruttando la versione al bordo del lemma di Schwarz-Pick, serve poter tradurre informazioni sull'andamento di $f$ vicino a $1$ in informazioni sull'andamento di $f^h$. La seguente proposizione ci permette proprio di fare queato passaggio.

\begin{prop} \label{o^3->o^2}
  Sia $f:\mathbb{D} \longrightarrow \mathbb{D}$ una funzione tale che
  \begin{equation} \label{o^3}
    f(z)=1+(z-1)+o((z-1)^3)
  \end{equation}
  per $z \longrightarrow 1$ non tangenzialmente. Allora
  \begin{equation} \label{o^2}
    f^h(z)=1+o((z-1)^2)
  \end{equation}
  per $z \longrightarrow 1$ non tangenzialmente.
\end{prop}

\begin{proof}

  Sia $S$ un settore di vertice $1$ e angolo d'apertura $2\alpha$, e $S'$ uno un po' più grande di vertice $1$ e angolo $2\beta$, $\beta>\alpha$. Per $z \in S$, sia $C(z)$ il cerchio di centro $z$ e raggio $r(z)=\text{dist}(z, \partial S')$ (la distanza di $z$ dal bordo di $S'$). Allora per la formula integrale di Cauchy
  \begin{align*}
    f'(z) & =\frac{1}{2\pi i} \int_{C(z)} \frac{f(w)}{(w-z)^2}\diff w \\
    & =\frac{1}{2\pi i} \int_{C(z)} \frac{w-1+(f(w)-w)}{(w-z)^2}\diff w \\
    & =\frac{1}{2\pi i} \int_{C(z)} \frac{1}{w-z}\diff w+\frac{1}{2\pi i} \int_{C(z)} \frac{z-1+f(w)-w}{(w-z)^2}\diff w \\
    & =1+\frac{1}{2\pi i} \int_{C(z)} \frac{f(w)-w}{(w-z)^2}\diff w=:1+I(z).
  \end{align*}

  \begin{center}
    \definecolor{qqffqq}{rgb}{0,1,0}
    \definecolor{qqqqff}{rgb}{0,0,1}
    \definecolor{uququq}{rgb}{0.25,0.25,0.25}
    \definecolor{ffqqqq}{rgb}{1,0,0}
    \begin{tikzpicture}[line cap=round,line join=round,>=triangle 45,x=3.0cm,y=3.0cm]
      \draw[->,color=black] (-1.12,0) -- (1.2,0);
      \foreach \x in {-1,1}
      \draw[shift={(\x,0)},color=black] (0pt,2pt) -- (0pt,-2pt);
      \draw[->,color=black] (0,-1.11) -- (0,1.11);
      \foreach \y in {-1,1}
      \draw[shift={(0,\y)},color=black] (2pt,0pt) -- (-2pt,0pt);
      \clip(-1.12,-1.11) rectangle (1.2,1.11);
      \draw [shift={(1,0)},color=ffqqqq,fill=ffqqqq,fill opacity=0.1] (0,0) -- (180:0.26) arc (180:244.77:0.26) -- cycle;
      \draw [shift={(1,0)},color=qqqqff,fill=qqqqff,fill opacity=0.1] (0,0) -- (-180:0.14) arc (-180:-127.09:0.14) -- cycle;
      \draw [shift={(1,0)},color=qqffqq,fill=qqffqq,fill opacity=0.1] (0,0) -- (115.23:0.33) arc (115.23:143.3:0.33) -- cycle;
      \draw(0,0) circle (3cm);
      \draw (1,0)-- (0.27,0.96);
      \draw (0.64,0.77)-- (1,0);
      \draw (0.27,-0.96)-- (1,0);
      \draw (1,0)-- (0.64,-0.77);
      \draw [shift={(1,0)},color=ffqqqq] (180:0.26) arc (180:244.77:0.26);
      \draw [shift={(1,0)},color=ffqqqq] (180:0.23) arc (180:244.77:0.23);
      \draw(0.5,0.37) circle (0.888cm);
      \draw (0.5,0.37)-- (0.77,0.49);
      \draw (0.27,0.54)-- (1,0);
      \draw [shift={(1,0)},color=qqffqq] (115.23:0.33) arc (115.23:143.3:0.33);
      \draw [shift={(1,0)},color=qqffqq] (115.23:0.3) arc (115.23:143.3:0.3);
      \draw [shift={(1,0)},color=qqffqq] (115.23:0.28) arc (115.23:143.3:0.28);
      \begin{scriptsize}
        \draw[color=ffqqqq] (0.84,-0.09) node {$\beta$};
        \fill [color=black] (0.5,0.37) circle (1.5pt);
        \draw[color=black] (0.48,0.32) node {$z$};
        \draw[color=black] (0.18,0.14) node {$C(z)$};
        \draw[color=black] (0.54,0.465) node {$r(z)$};
        \fill [color=black] (1,0) circle (1.5pt);
        \draw[color=black] (1.04,0.05) node {$1$};
        \fill [color=black] (0.26,0.545) circle (1.5pt);
        \draw[color=black] (0.20,0.58) node {$A$};
        \fill [color=black] (1,0) circle (1.5pt);
        \draw[color=qqqqff] (0.92,-0.03) node {$\alpha$};
        \draw[color=qqffqq] (0.83,0.16) node {$\gamma$};
      \end{scriptsize}
    \end{tikzpicture}
  \end{center}

  Dato $\epsilon>0$ fissato, per ipotesi esiste $\delta>0$ tale che $|f(w)-w|<\epsilon|1-w|^3$ per ogni $w \in S'$ con $|w-1|<\delta$. Se $|z-1|<\delta/2$, $r(z)<|z-1|<\delta/2$, dunque per ogni $w \in C(z)$ abbiamo effettivamente $|w-1| \le |w-z|+|z-1|=r(z)+|z-1|<\delta$. Per questi $z$ vale che
  \begin{align*}
    |I(z)| & \le \frac{\epsilon}{2\pi} \int_0^{2\pi} \frac{|1-(z+r(z)e^{i\theta})|^3}{|(z+r(z)e^{i\theta})-z|^2}r(z)\diff\theta \\
    & \le \frac{\epsilon}{r(z)}\max_{\theta \in [0,2\pi]} |1-(z+r(z)e^{i\theta})|^3 \\
    & =\frac{\epsilon}{r(z)}\max_{w \in C(z)}|1-w|^3.
  \end{align*}
  Il massimo è raggiunto per l'intersezione più lontana da $1$ tra la retta passante per $z$ e $1$ e la circonferenza $C(z)$ (il punto $A$ in figura), perciò
  \begin{align*}
    |I(z)| & \le \frac{\epsilon}{r(z)}(r(z)+|z-1|)^3 \\
    & =\epsilon r(z)^2\left(1+\frac{|z-1|}{r(z)}\right)^3 \\
    & =\epsilon r(z)^2(1+\csc\gamma)^3 \\
    & \le \epsilon r(z)^2(1+\csc(\beta-\alpha))^3 \\
    & \le \epsilon |z-1|^2(1+\csc(\beta-\alpha))^3,
  \end{align*}
  da cui otteniamo $f'(z)=1+o((z-1)^2)$ per $z \longrightarrow 1$ non tangenzialmente. Inoltre, per ipotesi
  \marginpar{$|z-1|$ e $1-|z|$ non tangenzialmente hanno gli stessi $o$-piccoli (chiedere conferma ad Abate): devo scrivere la dimostrazione?}
  $$\frac{1-|f(z)|}{1-|z|}=\frac{1-|z|+o((z-1)^3)}{1-|z|}=1+o((z-1)^2)$$
  per $z \longrightarrow 1$ non tangenzialmente (perché in tal caso $|z-1|$ e $1-|z|$ hanno gli stessi $o$-piccoli). \\
  %Accenno della dim: 1-|z| \le |z-1| segue dalla disuguaglianza triangolare, |z-1| \le (1-|z|)*costante si fa scrivendo |z-1|=r(z)*(|z-1|/r(z)), ripetere la dim di sopra per far apparire la costante csc(beta-alpha) e osservando che r(z) \le 1-|z| perché C(z) sta dentro \mathbb{D}
  Possiamo quindi concludere che
  $$f^h(z)=|f'(z)|\frac{1-|z|^2}{1-|f(z)|^2}=1+o((z-1)^2)$$
  per $z \longrightarrow 1$ non tangenzialmente.
  \marginpar{Gli ultimi passaggi con $o$-piccoli: devo spiegarli meglio? Forse non sono così ovvi}
\end{proof}

Siamo ora pronti a dimostrare il teorema 2.1 di \cite{BK}.

\begin{thm} \label{burns_krantz}
  (Burns-Krantz, 1994) Sia $f:\mathbb{D} \longrightarrow \mathbb{D}$ una funzione olomorfa dal disco in sé tale che
  \begin{equation} \label{O^4}
    f(z)=1+(z-1)+\mathcal{O}((z-1)^4)
  \end{equation}
  per $z \longrightarrow 1$. Allora $f$ è l'identità sul disco.
\end{thm}

\begin{proof}
  \marginpar{È comprensibile?}

  Chiaramente, se vale \eqref{O^4} per $z \longrightarrow 1$ vale anche \eqref{o^3}, in particolare per $z \longrightarrow 1$ non tangenzialmente.
  Dalla proposizione \ref{o^3->o^2} segue che anche \eqref{o^2} vale per $z \longrightarrow 1$ non tangenzialmente, quindi esiste una successione $z_n$ che soddisfa le ipotesi del teorema \ref{boundary_schwarz_pick} (usiamo di nuovo che, non tangenzialmente, $|z-1|$ e $1-|z|$ hanno gli stessi $o$-piccoli), da cui la tesi.
\end{proof}

Il termine $\mathcal{O}((z-1)^4)$ non è migliorabile, come mostra il seguente controesempio.

\begin{ex}
  $f(z)=\dfrac{1+3z^2}{3+z^2}$. Osserviamo che $f$ è una funzione olomorfa su $\mathbb{C} \setminus \{\pm i\sqrt{3}\}$, quindi in particolare è ben definita su $\mathbb{D}$. Verifichiamo che l'immagine è contenuta in $\mathbb{D}$:
  \begin{align*}
    |f(z)|^2<1 \\
    \dfrac{(1+3z^2)(1+3\bar{z}^2)}{(3+z^2)(3+\bar{z}^2)} < 1 \\
    (1+3z^2)(1+3\bar{z}^2) < (3+z^2)(3+\bar{z}^2) \\
    1+3z^2+3\bar{z}^2+9|z|^4 < 9+3z^2+3\bar{z}^2+|z|^4 \\
    1-|z|^4 < 9(1-|z|^4)
  \end{align*}
  e l'ultima disuguaglianza è verificata perché $z \in \mathbb{D} \implies |z|<1 \implies 1-|z|^4>0$. \\
  Ovviamente $f$ non può essere iniettiva su $\mathbb{D}$ perché $f(z)=f(-z)$, dunque non è un automorfismo. Adesso mostriamo che $f(z)-1-(z-1)$ è $\mathcal{O}((z-1)^3)$ ma non $\mathcal{O}((z-1)^4)$ per $z \longrightarrow 1$:
  \begin{align*}
    f(z)-z & =\frac{1+3z^2}{3+z^2}-z \\
    & =\frac{1+3z^2-3z-z^3}{3+z^2} \\
    & =\frac{(1-z)^3}{3+z^2}=:g(z).
  \end{align*}
  Poiché $\displaystyle \lim_{z \longrightarrow 1} g(z)/(z-1)^3=-1/4$, $g(z)$ è $\mathcal{O}((z-1)^3)$ ma non $\mathcal{O}((z-1)^4)$ per $z \longrightarrow 1$.
\end{ex}
