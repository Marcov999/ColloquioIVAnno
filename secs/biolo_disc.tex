Lo scopo di questo paragrafo è mostrare che quasi tutti i domini semplicemente connessi di $\mathbb{C}$ sono biolomorfi al disco. Il teorema di uniformizzazione di Riemann, di cui riporteremo solo l'enunciato, caratterizza i biolomorfismi delle superfici di Riemann, in particolare caratterizza completamente i biolomorfismi di quelle semplicemente connesse.

\begin{lm}
  Sia $\Omega \subset \mathbb{D}$ dominio limitato con $\Omega \not=\mathbb{D}$ e $0 \in \Omega$.
  Allora esiste $f$ t.c. $f(0)=0, f'(0) \in \mathbb{R}, f'(0)>0$, $\Omega \subseteq f(\mathbb{C})$ e, se $\Omega_f$ è la componente connessa di $f^{-1}(\Omega)$ contentente $0$, $f\restrict{\Omega_f}:\Omega_f \longrightarrow \Omega$ è un rivestimento.
  Inoltre, $\displaystyle d_1=\inf_{z \not\in \Omega_f} |z|>\inf_{z \not\in \Omega} |z|=d$.
\end{lm}

\begin{proof}
  Sia $a \in \mathbb{D}\setminus\Omega$, $b \in \mathbb{D}$ t.c. $b^2=-a$. Siano $\varphi, \psi \in \text{Aut}(\mathbb{D})$, $\varphi(z)=\dfrac{z+a}{1+\bar{a}z}, \psi(z)=\dfrac{z+b}{1+\bar{b}z}$.
  Poniamo $f \in \text{Hol}(\mathbb{D}, \mathbb{D})$ t.c. $f(z)=\dfrac{\bar{b}}{|b|}\varphi(\psi(z)^2)$. $f(\mathbb{D})=\mathbb{D}, f(0)=0$. $f'(0)=2|b|\dfrac{1-|b|^2}{1-|a|^2}>0$.
  Siccome $w \longmapsto w^2$ è un rivestimento di $\mathbb{D}^*=\mathbb{D}\setminus\{0\}$ e $\varphi^{-1}(\Omega) \subseteq \mathbb{D}^*$, $f$ è un rivestimento da $\Omega_f$ a $\Omega$. Se $d_1=1$ abbiamo finito in quanto $d \le |a|<1$.
  Se invece $d_1<1$, esiste $z_1 \in \partial\Omega_f \cap \mathbb{D}$ con $|z_1|=d_1 \implies f(z_1) \not\in \Omega \implies |f(z_1)| \ge d$. Allora per il lemma di Schwarz abbiamo $d \le |f(z_1)|<|z_1|=d_1$.
\end{proof}

\begin{thm}
  (Osgood, Koebe) Sia $\Omega \subset \subset \mathbb{C}$ un dominio limitato, $z_0 \in \Omega$. Allora esiste un unico rivestimento olomorfo $f_0:\mathbb{D} \longrightarrow \Omega$ t.c. $f_0(0)=z_0$ e $f_0'(0) \in \mathbb{R}, f'(0)>0$.
\end{thm}

\begin{proof}
  Possiamo supporre $z_0=0 \in \Omega$ e $\Omega \subset \subset \mathbb{D}$. Sia $\mathcal{F} \subset \text{Hol}(\mathbb{D}, \mathbb{D})$ t.c.
  $\mathcal{F}=\{f \in \text{Hol}(\mathbb{D}, \mathbb{D}) \mid f(0)=0, f'(0) \in \mathbb{R}, f'(0)>0; \Omega \subseteq f(\mathbb{D});$ se $\Omega_f$ è la componente connessa di $f^{-1}(\Omega)$ contenente $0$ allora $f\restrict{\Omega_f}:\Omega_f \longrightarrow \Omega$ è un rivestimento$\}$.
  Se esiste $f_0 \in \mathcal{F}$ con $\Omega_{f_0}=\mathbb{D}$ ci resta da dimostrare solo l'unicità. Poniamo per ogni $f \in \mathcal{F}$ $\displaystyle d_f=\inf_{z \not\in \Omega_f} |z|=\min_{z \in \partial\Omega_f} |z| \le 1$. Abbiamo $d_f=1 \iff \Omega_f=\mathbb{D}$.
  Dobbiamo trovare $f_0 \in \mathcal{F}$ con $d_{f_0}=1$. Sia $\displaystyle d=\sup_{f \in \mathcal{F}} d_f \le 1$. Sia $\{f_n\} \subset \mathcal{F}$ t.c. $d_{f_n} \longrightarrow d$.
  Per il teorema di Montel possiamo supporre che $f_n \longrightarrow f_0 \in \text{Hol}(\mathbb{D}, \mathbb{C})$ con immaggine in $\overline {\mathbb{D}}$. Vogliamo $f_0 \in \mathcal{F}$. Chiaramente $f_0(0)=0, f_0'(0) \in \mathbb{R}, f'(0) \ge 0$.
  \begin{enumerate}
    \item $f_0$ non è costante e $f'(0)>0$: sia $r>0$ t.c. $\mathbb{D}_r \subset \subset \Omega$. Siccome $f_n:\Omega_{f_n} \longrightarrow \Omega$ è un rivestimento e $0 \in \Omega_{f_n}$, esiste un unico $h_n:\mathbb{D}_r \longrightarrow \Omega_{f_n}$ olomorfa t.c. $f_n \circ h_n=\id_{\mathbb{D}_r}$ e $h_r(0)=0$.
    Sempre per il teorema di Montel, a meno di sottosuccessioni possiamo supporre $h_n \longrightarrow h_0 \in \text{Hol}(\mathbb{D}_r, \overline{\mathbb{D}})$ t.c. $f_0 \circ h_0=\id_{\mathbb{D}_r}$.
    Dunque $h_0$ è iniettiva, quindi per il teorema dell'applicazione aperta è aperta, perciò $h_0(\mathbb{D}_r) \subseteq \mathbb{D}$, e $f_0$ non è costante, dunque per il primo teorema di Hurwitz $f_0(\mathbb{D}) \subseteq \mathbb{D}$. $1=\id_{\mathbb{D}_r}'(0)=(f_0 \circ h_0)'(0)=f_0'(h_0(0))\cdot h_0'(0)=f_0'(0) \cdot h_0'(0) \implies f_0'(0) \not=0 \implies f_0'(0)>0$.
    \item $f_0(\Omega_{f_0})=\Omega$ dove $\Omega_{f_0}$ è la componente connessa di $f_0^{-1}(\Omega)$ contenente $0$. Sia infatti $z_0 \in \Omega$ e sia $\gamma$ una curva in $\Omega$ da $0$ a $z_0$.
    Ricopriamo $\gamma$ con dischi $D_0=\mathbb{D}_r, D_1, \dots, D_k$ con $D_j \cap D_{j+1} \not=\emptyset$ e $z_0 \in D_k$. Sia $h_{n, 0}$ l'inversa di $f_n$ su $D_0$ t.c. $h_{n,0}=0$.
    Per ogni $j$ sia $h_{n, j}$ l'inversa di $f_n$ su $D_j$ che coincide con $h_{n, j-1}$ su $D_{j-1} \cap D_j$. Per Montel, a meno di sottosuccessioni $h_{n, 0} \longrightarrow h_{0, 0} \in \text{Hol}(D_0, \mathbb{D})$.
    Per il teorema di Vitali, $h_{n, 1} \longrightarrow h_{0, 1} \in \text{Hol}(D_1, \mathbb{D})$ e, per induzione, $h_{n, k} \longrightarrow h_{0, k} \in \text{Hol}(D_k, \mathbb{D})$ con $f_0 \circ h_{0, k}=\id_{D_k} \implies f_0(h_{0, k}(z_0))=z_0 \implies z_0 \in f_0(\mathbb{D})$.
    In realtà $h_{0, k}(z_0) \in \Omega_{f_0}$ perché è immagine della curva ottenuta con $h_{0, j} \circ \gamma$ che parte da $0$ e quindi è contenuta nella componente connessa di $\Omega$ contenete $0$.
    \item $f_0:\Omega_{f_0} \longrightarrow \Omega$ è un rivestimento.
  \end{enumerate}

  Per l'unicità, si veda l'esercizio \ref{un_biolo}.
\end{proof}

\begin{exc} \label{un_biolo}
  Se $f_1, f_2:\mathbb{D} \longrightarrow \Omega$ sono rivestimenti con $f_1(0)=f_1(0)$ e $f_1'(0), f_2'(0)>0$, allora $f_1 \equiv f_2$. Hint: dato che sono rivestimenti, si sfruttano $h$ e $h^{-1}$ date dal seguente diagramma commutativo:
  \begin{center}
    \begin{tikzcd}
      & \mathbb{D} \arrow[dl, "h"', shift right] \arrow[d, "f_1"]\\
      \mathbb{D} \arrow[ru, "h^{-1}" right, shift right] \arrow[r, "f_2"'] & \Omega
    \end{tikzcd}
  \end{center}
\end{exc}

\begin{thm}
  (Riemann) Se $\Omega \subset \mathbb{C}$ è un dominio semplicemente connesso con $\Omega \not=\mathbb{C}$, allora $\Omega$ è biolomorfo a $\mathbb{D}$.
\end{thm}

\begin{proof}
  In settimana.
\end{proof}

Per il teorema di Liouville abbiamo che $\mathbb{C}$ non è biolomorfo a $\mathbb{D}$. Dato che $\widehat{\mathbb{C}}$ è compatta, abbiamo che non è biolomorfa né a $\mathbb{D}$ né a $\mathbb{C}$.

\begin{thm}
  (Uniformizzazione di Riemann) Se $X$ è una superficie di Riemann semplicemente connessa, allora $X$ è biolomorfo a $\widehat{\mathbb{C}}$, $\mathbb{C}$ o $\mathbb{D}$. Più in generale, se $X$ è una superficie di Riemann qualsiasi e $\pi:\widetilde{X} \longrightarrow X$ è un rivestimento universale, allora $\widetilde{X}$ è una superficie di Riemann e
  \begin{nlist}
    \item se $\widetilde{X}$ è biolomorfo a $\widehat{\mathbb{C}}$, allora anche $X$ è biolomorfo a $\widehat{\mathbb{C}}$ (caso ellittico);
    \item se $\widetilde{X}$ è biolomorfo a $\mathbb{C}$, allora $X$ è biolomorfo a $\mathbb{C}$, $\mathbb{C}^*$, oppure un toro $T_{\tau}=\faktor{\mathbb{C}}{(\mathbb{Z}+\tau \mathbb{Z})}$ con $\mathfrak{Im}\tau>0$ (caso parabolico);
    \item in tutti gli altri casi $\widetilde{X}$ è biolomorfo a $\mathbb{D}$ (caso iperbolico).
  \end{nlist}
\end{thm}

Quindi, se $\Omega \subset\subset \mathbb{C}$ è limitato e semplicemente connesso, abbiamo per il teorema di Riemann che esiste $f:\mathbb{D} \longrightarrow \Omega$ biolomorfismo. Domanda: possiamo estendere $f$ a un omeomorfismo da $\overline{\mathbb{D}}$ a $\overline{\Omega}$?

\begin{thm}
  (Carathéodory) Un biolomorfismo $\mathbb{D} \longrightarrow \Omega$ si estende continuo da $\overline{\mathbb{D}}$ a $\overline{\Omega}$ se e solo se $\partial \Omega$ è localmente connesso.
\end{thm}

\begin{cor}
  Si estende a un omeomorfismo se e solo se $\partial\Omega$ è una curva di Jordan (cioè immagine omeomorfa di $S^1$).
\end{cor}

Esiste una condizione su $\partial\Omega$ diversa che è equivalente all'estendibilità di $f^{-1}:\Omega \longrightarrow \mathbb{D}$ al bordo.
