Vediamo ora alcuni risultati e definizioni preliminari, da considerarsi comunque come prerequisiti per altri risultati più interessanti che vedremo più avanti nel corso.

\begin{thm}
  (Weierstrass) Sia $\{f_n\} \subset \mathcal{O}(\Omega)$ t.c. $f_n \rightarrow f \in C^0(\Omega, \mathcal{C})$ uniformemente sui compatti. Allora:
  \begin{nlist}
    \item $f \in \mathcal{O}(\Omega)$;
    \item $f_n' \rightarrow f'$ uniformemente sui compatti.
  \end{nlist}
\end{thm}

\begin{proof}
  \begin{nlist}
    \item Sia $a \in \Omega$, $0<r<d(a, \partial\Omega)$ t.c. $D=D(a, r) \subset \subset \Omega$. $\displaystyle f_n(z)=\frac{1}{2\pi i} \int_{\partial D} \frac{f_n(\zeta)}{\zeta-z} \diff \zeta$ per ogni $z \in D(a, \rho)$ per ogni $0<\rho<r$.
    Allora $\dfrac{1}{|\zeta-z|} \le \dfrac{1}{r-\rho}$ per ogni $z \in D(a, \rho), \zeta \in \partial{D}$.
    Per ogni $\displaystyle z \in D(a, \rho), f(z)=\lim_{n \rightarrow +\infty} \frac{1}{2\pi i} \int_{\partial D} \frac{f_n(\zeta)}{\zeta-z} \diff \zeta$.
    Adesso, per uniforme convergenza e uniforme limitatezza si può portare il limite dentro, perciò $\displaystyle f(z)=\frac{1}{2\pi i}\int_{\partial D} \frac{f(\zeta)}{\zeta-z} \diff \zeta$, ma questo, per il teorema di Cauchy-Goursat+Morera, implica $f \in \mathcal{O}(\Omega)$.
    \item $\displaystyle f_n'(z)=\frac{1}{2\pi i} \int_{\partial D} \frac{f_n'(\zeta)}{(\zeta-z)^2}\diff\zeta \rightarrow \frac{1}{2\pi i} \int_{\partial D} \frac{f(\zeta)}{\zeta-z} \diff\zeta=f'(z)$.
    $f_n' \rightarrow f'$ uniformemente sui dischi e ogni compatto è coperto da un numero finito di dischi $\implies$ $f_n' \rightarrow f'$ uniformemente sui compatti.
  \end{nlist}
\end{proof}

\begin{thm}
  (Montel) $\Omega \subseteq \mathbb{C}$ aperto, $\mathcal{F} \subseteq \mathcal{O}(\Omega)$ t.c. per ogni $K \subset \subset C$ compatto esiste $M_K>0$ t.c. $\|f\|_K \le M_K$ per ogni $f \in \mathcal{F}$ (si diche che $\mathcal{F}$ è \textit{uniformemente limitata sui compatti}).
  Allora $\mathcal{F}$ è relativamente compatta in $\mathcal{O}(\Omega)$.
\end{thm}

\begin{proof}
  Basta vedere che ogni successione $\{f_n\} \subseteq \mathcal{F}$ ha una sottosuccessione convergente (segue dal fatto che, nelle ipotesi del teorema di Montel, la topologia compatta aperta è metrizzabile). \\
  Dati $a \in \Omega, 0<r<d(a, \partial \Omega), f \in \mathcal{O}(\Omega)$, sia $c_n(f)=\dfrac{f^{(n)}(a)}{n!}$, allora $\displaystyle f(z)=\sum_{n=0}^{+\infty} c_n(f)(z-a)^n$ in $\overline{D(a, r)}$.
  Inoltre, se $\|f\|_{\overline{D(a, r)}} \le M$, allora per le disuguaglianze di Cauchy $|c_n(f)| \le \dfrac{M}{r^n}$ per ogni $n \ge 0$.
  Sia $\{f_n\} \subseteq \mathcal{F}$. Per ipotesi, esiste $M$ t.c. $\|f_n\|_{\overline{D(a, r)}} \le M$ per ogni $n$ $\implies$ $|c_0(f_n)| \le M$ per ogni $n$ $\implies$ esiste una sottosuccessione $c_0(f_{n_j^{(0)}})$ che tende a $c_0 \in \mathbb{C}$.
  Per  induzione, da $\{f_{n_j^{(k-1)}}\}$ possiamo estrarre una sottosuccessione $\{f_{n_j^{(k)}}\}$ t.c. $c_k(f_{n_j^{(k)}}) \rightarrow c_k \in \mathbb{C}$. Consideriamo $\{f_{n_j^{(j)}}\}$, allora $c_k(f_{n_j^{(j)}}) \rightarrow c_k \in \mathbb{C}$ per ogni $k$.
  Sia $f_{\nu_j}=f_{n_j^{(j)}}$. Poniamo $D_a=\overline{D(a, r/2)}$ e sia $z \in D_a$. Vogliamo $\displaystyle f_{\nu_j} \rightarrow f(z)=\sum_{n=0}^{+\infty} c_n(z-a)^n$ in $D_a$. Basta vedere che $f_{\nu_j}$ è di Cauchy uniformemente in $D_a$.
  $\displaystyle |f_{\nu_h}(z)-f_{\nu_k}(z)| \le \sum_{n=0}^{+\infty} |c_n(f_{\nu_h})-c_n(f_{\nu_k})||z-a|^n=\sum_{n=0}^N |c_n(f_{\nu_h})-c_n(f_{\nu_k})||z-a|^n+\sum_{n>N} |c_n(f_{\nu_h})-c_n(f_{\nu_k})||z-a|^n$.
  Sappiamo che $|c_n(f_{\nu_k})| \le \dfrac{M}{r^n}$ e $z \in D_a \implies |z-a| \le \dfrac{r}{2}$.
  Allora $\displaystyle \sum_{n>N} |c_n(f_{\nu_h})-c_n(f_{\nu_k})||z-a|^n \le \sum_{n>N} \frac{2M}{r^n}\left(\frac{r}{2}\right)^n=\frac{M}{2^{N-1}}$.
  $\displaystyle \sum_{n=0}^{+\infty} |c_n(f_{\nu_h})-c_n(f_{\nu_k})||z-a|^n \le \sum_{n=0}^{+\infty} |c_n(f_{\nu_h})-c_n(f_{\nu_k})|\left(\dfrac{r}{2}\right)^n$.
  Dato $\epsilon>0$, scegliamo $N>>1$ t.c. $\dfrac{M}{2^{N-1}}<\epsilon/2$ e $n_0$ t.c. per ogni $h, k \ge n_0, |c_n(f_{\nu_h})-c_n(f_{\nu_k})|\left(\dfrac{r}{2}\right)^n<\dfrac{\epsilon}{2(N+1)}$
  (possiamo farlo, una volta fissato $N$, perché gli $n$ tra $0$ e $N$ sono in numero finito e le successioni $c_n(f_{\nu_j})$ convergono, dunque si sceglie un indice per ogni successione e si prende come $n_0$ il massimo di questi indici).
  Mettendo insieme le disuguaglianze si ha che per ogni $\epsilon>0$ esiste $n_0$ t.c. per ogni $h, k \ge n_0$ e per ogni $z \in D_a$, $|f_{\nu_k}(z)-f_{\nu_k}(z)|<\epsilon$, dunque la sottosuccessione $f_{\nu_j}$ è di Cauchy e converge uniformemente su $D_a$.
  Deve convergere a $f$ perché, per il teorema di Weierstrass, le derivate convergono al valore della derivata limite, e questo ci dice che i coefficienti della serie della funzione limite sono proprio quelli di $f$. \\
  $\Omega$ è a base numerabile, dunque possiamo estrarre un sottoricoprimento numerabile da $\{D_a | a \in \Omega\}$. Sia dunque $\{a_j\} \subseteq \Omega$ t.c. $\displaystyle \bigcup_j D_{a_j}=\Omega$. Per quanto dimostrato finora, possiamo estrarre da $\{f_n\}$ una sottosuccessione $\{f_{n_j^{(0)}}\}$ convergente uniformemente in $D_{a_0}$.
  Per induzione, da $\{f_{n_j^{(k-1)}}\}$ estraiamo una sottosuccessione $\{f_{n_j^{(k)}}\}$ convergente uniformemente in $D_{a_0} \cup \dots \cup D_{a_k}$. Prendiamo $\{f_{n_j^{(j)}}\}$ che converge uniformemente in ogni $D_{a_k}$.
  Adesso, ogni compatto è coperto da un numero finito di $D_{a_k}$, quindi (scegliendo per ogni $\epsilon$ il massimo degli indici t.c. le cose che vogliamo valgono in quei $D_{a_k}$) $\{f_{n_j^{(j)}}\}$ converge uniformemente sui compatti.
\end{proof}

\begin{thm}
  (Vitali) $\Omega \subseteq \mathbb{C}$ dominio, $A \subseteq \Omega$ con almeno un punto di accumulazione in $\Omega$. Sia $\{f_n\} \subset \mathcal{O}(\Omega)$ uniformemente limitata sui compatti. Supponiamo che, per ogni $a \in A$, $\{f_n(a)\}$ converge (cioe $f_n$ converge puntualmente). Allora esiste $f \in \mathcal{O}(\Omega)$ t.c. $f_n \rightarrow f$ uniformemente sui compatti di $\Omega$.
\end{thm}

\begin{proof}
  Facciamola per assurdo. Supponiamo che esistono $K \subset\subset \Omega, \\ \{n_k\}, \{m_k\} \subset \mathbb{N}, \{z_k\} \subset K, \delta>0$ t.c. $|f_{n_k}(z_k)-f_{m_k}(z_k)| \ge \delta$. A meno di sottosuccessioni, $z_k \rightarrow z_0 \in K$.
  Per il teorema di Montel, a meno di sottosuccessioni $f_{n_k} \rightarrow g_1 \in \Omega$ e $f_{m_k} \rightarrow g_2 \in \Omega$ con $|g_1(z_0)-g_2(z_0)| \ge \delta$ (passando al limite). Per ipotesi,  $g_1(a)=g_2(a)$ per ogni $a \in A$. Per il principio di identità, $g_1 \equiv g_2$, assurdo.
\end{proof}

\begin{thm}
  (Sviluppo di Laurent) Siano $0 \le r_1 < r_2 \le +\infty, A(r_1, r_2):=\{z \in \mathbb{C} | r_1 < |z| < r_2 \}$. Sia $f \in \mathcal{O}(A(r_1, r_2))$, allora $\displaystyle f(z)=\sum_{n=-\infty}^{+\infty} c_nz^n$ e converge uniformemente e assolutamente sui compatti di $A(r_1, r_2)$.
  In particolare, se $\Omega \subseteq \mathbb{C}$ aperto, $a \in \Omega$ e $\displaystyle f \in \mathcal{O}(\Omega \setminus \{a\}), f(z)=\sum_{n=-\infty}^{+\infty} c_n(z-a)^n$ in $\{0<|z-a|<r\} \subset \Omega$.
\end{thm}

\begin{cor}
  (Teorema di estensione di Riemann) $f \in \mathcal{O}(\Omega \setminus \{a\})$ si estende olomorficamente ad $a$ $\Leftrightarrow$ $\displaystyle \lim_{z \rightarrow a} (z-a)f(z)=0$.
\end{cor}

\begin{proof}
  Per lo sviluppo di Laurent, $\displaystyle (z-a)f(z)=\sum_{n=-\infty}^{+\infty} c_n(z-a)^{n+1} \rightarrow 0 \Leftrightarrow c_n=0$ per ogni $n \le -1$.
\end{proof}

\begin{thm} \label{biolo}
  \begin{nlist}
    \item $f \in \text{Hol}(\Omega, \Omega_1)$ biettiva $\implies$ $f^{-1}$ è olomorfa e $f'$ non si annulla mai;
    \item $f \in \mathcal{O}(\Omega)$ t.c. $f'(z_0) \not=0$ $\implies$ $f$ è iniettiva vicino a $z_0$.
  \end{nlist}
\end{thm}

\begin{proof}
  \begin{nlist}
    \item Per il teorema dell'applicazione aperta, $f$ è aperta $\implies$ $f$ omeomorfismo. $g=f^{-1}$. Sia $w_0 \in \Omega_1$ t.c. $f'(g(w_0)) \not=0$. Allora
    $$ \frac{g(w)-g(w_0)}{w-w_0}=\frac{1}{\frac{w-w_0}{g(w)-g(w_0)}}=\frac{1}{\frac{f(g(w))-f(g(w_0))}{g(w)-g(w_0)}}=\frac{1}{f'(g(w_0))}. $$
    Quindi $g$ è olomorfa in $\Omega_1 \setminus f(\{f'=0\})$. Per il corollario \ref{olo_discr} $\{f'=0\}$ è discreto in $\Omega$. $f$ omeomorfismo $\implies$ $f(\{f'=0\})$ discreto in $\Omega_1$.
    Ma $g$ è continua (quindi localmente limitata) in $\Omega_1$, dunque per il teorema di estensione di Riemann $g \in \mathcal{O}(\Omega_1)$. $(f' \circ g)g' \equiv 1$ su $\Omega_1 \setminus f(\{f'=0\})$ $\implies$ vale su $\Omega_1$ $\implies$ $f'\circ g \not=0$ sempre.
    \item Possiamo supporre $z_0=0$. $\displaystyle f(z)=\sum_{n=0}^{+\infty} c_nz^n$. Per ipotesi, $c_1 \not=0$. \\
    $\displaystyle f(z)-f(w)=c_1(z-w)+(z-w)\sum_{n=2}^{+\infty} c_n\sum_{k=1}^n w^{k-1}z^{n-k}$. \\
    $\displaystyle |f(z)-f(w)| \ge |c_1||z-w|-|z-w|\sum_{n=2}^{+\infty} |c_n|\sum_{k=1}^n |w|^{k-1}|z|^{n-k}$. \\
    Prendiamo $z, w \in D(0, r)$, allora \\
    $\displaystyle |c_1||z-w|-|z-w|\sum_{n=2}^{+\infty} |c_n|\sum_{k=1}^n |w|^{k-1}|z|^{n-k} \ge \\
    \ge |c_1||z-w|-|z-w|\sum_{n=2}^{+\infty} |c_n|nr^{n-1}=(|c_1|-\sum_{n=2}^{+\infty} |c_n|nr^{n-1})|z-w|$.
    Scegliamo $r$ t.c. $\displaystyle \sum_{n=2}^{+\infty} |c_n|nr^{n-1} \le \frac{|c_1|}{2}$, allora $\displaystyle (|c_1|-\sum_{n=2}^{+\infty} |c_n|nr^{n-1})|z-w| \ge \frac{|c_1|}{2}|z-w|$.
    Dato che $c_1 \not=0$, si ha quindi (concatenando le disuguaglianze) che $z \not=w \implies |f(z)-f(w)| \ge \dfrac{|c_1|}{2}|z-w|>0 \implies f(z) \not= f(w)$.
  \end{nlist}
\end{proof}

\begin{defn}
  Se $f: \Omega_1 \rightarrow \Omega_2$ è olomorfa e biettiva (quindi con inversa olomorfa per il teorema \ref{biolo}) si chiama \textsc{biolomorfismo}.
\end{defn}

\begin{defn}
  $f:\Omega_1 \rightarrow \mathbb{C}$ è un \textsc{biolomorfismo locale} se ogni $a \in \Omega_1$ ha un intorno $U \ni a$ t.c. $f\restrict{U}:U \rightarrow f(U)$ è un biolomorfismo.
\end{defn}

Per il teorema \ref{biolo}, $f$ è un biolomorfismo locale se e solo se $f'$ non si annulla mai.

\begin{defn}
  Sia $\displaystyle f(z)=\sum_{n=-\infty}^{+\infty} c_n(z-a)^n$ in $D^*=D(a, r) \setminus \{a\}$. $ord_a(f):=\inf\{n \in \mathbb{Z} | c_n \not=0\}$ è detto \textsc{ordine di $f$ in $a$}. $ord_a(f) \ge 0 \Leftrightarrow f$ è olomorfa in $a$.
  Se $0>ord_a(f)>-\infty$ diremo che $a$ è un \textsc{polo} di $f$. Se $ord_a(f)=-\infty$ $a$ è una \textsc{singolarità essenziale}.
\end{defn}

\begin{thm}
  (Casorati-Weierstrass) Se $a$ è una singolarità essenziale, $f(D^*)$ è denso in $\mathbb{C}$.
\end{thm}

\begin{defn}
  $c_{-1}=:res_f(a)$ è detto \textsc{residuo di $f$ in $a$}.
\end{defn}

\begin{oss} \label{int_res}
  $\gamma(t)=a+\rho e^{2\pi i t}, 0<\rho<r$. \\
  $\displaystyle \frac{1}{2\pi i} \int_{\gamma} f \diff z=\frac{1}{2\pi i} \sum_{n=-\infty}^{+\infty} c_n \int_{\gamma} (z-a)^n \diff z= \\ \frac{1}{2\pi i} \sum_{n=-\infty}^{+\infty} c_n \int_0^1 \rho^ne^{2\pi i n t} \rho 2 \pi i e^{2\pi i t} \diff t=\sum_{n=-\infty}^{+\infty} c_n \rho^{n+1} \int_0^1 e^{2\pi i (n+1)t} \diff t=c_{-1}$.
\end{oss}

\begin{prop}
  $\Omega \subseteq \mathbb{C}$ aperto, $E \subset \Omega$ discreto e chiuso in $\Omega$, $D \subset \subset \Omega$ disco chiuso t.c. $E \cap \partial D=\emptyset$, $f\in \mathcal{O}(\Omega \setminus E)$.
  Allora $\displaystyle \frac{1}{2\pi i}\int_{\partial D} f \diff z=\sum_{a \in D \cap E} res_f(a)$.
\end{prop}

\begin{proof}
  Traccia: si dimostra che $E \cap D$ è finito e si applica una versione leggermente più forte del teorema di Cauchy-Goursat+Morera, prendendo per ogni punto di $E$ un dischetto tutto contenuto in $D$ che lo isoli dagli altri e considerando la regione $D$ meno quei dischetti. Il bordo di questa regione è considerato il bordo di $D$ meno il bordo dei dischetti. Questo bordo, a meno di aggiungere dei tratti lineari che uniscono una circonferenza all'altra (che quindi verranno percorsi in entrambi i sensi nell'integrale e non daranno contributo), è percorribile con un solo cammino omotopo al cammino costante in $\Omega \setminus E$, il cui integrale fa $0$ per la versione forte del teorema di C-G+M, dunque l'integrale sul bordo di $D$ meno l'integrale sul bordo dei dischetti (occhio al verso di percorrenza di uno e degli altri!) deve essere uguale a $0$. Per l'osservazione \ref{int_res} si ha la tesi.
\end{proof}

\begin{oss} \label{wn}
  $\gamma:[0, 1] \rightarrow \mathbb{C}$ chiusa ($\gamma(0)=\gamma(1))$, $a \not\in \gamma([0, 1])$. $p_s: \mathbb{C} \rightarrow \mathbb{C} \setminus \{a\}$, $p_a(z)=a+e^z$ è un rivestimento.
  \begin{center}
    \begin{tikzcd}
            & \mathbb{C} \arrow[d, "p_a"]\\
            \left[0,1\right] \arrow[ru, "\tilde{\gamma}"] \arrow[r, "\gamma"'] & \mathbb{C} \setminus \{a\}
     \end{tikzcd}
  \end{center}
  Sia $\tilde{\gamma}$ un sollevamento di $\gamma$ rispetto a $p_a$, $p_a(\tilde{\gamma}(1))=\gamma(1)=\gamma(0)=p_a(\tilde{\gamma}(0)) \iff e^{\tilde{\gamma}(1)}=e^{\tilde{\gamma}(0)} \iff \tilde{\gamma}(1)-\tilde{\gamma}(0) \in 2\pi i \mathbb{Z}$.
\end{oss}

\begin{defn}
  L'\textsc{indice di avvolgimento $\gamma$ rispetto ad $a$} (\textit{winding number} in inglese) è dato dall'osservazione \ref{wn}: $n(\gamma, a):=\dfrac{1}{2\pi i}(\tilde{\gamma}(1)-\tilde{\gamma}(0)) \in \mathbb{Z}$.
\end{defn}

\begin{thm}
  \begin{nlist}
    \item $n(\gamma, a)$ dipende solo da $a$ e da $\gamma$ e non dal sollevamento scelto;
    \item $n(\gamma, a) \in \mathbb{Z}$;
    \item $\displaystyle n(\gamma, a)=\frac{1}{2\pi i}\int_{\gamma} \frac{1}{z-a} \diff z$;
    \item $a \mapsto n(\gamma, a)$ è costante sulle componente connesse di $\mathbb{C} \setminus \gamma([0, 1])$. In particolare $n(\gamma, a)=0$ sulla componente connessa illimitata di $\mathbb{C} \setminus \gamma([0, 1])$;
    \item $\gamma(t)=a_0+re^{2\pi i t} \implies n(a, \gamma)=1$ per ogni $a \in D(a_0, r)$;
    \item $\gamma_1$ e $\gamma_2$ chiuse con $\gamma_1(0)=\gamma_2(0)=p_0$ omotope (tramite omotopia che fissa il punto base $p_0$) e $a \not\in \gamma_1([0, 1]) \cup \gamma_2([0, 1])$, se l'omotopia è in $\mathbb{C} \setminus \{a\}$ allora $n(\gamma_1, a)=n(\gamma_2, a)$.
  \end{nlist}
\end{thm}

\begin{thm}
  (Teorema dei residui) $\Omega \subseteq \mathbb{C}$ aperto, $E \subset \Omega$ discreto e chiuso in $\Omega$, $\gamma$ curva chiusa in $\Omega \setminus E$ omotopa a una costante in $\Omega$.
  Allora per ogni $f \in \mathcal{O}(\Omega \setminus E)$ $\displaystyle \frac{1}{2\pi i} \int_{\gamma} f \diff z=\sum_{a \in E} res_f(a) \cdot n(\gamma, a)$.
\end{thm}

\begin{defn}
  $\Omega \subseteq \mathbb{C}$, $f$ è \textsc{meromorfa} in $\Omega$ se esiste $E \subset \Omega$ discreto e chiuso in $\Omega$ t.c. $f \in \mathcal{O}(\Omega \setminus E)$ e nessun punto di $E$ è una singolarità essenziale. Scriveremo che $f \in \mathcal{M}(\Omega)$.
\end{defn}

\begin{prop}
  \begin{nlist}
    \item $f \in \mathcal{O}(\Omega \setminus E)$ è meromorfa $\iff$ localmente è quoziente di due funzioni olomorfe;
    \item $f \in \mathcal{O}(\Omega \setminus E)$ è meromorfa $\iff$ per ogni $a \in E$ o $|f|$ è limitato vicino ad $a$ o $\displaystyle \lim_{z \rightarrow a} |f(z)|=+\infty$.
  \end{nlist}
\end{prop}

\begin{proof}
  \begin{nlist}
    \item ($\implies$) Se $a \in \Omega \setminus E$ banalmente $f=\dfrac{f}{1}$ vicino ad $a$. \\
    Se $a \in E$, $\displaystyle f(z)=\sum_{n \ge n_0} c_n(z-a)^n=(z-a)^{n_0}(c_{n_0}+h(z))$, $h$ olomorfa vicino ad $a$. Se $n_0<0$, $f(z)=\dfrac{c_{n_0}+h(z)}{(z-a)^{-n_0}}$.

    ($\Leftarrow$) Se $\displaystyle f(z)=\frac{h_1(z)}{h_2(z)}=\frac{\sum_{n \ge n_1} b_n(z-a)^n}{\sum_{m \ge n_2} c_m(z-a)^m}=(z-a)^{n_1-n_2}k(z)$, $k$ olomorfa vicino ad $a$.
    \item Per Casorati-Weierstrass, $a \in E$ è singolarità essenziale $\iff$ $\displaystyle \lim_{z \rightarrow a} |f(z)|$ non esiste. Per lo stesso motivo, è un polo $\iff$ $\displaystyle \lim_{z \rightarrow a} |f(z)|=+\infty$.
  \end{nlist}
\end{proof}

\begin{thm}
  (Principio dell'argomento) $\Omega \subseteq \mathbb{C}$, $f \in \mathcal{M}(\Omega)$. $Z_f:=\{\text{zeri di } f\}, P_f:=\{\text{poli di } f\}$. $\gamma$ curva chiusa in $\Omega \setminus (Z_f \cup P_f)$ omotopa a una costante in $\Omega$.
  Allora $\displaystyle \sum_{a \in Z_f \cup P_f} n(\gamma, a) \cdot ord_a(f)=\frac{1}{2\pi i}\int_{\gamma} \frac{f'}{f} \diff z$.
\end{thm}

\begin{proof}
  $ord_a(f)=res_{f'/f}(a)$. Infatti $f(z)=(z-a)^mh(z)$ con $m=ord_a(f)$, $h(a) \not=0$ e $h$ olomorfa. $f'(z)=m(z-a)^{m-1}h(z)+(z-a)^mh'(z)$. Allora $\dfrac{f'}{f}=\dfrac{m}{z-a}+\dfrac{h'(z)}{h(z)}$ e $\dfrac{h'}{h}$ è olomorfa $\implies$ $res_{f'/f}(a)=m=ord_a(f)$. La tesi segue allora dal teorema dei residui.
\end{proof}

\begin{prop}
  (Versione semplice del teorema di Rouché) $\Omega \subseteq \mathbb{C}$, $f, g \in \mathcal{O}(\Omega)$, $D$ disco con $\overline{D} \subset \Omega$. Supponiamo che $|f-g|<|g|$ su $\partial D$ (questo implica anche che non si annullano mai su $\partial D$). Allora $f$ e $g$ hanno lo stesso numero di zeri (contati con molteplicità) su $D$.
\end{prop}

\begin{proof}
  Per $t \in [0, 1]$ poniamo $f_t=g+t(f-g)$ ($f_0=g$, $f_1=f$). Se $z \in \partial D$, $0<|g(z)|-|f(z)-g(z)| \le |g(z)|-t|f(z)-g(z)| \le |f_t(z)|$. Sia $\displaystyle a_t=\sum_{a \in \overline{D}} ord_a(f_t)=\text{numero di zeri di } f_t \text{ in } \overline{D}$.
  Non ci sono poli, dunque $a_t \in \mathbb{N}$, quindi per il principio dell'argomento $\displaystyle a_t=\frac{1}{2\pi i} \int_{\partial D} \frac{f_t'}{f_t} \diff z=\frac{1}{2\pi i} \int_{\partial D} \frac{g'+t(f'-g')}{g+t(f-g)} \diff z$,
  che dipende con continuità da $t$ $\implies$ $a_t$ è costante (è a valori in $\mathbb{N}$) $\implies$ $a_0=a_1$ come voluto.
\end{proof}

\begin{cor}
  (Teorema di Ritt) Sia $h \in \mathcal{O}(\mathbb{D})$ ($\mathbb{D}:=\{|z|<1\}$) t.c. $h(\mathbb{D}) \subset \subset \mathbb{D}$. Allora $h$ ha un punto fisso.
\end{cor}

\begin{proof}
  Esiste $0<r<1$ t.c. $|h(z)|<r$ per ogni $z \in \mathbb{D}$. Sia $\mathbb{D}_r:=\{|z|<r\}$. Su $\partial \mathbb{D}_r$, $|z-(z-h(z))|=|h(z)|<r=|z|$.
  Per il teorema di Rouché su $g(z)=z, f(z)=z-h(z)$, $g$ e $f$ hanno lo stesso numero di zeri in $\mathbb{D}$, ma $g$ ha un unico zero $\implies$ $f=\id_{\mathbb{D}}-h$ ha un unico zero $z_0$ $\implies$ $h(z_0)=z_0$.
\end{proof}
