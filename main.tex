\documentclass{article}
\usepackage{mstyle}
\usepackage{pgfplots}
\usetikzlibrary{intersections, pgfplots.fillbetween}

\title{Teoremi di rigidità per funzioni olomorfe nel disco}
\date{}
\author{Candidato: Marco Vergamini \qquad Relatore: Prof. Marco Abate}

\begin{document}
\maketitle
\newpage
\tableofcontents
\newpage


\section*{Introduzione}
\addcontentsline{toc}{section}{Introduzione}
Da scrivere alla fine


\newpage

\section{Prerequisiti}

\subsection{Lemma di Schwarz e distanza di Poincaré}
\begin{defn}
  Sia $\Omega \subset \mathbb{C}$ un aperto. Una funzione $f:\Omega \longrightarrow \mathbb{C}$ si dice \textit{olomorfa} in $\Omega$ se è derivabile in senso complesso per ogni $z \in \Omega$.
\end{defn}

\begin{defn}
  Se $f: \Omega \longrightarrow \Omega$ olomorfa è biettiva, allora si può dimostrare che anche $f^{-1}$ è olomorfa. In tal caso $f$ è detta \textit{automorfismo} (in senso olomorfo di $\Omega$).
\end{defn}

Com'è noto, la condizione di olomorfia per funzioni a valori complessi è molto più forte della derivabilità in senso reale (in particolare, è equivalente all'analiticità). Fra i vari risultati che si possono dimostrare per le funzioni olomorfe, ci interessa studiare il lemma di Schwarz-Pick. \\

Notazione: indichiamo il disco unitario con $\mathbb{D}=\{z \in \mathbb{C} \mid |z|<1\}$.

\begin{lm}
  (Schwarz) Sia $f:\mathbb{D} \longrightarrow \mathbb{D}$ una funzione olomorfa t.c. $f(0)=0$. Allora per ogni $z \in \mathbb{D}$ $|f(z)| \le |z|$ e $|f'(0)| \le 1$; inoltre, se vale l'uguale nella prima per $z \not=0$ oppure nella seconda allora $f(z)=e^{i\theta}z, \theta \in \mathbb{R}$.
\end{lm}

\begin{lm}
  (Schwarz-Pick) Sia $f:\mathbb{D} \longrightarrow \mathbb{D}$ una funzione olomorfa.
  Allora per ogni $z, w \in \mathbb{D}$
  $$\left|\frac{f(z)-f(w)}{1-\overline{f(w)}f(z)}\right| \le \left|\frac{z-w}{1-\bar{w}z}\right|, \qquad \frac{|f'(z)|}{1-|f(z)|^2} \le \frac{1}{1-|z|^2}.$$
  Inoltre se vale l'uguale nella prima per $z_0, w_0$ con $z_0 \not=w_0$ o nella seconda per $z_0$ allora $f$ è un automorfismo e vale l'uguale sempre.
\end{lm}

Il lemma di Schwarz-Pick può essere riformulato usando due funzioni di due variabili sul disco, una delle quali è nota come distanza iperbolica (in realtà, anche l'altra è una distanza, ma questo non lo useremo). Con queste funzioni dimostreremo una serie di disuguaglianze che ci permetteranno di dimostrare la disuguaglianza di Golusin, dalla quale seguirà una versione al bordo del lemma.

\begin{defn}
  Dati $z, w \in \mathbb{D}$ poniamo
  $$[z,w]:=\frac{z-w}{1-\bar{w}z}, \qquad p(z,w):=|[z,w]|, \qquad d(z,w):=\log\left(\frac{1+p(z,w)}{1-p(z,w)}\right).$$
\end{defn}

$d$ è ben definita, in quanto $p(z,w)<1$. Infatti, dobbiamo verificare che
  $$\frac{|z-w|}{|1-\bar{w}z|} < 1$$
  $$|z-w|^2 < |1-\bar{w}z|^2$$
  $$|z|^2+|w|^2-\bar{w}-w\bar{z} < 1+|wz|^2-\bar{w}z-w\bar{z}$$
  $$1+|wz|^2-|z|^2-|w|^2 > 0$$
  $$(1-|w|^2)(1-|z|^2) > 0,$$
che è vera perché $z, w \in \mathbb{D}$.

\begin{prop}
  $d$ è una distanza (la sopracitata distanza iperbolica).
\end{prop}

\begin{proof}
  Mostriamo preliminarmente che $p$ è una distanza. In entrambi i casi, l'unica cosa non ovvia da controllare è la disuguaglianza triangolare. Perciò, dati $z_0, z_1, z_2 \in \mathbb{D}$, vogliamo $p(z_1,z_2) \le p(z_1,z_0)+p(z_0,z_2)$. Osserviamo che, per il lemma di Schwarz-Pick, $p$ è invariante applicando automorfismi, perciò supponiamo senza perdita di generalità $z_1=0$ (possiamo farlo perché il gruppo degli automorfismidi $\mathbb{D}$ è transitivo). A questo punto la disuguaglianza da dimostrare diventa
  \marginpar{Come si dimostra? Qui c'è una dim, ma ponendo $z_0=0$: https://mathoverflow.net/questions/21604/nice-proof-of-the-triangle-inequality-for-the-metric-of-the-hyperbolic-plane}
  \begin{align*}
    |z_2| \le |z_0|+\frac{|z_0-z_2|}{|1-\bar{z}_2z_0|}.
  \end{align*}
  (c'è da fare la dimostrazione) \\
  A questo punto, possiamo osservare che $d(z,w) =2\,\text{arctanh}\,(p(z,w))$, perciò... ACHTUNG: LA DIM DEGLI APPUNTI DI ECA SEMBRA ESSERE FALLACE, ARCTANH NON È SUBADDITIVA SUI POSITIVI
  %Altrimenti, si può brutalmente espandere il conto e ricondursi a una disuguaglianza la cui dimostrazione si trova qui: https://www.math3ma.com/blog/the-pseudo-hyperbolic-metric-and-lindelofs-inequality
\end{proof}

\begin{defn}
  Data una funzione $f: \mathbb{D} \longrightarrow \mathbb{D}$, poniamo
  $$f^*(z,w):=\frac{[f(z),f(w)]}{[z,w]}$$
  e
  $$f^h(z):=|f^*(z,z)|:=\left|\lim_{w \longrightarrow z} f^*(z,w)\right|=\left|\lim_{w \longrightarrow z} \frac{[f(z),f(w)]}{[z,w]}\right|=\frac{|f'(z)|(1-|z|^2)}{1-|f(z)|^2}.$$
\end{defn}

\begin{oss}
  \begin{nlist}
    \item la disuguaglianza del lemma di Schwarz-Pick può essere riscritta come $|f^*(z,w)| \le 1$;
    \item  un altro modo di scrivere la disuguaglianza del lemma di Schwarz-Pick è $p(f(z),f(w)) \le p(z,w)$;
    \item $p(0,z)=|z| \implies d(0,z)=d(0,|z|)$;
    \item per definizione, $|f^*(z,w)|=|f^*(w,z)|$ e $f^h(z)$ è reale non negativo.
  \end{nlist}
  Questi risultati verranno usati nelle varie dimostrazioni e verranno esplicitati solo quando ciò che ne segue non è immediato.
\end{oss}

Seguono alcuni risultati noti di analisi complessa noti che verranno usati nelle dimostrazioni.

\begin{thm}
  (formula integrale di Cauchy) Sia $f$ olomorfa sull'aperto $\Omega$ e $D$ un disco chiuso di centro $a$ contenuto in $\Omega$. Allora
  \begin{equation}
    f^{(n)}(a)=\frac{n!}{2\pi i} \int_{\partial D} \frac{f(\zeta)}{(\zeta-a)^{n+1}}\diff\zeta.
  \end{equation}
\end{thm}

\begin{prop}
  Sia $f$ olomorfa sull'aperto $\Omega \setminus\{z_0\}$, con $z_0 \in \Omega$. Allora $f$ si estende a una funzione olomorfa $g$ definita su tutto $\Omega$ se e solo se è limitata in un intorno di $z_0$. In tal caso, $z_0$ è detta \textit{singolarità rimovibile}.
\end{prop}


\subsection{Limiti non tangenziali}
\begin{defn}
  Dati $\alpha \in (0,\pi/2)$ e $\sigma \in \partial\mathbb{D}$, chiamiamo \textit{settore di vertice $\sigma$ e angolo $2\alpha$} l'insieme $S(\sigma,\alpha) \subset \mathbb{D}$ tale che per ogni $z \in S(\sigma,\alpha)$ l'angolo compreso tra la retta congiungente $\sigma$ e $0$ e la retta congiungente $\sigma$ e $z$ è minore di $\alpha$.
  \begin{center}
    \begin{tikzpicture}[line cap=round,line join=round,>=triangle 45,x=2.5cm,y=2.5cm]
      \draw[->,color=black] (-1.13,0) -- (1.15,0);
      \foreach \x in {-1,1}
      \draw[shift={(\x,0)},color=black] (0pt,2pt) -- (0pt,-2pt);
      \draw[->,color=black] (0,-1.11) -- (0,1.12);
      \foreach \y in {-1,1}
      \draw[shift={(0,\y)},color=black] (2pt,0pt) -- (-2pt,0pt);
      \clip(-1.13,-1.11) rectangle (1.15,1.12);
      \draw(0,0) circle (2.5cm);
      \draw[name path=A] (0.49,0.87)-- (1,0);
      \draw[name path=B] (0.49,-0.87)-- (1,0);
      \tikzfillbetween[of=A and B]{blue, opacity=0.25};
      \draw[name path=C,smooth,samples=100,domain=-1:0.491] plot(\x,{sqrt(1-(\x)^2)});
      \draw[name path=D,smooth,samples=100,domain=-1:0.491] plot(\x,{0-sqrt(1-(\x)^2)});
      \tikzfillbetween[of=C and D]{blue, opacity=0.25};
    \end{tikzpicture}

    In blu, $S(1,2\pi/3)$
  \end{center}
\end{defn}

\begin{defn}
  INSERIRE DEFINIZIONE DI LIMITE NON-TANGENZIALE
\end{defn}

La seguente proposizione asserisce che, per $z \longrightarrow 1$ non tangenzialmente, un certo andamento di $f$ può essere tradotto nell'andamento di $f^h$. È questo che ci permetterà di dimostrare il teorema 2.1 di \cite{BK} passando per la versione al bordo del lemma di Schwarz-Pick.

\begin{prop} \label{o^3->o^2}
  Sia $f:\mathbb{D} \longrightarrow \mathbb{D}$ una funzione olomorfa tale che
  \begin{equation} \label{o^3}
    f(z)=1+(z-1)+o((z-1)^3)
  \end{equation}
  per $z \longrightarrow 1$ non tangenzialmente. Allora
  \begin{equation} \label{o^2}
    f^h(z)=1+o((z-1)^2)
  \end{equation}
  per $z \longrightarrow 1$ non tangenzialmente.
\end{prop}

\begin{proof}

  Sia $S$ un settore di vertice $1$ e angolo d'apertura $2\alpha$, e $S'$ uno un po' più grande di vertice $1$ e angolo $2\beta$, $\beta>\alpha$. Per $z \in S$, sia $C(z)$ il cerchio di centro $z$ e raggio $r(z)=\text{dist}(z, \partial S')$ (la distanza di $z$ dal bordo di $S'$). Allora per la formula integrale di Cauchy
  \begin{align*}
    f'(z) & =\frac{1}{2\pi i} \int_{C(z)} \frac{f(w)}{(w-z)^2}\diff w \\
    & =\frac{1}{2\pi i} \int_{C(z)} \frac{w-1+(f(w)-w)}{(w-z)^2}\diff w \\
    & =\frac{1}{2\pi i} \int_{C(z)} \frac{1}{w-z}\diff w+\frac{1}{2\pi i} \int_{C(z)} \frac{z-1+f(w)-w}{(w-z)^2}\diff w \\
    & =1+\frac{1}{2\pi i} \int_{C(z)} \frac{f(w)-w}{(w-z)^2}\diff w=:1+I(z).
  \end{align*}

  \begin{center}
    \definecolor{qqffqq}{rgb}{0,1,0}
    \definecolor{qqqqff}{rgb}{0,0,1}
    \definecolor{uququq}{rgb}{0.25,0.25,0.25}
    \definecolor{ffqqqq}{rgb}{1,0,0}
    \begin{tikzpicture}[line cap=round,line join=round,>=triangle 45,x=3.0cm,y=3.0cm]
      \draw[->,color=black] (-1.12,0) -- (1.2,0);
      \foreach \x in {-1,1}
      \draw[shift={(\x,0)},color=black] (0pt,2pt) -- (0pt,-2pt);
      \draw[->,color=black] (0,-1.11) -- (0,1.11);
      \foreach \y in {-1,1}
      \draw[shift={(0,\y)},color=black] (2pt,0pt) -- (-2pt,0pt);
      \clip(-1.12,-1.11) rectangle (1.2,1.11);
      \draw [shift={(1,0)},color=ffqqqq,fill=ffqqqq,fill opacity=0.1] (0,0) -- (180:0.26) arc (180:244.77:0.26) -- cycle;
      \draw [shift={(1,0)},color=qqqqff,fill=qqqqff,fill opacity=0.1] (0,0) -- (-180:0.14) arc (-180:-127.09:0.14) -- cycle;
      \draw [shift={(1,0)},color=qqffqq,fill=qqffqq,fill opacity=0.1] (0,0) -- (115.23:0.33) arc (115.23:143.3:0.33) -- cycle;
      \draw(0,0) circle (3cm);
      \draw (1,0)-- (0.27,0.96);
      \draw (0.64,0.77)-- (1,0);
      \draw (0.27,-0.96)-- (1,0);
      \draw (1,0)-- (0.64,-0.77);
      \draw [shift={(1,0)},color=ffqqqq] (180:0.26) arc (180:244.77:0.26);
      \draw [shift={(1,0)},color=ffqqqq] (180:0.23) arc (180:244.77:0.23);
      \draw(0.5,0.37) circle (0.888cm);
      \draw (0.5,0.37)-- (0.77,0.49);
      \draw (0.27,0.54)-- (1,0);
      \draw [shift={(1,0)},color=qqffqq] (115.23:0.33) arc (115.23:143.3:0.33);
      \draw [shift={(1,0)},color=qqffqq] (115.23:0.3) arc (115.23:143.3:0.3);
      \draw [shift={(1,0)},color=qqffqq] (115.23:0.28) arc (115.23:143.3:0.28);
      \begin{scriptsize}
        \draw[color=ffqqqq] (0.84,-0.09) node {$\beta$};
        \fill [color=black] (0.5,0.37) circle (1.5pt);
        \draw[color=black] (0.48,0.32) node {$z$};
        \draw[color=black] (0.18,0.14) node {$C(z)$};
        \draw[color=black] (0.54,0.465) node {$r(z)$};
        \fill [color=black] (1,0) circle (1.5pt);
        \draw[color=black] (1.04,0.05) node {$1$};
        \fill [color=black] (0.26,0.545) circle (1.5pt);
        \draw[color=black] (0.20,0.58) node {$A$};
        \fill [color=black] (1,0) circle (1.5pt);
        \draw[color=qqqqff] (0.92,-0.03) node {$\alpha$};
        \draw[color=qqffqq] (0.83,0.16) node {$\gamma$};
      \end{scriptsize}
    \end{tikzpicture}
  \end{center}

  Dato $\epsilon>0$ fissato, per ipotesi esiste $\delta>0$ tale che $|f(w)-w|<\epsilon|1-w|^3$ per ogni $w \in S'$ con $|w-1|<\delta$. $B(z,r(z)) \subset \mathbb{D} \implies r(z) \le 1-|z|$. Se $|z-1|<\delta/2$, $r(z) \le 1-|z|=|z-1-z|-|z| \le |z-1|+|z|-|z|=|z-1|<\delta/2$, dunque per ogni $w \in C(z)$ abbiamo $|w-1| \le |w-z|+|z-1|=r(z)+|z-1|<\delta$. Per questi $z$ vale che
  \begin{align*}
    |I(z)| & \le \frac{\epsilon}{2\pi} \int_0^{2\pi} \frac{|1-(z+r(z)e^{i\theta})|^3}{|(z+r(z)e^{i\theta})-z|^2}r(z)\diff\theta \\
    & \le \frac{\epsilon}{r(z)}\max_{\theta \in [0,2\pi]} |1-(z+r(z)e^{i\theta})|^3 \\
    & =\frac{\epsilon}{r(z)}\max_{w \in C(z)}|1-w|^3.
  \end{align*}
  Il massimo è raggiunto per l'intersezione più lontana da $1$ tra la circonferenza $C(z)$ e la retta passante per $1$ e $z$ (il punto $A$ in figura), perciò, detto $\gamma$ l'angolo tra $\partial S'$ (per essere precisi, la retta contenente il tratto affine più vicino a $z$) e la retta congiungente $1$ e $z$:
  \begin{align*}
    |I(z)| & \le \frac{\epsilon}{r(z)}(r(z)+|z-1|)^3 \\
    & =\epsilon r(z)^2\left(1+\frac{|z-1|}{r(z)}\right)^3 \\
    & =\epsilon r(z)^2(1+\csc\gamma)^3 \\
    & \le \epsilon r(z)^2(1+\csc(\beta-\alpha))^3 \\
    & \le \epsilon |z-1|^2(1+\csc(\beta-\alpha))^3.
  \end{align*}
  La penultima disuguaglianza segue da $\gamma \ge \beta-\alpha$ e dal fatto che $\csc$ è decrescente sui positivi, mentre l'ultima segue da quanto visto sopra. Otteniamo dunque $f'(z)=1+o((z-1)^2)$ per $z \longrightarrow 1$ non tangenzialmente.

  Adesso ci servirà il seguente lemma.

  \begin{lm} \label{opiccoli}
    Per $z \longrightarrow 1$ non tangenzialmente, $|z-1|$ e $1-|z|$ hanno gli stessi $o$-piccoli.
  \end{lm}

  \begin{proof}
    $1-|z| \le |z-1|$ l'abbiamo già visto. Per concludere ci basta dunque mostrare che, per $z$ appartenente a un settore $S$ di angolo $2\alpha$ fissato, vale una disuguaglianza opposta, a meno di una qualche costante.
    $$|z-1|=r(z)\frac{|z-1|}{r(z)} \le r(z)\csc(\beta-\alpha) \le (1-|z|)\csc(\beta-\alpha),$$
    dove $\beta>\alpha$ è stato scelto come sopra e le disuguaglianze le abbiamo già viste.
  \end{proof}

  Per ipotesi
  $$\frac{1-|f(z)|}{1-|z|}=\frac{1-|z|+o((z-1)^3)}{1-|z|}=1+o((z-1)^2)$$
  per $z \longrightarrow 1$ non tangenzialmente (abbiamo usato il lemma \ref{opiccoli} per poter usare indipendentemente $z-1$ o $1-|z|$ negli $o$-piccoli). \\
  Possiamo quindi concludere che
  $$f^h(z)=|f'(z)|\frac{1-|z|^2}{1-|f(z)|^2}=1+o((z-1)^2)$$
  per $z \longrightarrow 1$ non tangenzialmente.
\end{proof}


\subsection{Verso la disuguaglianza di Golusin}
\begin{prop} \label{24}
  Siano $f:\mathbb{D} \longrightarrow \mathbb{D}$ una funzione olomorfa che non è un automorfismo, $v \in \mathbb{D}$. Allora per ogni $z \in \mathbb{D}$ si ha che $f^*(z,v) \in \mathbb{D}$ e la funzione $z \longmapsto f^*(z,v)$ è olomorfa.
\end{prop}

\begin{proof}
  Per quanto riguarda l'olomorfia, dalla definizione sappiamo che l'unico punto che potrebbe dar problemi è $v$, ma abbiamo visto che la funzione ammette limite finito per $z \longrightarrow v$, perciò $v$ è una singolarità rimovibile. Per il lemma di Schwarz-Pick, $|f^*(z,w)| \le 1$, inoltre vale l'uguale in qualche punto solo se $f$ è un automorfismo, dunque con le ipotesi su $f$ abbiamo che vale la disuguaglianza stretta sempre, cioè $f^*(z,v) \in \mathbb{D}$ per ogni $z \in \mathbb{D}$.
\end{proof}

\begin{thm} \label{31}
  Sia $f:\mathbb{D} \longrightarrow \mathbb{D}$ una funzione olomorfa che non è un automorfismo. Allora per ogni $z, w, v \in \mathbb{D}$ vale
  \begin{equation} \label{3.1}
    d(f^*(z,v),f^*(w,v)) \le d(z,w).
  \end{equation}
\end{thm}

\begin{proof}
  Poiché $f$ non è un automorfismo, per la proposizione \ref{24} la mappa $z \longmapsto f^*(z,v)$ è olomorfa dal disco unitario in sé, perciò il membro sinistro della disuguaglianza \eqref{3.1} è ben definito. Per quanto riguarda la disuguaglianza,
  \begin{align*}
    p(f^*(z,v), f^*(w,v)) \le p(z,w) \\
    2\,\text{arctanh}\,(p(f^*(z,v), f^*(w,v))) \le 2\,\text{arctanh}\,(p(z,w)) \\
    d(f^*(z,v), f^*(w,v)) \le d(z,w),
  \end{align*}
  dove la prima riga segue dal lemma di Schwarz-Pick applicato alla funzione di cui sopra, il passaggio dalla prima alla seconda è perché $\text{arctanh}$ è crescente e dalla seconda all'ultima è la definizione di $d$.
\end{proof}

\begin{cor} \label{32}
  Sia $f:\mathbb{D} \longrightarrow \mathbb{D}$ una funzione olomorfa che non è un automorfismo. Allora per ogni $z, w, v \in \mathbb{D}$ vale
  \begin{equation}
    d(0, f^*(z,v)) \le d(0,f^*(w,v))+d(z,w).
  \end{equation}
\end{cor}

\begin{proof}
  \begin{align*}
    d(0,f^*(z,v)) & \le d(0,f^*(w,v))+d(f^*(w,v),f^*(z,v)) \\
    & \le d(0,f^*(w,v))+d(z,w),
  \end{align*}
  dove la prima è la disuguaglianza triangolare per la distanza $d$ e la seconda segue dal teorema \ref{31}.
\end{proof}

\begin{cor} \label{33}
  Sia $f:\mathbb{D} \longrightarrow \mathbb{D}$ una funzione olomorfa che non è un automorfismo. Allora per ogni $z, w, v, u \in \mathbb{D}$ vale
  \begin{equation}
    d(0, f^*(z,v)) \le d(0, f^*(u,w))+d(z,w)+d(v,u).
  \end{equation}
\end{cor}

\begin{proof}
  \begin{align*}
    d(0,f^*(z,v)) & \le d(0,f^*(w,v))+d(z,w) \\
    & =d(0,f^*(v,w))+d(z,w) \\
    & \le d(0,f^*(u,w))+d(z,w)+d(v,u),
  \end{align*}
  dove le due disuguaglianze seguono dal corollario \ref{32}.
\end{proof}

\begin{cor} \label{quasigolusin}
  Sia $f:\mathbb{D} \longrightarrow \mathbb{D}$ una funzione olomorfa che non è un automorfismo. Allora per ogni $z, w \in \mathbb{D}$ vale
  \begin{equation} \label{quasigol}
    d(f^h(z), f^h(w)) \le 2d(z,w).
  \end{equation}
\end{cor}

\begin{proof}
  Siano $z, w \in \mathbb{D}$, senza perdita di generalità possiamo supporre $f^h(z) \ge f^h(w)$. Allora
  \begin{align*}
    d(f^h(z), f^h(w)) & =\log\left(\frac{1+\frac{f^h(z)-f^h(w)}{1-f^h(w)f^h(z)}}{1-\frac{f^h(z)-f^h(w)}{1-f^h(w)f^h(z)}}\right) \\
    & =\log\left(\frac{1-f^h(w)f^h(z)+f^h(z)-f^h(w)}{1-f^h(w)f^h(z)+f^h(w)-f^h(z)}\right) \\
    & =\log\left(\frac{1+f^h(z)}{1-f^h(z)}\cdot\frac{1-f^h(w)}{1+f^h(w)}\right) \\
    & =\log\left(\frac{1+f^h(z)}{1-f^h(z)}\right)-\log\left(\frac{1+f^h(w)}{1-f^h(w)}\right) \\
    & =d(0,f^h(z))-d(0,f^h(w)) \le 2d(z,w).
  \end{align*}
  dove la disuguaglianza finale segue dal corollario \ref{33} ponendo $u=w, v=z$.
\end{proof}

Ponendo $w=0$ in \eqref{quasigol} otteniamo la disuguaglianza di Golusin, che ci servirà per dimostrare il risultato a cui puntiamo.


\newpage

\section{Dalla disuguaglianza di Golusin al teorema di Burns-Krantz}

\subsection{Rigidità al bordo}
Dalla disuguaglianza di Golusin possiamo dimostrare una versione al bordo del lemma di Schwarz-Pick, seguendo la traccia data nel remark 5.6 di \cite{BKR}.

\begin{thm} \label{boundary_schwarz_pick}
  (lemma di Schwarz-Pick al bordo) Sia $f:\mathbb{D} \longrightarrow \mathbb{D}$ una funzione olomorfa dal disco in sé tale che
  \begin{equation} \label{n_o^2}
    f^h(z_n)=1+o((|z_n|-1)^2)
  \end{equation}
  per qualche successione $\{z_n\}_{n \in \mathbb{N}} \subset \mathbb{D}$ con $|z_n| \longrightarrow 1$. Allora $f$ è un automorfismo.
\end{thm}

\begin{proof}
  Supponiamo per assurdo che $f$ non sia un automorfismo. Allora possiamo applicare il corollario \ref{quasigolusin}, che per $w=0$ ci dà
  \begin{align*}
    d(f^h(z), f^h(0)) \le 2d(z,0) \\
    \log{\left(\frac{1+\left|\frac{f^h(z)-f^h(0)}{1-f^h(z)f^h(0)}\right|}{1-\left|\frac{f^h(z)-f^h(0)}{1-f^h(z)f^h(0)}\right|}\right)} \le 2\log{\left(\frac{1+|z|}{1-|z|}\right)} \\
    \frac{|1-f^h(z)f^h(0)|+|f^h(z)-f^h(0)|}{|1-f^h(z)f^h(0)|-|f^h(z)-f^h(0)|} \le \frac{(1+|z|)^2}{(1-|z|)^2}.
  \end{align*}
  Ricordiamo che per definizione $f^h(z) \ge 0$ e inoltre per il lemma di Schwarz-Pick $f^h(z) \le 1$, per ogni $z \in \mathbb{D}$. Sempre per il lemma originale, se valesse $f^h(0)=1$ avremmo che $f$ è un automorfismo, contraddizione. Perciò dev'essere $f^h(0)<1$, ma $\displaystyle \lim_{n \longrightarrow +\infty} f^h(z_n)=1$, quindi definitivamente $f^h(z_n)-f^h(0)>0$ e $1-f^h(z_n)f^h(0)>0$, da cui
  \begin{align*}
    \frac{(1-f^h(0))(1+f^h(z_n))}{(1-f^h(z_n))(1+f^h(0))} \le \frac{(1+|z_n|)^2}{(1-|z_n|)^2} \\
    \frac{1+f^h(0)}{(1-f^h(0))(1+f^h(z_n))}(1-f^h(z_n)) \ge \frac{(1-|z_n|)^2}{(1+|z_n|)^2}.
  \end{align*}
  Per ipotesi vale \eqref{n_o^2}, dunque
  \begin{align*}
    \frac{1+f^h(0)}{(1-f^h(0))(1+f^h(z_n))}o((|z_n|-1)^2) \ge \frac{(1-|z_n|)^2}{(1+|z_n|)^2} \\
    \frac{(1+f^h(0))(1+|z_n|)^2}{(1-f^h(0))(1+f^h(z_n))}o((|z_n|-1)^2) \ge 1.
  \end{align*}
  Poiché $\displaystyle \lim_{n \longrightarrow +\infty} \frac{(1+f^h(0))(1+|z_n|)^2}{(1-f^h(0))(1+f^h(z_n))}=\frac{2(1+f^h(0))}{1-f^h(0)} < +\infty$, otteniamo di nuovo una contraddizione.
\end{proof}

Siamo ora pronti a dimostrare il teorema 2.1 di \cite{BK}.


\subsection{Teorema di Burns-Krantz}
\marginpar{Magari dire qualcosa sull'utilità di questa proposizione, che forse verrà spostata nei prerequisiti}

\begin{prop} \label{o^3->o^2}
  Sia $f:\mathbb{D} \longrightarrow \mathbb{D}$ una funzione tale che
  \begin{equation} \label{o^3}
    f(z)=1+(z-1)+o(|z-1|^3)
  \end{equation}
  per $z \longrightarrow 1$ non tangenzialmente. Allora
  \begin{equation} \label{o^2}
    f^h(z)=1+o(|z-1|^2)
  \end{equation}
  per $z \longrightarrow 1$ non tangenzialmente.
\end{prop}

\marginpar{ricordati di definire $f^h$, con la notazione di BKR, quindi occhio quando scrivi tutti i risultati e le dimostrazioni in BM}

\begin{proof}
  Da scrivere, praticamente va copiata.
\end{proof}

Siamo ora pronti a dimostrare il teorema 2.1 di \cite{BK}.

\begin{thm} \label{burns_krantz}
  (Burns-Krantz, 1994) Sia $f:\mathbb{D} \longrightarrow \mathbb{D}$ una funzione olomorfa dal disco in sé tale che
  \begin{equation} \label{O^4}
    f(z)=1+(z-1)+\mathcal{O}(|z-1|^4)
  \end{equation}
  per $z \longrightarrow 1$. Allora $f$ è l'identità sul disco.
\end{thm}

\marginpar{Servono i risultati visti in BKR; poi: è comprensibile? Da rivedere in seguito}

\begin{proof}
  Se vale l'ipotesi \eqref{O^4} per $z \longrightarrow 1$ vale anche \eqref{o^3}, in particolare per $z \longrightarrow 1$ non tangenzialmente. Dalla proposizione \ref{o^3->o^2} segue che anche \eqref{o^2} vale per $z \longrightarrow 1$ non tangenzialmente, quindi esiste una successione $z_n$ che soddisfa le ipotesi del teorema \ref{boundary_schwarz_pick}, da cui la tesi.
\end{proof}

Aggiungere controesempio


\newpage

\begin{thebibliography}{widest entry}
  \bibitem[BK]{BK} D. M. Burns, S. G. Krantz: Rigidity of holomorphic mappings and a new Schwarz lemma at the boundary. \textit{Journal of the American Mathematical Society}, \textbf{Volume 7} (1994), no. 3, 661--676
  \bibitem[BKR]{BKR} F. Bracci, D. Kraus, O. Roth: A new Schwarz-Pick Lemma at the boundary and rigidity of holomorphic maps. Preprint, ArXiv:2003.02019v1 (2020)
  \bibitem[BM]{BM} A. F. Beardon, D. Minda: A multi-point Schwarz-Pick lemma. \textit{Journal d'Analyse Mathématique}, \textbf{Volume 92} (2004), 81--104
  \bibitem[NN]{NN} N. Narasimhan, Y. Nievergelt: \textbf{Complex analysis in one variable (2nd edition)}. Springer, INSERIRE CITTÀ, 2001
\end{thebibliography}


\section*{Ringraziamenti}
\addcontentsline{toc}{section}{Ringraziamenti}
Un sentito grazie al professor Marco Abate, che è riuscito a tenere il suo interessantissimo corso ugualmente appassionante anche con la didattica a distanza.


\end{document}
