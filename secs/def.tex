\begin{frame}
  \frametitle{Domini strettamente pseudoconvessi e metrica di Kobayashi}
  \only<1>{\begin{set}
      Fissiamo $\Omega \subseteq \mathbb{C}^n, n \ge 2$ un dominio limitato con bordo $C^2$, cioè esiste $\rho \in C^2(\mathbb{C}^n)$ tale che $\Omega=\{\rho(z)<0\}$.
      Dato $p \in \partial\Omega$, lo \textit{spazio tangente complesso} a $\partial\Omega$ in $p$ è $H_p\partial\Omega=\{Z \in \mathbb{C}^n \mid \langle \bar{\partial}\rho(p),Z\rangle=0\}$. Assumiamo che $\Omega$ sia \textit{strettamente pseudoconvesso}, cioè la \textit{forma di Levi}
      $$L_{\rho}(p;Z)=\sum_{\nu,\mu=1}^n \frac{\partial\rho^2}{\partial z_\nu\partial\bar{z}_\mu}(p)Z_\nu\bar{Z}_\mu, \quad Z=(Z_1,\dots,Z_n) \in \mathbb{C}^n$$
      è definita positiva in $H_p\partial\Omega$ per ogni $p \in \Omega$.
  \end{set}}
  \only<2>{
  \begin{defn}
    Sia $\mathbb{D}$ il disco unitario in $\mathbb{C}$, data $f:\mathbb{D} \longrightarrow \mathbb{C}^n$ olomorfa indichiamo con $Df(z)$ il differenziale di $f$ in $z \in \mathbb{D}$. La \textit{metrica di Kobayashi} su $\Omega$ è
    $$K(x;Z)=\inf\{|v| \mid v \in \mathbb{C}, \text{ esiste }f:\mathbb{D} \longrightarrow \Omega$$
    $$\text{ olomorfa con } f(0)=x, Df(0)v=Z\},$$
    che induce la \textit{distanza di Kobayashi} $d_K$.
  \end{defn}
  }
\end{frame}

\begin{frame}[t]
  \frametitle{Domini strettamente pseudoconvessi e iperbolicità di Gromov}
  \begin{defn}
    Sia $(X,d)$ uno spazio metrico, dati $x,y \in X$ il \textit{prodotto di Gromov} con punto base $w$ è $(x,y)_w=\frac{1}{2}\big(d(x,w)+d(y,w)-d(x,y)\big)$. Dato $\delta \ge 0$, diciamo che $X$ è \textit{$\delta$-iperbolico} se
  $$(x,y)_w \ge (x,z)_w\wedge(y,z)_w-\delta\text{ per ogni }x,y,z,w \in X.$$
  Fissato $w \in X$, il \textit{bordo iperbolico} è $\partial_GX$ costruito come classe di equivalenza delle successioni $(x_i)$ tali che $\displaystyle\lim_{i,j\rightarrow \infty}(x_i,x_j)=\infty$; due tali successioni $(x_i), (y_i)$ sono equivalenti se $\displaystyle\lim_{i\rightarrow \infty}(x_i,y_i)=\infty$.
  \end{defn}
  \pause
  \begin{thm}
    (Balogh-Bonk) $(\Omega,d_K)$ è Gromov iperbolico, e il bordo iperbolico $\partial_G\Omega$ può essere identificato con il bordo euclideo $\partial\Omega$.
  \end{thm}
\end{frame}
